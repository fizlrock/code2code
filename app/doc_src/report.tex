
\documentclass[a4paper, 12pt]{article}
\usepackage[left=3cm,right=1.5cm,top=2cm,bottom=2cm]{geometry}
\usepackage[T2A]{fontenc}
\usepackage{graphicx}
\usepackage[table]{xcolor}

%Hyphenation rules
%--------------------------------------
\usepackage{hyphenat}
\hyphenation{ма-те-ма-ти-ка вос-ста-нав-ли-вать}
%--------------------------------------
\usepackage[english, russian]{babel}
\begin{document}
 
\tableofcontents

\pagebreak

\begin{abstract}
  Это вводный абзац в начале документа.
\end{abstract}
 
\section{Задание}
\begin{enumerate}
\item Составить таблицу кодов блоков для метода Хаффмана с блокированием. Вероятности букв считать по фрагменту сообщения в задании. Длина блока указана. Вычислить EX, ML(X), ML(Xбл). Здесь EX – энтропия алфавита из букв сообщения, ML(X) – среднее количество элементарных символов на букву при сжатии методом Хаффмана, ML(Xбл) – среднее количество элементарных символов на букву при сжатии методом Хаффмана с блокированием. 
\item Сжать сообщение адаптивным методом Хаффмана. 
\item Сжать сообщение методами LZ77, LZSS, LZ78  Для методов LZ77, LZSS размер словаря – 10 символов, буфера – 6 символов. Для метода LZ78 размер словаря 32 записи. 
\item Сжать сообщение из задания №2 арифметическим методом. 
\item Распаковать сообщения, сжатые адаптивным методом Хаффмана, методами LZ77, LZSS, LZ78 и арифметическим методом. Для методов LZ77, LZSS размер словаря – 10 символов. Для метода LZ78 размер словаря – 16 записей. При декодировании таблица состоит из следующих столбцов: «Код», «Словарь» и «Выходной поток».
\end{enumerate}
\pagebreak
\section{Решение}
\subsection{Вариант №1}
\paragraph{Задание 1. Блочный хаффман \\}

Строка ББААББББББ, размер блока: 3
\begin{center}
 \begin{tabular}{ |c|c|l| } 
  \hline
     Буква & Вероятность & Код\\ \hline
Б & 0.80 & 1\\\hline
А & 0.20 & 0
\\ \hline \end{tabular}
\end{center}
Энтропия алфавита: 0.7219
\begin{center}
 \begin{tabular}{ |c|c|l| } 
  \hline
     Блок & Вероятность & Код\\ \hline
БББ & 0.51 & 1\\\hline
БАБ & 0.13 & 001\\\hline
АББ & 0.13 & 010\\\hline
ББА & 0.13 & 011\\\hline
АБА & 0.03 & 00001\\\hline
ААБ & 0.03 & 00010\\\hline
БАА & 0.03 & 00011\\\hline
ААА & 0.01 & 00000
\\ \hline \end{tabular}
\end{center}
Бит на символ при посимвольном кодировании: 1.0000, при блочном: 0.7280


\pagebreak
\paragraph{Задание 2. Сжать адаптивным хаффманом\\}

Строка: 
КЕЕЕНООННН\\
Результат: 'К' 0'Е' 01 1 00'Н' 000'О' 0101 1111 111 10










\pagebreak
\paragraph{Задание 3.1}

Закодировать сообщение методом LZ77\\
Строка:КУКУКУ\_КУКУШКА\_КУКИШ\\
Результат: <0,0,К> <0,0,У> <8,4,\_> <3,4,Ш> <0,1,А> <2,4,И> <0,0,Ш>\\
\begin{table}[h!]
\centering
\begin{tabular}{|c|c|c|c|c|c|c|c|c|c|c|c|c|c|c|c|c|} 
\hline
\multicolumn{10}{|c|}{Cловарь} & \multicolumn{6}{c|}{Буфер} & Код  \\ \hline
  &   &   &   &   &   &   &   &   &   & \cellcolor[HTML]{8CE4F6} К & У & К & У & К & У & <0,0,К>
\\ \hline
  &   &   &   &   &   &   &   &   & К & \cellcolor[HTML]{8CE4F6} У & К & У & К & У &   & <0,0,У>
\\ \hline
  &   &   &   &   &   &   &   & \cellcolor[HTML]{FFFF00} К & \cellcolor[HTML]{FFFF00} У & \cellcolor[HTML]{FFFF00} К & \cellcolor[HTML]{FFFF00} У & \cellcolor[HTML]{FFFF00} К & \cellcolor[HTML]{FFFF00} У & \cellcolor[HTML]{8CE4F6}   & К & <8,4,\_>
\\ \hline
  &   &   & \cellcolor[HTML]{FFFF00} К & \cellcolor[HTML]{FFFF00} У & \cellcolor[HTML]{FFFF00} К & \cellcolor[HTML]{FFFF00} У & К & У &   & \cellcolor[HTML]{FFFF00} К & \cellcolor[HTML]{FFFF00} У & \cellcolor[HTML]{FFFF00} К & \cellcolor[HTML]{FFFF00} У & \cellcolor[HTML]{8CE4F6} Ш & К & <3,4,Ш>
\\ \hline
\cellcolor[HTML]{FFFF00} К & У & К & У &   & К & У & К & У & Ш & \cellcolor[HTML]{FFFF00} К & \cellcolor[HTML]{8CE4F6} А &   & К & У & К & <0,1,А>
\\ \hline
К & У & \cellcolor[HTML]{FFFF00}   & \cellcolor[HTML]{FFFF00} К & \cellcolor[HTML]{FFFF00} У & \cellcolor[HTML]{FFFF00} К & У & Ш & К & А & \cellcolor[HTML]{FFFF00}   & \cellcolor[HTML]{FFFF00} К & \cellcolor[HTML]{FFFF00} У & \cellcolor[HTML]{FFFF00} К & \cellcolor[HTML]{8CE4F6} И & Ш & <2,4,И>
\\ \hline
К & У & Ш & К & А &   & К & У & К & И & \cellcolor[HTML]{8CE4F6} Ш &   &   &   &   &   & <0,0,Ш>
\\ \hline
\end{tabular}
\end{table}

\paragraph{Задание 3.3}

Закодировать сообщение методом LZ78\\
Строка:КУКУКУ\_КУКУШКА\_КУКИШ\\
\begin{table}[h!]
\centering
\begin{tabular}{|c|c|c|} 
\hline
 Входная фраза (в словарь) & Код & Позиция словаря \\ \hline

 &  & 0 \\ \hline
К & 0'К' & 1 \\ \hline
У & 0'У' & 2 \\ \hline
КУ & 1'У' & 3 \\ \hline
КУ\_ & 3'\_' & 4 \\ \hline
КУК & 3'К' & 5 \\ \hline
УШ & 2'Ш' & 6 \\ \hline
КА & 1'А' & 7 \\ \hline
\_ & 0'\_' & 8 \\ \hline
КУКИ & 5'И' & 9 \\ \hline
Ш & 0'Ш' & 10 \\ \hline
\end{tabular}
\end{table}

Результат: 0'К' 0'У' 1'У' 3'\_' 3'К' 2'Ш' 1'А' 0'\_' 5'И' 0'Ш'\\
\pagebreak
\paragraph{Задание 4. Арифметическое кодирование\\}

Исходная строка: КЕЕЕНООННН\
\begin{center}
 \begin{tabular}{ |c|c| } 
  \hline
     Буква & Вероятность \\ \hline
Н & 0.40\\\hline
Е & 0.30\\\hline
О & 0.20\\\hline
К & 0.10
\\ \hline \end{tabular}
\end{center}
\begin{center}
 \begin{tabular}{ |c|c|c| } 
  \hline
     Буква & Начало & Конец \\ \hline
Н & 0.00 & 0.40\\\hline
Е & 0.40 & 0.70\\\hline
О & 0.70 & 0.90\\\hline
К & 0.90 & 1.00
\\ \hline \end{tabular}
\end{center}
\begin{center}
 \begin{tabular}{ |c|c|c|c| } 
  \hline
     Буква & delta & min & max \\ \hline
К & 0.1000000000 & 0.9000000000 & 1.0000000000\\\hline
Е & 0.0300000000 & 0.9400000000 & 0.9700000000\\\hline
Е & 0.0090000000 & 0.9520000000 & 0.9610000000\\\hline
Е & 0.0027000000 & 0.9556000000 & 0.9583000000\\\hline
Н & 0.0010800000 & 0.9556000000 & 0.9566800000\\\hline
О & 0.0002160000 & 0.9563560000 & 0.9565720000\\\hline
О & 0.0000432000 & 0.9565072000 & 0.9565504000\\\hline
Н & 0.0000172800 & 0.9565072000 & 0.9565244800\\\hline
Н & 0.0000069120 & 0.9565072000 & 0.9565141120\\\hline
Н & 0.0000027648 & 0.9565072000 & 0.9565099648
\\ \hline \end{tabular}
\end{center}
Результат: 956508
\pagebreak
\paragraph{Задание 5.1}

\\ 

Декодировать сообщение методом адаптивного хаффмана \\
Строка: 
'О'0'Р'00'П'100'Н'11011001001111\\
Результат: ОРПНРПППНН










\paragraph{Задание 5.3 Декодировать строку(LZ78)\\}

Исходная строка: [0'д'] [0'о'] [0'р'] [2'г'] [2' '] [3'о'] [0'г'] [0'а'] [0' '] [7'о'] [3'а'] [9'р'] [2'г']\\
\begin{table}[h!]
\centering
\begin{tabular}{|c|c|c|} 
\hline
 Код & Словарь & Выходной поток 
\hline

 & [] & 
\\ \hline
0'д' & [, д] & д
\\ \hline
0'о' & [, д, о] & о
\\ \hline
0'р' & [, д, о, р] & р
\\ \hline
2'г' & [, д, о, р, ог] & ог
\\ \hline
2' ' & [, д, о, р, ог, о ] & о 
\\ \hline
3'о' & [, д, о, р, ог, о , ро] & ро
\\ \hline
0'г' & [, д, о, р, ог, о , ро, г] & г
\\ \hline
0'а' & [, д, о, р, ог, о , ро, г, а] & а
\\ \hline
0' ' & [, д, о, р, ог, о , ро, г, а,  ] &  
\\ \hline
7'о' & [, д, о, р, ог, о , ро, г, а,  , го] & го
\\ \hline
3'а' & [, д, о, р, ог, о , ро, г, а,  , го, ра] & ра
\\ \hline
9'р' & [, д, о, р, ог, о , ро, г, а,  , го, ра,  р] &  р
\\ \hline
2'г' & [, д, о, р, ог, о , ро, г, а,  , го, ра,  р, ог] & ог
\\ \hline
\end{tabular}
\end{table}

Результат: дорого рога гора рог
\pagebreak
\subsection{Вариант №2}
\paragraph{Задание 1. Блочный хаффман \\}

Строка САСВВАВВВС, размер блока: 2
\begin{center}
 \begin{tabular}{ |c|c|l| } 
  \hline
     Буква & Вероятность & Код\\ \hline
В & 0.50 & 0\\\hline
С & 0.30 & 11\\\hline
А & 0.20 & 10
\\ \hline \end{tabular}
\end{center}
Энтропия алфавита: 1.4855
\begin{center}
 \begin{tabular}{ |c|c|l| } 
  \hline
     Блок & Вероятность & Код\\ \hline
ВВ & 0.25 & 01\\\hline
СВ & 0.15 & 101\\\hline
ВС & 0.15 & 110\\\hline
АВ & 0.10 & 000\\\hline
ВА & 0.10 & 001\\\hline
СС & 0.09 & 1111\\\hline
АС & 0.06 & 1001\\\hline
СА & 0.06 & 1110\\\hline
АА & 0.04 & 1000
\\ \hline \end{tabular}
\end{center}
Бит на символ при посимвольном кодировании: 1.5000, при блочном: 1.5000


\pagebreak
\paragraph{Задание 2. Сжать адаптивным хаффманом\\}

Строка: 
АББААСКААС\\
Результат: 'А' 0'Б' 01 01 01 00'С' 000'К' 0 0 001










\pagebreak

\paragraph{Задание 3.3}

Закодировать сообщение методом LZ78\\
Строка:ЛЯЛЯЛЯ\_ЛЯЛЯ\_ЯЛИК\_МЯЛ\\
\begin{table}[h!]
\centering
\begin{tabular}{|c|c|c|} 
\hline
 Входная фраза (в словарь) & Код & Позиция словаря \\ \hline

 &  & 0 \\ \hline
Л & 0'Л' & 1 \\ \hline
Я & 0'Я' & 2 \\ \hline
ЛЯ & 1'Я' & 3 \\ \hline
ЛЯ\_ & 3'\_' & 4 \\ \hline
ЛЯЛ & 3'Л' & 5 \\ \hline
Я\_ & 2'\_' & 6 \\ \hline
ЯЛ & 2'Л' & 7 \\ \hline
И & 0'И' & 8 \\ \hline
К & 0'К' & 9 \\ \hline
\_ & 0'\_' & 10 \\ \hline
М & 0'М' & 11 \\ \hline
\end{tabular}
\end{table}

Результат: 0'Л' 0'Я' 1'Я' 3'\_' 3'Л' 2'\_' 2'Л' 0'И' 0'К' 0'\_' 0'М'\\
\pagebreak
\paragraph{Задание 4. Арифметическое кодирование\\}

Исходная строка: АББААСКААС\
\begin{center}
 \begin{tabular}{ |c|c| } 
  \hline
     Буква & Вероятность \\ \hline
А & 0.50\\\hline
Б & 0.20\\\hline
С & 0.20\\\hline
К & 0.10
\\ \hline \end{tabular}
\end{center}
\begin{center}
 \begin{tabular}{ |c|c|c| } 
  \hline
     Буква & Начало & Конец \\ \hline
А & 0.00 & 0.50\\\hline
Б & 0.50 & 0.70\\\hline
С & 0.70 & 0.90\\\hline
К & 0.90 & 1.00
\\ \hline \end{tabular}
\end{center}
\begin{center}
 \begin{tabular}{ |c|c|c|c| } 
  \hline
     Буква & delta & min & max \\ \hline
А & 0.5000000000 & 0.0000000000 & 0.5000000000\\\hline
Б & 0.1000000000 & 0.2500000000 & 0.3500000000\\\hline
Б & 0.0200000000 & 0.3000000000 & 0.3200000000\\\hline
А & 0.0100000000 & 0.3000000000 & 0.3100000000\\\hline
А & 0.0050000000 & 0.3000000000 & 0.3050000000\\\hline
С & 0.0010000000 & 0.3035000000 & 0.3045000000\\\hline
К & 0.0001000000 & 0.3044000000 & 0.3045000000\\\hline
А & 0.0000500000 & 0.3044000000 & 0.3044500000\\\hline
А & 0.0000250000 & 0.3044000000 & 0.3044250000\\\hline
С & 0.0000050000 & 0.3044175000 & 0.3044225000
\\ \hline \end{tabular}
\end{center}
Результат: 30442
\pagebreak
\paragraph{Задание 5.1}

\\ 

Декодировать сообщение методом адаптивного хаффмана \\
Строка: 
'R'0'T'01100'N'010111100'D'1001\\
Результат: RTTTNRRRDD










\paragraph{Задание 5.3 Декодировать строку(LZ78)\\}

Исходная строка: [0'м'] [0'и'] [0'р'] [0' '] [0'п'] [2'р'] [4'т'] [6' '] [0'т'] [2'г'] [0'р']\\
\begin{table}[h!]
\centering
\begin{tabular}{|c|c|c|} 
\hline
 Код & Словарь & Выходной поток 
\hline

 & [] & 
\\ \hline
0'м' & [, м] & м
\\ \hline
0'и' & [, м, и] & и
\\ \hline
0'р' & [, м, и, р] & р
\\ \hline
0' ' & [, м, и, р,  ] &  
\\ \hline
0'п' & [, м, и, р,  , п] & п
\\ \hline
2'р' & [, м, и, р,  , п, ир] & ир
\\ \hline
4'т' & [, м, и, р,  , п, ир,  т] &  т
\\ \hline
6' ' & [, м, и, р,  , п, ир,  т, ир ] & ир 
\\ \hline
0'т' & [, м, и, р,  , п, ир,  т, ир , т] & т
\\ \hline
2'г' & [, м, и, р,  , п, ир,  т, ир , т, иг] & иг
\\ \hline
0'р' & [, м, и, р,  , п, ир,  т, ир , т, иг, р] & р
\\ \hline
\end{tabular}
\end{table}

Результат: мир пир тир тигр
\pagebreak
\subsection{Вариант №3}
\paragraph{Задание 1. Блочный хаффман \\}

Строка ТИИИИККККК, размер блока: 2
\begin{center}
 \begin{tabular}{ |c|c|l| } 
  \hline
     Буква & Вероятность & Код\\ \hline
К & 0.50 & 0\\\hline
И & 0.40 & 11\\\hline
Т & 0.10 & 10
\\ \hline \end{tabular}
\end{center}
Энтропия алфавита: 1.3610
\begin{center}
 \begin{tabular}{ |c|c|l| } 
  \hline
     Блок & Вероятность & Код\\ \hline
КК & 0.25 & 10\\\hline
ИК & 0.20 & 00\\\hline
КИ & 0.20 & 01\\\hline
ИИ & 0.16 & 110\\\hline
ТК & 0.05 & 11101\\\hline
КТ & 0.05 & 11110\\\hline
ТИ & 0.04 & 111111\\\hline
ИТ & 0.04 & 11100\\\hline
ТТ & 0.01 & 111110
\\ \hline \end{tabular}
\end{center}
Бит на символ при посимвольном кодировании: 1.5000, при блочном: 1.3900


\pagebreak
\paragraph{Задание 2. Сжать адаптивным хаффманом\\}

Строка: 
ПРОВППРРРО\\
Результат: 'П' 0'Р' 00'О' 100'В' 10 11 10 10 11 101










\pagebreak
\paragraph{Задание 3.1}

Закодировать сообщение методом LZ77\\
Строка:ТАРАРА\_ТАРТАР\_ТАРТ\_ТАРА\\
Результат: <0,0,Т> <0,0,А> <0,0,Р> <8,3,\_> <3,3,Т> <0,2,\_> <3,4,\_> <1,3,А>\\
\begin{table}[h!]
\centering
\begin{tabular}{|c|c|c|c|c|c|c|c|c|c|c|c|c|c|c|c|c|} 
\hline
\multicolumn{10}{|c|}{Cловарь} & \multicolumn{6}{c|}{Буфер} & Код  \\ \hline
  &   &   &   &   &   &   &   &   &   & \cellcolor[HTML]{8CE4F6} Т & А & Р & А & Р & А & <0,0,Т>
\\ \hline
  &   &   &   &   &   &   &   &   & Т & \cellcolor[HTML]{8CE4F6} А & Р & А & Р & А &   & <0,0,А>
\\ \hline
  &   &   &   &   &   &   &   & Т & А & \cellcolor[HTML]{8CE4F6} Р & А & Р & А &   & Т & <0,0,Р>
\\ \hline
  &   &   &   &   &   &   & Т & \cellcolor[HTML]{FFFF00} А & \cellcolor[HTML]{FFFF00} Р & \cellcolor[HTML]{FFFF00} А & \cellcolor[HTML]{FFFF00} Р & \cellcolor[HTML]{FFFF00} А & \cellcolor[HTML]{8CE4F6}   & Т & А & <8,3,\_>
\\ \hline
  &   &   & \cellcolor[HTML]{FFFF00} Т & \cellcolor[HTML]{FFFF00} А & \cellcolor[HTML]{FFFF00} Р & А & Р & А &   & \cellcolor[HTML]{FFFF00} Т & \cellcolor[HTML]{FFFF00} А & \cellcolor[HTML]{FFFF00} Р & \cellcolor[HTML]{8CE4F6} Т & А & Р & <3,3,Т>
\\ \hline
\cellcolor[HTML]{FFFF00} А & \cellcolor[HTML]{FFFF00} Р & А & Р & А &   & Т & А & Р & Т & \cellcolor[HTML]{FFFF00} А & \cellcolor[HTML]{FFFF00} Р & \cellcolor[HTML]{8CE4F6}   & Т & А & Р & <0,2,\_>
\\ \hline
Р & А &   & \cellcolor[HTML]{FFFF00} Т & \cellcolor[HTML]{FFFF00} А & \cellcolor[HTML]{FFFF00} Р & \cellcolor[HTML]{FFFF00} Т & А & Р &   & \cellcolor[HTML]{FFFF00} Т & \cellcolor[HTML]{FFFF00} А & \cellcolor[HTML]{FFFF00} Р & \cellcolor[HTML]{FFFF00} Т & \cellcolor[HTML]{8CE4F6}   & Т & <3,4,\_>
\\ \hline
Р & \cellcolor[HTML]{FFFF00} Т & \cellcolor[HTML]{FFFF00} А & \cellcolor[HTML]{FFFF00} Р &   & Т & А & Р & Т &   & \cellcolor[HTML]{FFFF00} Т & \cellcolor[HTML]{FFFF00} А & \cellcolor[HTML]{FFFF00} Р & \cellcolor[HTML]{8CE4F6} А &   &   & <1,3,А>
\\ \hline
\end{tabular}
\end{table}

\paragraph{Задание 3.3}

Закодировать сообщение методом LZ78\\
Строка:ТАРАРА\_ТАРТАР\_ТАРТ\_ТАРА\\
\begin{table}[h!]
\centering
\begin{tabular}{|c|c|c|} 
\hline
 Входная фраза (в словарь) & Код & Позиция словаря \\ \hline

 &  & 0 \\ \hline
Т & 0'Т' & 1 \\ \hline
А & 0'А' & 2 \\ \hline
Р & 0'Р' & 3 \\ \hline
АР & 2'Р' & 4 \\ \hline
А\_ & 2'\_' & 5 \\ \hline
ТА & 1'А' & 6 \\ \hline
РТ & 3'Т' & 7 \\ \hline
АР\_ & 4'\_' & 8 \\ \hline
ТАР & 6'Р' & 9 \\ \hline
Т\_ & 1'\_' & 10 \\ \hline
ТАРА & 9'А' & 11 \\ \hline
\end{tabular}
\end{table}

Результат: 0'Т' 0'А' 0'Р' 2'Р' 2'\_' 1'А' 3'Т' 4'\_' 6'Р' 1'\_' 9'А'\\
\pagebreak
\paragraph{Задание 4. Арифметическое кодирование\\}

Исходная строка: ПРОВППРРРО\
\begin{center}
 \begin{tabular}{ |c|c| } 
  \hline
     Буква & Вероятность \\ \hline
Р & 0.40\\\hline
П & 0.30\\\hline
О & 0.20\\\hline
В & 0.10
\\ \hline \end{tabular}
\end{center}
\begin{center}
 \begin{tabular}{ |c|c|c| } 
  \hline
     Буква & Начало & Конец \\ \hline
Р & 0.00 & 0.40\\\hline
П & 0.40 & 0.70\\\hline
О & 0.70 & 0.90\\\hline
В & 0.90 & 1.00
\\ \hline \end{tabular}
\end{center}
\begin{center}
 \begin{tabular}{ |c|c|c|c| } 
  \hline
     Буква & delta & min & max \\ \hline
П & 0.3000000000 & 0.4000000000 & 0.7000000000\\\hline
Р & 0.1200000000 & 0.4000000000 & 0.5200000000\\\hline
О & 0.0240000000 & 0.4840000000 & 0.5080000000\\\hline
В & 0.0024000000 & 0.5056000000 & 0.5080000000\\\hline
П & 0.0007200000 & 0.5065600000 & 0.5072800000\\\hline
П & 0.0002160000 & 0.5068480000 & 0.5070640000\\\hline
Р & 0.0000864000 & 0.5068480000 & 0.5069344000\\\hline
Р & 0.0000345600 & 0.5068480000 & 0.5068825600\\\hline
Р & 0.0000138240 & 0.5068480000 & 0.5068618240\\\hline
О & 0.0000027648 & 0.5068576768 & 0.5068604416
\\ \hline \end{tabular}
\end{center}
Результат: 50686
\pagebreak
\paragraph{Задание 5.1}

\\ 

Декодировать сообщение методом адаптивного хаффмана \\
Строка: 
'S'0'D'00'A'1101000'R'011001001\\
Результат: SDADDDRAAR










\paragraph{Задание 5.3 Декодировать строку(LZ78)\\}

Исходная строка: [0'г'] [0'о'] [0'р'] [2'д'] [0' '] [1'о'] [3'а'] [5'р'] [4' '] [3'о'] [0'г']\\
\begin{table}[h!]
\centering
\begin{tabular}{|c|c|c|} 
\hline
 Код & Словарь & Выходной поток 
\hline

 & [] & 
\\ \hline
0'г' & [, г] & г
\\ \hline
0'о' & [, г, о] & о
\\ \hline
0'р' & [, г, о, р] & р
\\ \hline
2'д' & [, г, о, р, од] & од
\\ \hline
0' ' & [, г, о, р, од,  ] &  
\\ \hline
1'о' & [, г, о, р, од,  , го] & го
\\ \hline
3'а' & [, г, о, р, од,  , го, ра] & ра
\\ \hline
5'р' & [, г, о, р, од,  , го, ра,  р] &  р
\\ \hline
4' ' & [, г, о, р, од,  , го, ра,  р, од ] & од 
\\ \hline
3'о' & [, г, о, р, од,  , го, ра,  р, од , ро] & ро
\\ \hline
0'г' & [, г, о, р, од,  , го, ра,  р, од , ро, г] & г
\\ \hline
\end{tabular}
\end{table}

Результат: город гора род рог
\pagebreak
\subsection{Вариант №4}
\paragraph{Задание 1. Блочный хаффман \\}

Строка ДДУДУУУУУУ, размер блока: 3
\begin{center}
 \begin{tabular}{ |c|c|l| } 
  \hline
     Буква & Вероятность & Код\\ \hline
У & 0.70 & 1\\\hline
Д & 0.30 & 0
\\ \hline \end{tabular}
\end{center}
Энтропия алфавита: 0.8813
\begin{center}
 \begin{tabular}{ |c|c|l| } 
  \hline
     Блок & Вероятность & Код\\ \hline
УУУ & 0.34 & 11\\\hline
УДУ & 0.15 & 101\\\hline
ДУУ & 0.15 & 00\\\hline
УУД & 0.15 & 100\\\hline
УДД & 0.06 & 0101\\\hline
ДУД & 0.06 & 0110\\\hline
ДДУ & 0.06 & 0111\\\hline
ДДД & 0.03 & 0100
\\ \hline \end{tabular}
\end{center}
Бит на символ при посимвольном кодировании: 1.0000, при блочном: 0.9087


\pagebreak
\paragraph{Задание 2. Сжать адаптивным хаффманом\\}

Строка: 
АППРОПММММ\\
Результат: 'А' 0'П' 01 00'Р' 000'О' 0 1100'М' 1001 111 10










\pagebreak
\paragraph{Задание 3.1}

Закодировать сообщение методом LZ77\\
Строка:СЫР\_СЫН\_СЫРОК\_СЫНОК\\
Результат: <0,0,С> <0,0,Ы> <0,0,Р> <0,0,\_> <6,2,Н> <6,3,Р> <0,0,О> <0,0,К> <0,4,О> <0,0,К>\\
\begin{table}[h!]
\centering
\begin{tabular}{|c|c|c|c|c|c|c|c|c|c|c|c|c|c|c|c|c|} 
\hline
\multicolumn{10}{|c|}{Cловарь} & \multicolumn{6}{c|}{Буфер} & Код  \\ \hline
  &   &   &   &   &   &   &   &   &   & \cellcolor[HTML]{8CE4F6} С & Ы & Р &   & С & Ы & <0,0,С>
\\ \hline
  &   &   &   &   &   &   &   &   & С & \cellcolor[HTML]{8CE4F6} Ы & Р &   & С & Ы & Н & <0,0,Ы>
\\ \hline
  &   &   &   &   &   &   &   & С & Ы & \cellcolor[HTML]{8CE4F6} Р &   & С & Ы & Н &   & <0,0,Р>
\\ \hline
  &   &   &   &   &   &   & С & Ы & Р & \cellcolor[HTML]{8CE4F6}   & С & Ы & Н &   & С & <0,0,\_>
\\ \hline
  &   &   &   &   &   & \cellcolor[HTML]{FFFF00} С & \cellcolor[HTML]{FFFF00} Ы & Р &   & \cellcolor[HTML]{FFFF00} С & \cellcolor[HTML]{FFFF00} Ы & \cellcolor[HTML]{8CE4F6} Н &   & С & Ы & <6,2,Н>
\\ \hline
  &   &   & С & Ы & Р & \cellcolor[HTML]{FFFF00}   & \cellcolor[HTML]{FFFF00} С & \cellcolor[HTML]{FFFF00} Ы & Н & \cellcolor[HTML]{FFFF00}   & \cellcolor[HTML]{FFFF00} С & \cellcolor[HTML]{FFFF00} Ы & \cellcolor[HTML]{8CE4F6} Р & О & К & <6,3,Р>
\\ \hline
Ы & Р &   & С & Ы & Н &   & С & Ы & Р & \cellcolor[HTML]{8CE4F6} О & К &   & С & Ы & Н & <0,0,О>
\\ \hline
Р &   & С & Ы & Н &   & С & Ы & Р & О & \cellcolor[HTML]{8CE4F6} К &   & С & Ы & Н & О & <0,0,К>
\\ \hline
\cellcolor[HTML]{FFFF00}   & \cellcolor[HTML]{FFFF00} С & \cellcolor[HTML]{FFFF00} Ы & \cellcolor[HTML]{FFFF00} Н &   & С & Ы & Р & О & К & \cellcolor[HTML]{FFFF00}   & \cellcolor[HTML]{FFFF00} С & \cellcolor[HTML]{FFFF00} Ы & \cellcolor[HTML]{FFFF00} Н & \cellcolor[HTML]{8CE4F6} О & К & <0,4,О>
\\ \hline
С & Ы & Р & О & К &   & С & Ы & Н & О & \cellcolor[HTML]{8CE4F6} К &   &   &   &   &   & <0,0,К>
\\ \hline
\end{tabular}
\end{table}

\paragraph{Задание 3.3}

Закодировать сообщение методом LZ78\\
Строка:СЫР\_СЫН\_СЫРОК\_СЫНОК\\
\begin{table}[h!]
\centering
\begin{tabular}{|c|c|c|} 
\hline
 Входная фраза (в словарь) & Код & Позиция словаря \\ \hline

 &  & 0 \\ \hline
С & 0'С' & 1 \\ \hline
Ы & 0'Ы' & 2 \\ \hline
Р & 0'Р' & 3 \\ \hline
\_ & 0'\_' & 4 \\ \hline
СЫ & 1'Ы' & 5 \\ \hline
Н & 0'Н' & 6 \\ \hline
\_С & 4'С' & 7 \\ \hline
ЫР & 2'Р' & 8 \\ \hline
О & 0'О' & 9 \\ \hline
К & 0'К' & 10 \\ \hline
\_СЫ & 7'Ы' & 11 \\ \hline
НО & 6'О' & 12 \\ \hline
\end{tabular}
\end{table}

Результат: 0'С' 0'Ы' 0'Р' 0'\_' 1'Ы' 0'Н' 4'С' 2'Р' 0'О' 0'К' 7'Ы' 6'О'\\
\pagebreak
\paragraph{Задание 4. Арифметическое кодирование\\}

Исходная строка: АППРОПММММ\
\begin{center}
 \begin{tabular}{ |c|c| } 
  \hline
     Буква & Вероятность \\ \hline
М & 0.40\\\hline
П & 0.30\\\hline
А & 0.10\\\hline
Р & 0.10\\\hline
О & 0.10
\\ \hline \end{tabular}
\end{center}
\begin{center}
 \begin{tabular}{ |c|c|c| } 
  \hline
     Буква & Начало & Конец \\ \hline
М & 0.00 & 0.40\\\hline
П & 0.40 & 0.70\\\hline
А & 0.70 & 0.80\\\hline
Р & 0.80 & 0.90\\\hline
О & 0.90 & 1.00
\\ \hline \end{tabular}
\end{center}
\begin{center}
 \begin{tabular}{ |c|c|c|c| } 
  \hline
     Буква & delta & min & max \\ \hline
А & 0.1000000000 & 0.7000000000 & 0.8000000000\\\hline
П & 0.0300000000 & 0.7400000000 & 0.7700000000\\\hline
П & 0.0090000000 & 0.7520000000 & 0.7610000000\\\hline
Р & 0.0009000000 & 0.7592000000 & 0.7601000000\\\hline
О & 0.0000900000 & 0.7600100000 & 0.7601000000\\\hline
П & 0.0000270000 & 0.7600460000 & 0.7600730000\\\hline
М & 0.0000108000 & 0.7600460000 & 0.7600568000\\\hline
М & 0.0000043200 & 0.7600460000 & 0.7600503200\\\hline
М & 0.0000017280 & 0.7600460000 & 0.7600477280\\\hline
М & 0.0000006912 & 0.7600460000 & 0.7600466912
\\ \hline \end{tabular}
\end{center}
Результат: 760046
\pagebreak
\paragraph{Задание 5.1}

\\ 

Декодировать сообщение методом адаптивного хаффмана \\
Строка: 
'H'0'K'00'N'11100'F'1111110111\\
Результат: HKNKFNKNN









\paragraph{Задание 5.3 Декодировать строку(LZ78)\\}

Исходная строка: [0'л'] [0'о'] [0'с'] [2'с'] [0'ь'] [0' '] [1'о'] [3'ь'] [6'о'] [8' '] [4'а']\\
\begin{table}[h!]
\centering
\begin{tabular}{|c|c|c|} 
\hline
 Код & Словарь & Выходной поток 
\hline

 & [] & 
\\ \hline
0'л' & [, л] & л
\\ \hline
0'о' & [, л, о] & о
\\ \hline
0'с' & [, л, о, с] & с
\\ \hline
2'с' & [, л, о, с, ос] & ос
\\ \hline
0'ь' & [, л, о, с, ос, ь] & ь
\\ \hline
0' ' & [, л, о, с, ос, ь,  ] &  
\\ \hline
1'о' & [, л, о, с, ос, ь,  , ло] & ло
\\ \hline
3'ь' & [, л, о, с, ос, ь,  , ло, сь] & сь
\\ \hline
6'о' & [, л, о, с, ос, ь,  , ло, сь,  о] &  о
\\ \hline
8' ' & [, л, о, с, ос, ь,  , ло, сь,  о, сь ] & сь 
\\ \hline
4'а' & [, л, о, с, ос, ь,  , ло, сь,  о, сь , оса] & оса
\\ \hline
\end{tabular}
\end{table}

Результат: лосось лось ось оса
\pagebreak
\subsection{Вариант №5}
\paragraph{Задание 1. Блочный хаффман \\}

Строка ТОКООКККТК, размер блока: 2
\begin{center}
 \begin{tabular}{ |c|c|l| } 
  \hline
     Буква & Вероятность & Код\\ \hline
К & 0.50 & 0\\\hline
О & 0.30 & 11\\\hline
Т & 0.20 & 10
\\ \hline \end{tabular}
\end{center}
Энтропия алфавита: 1.4855
\begin{center}
 \begin{tabular}{ |c|c|l| } 
  \hline
     Блок & Вероятность & Код\\ \hline
КК & 0.25 & 01\\\hline
КО & 0.15 & 101\\\hline
ОК & 0.15 & 110\\\hline
ТК & 0.10 & 000\\\hline
КТ & 0.10 & 001\\\hline
ОО & 0.09 & 1111\\\hline
ОТ & 0.06 & 1001\\\hline
ТО & 0.06 & 1110\\\hline
ТТ & 0.04 & 1000
\\ \hline \end{tabular}
\end{center}
Бит на символ при посимвольном кодировании: 1.5000, при блочном: 1.5000


\pagebreak
\paragraph{Задание 2. Сжать адаптивным хаффманом\\}

Строка: 
РККЕАРРООО\\
Результат: 'Р' 0'К' 01 00'Е' 000'А' 10 10 1100'О' 11101 00










\pagebreak

\paragraph{Задание 3.3}

Закодировать сообщение методом LZ78\\
Строка:ОСЫ\_ОСЫ\_СЫПЬ\_НАСЫПЬ\\
\begin{table}[h!]
\centering
\begin{tabular}{|c|c|c|} 
\hline
 Входная фраза (в словарь) & Код & Позиция словаря \\ \hline

 &  & 0 \\ \hline
О & 0'О' & 1 \\ \hline
С & 0'С' & 2 \\ \hline
Ы & 0'Ы' & 3 \\ \hline
\_ & 0'\_' & 4 \\ \hline
ОС & 1'С' & 5 \\ \hline
Ы\_ & 3'\_' & 6 \\ \hline
СЫ & 2'Ы' & 7 \\ \hline
П & 0'П' & 8 \\ \hline
Ь & 0'Ь' & 9 \\ \hline
\_Н & 4'Н' & 10 \\ \hline
А & 0'А' & 11 \\ \hline
СЫП & 7'П' & 12 \\ \hline
\end{tabular}
\end{table}

Результат: 0'О' 0'С' 0'Ы' 0'\_' 1'С' 3'\_' 2'Ы' 0'П' 0'Ь' 4'Н' 0'А' 7'П'\\
\pagebreak
\paragraph{Задание 4. Арифметическое кодирование\\}

Исходная строка: РККЕАРРООО\
\begin{center}
 \begin{tabular}{ |c|c| } 
  \hline
     Буква & Вероятность \\ \hline
Р & 0.30\\\hline
О & 0.30\\\hline
К & 0.20\\\hline
А & 0.10\\\hline
Е & 0.10
\\ \hline \end{tabular}
\end{center}
\begin{center}
 \begin{tabular}{ |c|c|c| } 
  \hline
     Буква & Начало & Конец \\ \hline
Р & 0.00 & 0.30\\\hline
О & 0.30 & 0.60\\\hline
К & 0.60 & 0.80\\\hline
А & 0.80 & 0.90\\\hline
Е & 0.90 & 1.00
\\ \hline \end{tabular}
\end{center}
\begin{center}
 \begin{tabular}{ |c|c|c|c| } 
  \hline
     Буква & delta & min & max \\ \hline
Р & 0.3000000000 & 0.0000000000 & 0.3000000000\\\hline
К & 0.0600000000 & 0.1800000000 & 0.2400000000\\\hline
К & 0.0120000000 & 0.2160000000 & 0.2280000000\\\hline
Е & 0.0012000000 & 0.2268000000 & 0.2280000000\\\hline
А & 0.0001200000 & 0.2277600000 & 0.2278800000\\\hline
Р & 0.0000360000 & 0.2277600000 & 0.2277960000\\\hline
Р & 0.0000108000 & 0.2277600000 & 0.2277708000\\\hline
О & 0.0000032400 & 0.2277632400 & 0.2277664800\\\hline
О & 0.0000009720 & 0.2277642120 & 0.2277651840\\\hline
О & 0.0000002916 & 0.2277645036 & 0.2277647952
\\ \hline \end{tabular}
\end{center}
Результат: 2277646
\pagebreak
\paragraph{Задание 5.1}

\\ 

Декодировать сообщение методом адаптивного хаффмана \\
Строка: 
'D'0'C'00'B'101100'F'11011011011101001\\
Результат: DCBBDBFDBDBDBDCBB

















\paragraph{Задание 5.3 Декодировать строку(LZ78)\\}

Исходная строка: [0'л'] [0'е'] [0'с'] [0' '] [1'е'] [3'а'] [4'л'] [2'с'] [0'к'] [0'а'] [7'е'] [3'о'] [0'к']\\
\begin{table}[h!]
\centering
\begin{tabular}{|c|c|c|} 
\hline
 Код & Словарь & Выходной поток 
\hline

 & [] & 
\\ \hline
0'л' & [, л] & л
\\ \hline
0'е' & [, л, е] & е
\\ \hline
0'с' & [, л, е, с] & с
\\ \hline
0' ' & [, л, е, с,  ] &  
\\ \hline
1'е' & [, л, е, с,  , ле] & ле
\\ \hline
3'а' & [, л, е, с,  , ле, са] & са
\\ \hline
4'л' & [, л, е, с,  , ле, са,  л] &  л
\\ \hline
2'с' & [, л, е, с,  , ле, са,  л, ес] & ес
\\ \hline
0'к' & [, л, е, с,  , ле, са,  л, ес, к] & к
\\ \hline
0'а' & [, л, е, с,  , ле, са,  л, ес, к, а] & а
\\ \hline
7'е' & [, л, е, с,  , ле, са,  л, ес, к, а,  ле] &  ле
\\ \hline
3'о' & [, л, е, с,  , ле, са,  л, ес, к, а,  ле, со] & со
\\ \hline
0'к' & [, л, е, с,  , ле, са,  л, ес, к, а,  ле, со, к] & к
\\ \hline
\end{tabular}
\end{table}

Результат: лес леса леска лесок
\pagebreak
\subsection{Вариант №6}
\paragraph{Задание 1. Блочный хаффман \\}

Строка КООКЛЛЛЛЛЛ, размер блока: 2
\begin{center}
 \begin{tabular}{ |c|c|l| } 
  \hline
     Буква & Вероятность & Код\\ \hline
Л & 0.60 & 1\\\hline
К & 0.20 & 00\\\hline
О & 0.20 & 01
\\ \hline \end{tabular}
\end{center}
Энтропия алфавита: 1.3710
\begin{center}
 \begin{tabular}{ |c|c|l| } 
  \hline
     Блок & Вероятность & Код\\ \hline
ЛЛ & 0.36 & 11\\\hline
КЛ & 0.12 & 010\\\hline
ЛО & 0.12 & 011\\\hline
ОЛ & 0.12 & 100\\\hline
ЛК & 0.12 & 101\\\hline
КК & 0.04 & 0000\\\hline
ОО & 0.04 & 0001\\\hline
КО & 0.04 & 0010\\\hline
ОК & 0.04 & 0011
\\ \hline \end{tabular}
\end{center}
Бит на символ при посимвольном кодировании: 1.4000, при блочном: 1.4000


\pagebreak
\paragraph{Задание 2. Сжать адаптивным хаффманом\\}

Строка: 
СРОССКРРРР\\
Результат: 'С' 0'Р' 00'О' 0 0 000'К' 00 10 11 0










\pagebreak

\paragraph{Задание 3.3}

Закодировать сообщение методом LZ78\\
Строка:КУСКУС\_ КУСАКА\_СОБАКА\\
\begin{table}[h!]
\centering
\begin{tabular}{|c|c|c|} 
\hline
 Входная фраза (в словарь) & Код & Позиция словаря \\ \hline

 &  & 0 \\ \hline
К & 0'К' & 1 \\ \hline
У & 0'У' & 2 \\ \hline
С & 0'С' & 3 \\ \hline
КУ & 1'У' & 4 \\ \hline
С\_ & 3'\_' & 5 \\ \hline
  & 0' ' & 6 \\ \hline
КУС & 4'С' & 7 \\ \hline
А & 0'А' & 8 \\ \hline
КА & 1'А' & 9 \\ \hline
\_ & 0'\_' & 10 \\ \hline
СО & 3'О' & 11 \\ \hline
Б & 0'Б' & 12 \\ \hline
АК & 8'К' & 13 \\ \hline
\end{tabular}
\end{table}

Результат: 0'К' 0'У' 0'С' 1'У' 3'\_' 0' ' 4'С' 0'А' 1'А' 0'\_' 3'О' 0'Б' 8'К'\\
\pagebreak
\paragraph{Задание 4. Арифметическое кодирование\\}

Исходная строка: СРОССКРРРР\
\begin{center}
 \begin{tabular}{ |c|c| } 
  \hline
     Буква & Вероятность \\ \hline
Р & 0.50\\\hline
С & 0.30\\\hline
К & 0.10\\\hline
О & 0.10
\\ \hline \end{tabular}
\end{center}
\begin{center}
 \begin{tabular}{ |c|c|c| } 
  \hline
     Буква & Начало & Конец \\ \hline
Р & 0.00 & 0.50\\\hline
С & 0.50 & 0.80\\\hline
К & 0.80 & 0.90\\\hline
О & 0.90 & 1.00
\\ \hline \end{tabular}
\end{center}
\begin{center}
 \begin{tabular}{ |c|c|c|c| } 
  \hline
     Буква & delta & min & max \\ \hline
С & 0.3000000000 & 0.5000000000 & 0.8000000000\\\hline
Р & 0.1500000000 & 0.5000000000 & 0.6500000000\\\hline
О & 0.0150000000 & 0.6350000000 & 0.6500000000\\\hline
С & 0.0045000000 & 0.6425000000 & 0.6470000000\\\hline
С & 0.0013500000 & 0.6447500000 & 0.6461000000\\\hline
К & 0.0001350000 & 0.6458300000 & 0.6459650000\\\hline
Р & 0.0000675000 & 0.6458300000 & 0.6458975000\\\hline
Р & 0.0000337500 & 0.6458300000 & 0.6458637500\\\hline
Р & 0.0000168750 & 0.6458300000 & 0.6458468750\\\hline
Р & 0.0000084375 & 0.6458300000 & 0.6458384375
\\ \hline \end{tabular}
\end{center}
Результат: 64583
\pagebreak
\paragraph{Задание 5.1}

\\ 

Декодировать сообщение методом адаптивного хаффмана \\
Строка: 
'G'0'H'00'F'100'D'000'C'100110100100\\
Результат: GHFDCGFGCC










\paragraph{Задание 5.3 Декодировать строку(LZ78)\\}

Исходная строка: [0'с'] [0'о'] [1'у'] [0'д'] [0' '] [3'д'] [0'н'] [2' '] [6' '] [4'н'] [0'о']\\
\begin{table}[h!]
\centering
\begin{tabular}{|c|c|c|} 
\hline
 Код & Словарь & Выходной поток 
\hline

 & [] & 
\\ \hline
0'с' & [, с] & с
\\ \hline
0'о' & [, с, о] & о
\\ \hline
1'у' & [, с, о, су] & су
\\ \hline
0'д' & [, с, о, су, д] & д
\\ \hline
0' ' & [, с, о, су, д,  ] &  
\\ \hline
3'д' & [, с, о, су, д,  , суд] & суд
\\ \hline
0'н' & [, с, о, су, д,  , суд, н] & н
\\ \hline
2' ' & [, с, о, су, д,  , суд, н, о ] & о 
\\ \hline
6' ' & [, с, о, су, д,  , суд, н, о , суд ] & суд 
\\ \hline
4'н' & [, с, о, су, д,  , суд, н, о , суд , дн] & дн
\\ \hline
0'о' & [, с, о, су, д,  , суд, н, о , суд , дн, о] & о
\\ \hline
\end{tabular}
\end{table}

Результат: сосуд судно суд дно
\pagebreak
\subsection{Вариант №7}
\paragraph{Задание 1. Блочный хаффман \\}

Строка ТТУТТТТТТТ, размер блока: 3
\begin{center}
 \begin{tabular}{ |c|c|l| } 
  \hline
     Буква & Вероятность & Код\\ \hline
Т & 0.90 & 1\\\hline
У & 0.10 & 0
\\ \hline \end{tabular}
\end{center}
Энтропия алфавита: 0.4690
\begin{center}
 \begin{tabular}{ |c|c|l| } 
  \hline
     Блок & Вероятность & Код\\ \hline
ТТТ & 0.73 & 1\\\hline
ТУТ & 0.08 & 001\\\hline
ТТУ & 0.08 & 010\\\hline
УТТ & 0.08 & 011\\\hline
УУТ & 0.01 & 00011\\\hline
ТУУ & 0.01 & 00001\\\hline
УТУ & 0.01 & 00010\\\hline
УУУ & 0.00 & 00000
\\ \hline \end{tabular}
\end{center}
Бит на символ при посимвольном кодировании: 1.0000, при блочном: 0.5327


\pagebreak
\paragraph{Задание 2. Сжать адаптивным хаффманом\\}

Строка: 
ОРОПАВРРРР\\
Результат: 'О' 0'Р' 1 00'П' 000'А' 1100'В' 00 01 11 0










\pagebreak
\paragraph{Задание 3.1}

Закодировать сообщение методом LZ77\\
Строка:РОЗА\_РОЗАРИЙ\_ЗАРЯДКА\\
Результат: <0,0,Р> <0,0,О> <0,0,З> <0,0,А> <0,0,\_> <5,4,Р> <0,0,И> <0,0,Й> <2,1,З> <4,2,Я> <0,0,Д> <0,0,К> <0,0,А>\\
\begin{table}[h!]
\centering
\begin{tabular}{|c|c|c|c|c|c|c|c|c|c|c|c|c|c|c|c|c|} 
\hline
\multicolumn{10}{|c|}{Cловарь} & \multicolumn{6}{c|}{Буфер} & Код  \\ \hline
  &   &   &   &   &   &   &   &   &   & \cellcolor[HTML]{8CE4F6} Р & О & З & А &   & Р & <0,0,Р>
\\ \hline
  &   &   &   &   &   &   &   &   & Р & \cellcolor[HTML]{8CE4F6} О & З & А &   & Р & О & <0,0,О>
\\ \hline
  &   &   &   &   &   &   &   & Р & О & \cellcolor[HTML]{8CE4F6} З & А &   & Р & О & З & <0,0,З>
\\ \hline
  &   &   &   &   &   &   & Р & О & З & \cellcolor[HTML]{8CE4F6} А &   & Р & О & З & А & <0,0,А>
\\ \hline
  &   &   &   &   &   & Р & О & З & А & \cellcolor[HTML]{8CE4F6}   & Р & О & З & А & Р & <0,0,\_>
\\ \hline
  &   &   &   &   & \cellcolor[HTML]{FFFF00} Р & \cellcolor[HTML]{FFFF00} О & \cellcolor[HTML]{FFFF00} З & \cellcolor[HTML]{FFFF00} А &   & \cellcolor[HTML]{FFFF00} Р & \cellcolor[HTML]{FFFF00} О & \cellcolor[HTML]{FFFF00} З & \cellcolor[HTML]{FFFF00} А & \cellcolor[HTML]{8CE4F6} Р & И & <5,4,Р>
\\ \hline
Р & О & З & А &   & Р & О & З & А & Р & \cellcolor[HTML]{8CE4F6} И & Й &   & З & А & Р & <0,0,И>
\\ \hline
О & З & А &   & Р & О & З & А & Р & И & \cellcolor[HTML]{8CE4F6} Й &   & З & А & Р & Я & <0,0,Й>
\\ \hline
З & А & \cellcolor[HTML]{FFFF00}   & Р & О & З & А & Р & И & Й & \cellcolor[HTML]{FFFF00}   & \cellcolor[HTML]{8CE4F6} З & А & Р & Я & Д & <2,1,З>
\\ \hline
  & Р & О & З & \cellcolor[HTML]{FFFF00} А & \cellcolor[HTML]{FFFF00} Р & И & Й &   & З & \cellcolor[HTML]{FFFF00} А & \cellcolor[HTML]{FFFF00} Р & \cellcolor[HTML]{8CE4F6} Я & Д & К & А & <4,2,Я>
\\ \hline
З & А & Р & И & Й &   & З & А & Р & Я & \cellcolor[HTML]{8CE4F6} Д & К & А &   &   &   & <0,0,Д>
\\ \hline
А & Р & И & Й &   & З & А & Р & Я & Д & \cellcolor[HTML]{8CE4F6} К & А &   &   &   &   & <0,0,К>
\\ \hline
Р & И & Й &   & З & А & Р & Я & Д & К & \cellcolor[HTML]{8CE4F6} А &   &   &   &   &   & <0,0,А>
\\ \hline
\end{tabular}
\end{table}

\paragraph{Задание 3.3}

Закодировать сообщение методом LZ78\\
Строка:РОЗА\_РОЗАРИЙ\_ЗАРЯДКА\\
\begin{table}[h!]
\centering
\begin{tabular}{|c|c|c|} 
\hline
 Входная фраза (в словарь) & Код & Позиция словаря \\ \hline

 &  & 0 \\ \hline
Р & 0'Р' & 1 \\ \hline
О & 0'О' & 2 \\ \hline
З & 0'З' & 3 \\ \hline
А & 0'А' & 4 \\ \hline
\_ & 0'\_' & 5 \\ \hline
РО & 1'О' & 6 \\ \hline
ЗА & 3'А' & 7 \\ \hline
РИ & 1'И' & 8 \\ \hline
Й & 0'Й' & 9 \\ \hline
\_З & 5'З' & 10 \\ \hline
АР & 4'Р' & 11 \\ \hline
Я & 0'Я' & 12 \\ \hline
Д & 0'Д' & 13 \\ \hline
К & 0'К' & 14 \\ \hline
\end{tabular}
\end{table}

Результат: 0'Р' 0'О' 0'З' 0'А' 0'\_' 1'О' 3'А' 1'И' 0'Й' 5'З' 4'Р' 0'Я' 0'Д' 0'К'\\
\pagebreak
\paragraph{Задание 4. Арифметическое кодирование\\}

Исходная строка: ОРОПАВРРРР\
\begin{center}
 \begin{tabular}{ |c|c| } 
  \hline
     Буква & Вероятность \\ \hline
Р & 0.50\\\hline
О & 0.20\\\hline
А & 0.10\\\hline
В & 0.10\\\hline
П & 0.10
\\ \hline \end{tabular}
\end{center}
\begin{center}
 \begin{tabular}{ |c|c|c| } 
  \hline
     Буква & Начало & Конец \\ \hline
Р & 0.00 & 0.50\\\hline
О & 0.50 & 0.70\\\hline
А & 0.70 & 0.80\\\hline
В & 0.80 & 0.90\\\hline
П & 0.90 & 1.00
\\ \hline \end{tabular}
\end{center}
\begin{center}
 \begin{tabular}{ |c|c|c|c| } 
  \hline
     Буква & delta & min & max \\ \hline
О & 0.2000000000 & 0.5000000000 & 0.7000000000\\\hline
Р & 0.1000000000 & 0.5000000000 & 0.6000000000\\\hline
О & 0.0200000000 & 0.5500000000 & 0.5700000000\\\hline
П & 0.0020000000 & 0.5680000000 & 0.5700000000\\\hline
А & 0.0002000000 & 0.5694000000 & 0.5696000000\\\hline
В & 0.0000200000 & 0.5695600000 & 0.5695800000\\\hline
Р & 0.0000100000 & 0.5695600000 & 0.5695700000\\\hline
Р & 0.0000050000 & 0.5695600000 & 0.5695650000\\\hline
Р & 0.0000025000 & 0.5695600000 & 0.5695625000\\\hline
Р & 0.0000012500 & 0.5695600000 & 0.5695612500
\\ \hline \end{tabular}
\end{center}
Результат: 56956
\pagebreak
\paragraph{Задание 5.1}

\\ 

Декодировать сообщение методом адаптивного хаффмана \\
Строка: 
'V'0'B'00'C'100'N'11000'F'00001101\\
Результат: VBCNBFVCBV










\paragraph{Задание 5.3 Декодировать строку(LZ78)\\}

Исходная строка: [0'б'] [0'а'] [0'з'] [2'р'] [0' '] [1'а'] [0'р'] [5'з'] [4'я'] [5'а'] [0'м'] [6'р']\\
\begin{table}[h!]
\centering
\begin{tabular}{|c|c|c|} 
\hline
 Код & Словарь & Выходной поток 
\hline

 & [] & 
\\ \hline
0'б' & [, б] & б
\\ \hline
0'а' & [, б, а] & а
\\ \hline
0'з' & [, б, а, з] & з
\\ \hline
2'р' & [, б, а, з, ар] & ар
\\ \hline
0' ' & [, б, а, з, ар,  ] &  
\\ \hline
1'а' & [, б, а, з, ар,  , ба] & ба
\\ \hline
0'р' & [, б, а, з, ар,  , ба, р] & р
\\ \hline
5'з' & [, б, а, з, ар,  , ба, р,  з] &  з
\\ \hline
4'я' & [, б, а, з, ар,  , ба, р,  з, аря] & аря
\\ \hline
5'а' & [, б, а, з, ар,  , ба, р,  з, аря,  а] &  а
\\ \hline
0'м' & [, б, а, з, ар,  , ба, р,  з, аря,  а, м] & м
\\ \hline
6'р' & [, б, а, з, ар,  , ба, р,  з, аря,  а, м, бар] & бар
\\ \hline
\end{tabular}
\end{table}

Результат: базар бар заря амбар
\pagebreak
\subsection{Вариант №8}
\paragraph{Задание 1. Блочный хаффман \\}

Строка TОООTTTTTО, размер блока: 3
\begin{center}
 \begin{tabular}{ |c|c|l| } 
  \hline
     Буква & Вероятность & Код\\ \hline
T & 0.60 & 1\\\hline
О & 0.40 & 0
\\ \hline \end{tabular}
\end{center}
Энтропия алфавита: 0.9710
\begin{center}
 \begin{tabular}{ |c|c|l| } 
  \hline
     Блок & Вероятность & Код\\ \hline
TTT & 0.22 & 01\\\hline
ОTT & 0.14 & 100\\\hline
TОT & 0.14 & 101\\\hline
TTО & 0.14 & 110\\\hline
ООT & 0.10 & 001\\\hline
TОО & 0.10 & 1111\\\hline
ОTО & 0.10 & 000\\\hline
ООО & 0.06 & 1110
\\ \hline \end{tabular}
\end{center}
Бит на символ при посимвольном кодировании: 1.0000, при блочном: 0.9813


\pagebreak
\paragraph{Задание 2. Сжать адаптивным хаффманом\\}

Строка: 
РОПВПАРВВВ\\
Результат: 'Р' 0'О' 00'П' 100'В' 01 000'А' 00 101 00 11










\pagebreak
\paragraph{Задание 3.1}

Закодировать сообщение методом LZ77\\
Строка:ПОЛ\_ПОЛОВНИК\_ПОЛОВЕЦ\\
Результат: <0,0,П> <0,0,О> <0,0,Л> <0,0,\_> <6,3,О> <0,0,В> <0,0,Н> <0,0,И> <0,0,К> <1,5,В> <0,0,Е> <0,0,Ц>\\
\begin{table}[h!]
\centering
\begin{tabular}{|c|c|c|c|c|c|c|c|c|c|c|c|c|c|c|c|c|} 
\hline
\multicolumn{10}{|c|}{Cловарь} & \multicolumn{6}{c|}{Буфер} & Код  \\ \hline
  &   &   &   &   &   &   &   &   &   & \cellcolor[HTML]{8CE4F6} П & О & Л &   & П & О & <0,0,П>
\\ \hline
  &   &   &   &   &   &   &   &   & П & \cellcolor[HTML]{8CE4F6} О & Л &   & П & О & Л & <0,0,О>
\\ \hline
  &   &   &   &   &   &   &   & П & О & \cellcolor[HTML]{8CE4F6} Л &   & П & О & Л & О & <0,0,Л>
\\ \hline
  &   &   &   &   &   &   & П & О & Л & \cellcolor[HTML]{8CE4F6}   & П & О & Л & О & В & <0,0,\_>
\\ \hline
  &   &   &   &   &   & \cellcolor[HTML]{FFFF00} П & \cellcolor[HTML]{FFFF00} О & \cellcolor[HTML]{FFFF00} Л &   & \cellcolor[HTML]{FFFF00} П & \cellcolor[HTML]{FFFF00} О & \cellcolor[HTML]{FFFF00} Л & \cellcolor[HTML]{8CE4F6} О & В & Н & <6,3,О>
\\ \hline
  &   & П & О & Л &   & П & О & Л & О & \cellcolor[HTML]{8CE4F6} В & Н & И & К &   & П & <0,0,В>
\\ \hline
  & П & О & Л &   & П & О & Л & О & В & \cellcolor[HTML]{8CE4F6} Н & И & К &   & П & О & <0,0,Н>
\\ \hline
П & О & Л &   & П & О & Л & О & В & Н & \cellcolor[HTML]{8CE4F6} И & К &   & П & О & Л & <0,0,И>
\\ \hline
О & Л &   & П & О & Л & О & В & Н & И & \cellcolor[HTML]{8CE4F6} К &   & П & О & Л & О & <0,0,К>
\\ \hline
Л & \cellcolor[HTML]{FFFF00}   & \cellcolor[HTML]{FFFF00} П & \cellcolor[HTML]{FFFF00} О & \cellcolor[HTML]{FFFF00} Л & \cellcolor[HTML]{FFFF00} О & В & Н & И & К & \cellcolor[HTML]{FFFF00}   & \cellcolor[HTML]{FFFF00} П & \cellcolor[HTML]{FFFF00} О & \cellcolor[HTML]{FFFF00} Л & \cellcolor[HTML]{FFFF00} О & \cellcolor[HTML]{8CE4F6} В & <1,5,В>
\\ \hline
В & Н & И & К &   & П & О & Л & О & В & \cellcolor[HTML]{8CE4F6} Е & Ц &   &   &   &   & <0,0,Е>
\\ \hline
Н & И & К &   & П & О & Л & О & В & Е & \cellcolor[HTML]{8CE4F6} Ц &   &   &   &   &   & <0,0,Ц>
\\ \hline
\end{tabular}
\end{table}

\paragraph{Задание 3.3}

Закодировать сообщение методом LZ78\\
Строка:ПОЛ\_ПОЛОВНИК\_ПОЛОВЕЦ\\
\begin{table}[h!]
\centering
\begin{tabular}{|c|c|c|} 
\hline
 Входная фраза (в словарь) & Код & Позиция словаря \\ \hline

 &  & 0 \\ \hline
П & 0'П' & 1 \\ \hline
О & 0'О' & 2 \\ \hline
Л & 0'Л' & 3 \\ \hline
\_ & 0'\_' & 4 \\ \hline
ПО & 1'О' & 5 \\ \hline
ЛО & 3'О' & 6 \\ \hline
В & 0'В' & 7 \\ \hline
Н & 0'Н' & 8 \\ \hline
И & 0'И' & 9 \\ \hline
К & 0'К' & 10 \\ \hline
\_П & 4'П' & 11 \\ \hline
ОЛ & 2'Л' & 12 \\ \hline
ОВ & 2'В' & 13 \\ \hline
Е & 0'Е' & 14 \\ \hline
Ц & 0'Ц' & 15 \\ \hline
\end{tabular}
\end{table}

Результат: 0'П' 0'О' 0'Л' 0'\_' 1'О' 3'О' 0'В' 0'Н' 0'И' 0'К' 4'П' 2'Л' 2'В' 0'Е' 0'Ц'\\
\pagebreak
\paragraph{Задание 4. Арифметическое кодирование\\}

Исходная строка: РОПВПАРВВВ\
\begin{center}
 \begin{tabular}{ |c|c| } 
  \hline
     Буква & Вероятность \\ \hline
В & 0.40\\\hline
Р & 0.20\\\hline
П & 0.20\\\hline
А & 0.10\\\hline
О & 0.10
\\ \hline \end{tabular}
\end{center}
\begin{center}
 \begin{tabular}{ |c|c|c| } 
  \hline
     Буква & Начало & Конец \\ \hline
В & 0.00 & 0.40\\\hline
Р & 0.40 & 0.60\\\hline
П & 0.60 & 0.80\\\hline
А & 0.80 & 0.90\\\hline
О & 0.90 & 1.00
\\ \hline \end{tabular}
\end{center}
\begin{center}
 \begin{tabular}{ |c|c|c|c| } 
  \hline
     Буква & delta & min & max \\ \hline
Р & 0.2000000000 & 0.4000000000 & 0.6000000000\\\hline
О & 0.0200000000 & 0.5800000000 & 0.6000000000\\\hline
П & 0.0040000000 & 0.5920000000 & 0.5960000000\\\hline
В & 0.0016000000 & 0.5920000000 & 0.5936000000\\\hline
П & 0.0003200000 & 0.5929600000 & 0.5932800000\\\hline
А & 0.0000320000 & 0.5932160000 & 0.5932480000\\\hline
Р & 0.0000064000 & 0.5932288000 & 0.5932352000\\\hline
В & 0.0000025600 & 0.5932288000 & 0.5932313600\\\hline
В & 0.0000010240 & 0.5932288000 & 0.5932298240\\\hline
В & 0.0000004096 & 0.5932288000 & 0.5932292096
\\ \hline \end{tabular}
\end{center}
Результат: 593229
\pagebreak
\paragraph{Задание 5.1}

\\ 

Декодировать сообщение методом адаптивного хаффмана \\
Строка: 
'K'0'N'00'M'101100'H'110111010111111\\
Результат: KNMMKMHKMKNMKKM















\paragraph{Задание 5.3 Декодировать строку(LZ78)\\}

Исходная строка: [0'б'] [0'р'] [0'и'] [0'з'] [0' '] [1'р'] [0'а'] [5'б'] [7'р'] [5'р'] [7'б'] [5'а'] [0'р']\\
\begin{table}[h!]
\centering
\begin{tabular}{|c|c|c|} 
\hline
 Код & Словарь & Выходной поток 
\hline

 & [] & 
\\ \hline
0'б' & [, б] & б
\\ \hline
0'р' & [, б, р] & р
\\ \hline
0'и' & [, б, р, и] & и
\\ \hline
0'з' & [, б, р, и, з] & з
\\ \hline
0' ' & [, б, р, и, з,  ] &  
\\ \hline
1'р' & [, б, р, и, з,  , бр] & бр
\\ \hline
0'а' & [, б, р, и, з,  , бр, а] & а
\\ \hline
5'б' & [, б, р, и, з,  , бр, а,  б] &  б
\\ \hline
7'р' & [, б, р, и, з,  , бр, а,  б, ар] & ар
\\ \hline
5'р' & [, б, р, и, з,  , бр, а,  б, ар,  р] &  р
\\ \hline
7'б' & [, б, р, и, з,  , бр, а,  б, ар,  р, аб] & аб
\\ \hline
5'а' & [, б, р, и, з,  , бр, а,  б, ар,  р, аб,  а] &  а
\\ \hline
0'р' & [, б, р, и, з,  , бр, а,  б, ар,  р, аб,  а, р] & р
\\ \hline
\end{tabular}
\end{table}

Результат: бриз бра бар раб ар
\pagebreak
\subsection{Вариант №9}
\paragraph{Задание 1. Блочный хаффман \\}

Строка СОКККККООО, размер блока: 2
\begin{center}
 \begin{tabular}{ |c|c|l| } 
  \hline
     Буква & Вероятность & Код\\ \hline
К & 0.50 & 0\\\hline
О & 0.40 & 11\\\hline
С & 0.10 & 10
\\ \hline \end{tabular}
\end{center}
Энтропия алфавита: 1.3610
\begin{center}
 \begin{tabular}{ |c|c|l| } 
  \hline
     Блок & Вероятность & Код\\ \hline
КК & 0.25 & 10\\\hline
КО & 0.20 & 00\\\hline
ОК & 0.20 & 01\\\hline
ОО & 0.16 & 110\\\hline
КС & 0.05 & 11101\\\hline
СК & 0.05 & 11110\\\hline
ОС & 0.04 & 111111\\\hline
СО & 0.04 & 11100\\\hline
СС & 0.01 & 111110
\\ \hline \end{tabular}
\end{center}
Бит на символ при посимвольном кодировании: 1.5000, при блочном: 1.3900


\pagebreak
\paragraph{Задание 2. Сжать адаптивным хаффманом\\}

Строка: 
РОРНРПОООО\\
Результат: 'Р' 0'О' 1 00'Н' 1 000'П' 00 10 11 0










\pagebreak

\paragraph{Задание 3.3}

Закодировать сообщение методом LZ78\\
Строка:МУМУМУ\_МУКА\_МУРКА\\
\begin{table}[h!]
\centering
\begin{tabular}{|c|c|c|} 
\hline
 Входная фраза (в словарь) & Код & Позиция словаря \\ \hline

 &  & 0 \\ \hline
М & 0'М' & 1 \\ \hline
У & 0'У' & 2 \\ \hline
МУ & 1'У' & 3 \\ \hline
МУ\_ & 3'\_' & 4 \\ \hline
МУК & 3'К' & 5 \\ \hline
А & 0'А' & 6 \\ \hline
\_ & 0'\_' & 7 \\ \hline
МУР & 3'Р' & 8 \\ \hline
К & 0'К' & 9 \\ \hline
\end{tabular}
\end{table}

Результат: 0'М' 0'У' 1'У' 3'\_' 3'К' 0'А' 0'\_' 3'Р' 0'К'\\
\pagebreak
\paragraph{Задание 4. Арифметическое кодирование\\}

Исходная строка: РОРНРПОООО\
\begin{center}
 \begin{tabular}{ |c|c| } 
  \hline
     Буква & Вероятность \\ \hline
О & 0.50\\\hline
Р & 0.30\\\hline
Н & 0.10\\\hline
П & 0.10
\\ \hline \end{tabular}
\end{center}
\begin{center}
 \begin{tabular}{ |c|c|c| } 
  \hline
     Буква & Начало & Конец \\ \hline
О & 0.00 & 0.50\\\hline
Р & 0.50 & 0.80\\\hline
Н & 0.80 & 0.90\\\hline
П & 0.90 & 1.00
\\ \hline \end{tabular}
\end{center}
\begin{center}
 \begin{tabular}{ |c|c|c|c| } 
  \hline
     Буква & delta & min & max \\ \hline
Р & 0.3000000000 & 0.5000000000 & 0.8000000000\\\hline
О & 0.1500000000 & 0.5000000000 & 0.6500000000\\\hline
Р & 0.0450000000 & 0.5750000000 & 0.6200000000\\\hline
Н & 0.0045000000 & 0.6110000000 & 0.6155000000\\\hline
Р & 0.0013500000 & 0.6132500000 & 0.6146000000\\\hline
П & 0.0001350000 & 0.6144650000 & 0.6146000000\\\hline
О & 0.0000675000 & 0.6144650000 & 0.6145325000\\\hline
О & 0.0000337500 & 0.6144650000 & 0.6144987500\\\hline
О & 0.0000168750 & 0.6144650000 & 0.6144818750\\\hline
О & 0.0000084375 & 0.6144650000 & 0.6144734375
\\ \hline \end{tabular}
\end{center}
Результат: 61447
\pagebreak
\paragraph{Задание 5.1}

\\ 

Декодировать сообщение методом адаптивного хаффмана \\
Строка: 
'T'0'Y'00'H'100'G'0010111111111111\\
Результат: TYHGGHGTTG










\paragraph{Задание 5.3 Декодировать строку(LZ78)\\}

Исходная строка: [0'л'] [0'е'] [0'т'] [0'о'] [0' '] [3'о'] [0'н'] [5'т'] [4'н'] [0'у'] [0'с'] [5'у'] [11'ы']\\
\begin{table}[h!]
\centering
\begin{tabular}{|c|c|c|} 
\hline
 Код & Словарь & Выходной поток 
\hline

 & [] & 
\\ \hline
0'л' & [, л] & л
\\ \hline
0'е' & [, л, е] & е
\\ \hline
0'т' & [, л, е, т] & т
\\ \hline
0'о' & [, л, е, т, о] & о
\\ \hline
0' ' & [, л, е, т, о,  ] &  
\\ \hline
3'о' & [, л, е, т, о,  , то] & то
\\ \hline
0'н' & [, л, е, т, о,  , то, н] & н
\\ \hline
5'т' & [, л, е, т, о,  , то, н,  т] &  т
\\ \hline
4'н' & [, л, е, т, о,  , то, н,  т, он] & он
\\ \hline
0'у' & [, л, е, т, о,  , то, н,  т, он, у] & у
\\ \hline
0'с' & [, л, е, т, о,  , то, н,  т, он, у, с] & с
\\ \hline
5'у' & [, л, е, т, о,  , то, н,  т, он, у, с,  у] &  у
\\ \hline
11'ы' & [, л, е, т, о,  , то, н,  т, он, у, с,  у, сы] & сы
\\ \hline
\end{tabular}
\end{table}

Результат: лето тон тонус усы
\pagebreak
\subsection{Вариант №10}
\paragraph{Задание 1. Блочный хаффман \\}

Строка СТТТТССТТТ, размер блока: 3
\begin{center}
 \begin{tabular}{ |c|c|l| } 
  \hline
     Буква & Вероятность & Код\\ \hline
Т & 0.70 & 1\\\hline
С & 0.30 & 0
\\ \hline \end{tabular}
\end{center}
Энтропия алфавита: 0.8813
\begin{center}
 \begin{tabular}{ |c|c|l| } 
  \hline
     Блок & Вероятность & Код\\ \hline
ТТТ & 0.34 & 11\\\hline
СТТ & 0.15 & 101\\\hline
ТСТ & 0.15 & 00\\\hline
ТТС & 0.15 & 100\\\hline
СТС & 0.06 & 0101\\\hline
ССТ & 0.06 & 0110\\\hline
ТСС & 0.06 & 0111\\\hline
ССС & 0.03 & 0100
\\ \hline \end{tabular}
\end{center}
Бит на символ при посимвольном кодировании: 1.0000, при блочном: 0.9087


\pagebreak
\paragraph{Задание 2. Сжать адаптивным хаффманом\\}

Строка: 
КАВАПВПППА\\
Результат: 'К' 0'А' 00'В' 11 100'П' 111 001 01 11 111










\pagebreak
\paragraph{Задание 3.1}

Закодировать сообщение методом LZ77\\
Строка:КОК\_КОКЛЮШ\_КЛУБ\_КЛУБОК\\
Результат: <0,0,К> <0,0,О> <8,1,\_> <6,3,Л> <0,0,Ю> <0,0,Ш> <3,2,Л> <0,0,У> <0,0,Б> <5,5,О> <0,0,К>\\
\begin{table}[h!]
\centering
\begin{tabular}{|c|c|c|c|c|c|c|c|c|c|c|c|c|c|c|c|c|} 
\hline
\multicolumn{10}{|c|}{Cловарь} & \multicolumn{6}{c|}{Буфер} & Код  \\ \hline
  &   &   &   &   &   &   &   &   &   & \cellcolor[HTML]{8CE4F6} К & О & К &   & К & О & <0,0,К>
\\ \hline
  &   &   &   &   &   &   &   &   & К & \cellcolor[HTML]{8CE4F6} О & К &   & К & О & К & <0,0,О>
\\ \hline
  &   &   &   &   &   &   &   & \cellcolor[HTML]{FFFF00} К & О & \cellcolor[HTML]{FFFF00} К & \cellcolor[HTML]{8CE4F6}   & К & О & К & Л & <8,1,\_>
\\ \hline
  &   &   &   &   &   & \cellcolor[HTML]{FFFF00} К & \cellcolor[HTML]{FFFF00} О & \cellcolor[HTML]{FFFF00} К &   & \cellcolor[HTML]{FFFF00} К & \cellcolor[HTML]{FFFF00} О & \cellcolor[HTML]{FFFF00} К & \cellcolor[HTML]{8CE4F6} Л & Ю & Ш & <6,3,Л>
\\ \hline
  &   & К & О & К &   & К & О & К & Л & \cellcolor[HTML]{8CE4F6} Ю & Ш &   & К & Л & У & <0,0,Ю>
\\ \hline
  & К & О & К &   & К & О & К & Л & Ю & \cellcolor[HTML]{8CE4F6} Ш &   & К & Л & У & Б & <0,0,Ш>
\\ \hline
К & О & К & \cellcolor[HTML]{FFFF00}   & \cellcolor[HTML]{FFFF00} К & О & К & Л & Ю & Ш & \cellcolor[HTML]{FFFF00}   & \cellcolor[HTML]{FFFF00} К & \cellcolor[HTML]{8CE4F6} Л & У & Б &   & <3,2,Л>
\\ \hline
  & К & О & К & Л & Ю & Ш &   & К & Л & \cellcolor[HTML]{8CE4F6} У & Б &   & К & Л & У & <0,0,У>
\\ \hline
К & О & К & Л & Ю & Ш &   & К & Л & У & \cellcolor[HTML]{8CE4F6} Б &   & К & Л & У & Б & <0,0,Б>
\\ \hline
О & К & Л & Ю & Ш & \cellcolor[HTML]{FFFF00}   & \cellcolor[HTML]{FFFF00} К & \cellcolor[HTML]{FFFF00} Л & \cellcolor[HTML]{FFFF00} У & \cellcolor[HTML]{FFFF00} Б & \cellcolor[HTML]{FFFF00}   & \cellcolor[HTML]{FFFF00} К & \cellcolor[HTML]{FFFF00} Л & \cellcolor[HTML]{FFFF00} У & \cellcolor[HTML]{FFFF00} Б & \cellcolor[HTML]{8CE4F6} О & <5,5,О>
\\ \hline
К & Л & У & Б &   & К & Л & У & Б & О & \cellcolor[HTML]{8CE4F6} К &   &   &   &   &   & <0,0,К>
\\ \hline
\end{tabular}
\end{table}

\paragraph{Задание 3.3}

Закодировать сообщение методом LZ78\\
Строка:КОК\_КОКЛЮШ\_КЛУБ\_КЛУБОК\\
\begin{table}[h!]
\centering
\begin{tabular}{|c|c|c|} 
\hline
 Входная фраза (в словарь) & Код & Позиция словаря \\ \hline

 &  & 0 \\ \hline
К & 0'К' & 1 \\ \hline
О & 0'О' & 2 \\ \hline
К\_ & 1'\_' & 3 \\ \hline
КО & 1'О' & 4 \\ \hline
КЛ & 1'Л' & 5 \\ \hline
Ю & 0'Ю' & 6 \\ \hline
Ш & 0'Ш' & 7 \\ \hline
\_ & 0'\_' & 8 \\ \hline
КЛУ & 5'У' & 9 \\ \hline
Б & 0'Б' & 10 \\ \hline
\_К & 8'К' & 11 \\ \hline
Л & 0'Л' & 12 \\ \hline
У & 0'У' & 13 \\ \hline
БО & 10'О' & 14 \\ \hline
\end{tabular}
\end{table}

Результат: 0'К' 0'О' 1'\_' 1'О' 1'Л' 0'Ю' 0'Ш' 0'\_' 5'У' 0'Б' 8'К' 0'Л' 0'У' 10'О'\\
\pagebreak
\paragraph{Задание 4. Арифметическое кодирование\\}

Исходная строка: КАВАПВПППА\
\begin{center}
 \begin{tabular}{ |c|c| } 
  \hline
     Буква & Вероятность \\ \hline
П & 0.40\\\hline
А & 0.30\\\hline
В & 0.20\\\hline
К & 0.10
\\ \hline \end{tabular}
\end{center}
\begin{center}
 \begin{tabular}{ |c|c|c| } 
  \hline
     Буква & Начало & Конец \\ \hline
П & 0.00 & 0.40\\\hline
А & 0.40 & 0.70\\\hline
В & 0.70 & 0.90\\\hline
К & 0.90 & 1.00
\\ \hline \end{tabular}
\end{center}
\begin{center}
 \begin{tabular}{ |c|c|c|c| } 
  \hline
     Буква & delta & min & max \\ \hline
К & 0.1000000000 & 0.9000000000 & 1.0000000000\\\hline
А & 0.0300000000 & 0.9400000000 & 0.9700000000\\\hline
В & 0.0060000000 & 0.9610000000 & 0.9670000000\\\hline
А & 0.0018000000 & 0.9634000000 & 0.9652000000\\\hline
П & 0.0007200000 & 0.9634000000 & 0.9641200000\\\hline
В & 0.0001440000 & 0.9639040000 & 0.9640480000\\\hline
П & 0.0000576000 & 0.9639040000 & 0.9639616000\\\hline
П & 0.0000230400 & 0.9639040000 & 0.9639270400\\\hline
П & 0.0000092160 & 0.9639040000 & 0.9639132160\\\hline
А & 0.0000027648 & 0.9639076864 & 0.9639104512
\\ \hline \end{tabular}
\end{center}
Результат: 96391
\pagebreak
\paragraph{Задание 5.1}

\\ 

Декодировать сообщение методом адаптивного хаффмана \\
Строка: 
'K'0'J'00'N'100'M'000'H'0010001\\
Результат: KJNMHJJJJJ










\paragraph{Задание 5.3 Декодировать строку(LZ78)\\}

Исходная строка: [0'к'] [0'у'] [0'с'] [0'о'] [1' '] [3'о'] [1'о'] [0'л'] [0' '] [6'к'] [9'к'] [4'л']\\
\begin{table}[h!]
\centering
\begin{tabular}{|c|c|c|} 
\hline
 Код & Словарь & Выходной поток 
\hline

 & [] & 
\\ \hline
0'к' & [, к] & к
\\ \hline
0'у' & [, к, у] & у
\\ \hline
0'с' & [, к, у, с] & с
\\ \hline
0'о' & [, к, у, с, о] & о
\\ \hline
1' ' & [, к, у, с, о, к ] & к 
\\ \hline
3'о' & [, к, у, с, о, к , со] & со
\\ \hline
1'о' & [, к, у, с, о, к , со, ко] & ко
\\ \hline
0'л' & [, к, у, с, о, к , со, ко, л] & л
\\ \hline
0' ' & [, к, у, с, о, к , со, ко, л,  ] &  
\\ \hline
6'к' & [, к, у, с, о, к , со, ко, л,  , сок] & сок
\\ \hline
9'к' & [, к, у, с, о, к , со, ко, л,  , сок,  к] &  к
\\ \hline
4'л' & [, к, у, с, о, к , со, ко, л,  , сок,  к, ол] & ол
\\ \hline
\end{tabular}
\end{table}

Результат: кусок сокол сок кол
\pagebreak
\subsection{Вариант №11}
\paragraph{Задание 1. Блочный хаффман \\}

Строка ВВВАСССССС, размер блока: 2
\begin{center}
 \begin{tabular}{ |c|c|l| } 
  \hline
     Буква & Вероятность & Код\\ \hline
С & 0.60 & 1\\\hline
В & 0.30 & 01\\\hline
А & 0.10 & 00
\\ \hline \end{tabular}
\end{center}
Энтропия алфавита: 1.2955
\begin{center}
 \begin{tabular}{ |c|c|l| } 
  \hline
     Блок & Вероятность & Код\\ \hline
СС & 0.36 & 11\\\hline
СВ & 0.18 & 00\\\hline
ВС & 0.18 & 01\\\hline
ВВ & 0.09 & 1011\\\hline
АС & 0.06 & 1000\\\hline
СА & 0.06 & 1001\\\hline
АВ & 0.03 & 101011\\\hline
ВА & 0.03 & 10100\\\hline
АА & 0.01 & 101010
\\ \hline \end{tabular}
\end{center}
Бит на символ при посимвольном кодировании: 1.4000, при блочном: 1.3350


\pagebreak
\paragraph{Задание 2. Сжать адаптивным хаффманом\\}

Строка: 
ЕНКПКЕКИЕЕ\\
Результат: 'Е' 0'Н' 00'К' 100'П' 01 10 11 1100'И' 10 10










\pagebreak
\paragraph{Задание 3.1}

Закодировать сообщение методом LZ77\\
Строка:ВАРВАР\_ВАРИТ\_ВАРЕНЬЕ\\
Результат: <0,0,В> <0,0,А> <0,0,Р> <7,3,\_> <3,3,И> <0,0,Т> <4,4,Е> <0,0,Н> <0,0,Ь> <0,0,Е>\\
\begin{table}[h!]
\centering
\begin{tabular}{|c|c|c|c|c|c|c|c|c|c|c|c|c|c|c|c|c|} 
\hline
\multicolumn{10}{|c|}{Cловарь} & \multicolumn{6}{c|}{Буфер} & Код  \\ \hline
  &   &   &   &   &   &   &   &   &   & \cellcolor[HTML]{8CE4F6} В & А & Р & В & А & Р & <0,0,В>
\\ \hline
  &   &   &   &   &   &   &   &   & В & \cellcolor[HTML]{8CE4F6} А & Р & В & А & Р &   & <0,0,А>
\\ \hline
  &   &   &   &   &   &   &   & В & А & \cellcolor[HTML]{8CE4F6} Р & В & А & Р &   & В & <0,0,Р>
\\ \hline
  &   &   &   &   &   &   & \cellcolor[HTML]{FFFF00} В & \cellcolor[HTML]{FFFF00} А & \cellcolor[HTML]{FFFF00} Р & \cellcolor[HTML]{FFFF00} В & \cellcolor[HTML]{FFFF00} А & \cellcolor[HTML]{FFFF00} Р & \cellcolor[HTML]{8CE4F6}   & В & А & <7,3,\_>
\\ \hline
  &   &   & \cellcolor[HTML]{FFFF00} В & \cellcolor[HTML]{FFFF00} А & \cellcolor[HTML]{FFFF00} Р & В & А & Р &   & \cellcolor[HTML]{FFFF00} В & \cellcolor[HTML]{FFFF00} А & \cellcolor[HTML]{FFFF00} Р & \cellcolor[HTML]{8CE4F6} И & Т &   & <3,3,И>
\\ \hline
А & Р & В & А & Р &   & В & А & Р & И & \cellcolor[HTML]{8CE4F6} Т &   & В & А & Р & Е & <0,0,Т>
\\ \hline
Р & В & А & Р & \cellcolor[HTML]{FFFF00}   & \cellcolor[HTML]{FFFF00} В & \cellcolor[HTML]{FFFF00} А & \cellcolor[HTML]{FFFF00} Р & И & Т & \cellcolor[HTML]{FFFF00}   & \cellcolor[HTML]{FFFF00} В & \cellcolor[HTML]{FFFF00} А & \cellcolor[HTML]{FFFF00} Р & \cellcolor[HTML]{8CE4F6} Е & Н & <4,4,Е>
\\ \hline
В & А & Р & И & Т &   & В & А & Р & Е & \cellcolor[HTML]{8CE4F6} Н & Ь & Е &   &   &   & <0,0,Н>
\\ \hline
А & Р & И & Т &   & В & А & Р & Е & Н & \cellcolor[HTML]{8CE4F6} Ь & Е &   &   &   &   & <0,0,Ь>
\\ \hline
Р & И & Т &   & В & А & Р & Е & Н & Ь & \cellcolor[HTML]{8CE4F6} Е &   &   &   &   &   & <0,0,Е>
\\ \hline
\end{tabular}
\end{table}

\paragraph{Задание 3.3}

Закодировать сообщение методом LZ78\\
Строка:ВАРВАР\_ВАРИТ\_ВАРЕНЬЕ\\
\begin{table}[h!]
\centering
\begin{tabular}{|c|c|c|} 
\hline
 Входная фраза (в словарь) & Код & Позиция словаря \\ \hline

 &  & 0 \\ \hline
В & 0'В' & 1 \\ \hline
А & 0'А' & 2 \\ \hline
Р & 0'Р' & 3 \\ \hline
ВА & 1'А' & 4 \\ \hline
Р\_ & 3'\_' & 5 \\ \hline
ВАР & 4'Р' & 6 \\ \hline
И & 0'И' & 7 \\ \hline
Т & 0'Т' & 8 \\ \hline
\_ & 0'\_' & 9 \\ \hline
ВАРЕ & 6'Е' & 10 \\ \hline
Н & 0'Н' & 11 \\ \hline
Ь & 0'Ь' & 12 \\ \hline
Е & 0'Е' & 13 \\ \hline
\end{tabular}
\end{table}

Результат: 0'В' 0'А' 0'Р' 1'А' 3'\_' 4'Р' 0'И' 0'Т' 0'\_' 6'Е' 0'Н' 0'Ь' 0'Е'\\
\pagebreak
\paragraph{Задание 4. Арифметическое кодирование\\}

Исходная строка: ЕНКПКЕКИЕЕ\
\begin{center}
 \begin{tabular}{ |c|c| } 
  \hline
     Буква & Вероятность \\ \hline
Е & 0.40\\\hline
К & 0.30\\\hline
И & 0.10\\\hline
Н & 0.10\\\hline
П & 0.10
\\ \hline \end{tabular}
\end{center}
\begin{center}
 \begin{tabular}{ |c|c|c| } 
  \hline
     Буква & Начало & Конец \\ \hline
Е & 0.00 & 0.40\\\hline
К & 0.40 & 0.70\\\hline
И & 0.70 & 0.80\\\hline
Н & 0.80 & 0.90\\\hline
П & 0.90 & 1.00
\\ \hline \end{tabular}
\end{center}
\begin{center}
 \begin{tabular}{ |c|c|c|c| } 
  \hline
     Буква & delta & min & max \\ \hline
Е & 0.4000000000 & 0.0000000000 & 0.4000000000\\\hline
Н & 0.0400000000 & 0.3200000000 & 0.3600000000\\\hline
К & 0.0120000000 & 0.3360000000 & 0.3480000000\\\hline
П & 0.0012000000 & 0.3468000000 & 0.3480000000\\\hline
К & 0.0003600000 & 0.3472800000 & 0.3476400000\\\hline
Е & 0.0001440000 & 0.3472800000 & 0.3474240000\\\hline
К & 0.0000432000 & 0.3473376000 & 0.3473808000\\\hline
И & 0.0000043200 & 0.3473678400 & 0.3473721600\\\hline
Е & 0.0000017280 & 0.3473678400 & 0.3473695680\\\hline
Е & 0.0000006912 & 0.3473678400 & 0.3473685312
\\ \hline \end{tabular}
\end{center}
Результат: 347368
\pagebreak
\paragraph{Задание 5.1}

\\ 

Декодировать сообщение методом адаптивного хаффмана \\
Строка: 
'L'0'K'00'M'100'N'01000'B'10010111\\
Результат: LKMNMBBBB









\paragraph{Задание 5.3 Декодировать строку(LZ78)\\}

Исходная строка: [0'у'] [0'к'] [0'с'] [1'с'] [0' '] [1'к'] [4' '] [2'у'] [3'т'] [0'ы'] [5'к'] [4'т']\\
\begin{table}[h!]
\centering
\begin{tabular}{|c|c|c|} 
\hline
 Код & Словарь & Выходной поток 
\hline

 & [] & 
\\ \hline
0'у' & [, у] & у
\\ \hline
0'к' & [, у, к] & к
\\ \hline
0'с' & [, у, к, с] & с
\\ \hline
1'с' & [, у, к, с, ус] & ус
\\ \hline
0' ' & [, у, к, с, ус,  ] &  
\\ \hline
1'к' & [, у, к, с, ус,  , ук] & ук
\\ \hline
4' ' & [, у, к, с, ус,  , ук, ус ] & ус 
\\ \hline
2'у' & [, у, к, с, ус,  , ук, ус , ку] & ку
\\ \hline
3'т' & [, у, к, с, ус,  , ук, ус , ку, ст] & ст
\\ \hline
0'ы' & [, у, к, с, ус,  , ук, ус , ку, ст, ы] & ы
\\ \hline
5'к' & [, у, к, с, ус,  , ук, ус , ку, ст, ы,  к] &  к
\\ \hline
4'т' & [, у, к, с, ус,  , ук, ус , ку, ст, ы,  к, уст] & уст
\\ \hline
\end{tabular}
\end{table}

Результат: уксус укус кусты куст
\pagebreak
\subsection{Вариант №12}
\paragraph{Задание 1. Блочный хаффман \\}

Строка ТИИИКТКККТ, размер блока: 3
\begin{center}
 \begin{tabular}{ |c|c|l| } 
  \hline
     Буква & Вероятность & Код\\ \hline
К & 0.40 & 0\\\hline
Т & 0.30 & 10\\\hline
И & 0.30 & 11
\\ \hline \end{tabular}
\end{center}
Энтропия алфавита: 1.5710
\begin{center}
 \begin{tabular}{ |c|c|l| } 
  \hline
     Блок & Вероятность & Код\\ \hline
ККК & 0.06 & 1000\\\hline
ККИ & 0.05 & 0010\\\hline
ККТ & 0.05 & 0011\\\hline
КИК & 0.05 & 11110\\\hline
ИКК & 0.05 & 11111\\\hline
КТК & 0.05 & 0000\\\hline
ТКК & 0.05 & 0001\\\hline
ТТК & 0.04 & 10010\\\hline
ИИК & 0.04 & 10011\\\hline
КТТ & 0.04 & 10100\\\hline
КИИ & 0.04 & 10101\\\hline
ИКИ & 0.04 & 10110\\\hline
ИТК & 0.04 & 10111\\\hline
КТИ & 0.04 & 11000\\\hline
КИТ & 0.04 & 11001\\\hline
ИКТ & 0.04 & 11010\\\hline
ТКИ & 0.04 & 11011\\\hline
ТИК & 0.04 & 11100\\\hline
ТКТ & 0.04 & 11101\\\hline
ИТИ & 0.03 & 01000\\\hline
ТТИ & 0.03 & 01001\\\hline
ИТТ & 0.03 & 01010\\\hline
ИИИ & 0.03 & 01011\\\hline
ИИТ & 0.03 & 01100\\\hline
ТТТ & 0.03 & 01101\\\hline
ТИИ & 0.03 & 01110\\\hline
ТИТ & 0.03 & 01111
\\ \hline \end{tabular}
\end{center}
Бит на символ при посимвольном кодировании: 1.6000, при блочном: 1.5813


\pagebreak
\paragraph{Задание 2. Сжать адаптивным хаффманом\\}

Строка: 
УКВАУКВСАК\\
Результат: 'У' 0'К' 00'В' 100'А' 10 10 01 000'С' 001 10










\pagebreak
\paragraph{Задание 3.1}

Закодировать сообщение методом LZ77\\
Строка:СОКОЛ\_СОК\_КОЛ\_КОЛОСОК\\
Результат: <0,0,С> <0,0,О> <0,0,К> <8,1,Л> <0,0,\_> <4,3,\_> <2,4,К> <6,2,О> <0,0,С> <2,1,К>\\
\begin{table}[h!]
\centering
\begin{tabular}{|c|c|c|c|c|c|c|c|c|c|c|c|c|c|c|c|c|} 
\hline
\multicolumn{10}{|c|}{Cловарь} & \multicolumn{6}{c|}{Буфер} & Код  \\ \hline
  &   &   &   &   &   &   &   &   &   & \cellcolor[HTML]{8CE4F6} С & О & К & О & Л &   & <0,0,С>
\\ \hline
  &   &   &   &   &   &   &   &   & С & \cellcolor[HTML]{8CE4F6} О & К & О & Л &   & С & <0,0,О>
\\ \hline
  &   &   &   &   &   &   &   & С & О & \cellcolor[HTML]{8CE4F6} К & О & Л &   & С & О & <0,0,К>
\\ \hline
  &   &   &   &   &   &   & С & \cellcolor[HTML]{FFFF00} О & К & \cellcolor[HTML]{FFFF00} О & \cellcolor[HTML]{8CE4F6} Л &   & С & О & К & <8,1,Л>
\\ \hline
  &   &   &   &   & С & О & К & О & Л & \cellcolor[HTML]{8CE4F6}   & С & О & К &   & К & <0,0,\_>
\\ \hline
  &   &   &   & \cellcolor[HTML]{FFFF00} С & \cellcolor[HTML]{FFFF00} О & \cellcolor[HTML]{FFFF00} К & О & Л &   & \cellcolor[HTML]{FFFF00} С & \cellcolor[HTML]{FFFF00} О & \cellcolor[HTML]{FFFF00} К & \cellcolor[HTML]{8CE4F6}   & К & О & <4,3,\_>
\\ \hline
С & О & \cellcolor[HTML]{FFFF00} К & \cellcolor[HTML]{FFFF00} О & \cellcolor[HTML]{FFFF00} Л & \cellcolor[HTML]{FFFF00}   & С & О & К &   & \cellcolor[HTML]{FFFF00} К & \cellcolor[HTML]{FFFF00} О & \cellcolor[HTML]{FFFF00} Л & \cellcolor[HTML]{FFFF00}   & \cellcolor[HTML]{8CE4F6} К & О & <2,4,К>
\\ \hline
  & С & О & К &   & К & \cellcolor[HTML]{FFFF00} О & \cellcolor[HTML]{FFFF00} Л &   & К & \cellcolor[HTML]{FFFF00} О & \cellcolor[HTML]{FFFF00} Л & \cellcolor[HTML]{8CE4F6} О & С & О & К & <6,2,О>
\\ \hline
К &   & К & О & Л &   & К & О & Л & О & \cellcolor[HTML]{8CE4F6} С & О & К &   &   &   & <0,0,С>
\\ \hline
  & К & \cellcolor[HTML]{FFFF00} О & Л &   & К & О & Л & О & С & \cellcolor[HTML]{FFFF00} О & \cellcolor[HTML]{8CE4F6} К &   &   &   &   & <2,1,К>
\\ \hline
\end{tabular}
\end{table}

\paragraph{Задание 3.3}

Закодировать сообщение методом LZ78\\
Строка:СОКОЛ\_СОК\_КОЛ\_КОЛОСОК\\
\begin{table}[h!]
\centering
\begin{tabular}{|c|c|c|} 
\hline
 Входная фраза (в словарь) & Код & Позиция словаря \\ \hline

 &  & 0 \\ \hline
С & 0'С' & 1 \\ \hline
О & 0'О' & 2 \\ \hline
К & 0'К' & 3 \\ \hline
ОЛ & 2'Л' & 4 \\ \hline
\_ & 0'\_' & 5 \\ \hline
СО & 1'О' & 6 \\ \hline
К\_ & 3'\_' & 7 \\ \hline
КО & 3'О' & 8 \\ \hline
Л & 0'Л' & 9 \\ \hline
\_К & 5'К' & 10 \\ \hline
ОЛО & 4'О' & 11 \\ \hline
СОК & 6'К' & 12 \\ \hline
\end{tabular}
\end{table}

Результат: 0'С' 0'О' 0'К' 2'Л' 0'\_' 1'О' 3'\_' 3'О' 0'Л' 5'К' 4'О' 6'К'\\
\pagebreak
\paragraph{Задание 4. Арифметическое кодирование\\}

Исходная строка: УКВАУКВСАК\
\begin{center}
 \begin{tabular}{ |c|c| } 
  \hline
     Буква & Вероятность \\ \hline
К & 0.30\\\hline
А & 0.20\\\hline
В & 0.20\\\hline
У & 0.20\\\hline
С & 0.10
\\ \hline \end{tabular}
\end{center}
\begin{center}
 \begin{tabular}{ |c|c|c| } 
  \hline
     Буква & Начало & Конец \\ \hline
К & 0.00 & 0.30\\\hline
А & 0.30 & 0.50\\\hline
В & 0.50 & 0.70\\\hline
У & 0.70 & 0.90\\\hline
С & 0.90 & 1.00
\\ \hline \end{tabular}
\end{center}
\begin{center}
 \begin{tabular}{ |c|c|c|c| } 
  \hline
     Буква & delta & min & max \\ \hline
У & 0.2000000000 & 0.7000000000 & 0.9000000000\\\hline
К & 0.0600000000 & 0.7000000000 & 0.7600000000\\\hline
В & 0.0120000000 & 0.7300000000 & 0.7420000000\\\hline
А & 0.0024000000 & 0.7336000000 & 0.7360000000\\\hline
У & 0.0004800000 & 0.7352800000 & 0.7357600000\\\hline
К & 0.0001440000 & 0.7352800000 & 0.7354240000\\\hline
В & 0.0000288000 & 0.7353520000 & 0.7353808000\\\hline
С & 0.0000028800 & 0.7353779200 & 0.7353808000\\\hline
А & 0.0000005760 & 0.7353787840 & 0.7353793600\\\hline
К & 0.0000001728 & 0.7353787840 & 0.7353789568
\\ \hline \end{tabular}
\end{center}
Результат: 7353788
\pagebreak
\paragraph{Задание 5.1}

\\ 

Декодировать сообщение методом адаптивного хаффмана \\
Строка: 
'Y'0'T'00'R'100'F'01001111101111111\\
Результат: YTRFRFRYYR










\paragraph{Задание 5.3 Декодировать строку(LZ78)\\}

Исходная строка: [0'д'] [0'о'] [0'р'] [2'г'] [0'а'] [0' '] [0'г'] [2'р'] [5' '] [7'о'] [3'о'] [0'д']\\
\begin{table}[h!]
\centering
\begin{tabular}{|c|c|c|} 
\hline
 Код & Словарь & Выходной поток 
\hline

 & [] & 
\\ \hline
0'д' & [, д] & д
\\ \hline
0'о' & [, д, о] & о
\\ \hline
0'р' & [, д, о, р] & р
\\ \hline
2'г' & [, д, о, р, ог] & ог
\\ \hline
0'а' & [, д, о, р, ог, а] & а
\\ \hline
0' ' & [, д, о, р, ог, а,  ] &  
\\ \hline
0'г' & [, д, о, р, ог, а,  , г] & г
\\ \hline
2'р' & [, д, о, р, ог, а,  , г, ор] & ор
\\ \hline
5' ' & [, д, о, р, ог, а,  , г, ор, а ] & а 
\\ \hline
7'о' & [, д, о, р, ог, а,  , г, ор, а , го] & го
\\ \hline
3'о' & [, д, о, р, ог, а,  , г, ор, а , го, ро] & ро
\\ \hline
0'д' & [, д, о, р, ог, а,  , г, ор, а , го, ро, д] & д
\\ \hline
\end{tabular}
\end{table}

Результат: дорога гора город
\pagebreak
\subsection{Вариант №13}
\paragraph{Задание 1. Блочный хаффман \\}

Строка БОББББОБОО, размер блока: 3
\begin{center}
 \begin{tabular}{ |c|c|l| } 
  \hline
     Буква & Вероятность & Код\\ \hline
Б & 0.60 & 1\\\hline
О & 0.40 & 0
\\ \hline \end{tabular}
\end{center}
Энтропия алфавита: 0.9710
\begin{center}
 \begin{tabular}{ |c|c|l| } 
  \hline
     Блок & Вероятность & Код\\ \hline
БББ & 0.22 & 01\\\hline
ББО & 0.14 & 100\\\hline
ОББ & 0.14 & 101\\\hline
БОБ & 0.14 & 110\\\hline
ООБ & 0.10 & 001\\\hline
ОБО & 0.10 & 1111\\\hline
БОО & 0.10 & 000\\\hline
ООО & 0.06 & 1110
\\ \hline \end{tabular}
\end{center}
Бит на символ при посимвольном кодировании: 1.0000, при блочном: 0.9813


\pagebreak
\paragraph{Задание 2. Сжать адаптивным хаффманом\\}

Строка: 
ЛПРИРПТОРТ\\
Результат: 'Л' 0'П' 00'Р' 100'И' 01 01 000'Т' 0100'О' 01 001










\pagebreak

\paragraph{Задание 3.3}

Закодировать сообщение методом LZ78\\
Строка:ПЕС\_ПЕСОК\_СОКОЛ\_СКОЛ\\
\begin{table}[h!]
\centering
\begin{tabular}{|c|c|c|} 
\hline
 Входная фраза (в словарь) & Код & Позиция словаря \\ \hline

 &  & 0 \\ \hline
П & 0'П' & 1 \\ \hline
Е & 0'Е' & 2 \\ \hline
С & 0'С' & 3 \\ \hline
\_ & 0'\_' & 4 \\ \hline
ПЕ & 1'Е' & 5 \\ \hline
СО & 3'О' & 6 \\ \hline
К & 0'К' & 7 \\ \hline
\_С & 4'С' & 8 \\ \hline
О & 0'О' & 9 \\ \hline
КО & 7'О' & 10 \\ \hline
Л & 0'Л' & 11 \\ \hline
\_СК & 8'К' & 12 \\ \hline
ОЛ & 9'Л' & 13 \\ \hline
\end{tabular}
\end{table}

Результат: 0'П' 0'Е' 0'С' 0'\_' 1'Е' 3'О' 0'К' 4'С' 0'О' 7'О' 0'Л' 8'К' 9'Л'\\
\pagebreak
\paragraph{Задание 4. Арифметическое кодирование\\}

Исходная строка: ЛПРИРПТОРТ\
\begin{center}
 \begin{tabular}{ |c|c| } 
  \hline
     Буква & Вероятность \\ \hline
Р & 0.30\\\hline
Т & 0.20\\\hline
П & 0.20\\\hline
И & 0.10\\\hline
Л & 0.10\\\hline
О & 0.10
\\ \hline \end{tabular}
\end{center}
\begin{center}
 \begin{tabular}{ |c|c|c| } 
  \hline
     Буква & Начало & Конец \\ \hline
Р & 0.00 & 0.30\\\hline
Т & 0.30 & 0.50\\\hline
П & 0.50 & 0.70\\\hline
И & 0.70 & 0.80\\\hline
Л & 0.80 & 0.90\\\hline
О & 0.90 & 1.00
\\ \hline \end{tabular}
\end{center}
\begin{center}
 \begin{tabular}{ |c|c|c|c| } 
  \hline
     Буква & delta & min & max \\ \hline
Л & 0.1000000000 & 0.8000000000 & 0.9000000000\\\hline
П & 0.0200000000 & 0.8500000000 & 0.8700000000\\\hline
Р & 0.0060000000 & 0.8500000000 & 0.8560000000\\\hline
И & 0.0006000000 & 0.8542000000 & 0.8548000000\\\hline
Р & 0.0001800000 & 0.8542000000 & 0.8543800000\\\hline
П & 0.0000360000 & 0.8542900000 & 0.8543260000\\\hline
Т & 0.0000072000 & 0.8543008000 & 0.8543080000\\\hline
О & 0.0000007200 & 0.8543072800 & 0.8543080000\\\hline
Р & 0.0000002160 & 0.8543072800 & 0.8543074960\\\hline
Т & 0.0000000432 & 0.8543073448 & 0.8543073880
\\ \hline \end{tabular}
\end{center}
Результат: 85430735
\pagebreak
\paragraph{Задание 5.1}

\\ 

Декодировать сообщение методом адаптивного хаффмана \\
Строка: 
'D'0'C'00'V'1110111100'F'100111110\\
Результат: DCVCVVFFFF










\paragraph{Задание 5.3 Декодировать строку(LZ78)\\}

Исходная строка: [0'п'] [0'о'] [0'р'] [0'т'] [0' '] [1'о'] [3'а'] [5'р'] [0'а'] [6'р'] [0'т']\\
\begin{table}[h!]
\centering
\begin{tabular}{|c|c|c|} 
\hline
 Код & Словарь & Выходной поток 
\hline

 & [] & 
\\ \hline
0'п' & [, п] & п
\\ \hline
0'о' & [, п, о] & о
\\ \hline
0'р' & [, п, о, р] & р
\\ \hline
0'т' & [, п, о, р, т] & т
\\ \hline
0' ' & [, п, о, р, т,  ] &  
\\ \hline
1'о' & [, п, о, р, т,  , по] & по
\\ \hline
3'а' & [, п, о, р, т,  , по, ра] & ра
\\ \hline
5'р' & [, п, о, р, т,  , по, ра,  р] &  р
\\ \hline
0'а' & [, п, о, р, т,  , по, ра,  р, а] & а
\\ \hline
6'р' & [, п, о, р, т,  , по, ра,  р, а, пор] & пор
\\ \hline
0'т' & [, п, о, р, т,  , по, ра,  р, а, пор, т] & т
\\ \hline
\end{tabular}
\end{table}

Результат: порт пора рапорт
\pagebreak
\subsection{Вариант №14}
\paragraph{Задание 1. Блочный хаффман \\}

Строка КРООРТТТТТ, размер блока: 2
\begin{center}
 \begin{tabular}{ |c|c|l| } 
  \hline
     Буква & Вероятность & Код\\ \hline
Т & 0.50 & 0\\\hline
Р & 0.20 & 111\\\hline
О & 0.20 & 10\\\hline
К & 0.10 & 110
\\ \hline \end{tabular}
\end{center}
Энтропия алфавита: 1.7610
\begin{center}
 \begin{tabular}{ |c|c|l| } 
  \hline
     Блок & Вероятность & Код\\ \hline
ТТ & 0.25 & 10\\\hline
РТ & 0.10 & 1111\\\hline
ОТ & 0.10 & 000\\\hline
ТО & 0.10 & 001\\\hline
ТР & 0.10 & 010\\\hline
КТ & 0.05 & 11101\\\hline
ТК & 0.05 & 0110\\\hline
РР & 0.04 & 11000\\\hline
ОО & 0.04 & 11001\\\hline
ОР & 0.04 & 11010\\\hline
РО & 0.04 & 11011\\\hline
КО & 0.02 & 011111\\\hline
КР & 0.02 & 111000\\\hline
РК & 0.02 & 111001\\\hline
ОК & 0.02 & 01110\\\hline
КК & 0.01 & 011110
\\ \hline \end{tabular}
\end{center}
Бит на символ при посимвольном кодировании: 1.8000, при блочном: 1.7850


\pagebreak
\paragraph{Задание 2. Сжать адаптивным хаффманом\\}

Строка: 
СААВИПВАИИ\\
Результат: 'С' 0'А' 01 00'В' 000'И' 1100'П' 01 11 101 00










\pagebreak

\paragraph{Задание 3.3}

Закодировать сообщение методом LZ78\\
Строка:РАБ\_РАБА\_БАК\_БАКЕН\_БАК\\
\begin{table}[h!]
\centering
\begin{tabular}{|c|c|c|} 
\hline
 Входная фраза (в словарь) & Код & Позиция словаря \\ \hline

 &  & 0 \\ \hline
Р & 0'Р' & 1 \\ \hline
А & 0'А' & 2 \\ \hline
Б & 0'Б' & 3 \\ \hline
\_ & 0'\_' & 4 \\ \hline
РА & 1'А' & 5 \\ \hline
БА & 3'А' & 6 \\ \hline
\_Б & 4'Б' & 7 \\ \hline
АК & 2'К' & 8 \\ \hline
\_БА & 7'А' & 9 \\ \hline
К & 0'К' & 10 \\ \hline
Е & 0'Е' & 11 \\ \hline
Н & 0'Н' & 12 \\ \hline
\_БАК & 9'К' & 13 \\ \hline
\end{tabular}
\end{table}

Результат: 0'Р' 0'А' 0'Б' 0'\_' 1'А' 3'А' 4'Б' 2'К' 7'А' 0'К' 0'Е' 0'Н' 9'К'\\
\pagebreak
\paragraph{Задание 4. Арифметическое кодирование\\}

Исходная строка: СААВИПВАИИ\
\begin{center}
 \begin{tabular}{ |c|c| } 
  \hline
     Буква & Вероятность \\ \hline
А & 0.30\\\hline
И & 0.30\\\hline
В & 0.20\\\hline
С & 0.10\\\hline
П & 0.10
\\ \hline \end{tabular}
\end{center}
\begin{center}
 \begin{tabular}{ |c|c|c| } 
  \hline
     Буква & Начало & Конец \\ \hline
А & 0.00 & 0.30\\\hline
И & 0.30 & 0.60\\\hline
В & 0.60 & 0.80\\\hline
С & 0.80 & 0.90\\\hline
П & 0.90 & 1.00
\\ \hline \end{tabular}
\end{center}
\begin{center}
 \begin{tabular}{ |c|c|c|c| } 
  \hline
     Буква & delta & min & max \\ \hline
С & 0.1000000000 & 0.8000000000 & 0.9000000000\\\hline
А & 0.0300000000 & 0.8000000000 & 0.8300000000\\\hline
А & 0.0090000000 & 0.8000000000 & 0.8090000000\\\hline
В & 0.0018000000 & 0.8054000000 & 0.8072000000\\\hline
И & 0.0005400000 & 0.8059400000 & 0.8064800000\\\hline
П & 0.0000540000 & 0.8064260000 & 0.8064800000\\\hline
В & 0.0000108000 & 0.8064584000 & 0.8064692000\\\hline
А & 0.0000032400 & 0.8064584000 & 0.8064616400\\\hline
И & 0.0000009720 & 0.8064593720 & 0.8064603440\\\hline
И & 0.0000002916 & 0.8064596636 & 0.8064599552
\\ \hline \end{tabular}
\end{center}
Результат: 8064597
\pagebreak
\paragraph{Задание 5.1}

\\ 

Декодировать сообщение методом адаптивного хаффмана \\
Строка: 
'S'0'X'00'C'100'D'010011001001111\\
Результат: SXCDCDDDSS










\paragraph{Задание 5.3 Декодировать строку(LZ78)\\}

Исходная строка: [0'т'] [0'о'] [0'р'] [1' '] [0'с'] [2'р'] [4'с'] [6' '] [5'п'] [6'т']\\
\begin{table}[h!]
\centering
\begin{tabular}{|c|c|c|} 
\hline
 Код & Словарь & Выходной поток 
\hline

 & [] & 
\\ \hline
0'т' & [, т] & т
\\ \hline
0'о' & [, т, о] & о
\\ \hline
0'р' & [, т, о, р] & р
\\ \hline
1' ' & [, т, о, р, т ] & т 
\\ \hline
0'с' & [, т, о, р, т , с] & с
\\ \hline
2'р' & [, т, о, р, т , с, ор] & ор
\\ \hline
4'с' & [, т, о, р, т , с, ор, т с] & т с
\\ \hline
6' ' & [, т, о, р, т , с, ор, т с, ор ] & ор 
\\ \hline
5'п' & [, т, о, р, т , с, ор, т с, ор , сп] & сп
\\ \hline
6'т' & [, т, о, р, т , с, ор, т с, ор , сп, орт] & орт
\\ \hline
\end{tabular}
\end{table}

Результат: торт сорт сор спорт
\pagebreak
\subsection{Вариант №15}
\paragraph{Задание 1. Блочный хаффман \\}

Строка БИББББИИИБ, размер блока: 3
\begin{center}
 \begin{tabular}{ |c|c|l| } 
  \hline
     Буква & Вероятность & Код\\ \hline
Б & 0.60 & 1\\\hline
И & 0.40 & 0
\\ \hline \end{tabular}
\end{center}
Энтропия алфавита: 0.9710
\begin{center}
 \begin{tabular}{ |c|c|l| } 
  \hline
     Блок & Вероятность & Код\\ \hline
БББ & 0.22 & 01\\\hline
БИБ & 0.14 & 100\\\hline
ББИ & 0.14 & 101\\\hline
ИББ & 0.14 & 110\\\hline
ИИБ & 0.10 & 001\\\hline
ИБИ & 0.10 & 1111\\\hline
БИИ & 0.10 & 000\\\hline
ИИИ & 0.06 & 1110
\\ \hline \end{tabular}
\end{center}
Бит на символ при посимвольном кодировании: 1.0000, при блочном: 0.9813


\pagebreak
\paragraph{Задание 2. Сжать адаптивным хаффманом\\}

Строка: 
УВАААУУКПУ\\
Результат: 'У' 0'В' 00'А' 101 0 00 01 100'К' 1000'П' 11










\pagebreak

\paragraph{Задание 3.3}

Закодировать сообщение методом LZ78\\
Строка:ТАРА\_ТАРТАР\_ТАРЕЛКА\_ЕЛКА\\
\begin{table}[h!]
\centering
\begin{tabular}{|c|c|c|} 
\hline
 Входная фраза (в словарь) & Код & Позиция словаря \\ \hline

 &  & 0 \\ \hline
Т & 0'Т' & 1 \\ \hline
А & 0'А' & 2 \\ \hline
Р & 0'Р' & 3 \\ \hline
А\_ & 2'\_' & 4 \\ \hline
ТА & 1'А' & 5 \\ \hline
РТ & 3'Т' & 6 \\ \hline
АР & 2'Р' & 7 \\ \hline
\_ & 0'\_' & 8 \\ \hline
ТАР & 5'Р' & 9 \\ \hline
Е & 0'Е' & 10 \\ \hline
Л & 0'Л' & 11 \\ \hline
К & 0'К' & 12 \\ \hline
А\_Е & 4'Е' & 13 \\ \hline
ЛК & 11'К' & 14 \\ \hline
\end{tabular}
\end{table}

Результат: 0'Т' 0'А' 0'Р' 2'\_' 1'А' 3'Т' 2'Р' 0'\_' 5'Р' 0'Е' 0'Л' 0'К' 4'Е' 11'К'\\
\pagebreak
\paragraph{Задание 4. Арифметическое кодирование\\}

Исходная строка: УВАААУУКПУ\
\begin{center}
 \begin{tabular}{ |c|c| } 
  \hline
     Буква & Вероятность \\ \hline
У & 0.40\\\hline
А & 0.30\\\hline
В & 0.10\\\hline
К & 0.10\\\hline
П & 0.10
\\ \hline \end{tabular}
\end{center}
\begin{center}
 \begin{tabular}{ |c|c|c| } 
  \hline
     Буква & Начало & Конец \\ \hline
У & 0.00 & 0.40\\\hline
А & 0.40 & 0.70\\\hline
В & 0.70 & 0.80\\\hline
К & 0.80 & 0.90\\\hline
П & 0.90 & 1.00
\\ \hline \end{tabular}
\end{center}
\begin{center}
 \begin{tabular}{ |c|c|c|c| } 
  \hline
     Буква & delta & min & max \\ \hline
У & 0.4000000000 & 0.0000000000 & 0.4000000000\\\hline
В & 0.0400000000 & 0.2800000000 & 0.3200000000\\\hline
А & 0.0120000000 & 0.2960000000 & 0.3080000000\\\hline
А & 0.0036000000 & 0.3008000000 & 0.3044000000\\\hline
А & 0.0010800000 & 0.3022400000 & 0.3033200000\\\hline
У & 0.0004320000 & 0.3022400000 & 0.3026720000\\\hline
У & 0.0001728000 & 0.3022400000 & 0.3024128000\\\hline
К & 0.0000172800 & 0.3023782400 & 0.3023955200\\\hline
П & 0.0000017280 & 0.3023937920 & 0.3023955200\\\hline
У & 0.0000006912 & 0.3023937920 & 0.3023944832
\\ \hline \end{tabular}
\end{center}
Результат: 302394
\pagebreak
\paragraph{Задание 5.1}

\\ 

Декодировать сообщение методом адаптивного хаффмана \\
Строка: 
'K'0'C'00'B'100'V'100110100111110\\
Результат: KCBVKBBVVV










\paragraph{Задание 5.3 Декодировать строку(LZ78)\\}

Исходная строка: [0'т'] [0'о'] [0'н'] [0'и'] [0'к'] [0' '] [1'о'] [3' '] [7'н'] [3'а']\\
\begin{table}[h!]
\centering
\begin{tabular}{|c|c|c|} 
\hline
 Код & Словарь & Выходной поток 
\hline

 & [] & 
\\ \hline
0'т' & [, т] & т
\\ \hline
0'о' & [, т, о] & о
\\ \hline
0'н' & [, т, о, н] & н
\\ \hline
0'и' & [, т, о, н, и] & и
\\ \hline
0'к' & [, т, о, н, и, к] & к
\\ \hline
0' ' & [, т, о, н, и, к,  ] &  
\\ \hline
1'о' & [, т, о, н, и, к,  , то] & то
\\ \hline
3' ' & [, т, о, н, и, к,  , то, н ] & н 
\\ \hline
7'н' & [, т, о, н, и, к,  , то, н , тон] & тон
\\ \hline
3'а' & [, т, о, н, и, к,  , то, н , тон, на] & на
\\ \hline
\end{tabular}
\end{table}

Результат: тоник тон тонна
\pagebreak
\subsection{Вариант №16}
\paragraph{Задание 1. Блочный хаффман \\}

Строка ДЕЕДКУДДКК, размер блока: 2
\begin{center}
 \begin{tabular}{ |c|c|l| } 
  \hline
     Буква & Вероятность & Код\\ \hline
Д & 0.40 & 0\\\hline
К & 0.30 & 10\\\hline
Е & 0.20 & 111\\\hline
У & 0.10 & 110
\\ \hline \end{tabular}
\end{center}
Энтропия алфавита: 1.8464
\begin{center}
 \begin{tabular}{ |c|c|l| } 
  \hline
     Блок & Вероятность & Код\\ \hline
ДД & 0.16 & 110\\\hline
ДК & 0.12 & 010\\\hline
КД & 0.12 & 011\\\hline
КК & 0.09 & 000\\\hline
ДЕ & 0.08 & 1011\\\hline
ЕД & 0.08 & 1110\\\hline
ЕК & 0.06 & 1000\\\hline
КЕ & 0.06 & 1001\\\hline
ЕЕ & 0.04 & 11110\\\hline
ДУ & 0.04 & 11111\\\hline
УД & 0.04 & 0010\\\hline
КУ & 0.03 & 00111\\\hline
УК & 0.03 & 10100\\\hline
ЕУ & 0.02 & 101011\\\hline
УЕ & 0.02 & 00110\\\hline
УУ & 0.01 & 101010
\\ \hline \end{tabular}
\end{center}
Бит на символ при посимвольном кодировании: 1.9000, при блочном: 1.8650


\pagebreak
\paragraph{Задание 2. Сжать адаптивным хаффманом\\}

Строка: 
РПЕАКАРРПП\\
Результат: 'Р' 0'П' 00'Е' 100'А' 000'К' 111 01 01 00 00










\pagebreak
\paragraph{Задание 3.1}

Закодировать сообщение методом LZ77\\
Строка:УКУС\_КУСКУС\_УКСУС\_КСИ\\
Результат: <0,0,У> <0,0,К> <8,1,С> <0,0,\_> <6,3,К> <3,3,У> <2,1,С> <4,3,К> <1,1,И>\\
\begin{table}[h!]
\centering
\begin{tabular}{|c|c|c|c|c|c|c|c|c|c|c|c|c|c|c|c|c|} 
\hline
\multicolumn{10}{|c|}{Cловарь} & \multicolumn{6}{c|}{Буфер} & Код  \\ \hline
  &   &   &   &   &   &   &   &   &   & \cellcolor[HTML]{8CE4F6} У & К & У & С &   & К & <0,0,У>
\\ \hline
  &   &   &   &   &   &   &   &   & У & \cellcolor[HTML]{8CE4F6} К & У & С &   & К & У & <0,0,К>
\\ \hline
  &   &   &   &   &   &   &   & \cellcolor[HTML]{FFFF00} У & К & \cellcolor[HTML]{FFFF00} У & \cellcolor[HTML]{8CE4F6} С &   & К & У & С & <8,1,С>
\\ \hline
  &   &   &   &   &   & У & К & У & С & \cellcolor[HTML]{8CE4F6}   & К & У & С & К & У & <0,0,\_>
\\ \hline
  &   &   &   &   & У & \cellcolor[HTML]{FFFF00} К & \cellcolor[HTML]{FFFF00} У & \cellcolor[HTML]{FFFF00} С &   & \cellcolor[HTML]{FFFF00} К & \cellcolor[HTML]{FFFF00} У & \cellcolor[HTML]{FFFF00} С & \cellcolor[HTML]{8CE4F6} К & У & С & <6,3,К>
\\ \hline
  & У & К & \cellcolor[HTML]{FFFF00} У & \cellcolor[HTML]{FFFF00} С & \cellcolor[HTML]{FFFF00}   & К & У & С & К & \cellcolor[HTML]{FFFF00} У & \cellcolor[HTML]{FFFF00} С & \cellcolor[HTML]{FFFF00}   & \cellcolor[HTML]{8CE4F6} У & К & С & <3,3,У>
\\ \hline
С &   & \cellcolor[HTML]{FFFF00} К & У & С & К & У & С &   & У & \cellcolor[HTML]{FFFF00} К & \cellcolor[HTML]{8CE4F6} С & У & С &   & К & <2,1,С>
\\ \hline
К & У & С & К & \cellcolor[HTML]{FFFF00} У & \cellcolor[HTML]{FFFF00} С & \cellcolor[HTML]{FFFF00}   & У & К & С & \cellcolor[HTML]{FFFF00} У & \cellcolor[HTML]{FFFF00} С & \cellcolor[HTML]{FFFF00}   & \cellcolor[HTML]{8CE4F6} К & С & И & <4,3,К>
\\ \hline
У & \cellcolor[HTML]{FFFF00} С &   & У & К & С & У & С &   & К & \cellcolor[HTML]{FFFF00} С & \cellcolor[HTML]{8CE4F6} И &   &   &   &   & <1,1,И>
\\ \hline
\end{tabular}
\end{table}

\paragraph{Задание 3.3}

Закодировать сообщение методом LZ78\\
Строка:УКУС\_КУСКУС\_УКСУС\_КСИ\\
\begin{table}[h!]
\centering
\begin{tabular}{|c|c|c|} 
\hline
 Входная фраза (в словарь) & Код & Позиция словаря \\ \hline

 &  & 0 \\ \hline
У & 0'У' & 1 \\ \hline
К & 0'К' & 2 \\ \hline
УС & 1'С' & 3 \\ \hline
\_ & 0'\_' & 4 \\ \hline
КУ & 2'У' & 5 \\ \hline
С & 0'С' & 6 \\ \hline
КУС & 5'С' & 7 \\ \hline
\_У & 4'У' & 8 \\ \hline
КС & 2'С' & 9 \\ \hline
УС\_ & 3'\_' & 10 \\ \hline
КСИ & 9'И' & 11 \\ \hline
\end{tabular}
\end{table}

Результат: 0'У' 0'К' 1'С' 0'\_' 2'У' 0'С' 5'С' 4'У' 2'С' 3'\_' 9'И'\\
\pagebreak
\paragraph{Задание 4. Арифметическое кодирование\\}

Исходная строка: РПЕАКАРРПП\
\begin{center}
 \begin{tabular}{ |c|c| } 
  \hline
     Буква & Вероятность \\ \hline
Р & 0.30\\\hline
П & 0.30\\\hline
А & 0.20\\\hline
Е & 0.10\\\hline
К & 0.10
\\ \hline \end{tabular}
\end{center}
\begin{center}
 \begin{tabular}{ |c|c|c| } 
  \hline
     Буква & Начало & Конец \\ \hline
Р & 0.00 & 0.30\\\hline
П & 0.30 & 0.60\\\hline
А & 0.60 & 0.80\\\hline
Е & 0.80 & 0.90\\\hline
К & 0.90 & 1.00
\\ \hline \end{tabular}
\end{center}
\begin{center}
 \begin{tabular}{ |c|c|c|c| } 
  \hline
     Буква & delta & min & max \\ \hline
Р & 0.3000000000 & 0.0000000000 & 0.3000000000\\\hline
П & 0.0900000000 & 0.0900000000 & 0.1800000000\\\hline
Е & 0.0090000000 & 0.1620000000 & 0.1710000000\\\hline
А & 0.0018000000 & 0.1674000000 & 0.1692000000\\\hline
К & 0.0001800000 & 0.1690200000 & 0.1692000000\\\hline
А & 0.0000360000 & 0.1691280000 & 0.1691640000\\\hline
Р & 0.0000108000 & 0.1691280000 & 0.1691388000\\\hline
Р & 0.0000032400 & 0.1691280000 & 0.1691312400\\\hline
П & 0.0000009720 & 0.1691289720 & 0.1691299440\\\hline
П & 0.0000002916 & 0.1691292636 & 0.1691295552
\\ \hline \end{tabular}
\end{center}
Результат: 1691293
\pagebreak
\paragraph{Задание 5.1}

\\ 

Декодировать сообщение методом адаптивного хаффмана \\
Строка: 
'Z'0'X'00'Y'10110110111100'D'11110\\
Результат: ZXYYZZZZZZZDZZZZ
















\paragraph{Задание 5.3 Декодировать строку(LZ78)\\}

Исходная строка: [0'с'] [0'и'] [0'л'] [0'а'] [0' '] [3'а'] [1'к'] [4' '] [6'с'] [0'т'] [8'с'] [10'а'] [0'н']\\
\begin{table}[h!]
\centering
\begin{tabular}{|c|c|c|} 
\hline
 Код & Словарь & Выходной поток 
\hline

 & [] & 
\\ \hline
0'с' & [, с] & с
\\ \hline
0'и' & [, с, и] & и
\\ \hline
0'л' & [, с, и, л] & л
\\ \hline
0'а' & [, с, и, л, а] & а
\\ \hline
0' ' & [, с, и, л, а,  ] &  
\\ \hline
3'а' & [, с, и, л, а,  , ла] & ла
\\ \hline
1'к' & [, с, и, л, а,  , ла, ск] & ск
\\ \hline
4' ' & [, с, и, л, а,  , ла, ск, а ] & а 
\\ \hline
6'с' & [, с, и, л, а,  , ла, ск, а , лас] & лас
\\ \hline
0'т' & [, с, и, л, а,  , ла, ск, а , лас, т] & т
\\ \hline
8'с' & [, с, и, л, а,  , ла, ск, а , лас, т, а с] & а с
\\ \hline
10'а' & [, с, и, л, а,  , ла, ск, а , лас, т, а с, та] & та
\\ \hline
0'н' & [, с, и, л, а,  , ла, ск, а , лас, т, а с, та, н] & н
\\ \hline
\end{tabular}
\end{table}

Результат: сила ласка ласта стан
\pagebreak
\subsection{Вариант №17}
\paragraph{Задание 1. Блочный хаффман \\}

Строка ГНННОООМНГ, размер блока: 2
\begin{center}
 \begin{tabular}{ |c|c|l| } 
  \hline
     Буква & Вероятность & Код\\ \hline
Н & 0.40 & 0\\\hline
О & 0.30 & 10\\\hline
Г & 0.20 & 111\\\hline
М & 0.10 & 110
\\ \hline \end{tabular}
\end{center}
Энтропия алфавита: 1.8464
\begin{center}
 \begin{tabular}{ |c|c|l| } 
  \hline
     Блок & Вероятность & Код\\ \hline
НН & 0.16 & 110\\\hline
НО & 0.12 & 010\\\hline
ОН & 0.12 & 011\\\hline
ОО & 0.09 & 000\\\hline
ГН & 0.08 & 1011\\\hline
НГ & 0.08 & 1110\\\hline
ГО & 0.06 & 1000\\\hline
ОГ & 0.06 & 1001\\\hline
ГГ & 0.04 & 11110\\\hline
МН & 0.04 & 11111\\\hline
НМ & 0.04 & 0010\\\hline
МО & 0.03 & 00111\\\hline
ОМ & 0.03 & 10100\\\hline
ГМ & 0.02 & 101011\\\hline
МГ & 0.02 & 00110\\\hline
ММ & 0.01 & 101010
\\ \hline \end{tabular}
\end{center}
Бит на символ при посимвольном кодировании: 1.9000, при блочном: 1.8650


\pagebreak
\paragraph{Задание 2. Сжать адаптивным хаффманом\\}

Строка: 
ГНРНГРНПРР\\
Результат: 'Г' 0'Н' 00'Р' 11 11 101 0 100'П' 111 10










\pagebreak
\paragraph{Задание 3.1}

Закодировать сообщение методом LZ77\\
Строка:ДОМ\_ДОМИК\_ОМИК\_МИР\\
Результат: <0,0,Д> <0,0,О> <0,0,М> <0,0,\_> <6,3,И> <0,0,К> <4,1,О> <5,4,М> <1,1,Р>\\
\begin{table}[h!]
\centering
\begin{tabular}{|c|c|c|c|c|c|c|c|c|c|c|c|c|c|c|c|c|} 
\hline
\multicolumn{10}{|c|}{Cловарь} & \multicolumn{6}{c|}{Буфер} & Код  \\ \hline
  &   &   &   &   &   &   &   &   &   & \cellcolor[HTML]{8CE4F6} Д & О & М &   & Д & О & <0,0,Д>
\\ \hline
  &   &   &   &   &   &   &   &   & Д & \cellcolor[HTML]{8CE4F6} О & М &   & Д & О & М & <0,0,О>
\\ \hline
  &   &   &   &   &   &   &   & Д & О & \cellcolor[HTML]{8CE4F6} М &   & Д & О & М & И & <0,0,М>
\\ \hline
  &   &   &   &   &   &   & Д & О & М & \cellcolor[HTML]{8CE4F6}   & Д & О & М & И & К & <0,0,\_>
\\ \hline
  &   &   &   &   &   & \cellcolor[HTML]{FFFF00} Д & \cellcolor[HTML]{FFFF00} О & \cellcolor[HTML]{FFFF00} М &   & \cellcolor[HTML]{FFFF00} Д & \cellcolor[HTML]{FFFF00} О & \cellcolor[HTML]{FFFF00} М & \cellcolor[HTML]{8CE4F6} И & К &   & <6,3,И>
\\ \hline
  &   & Д & О & М &   & Д & О & М & И & \cellcolor[HTML]{8CE4F6} К &   & О & М & И & К & <0,0,К>
\\ \hline
  & Д & О & М & \cellcolor[HTML]{FFFF00}   & Д & О & М & И & К & \cellcolor[HTML]{FFFF00}   & \cellcolor[HTML]{8CE4F6} О & М & И & К &   & <4,1,О>
\\ \hline
О & М &   & Д & О & \cellcolor[HTML]{FFFF00} М & \cellcolor[HTML]{FFFF00} И & \cellcolor[HTML]{FFFF00} К & \cellcolor[HTML]{FFFF00}   & О & \cellcolor[HTML]{FFFF00} М & \cellcolor[HTML]{FFFF00} И & \cellcolor[HTML]{FFFF00} К & \cellcolor[HTML]{FFFF00}   & \cellcolor[HTML]{8CE4F6} М & И & <5,4,М>
\\ \hline
М & \cellcolor[HTML]{FFFF00} И & К &   & О & М & И & К &   & М & \cellcolor[HTML]{FFFF00} И & \cellcolor[HTML]{8CE4F6} Р &   &   &   &   & <1,1,Р>
\\ \hline
\end{tabular}
\end{table}

\paragraph{Задание 3.3}

Закодировать сообщение методом LZ78\\
Строка:ДОМ\_ДОМИК\_ОМИК\_МИР\\
\begin{table}[h!]
\centering
\begin{tabular}{|c|c|c|} 
\hline
 Входная фраза (в словарь) & Код & Позиция словаря \\ \hline

 &  & 0 \\ \hline
Д & 0'Д' & 1 \\ \hline
О & 0'О' & 2 \\ \hline
М & 0'М' & 3 \\ \hline
\_ & 0'\_' & 4 \\ \hline
ДО & 1'О' & 5 \\ \hline
МИ & 3'И' & 6 \\ \hline
К & 0'К' & 7 \\ \hline
\_О & 4'О' & 8 \\ \hline
МИК & 6'К' & 9 \\ \hline
\_М & 4'М' & 10 \\ \hline
И & 0'И' & 11 \\ \hline
Р & 0'Р' & 12 \\ \hline
\end{tabular}
\end{table}

Результат: 0'Д' 0'О' 0'М' 0'\_' 1'О' 3'И' 0'К' 4'О' 6'К' 4'М' 0'И' 0'Р'\\
\pagebreak
\paragraph{Задание 4. Арифметическое кодирование\\}

Исходная строка: ГНРНГРНПРР\
\begin{center}
 \begin{tabular}{ |c|c| } 
  \hline
     Буква & Вероятность \\ \hline
Р & 0.40\\\hline
Н & 0.30\\\hline
Г & 0.20\\\hline
П & 0.10
\\ \hline \end{tabular}
\end{center}
\begin{center}
 \begin{tabular}{ |c|c|c| } 
  \hline
     Буква & Начало & Конец \\ \hline
Р & 0.00 & 0.40\\\hline
Н & 0.40 & 0.70\\\hline
Г & 0.70 & 0.90\\\hline
П & 0.90 & 1.00
\\ \hline \end{tabular}
\end{center}
\begin{center}
 \begin{tabular}{ |c|c|c|c| } 
  \hline
     Буква & delta & min & max \\ \hline
Г & 0.2000000000 & 0.7000000000 & 0.9000000000\\\hline
Н & 0.0600000000 & 0.7800000000 & 0.8400000000\\\hline
Р & 0.0240000000 & 0.7800000000 & 0.8040000000\\\hline
Н & 0.0072000000 & 0.7896000000 & 0.7968000000\\\hline
Г & 0.0014400000 & 0.7946400000 & 0.7960800000\\\hline
Р & 0.0005760000 & 0.7946400000 & 0.7952160000\\\hline
Н & 0.0001728000 & 0.7948704000 & 0.7950432000\\\hline
П & 0.0000172800 & 0.7950259200 & 0.7950432000\\\hline
Р & 0.0000069120 & 0.7950259200 & 0.7950328320\\\hline
Р & 0.0000027648 & 0.7950259200 & 0.7950286848
\\ \hline \end{tabular}
\end{center}
Результат: 795026
\pagebreak
\paragraph{Задание 5.1}

\\ 

Декодировать сообщение методом адаптивного хаффмана \\
Строка: 
'R'0'F'00'T'100'D'101111011111101001\\
Результат: RFTDRRDTDDD











\paragraph{Задание 5.3 Декодировать строку(LZ78)\\}

Исходная строка: [0'т'] [0'о'] [0'с'] [0'к'] [0'а'] [0' '] [3'к'] [5'л'] [5' '] [0'л'] [5'с'] [1'и'] [0'к']\\
\begin{table}[h!]
\centering
\begin{tabular}{|c|c|c|} 
\hline
 Код & Словарь & Выходной поток 
\hline

 & [] & 
\\ \hline
0'т' & [, т] & т
\\ \hline
0'о' & [, т, о] & о
\\ \hline
0'с' & [, т, о, с] & с
\\ \hline
0'к' & [, т, о, с, к] & к
\\ \hline
0'а' & [, т, о, с, к, а] & а
\\ \hline
0' ' & [, т, о, с, к, а,  ] &  
\\ \hline
3'к' & [, т, о, с, к, а,  , ск] & ск
\\ \hline
5'л' & [, т, о, с, к, а,  , ск, ал] & ал
\\ \hline
5' ' & [, т, о, с, к, а,  , ск, ал, а ] & а 
\\ \hline
0'л' & [, т, о, с, к, а,  , ск, ал, а , л] & л
\\ \hline
5'с' & [, т, о, с, к, а,  , ск, ал, а , л, ас] & ас
\\ \hline
1'и' & [, т, о, с, к, а,  , ск, ал, а , л, ас, ти] & ти
\\ \hline
0'к' & [, т, о, с, к, а,  , ск, ал, а , л, ас, ти, к] & к
\\ \hline
\end{tabular}
\end{table}

Результат: тоска скала ластик
\pagebreak
\subsection{Вариант №18}
\paragraph{Задание 1. Блочный хаффман \\}

Строка КРРРАККККК, размер блока: 2
\begin{center}
 \begin{tabular}{ |c|c|l| } 
  \hline
     Буква & Вероятность & Код\\ \hline
К & 0.60 & 1\\\hline
Р & 0.30 & 01\\\hline
А & 0.10 & 00
\\ \hline \end{tabular}
\end{center}
Энтропия алфавита: 1.2955
\begin{center}
 \begin{tabular}{ |c|c|l| } 
  \hline
     Блок & Вероятность & Код\\ \hline
КК & 0.36 & 11\\\hline
КР & 0.18 & 00\\\hline
РК & 0.18 & 01\\\hline
РР & 0.09 & 1011\\\hline
КА & 0.06 & 1000\\\hline
АК & 0.06 & 1001\\\hline
РА & 0.03 & 101011\\\hline
АР & 0.03 & 10100\\\hline
АА & 0.01 & 101010
\\ \hline \end{tabular}
\end{center}
Бит на символ при посимвольном кодировании: 1.4000, при блочном: 1.3350


\pagebreak
\paragraph{Задание 2. Сжать адаптивным хаффманом\\}

Строка: 
ГНРПАНПППП\\
Результат: 'Г' 0'Н' 00'Р' 100'П' 000'А' 00 111 01 11 0










\pagebreak
\paragraph{Задание 3.1}

Закодировать сообщение методом LZ77\\
Строка:РИМ\_РОМ\_МУРОМ\_МУРКА\\
Результат: <0,0,Р> <0,0,И> <0,0,М> <0,0,\_> <6,1,О> <6,2,М> <0,0,У> <4,5,У> <4,1,К> <0,0,А>\\
\begin{table}[h!]
\centering
\begin{tabular}{|c|c|c|c|c|c|c|c|c|c|c|c|c|c|c|c|c|} 
\hline
\multicolumn{10}{|c|}{Cловарь} & \multicolumn{6}{c|}{Буфер} & Код  \\ \hline
  &   &   &   &   &   &   &   &   &   & \cellcolor[HTML]{8CE4F6} Р & И & М &   & Р & О & <0,0,Р>
\\ \hline
  &   &   &   &   &   &   &   &   & Р & \cellcolor[HTML]{8CE4F6} И & М &   & Р & О & М & <0,0,И>
\\ \hline
  &   &   &   &   &   &   &   & Р & И & \cellcolor[HTML]{8CE4F6} М &   & Р & О & М &   & <0,0,М>
\\ \hline
  &   &   &   &   &   &   & Р & И & М & \cellcolor[HTML]{8CE4F6}   & Р & О & М &   & М & <0,0,\_>
\\ \hline
  &   &   &   &   &   & \cellcolor[HTML]{FFFF00} Р & И & М &   & \cellcolor[HTML]{FFFF00} Р & \cellcolor[HTML]{8CE4F6} О & М &   & М & У & <6,1,О>
\\ \hline
  &   &   &   & Р & И & \cellcolor[HTML]{FFFF00} М & \cellcolor[HTML]{FFFF00}   & Р & О & \cellcolor[HTML]{FFFF00} М & \cellcolor[HTML]{FFFF00}   & \cellcolor[HTML]{8CE4F6} М & У & Р & О & <6,2,М>
\\ \hline
  & Р & И & М &   & Р & О & М &   & М & \cellcolor[HTML]{8CE4F6} У & Р & О & М &   & М & <0,0,У>
\\ \hline
Р & И & М &   & \cellcolor[HTML]{FFFF00} Р & \cellcolor[HTML]{FFFF00} О & \cellcolor[HTML]{FFFF00} М & \cellcolor[HTML]{FFFF00}   & \cellcolor[HTML]{FFFF00} М & У & \cellcolor[HTML]{FFFF00} Р & \cellcolor[HTML]{FFFF00} О & \cellcolor[HTML]{FFFF00} М & \cellcolor[HTML]{FFFF00}   & \cellcolor[HTML]{FFFF00} М & \cellcolor[HTML]{8CE4F6} У & <4,5,У>
\\ \hline
М &   & М & У & \cellcolor[HTML]{FFFF00} Р & О & М &   & М & У & \cellcolor[HTML]{FFFF00} Р & \cellcolor[HTML]{8CE4F6} К & А &   &   &   & <4,1,К>
\\ \hline
М & У & Р & О & М &   & М & У & Р & К & \cellcolor[HTML]{8CE4F6} А &   &   &   &   &   & <0,0,А>
\\ \hline
\end{tabular}
\end{table}

\paragraph{Задание 3.3}

Закодировать сообщение методом LZ78\\
Строка:РИМ\_РОМ\_МУРОМ\_МУРКА\\
\begin{table}[h!]
\centering
\begin{tabular}{|c|c|c|} 
\hline
 Входная фраза (в словарь) & Код & Позиция словаря \\ \hline

 &  & 0 \\ \hline
Р & 0'Р' & 1 \\ \hline
И & 0'И' & 2 \\ \hline
М & 0'М' & 3 \\ \hline
\_ & 0'\_' & 4 \\ \hline
РО & 1'О' & 5 \\ \hline
М\_ & 3'\_' & 6 \\ \hline
МУ & 3'У' & 7 \\ \hline
РОМ & 5'М' & 8 \\ \hline
\_М & 4'М' & 9 \\ \hline
У & 0'У' & 10 \\ \hline
РК & 1'К' & 11 \\ \hline
А & 0'А' & 12 \\ \hline
\end{tabular}
\end{table}

Результат: 0'Р' 0'И' 0'М' 0'\_' 1'О' 3'\_' 3'У' 5'М' 4'М' 0'У' 1'К' 0'А'\\
\pagebreak
\paragraph{Задание 4. Арифметическое кодирование\\}

Исходная строка: ГНРПАНПППП\
\begin{center}
 \begin{tabular}{ |c|c| } 
  \hline
     Буква & Вероятность \\ \hline
П & 0.50\\\hline
Н & 0.20\\\hline
Р & 0.10\\\hline
А & 0.10\\\hline
Г & 0.10
\\ \hline \end{tabular}
\end{center}
\begin{center}
 \begin{tabular}{ |c|c|c| } 
  \hline
     Буква & Начало & Конец \\ \hline
П & 0.00 & 0.50\\\hline
Н & 0.50 & 0.70\\\hline
Р & 0.70 & 0.80\\\hline
А & 0.80 & 0.90\\\hline
Г & 0.90 & 1.00
\\ \hline \end{tabular}
\end{center}
\begin{center}
 \begin{tabular}{ |c|c|c|c| } 
  \hline
     Буква & delta & min & max \\ \hline
Г & 0.1000000000 & 0.9000000000 & 1.0000000000\\\hline
Н & 0.0200000000 & 0.9500000000 & 0.9700000000\\\hline
Р & 0.0020000000 & 0.9640000000 & 0.9660000000\\\hline
П & 0.0010000000 & 0.9640000000 & 0.9650000000\\\hline
А & 0.0001000000 & 0.9648000000 & 0.9649000000\\\hline
Н & 0.0000200000 & 0.9648500000 & 0.9648700000\\\hline
П & 0.0000100000 & 0.9648500000 & 0.9648600000\\\hline
П & 0.0000050000 & 0.9648500000 & 0.9648550000\\\hline
П & 0.0000025000 & 0.9648500000 & 0.9648525000\\\hline
П & 0.0000012500 & 0.9648500000 & 0.9648512500
\\ \hline \end{tabular}
\end{center}
Результат: 96485
\pagebreak
\paragraph{Задание 5.1}

\\ 

Декодировать сообщение методом адаптивного хаффмана \\
Строка: 
'K'0'L'00'N'100'B'0011111110111001\\
Результат: KLNBBBNNNL










\paragraph{Задание 5.3 Декодировать строку(LZ78)\\}

Исходная строка: [0'к'] [0'о'] [0'с'] [0'т'] [0'ь'] [0' '] [1'о'] [3'а'] [6'о'] [8' '] [2'к'] [0'о']\\
\begin{table}[h!]
\centering
\begin{tabular}{|c|c|c|} 
\hline
 Код & Словарь & Выходной поток 
\hline

 & [] & 
\\ \hline
0'к' & [, к] & к
\\ \hline
0'о' & [, к, о] & о
\\ \hline
0'с' & [, к, о, с] & с
\\ \hline
0'т' & [, к, о, с, т] & т
\\ \hline
0'ь' & [, к, о, с, т, ь] & ь
\\ \hline
0' ' & [, к, о, с, т, ь,  ] &  
\\ \hline
1'о' & [, к, о, с, т, ь,  , ко] & ко
\\ \hline
3'а' & [, к, о, с, т, ь,  , ко, са] & са
\\ \hline
6'о' & [, к, о, с, т, ь,  , ко, са,  о] &  о
\\ \hline
8' ' & [, к, о, с, т, ь,  , ко, са,  о, са ] & са 
\\ \hline
2'к' & [, к, о, с, т, ь,  , ко, са,  о, са , ок] & ок
\\ \hline
0'о' & [, к, о, с, т, ь,  , ко, са,  о, са , ок, о] & о
\\ \hline
\end{tabular}
\end{table}

Результат: кость коса оса око
\pagebreak
\subsection{Вариант №19}
\paragraph{Задание 1. Блочный хаффман \\}

Строка КУКУУУУУУУ, размер блока: 3
\begin{center}
 \begin{tabular}{ |c|c|l| } 
  \hline
     Буква & Вероятность & Код\\ \hline
У & 0.80 & 1\\\hline
К & 0.20 & 0
\\ \hline \end{tabular}
\end{center}
Энтропия алфавита: 0.7219
\begin{center}
 \begin{tabular}{ |c|c|l| } 
  \hline
     Блок & Вероятность & Код\\ \hline
УУУ & 0.51 & 1\\\hline
КУУ & 0.13 & 001\\\hline
УУК & 0.13 & 010\\\hline
УКУ & 0.13 & 011\\\hline
УКК & 0.03 & 00001\\\hline
ККУ & 0.03 & 00010\\\hline
КУК & 0.03 & 00011\\\hline
ККК & 0.01 & 00000
\\ \hline \end{tabular}
\end{center}
Бит на символ при посимвольном кодировании: 1.0000, при блочном: 0.7280


\pagebreak
\paragraph{Задание 2. Сжать адаптивным хаффманом\\}

Строка: 
ВУКАУВУААА\\
Результат: 'В' 0'У' 00'К' 100'А' 11 10 11 1101 111 10










\pagebreak
\paragraph{Задание 3.1}

Закодировать сообщение методом LZ77\\
Строка:ОЛОВО\_ЛОВЕЦ\_ОВЦА\_ЦАП\\
Результат: <0,0,О> <0,0,Л> <8,1,В> <6,1,\_> <5,3,Е> <0,0,Ц> <4,1,О> <0,1,Ц> <0,0,А> <5,1,Ц> <7,1,П>\\
\begin{table}[h!]
\centering
\begin{tabular}{|c|c|c|c|c|c|c|c|c|c|c|c|c|c|c|c|c|} 
\hline
\multicolumn{10}{|c|}{Cловарь} & \multicolumn{6}{c|}{Буфер} & Код  \\ \hline
  &   &   &   &   &   &   &   &   &   & \cellcolor[HTML]{8CE4F6} О & Л & О & В & О &   & <0,0,О>
\\ \hline
  &   &   &   &   &   &   &   &   & О & \cellcolor[HTML]{8CE4F6} Л & О & В & О &   & Л & <0,0,Л>
\\ \hline
  &   &   &   &   &   &   &   & \cellcolor[HTML]{FFFF00} О & Л & \cellcolor[HTML]{FFFF00} О & \cellcolor[HTML]{8CE4F6} В & О &   & Л & О & <8,1,В>
\\ \hline
  &   &   &   &   &   & \cellcolor[HTML]{FFFF00} О & Л & О & В & \cellcolor[HTML]{FFFF00} О & \cellcolor[HTML]{8CE4F6}   & Л & О & В & Е & <6,1,\_>
\\ \hline
  &   &   &   & О & \cellcolor[HTML]{FFFF00} Л & \cellcolor[HTML]{FFFF00} О & \cellcolor[HTML]{FFFF00} В & О &   & \cellcolor[HTML]{FFFF00} Л & \cellcolor[HTML]{FFFF00} О & \cellcolor[HTML]{FFFF00} В & \cellcolor[HTML]{8CE4F6} Е & Ц &   & <5,3,Е>
\\ \hline
О & Л & О & В & О &   & Л & О & В & Е & \cellcolor[HTML]{8CE4F6} Ц &   & О & В & Ц & А & <0,0,Ц>
\\ \hline
Л & О & В & О & \cellcolor[HTML]{FFFF00}   & Л & О & В & Е & Ц & \cellcolor[HTML]{FFFF00}   & \cellcolor[HTML]{8CE4F6} О & В & Ц & А &   & <4,1,О>
\\ \hline
\cellcolor[HTML]{FFFF00} В & О &   & Л & О & В & Е & Ц &   & О & \cellcolor[HTML]{FFFF00} В & \cellcolor[HTML]{8CE4F6} Ц & А &   & Ц & А & <0,1,Ц>
\\ \hline
  & Л & О & В & Е & Ц &   & О & В & Ц & \cellcolor[HTML]{8CE4F6} А &   & Ц & А & П &   & <0,0,А>
\\ \hline
Л & О & В & Е & Ц & \cellcolor[HTML]{FFFF00}   & О & В & Ц & А & \cellcolor[HTML]{FFFF00}   & \cellcolor[HTML]{8CE4F6} Ц & А & П &   &   & <5,1,Ц>
\\ \hline
В & Е & Ц &   & О & В & Ц & \cellcolor[HTML]{FFFF00} А &   & Ц & \cellcolor[HTML]{FFFF00} А & \cellcolor[HTML]{8CE4F6} П &   &   &   &   & <7,1,П>
\\ \hline
\end{tabular}
\end{table}

\paragraph{Задание 3.3}

Закодировать сообщение методом LZ78\\
Строка:ОЛОВО\_ЛОВЕЦ\_ОВЦА\_ЦАП\\
\begin{table}[h!]
\centering
\begin{tabular}{|c|c|c|} 
\hline
 Входная фраза (в словарь) & Код & Позиция словаря \\ \hline

 &  & 0 \\ \hline
О & 0'О' & 1 \\ \hline
Л & 0'Л' & 2 \\ \hline
ОВ & 1'В' & 3 \\ \hline
О\_ & 1'\_' & 4 \\ \hline
ЛО & 2'О' & 5 \\ \hline
В & 0'В' & 6 \\ \hline
Е & 0'Е' & 7 \\ \hline
Ц & 0'Ц' & 8 \\ \hline
\_ & 0'\_' & 9 \\ \hline
ОВЦ & 3'Ц' & 10 \\ \hline
А & 0'А' & 11 \\ \hline
\_Ц & 9'Ц' & 12 \\ \hline
АП & 11'П' & 13 \\ \hline
\end{tabular}
\end{table}

Результат: 0'О' 0'Л' 1'В' 1'\_' 2'О' 0'В' 0'Е' 0'Ц' 0'\_' 3'Ц' 0'А' 9'Ц' 11'П'\\
\pagebreak
\paragraph{Задание 4. Арифметическое кодирование\\}

Исходная строка: ВУКАУВУААА\
\begin{center}
 \begin{tabular}{ |c|c| } 
  \hline
     Буква & Вероятность \\ \hline
А & 0.40\\\hline
У & 0.30\\\hline
В & 0.20\\\hline
К & 0.10
\\ \hline \end{tabular}
\end{center}
\begin{center}
 \begin{tabular}{ |c|c|c| } 
  \hline
     Буква & Начало & Конец \\ \hline
А & 0.00 & 0.40\\\hline
У & 0.40 & 0.70\\\hline
В & 0.70 & 0.90\\\hline
К & 0.90 & 1.00
\\ \hline \end{tabular}
\end{center}
\begin{center}
 \begin{tabular}{ |c|c|c|c| } 
  \hline
     Буква & delta & min & max \\ \hline
В & 0.2000000000 & 0.7000000000 & 0.9000000000\\\hline
У & 0.0600000000 & 0.7800000000 & 0.8400000000\\\hline
К & 0.0060000000 & 0.8340000000 & 0.8400000000\\\hline
А & 0.0024000000 & 0.8340000000 & 0.8364000000\\\hline
У & 0.0007200000 & 0.8349600000 & 0.8356800000\\\hline
В & 0.0001440000 & 0.8354640000 & 0.8356080000\\\hline
У & 0.0000432000 & 0.8355216000 & 0.8355648000\\\hline
А & 0.0000172800 & 0.8355216000 & 0.8355388800\\\hline
А & 0.0000069120 & 0.8355216000 & 0.8355285120\\\hline
А & 0.0000027648 & 0.8355216000 & 0.8355243648
\\ \hline \end{tabular}
\end{center}
Результат: 835522
\pagebreak
\paragraph{Задание 5.1}

\\ 

Декодировать сообщение методом адаптивного хаффмана \\
Строка: 
Ошибка декодирования\\
Результат: Ошибка декодирования
\paragraph{Задание 5.3 Декодировать строку(LZ78)\\}

Исходная строка: [0'б'] [0'е'] [0'р'] [2'т'] [0' '] [1'е'] [3'е'] [0'г'] [5'б'] [7'г']\\
\begin{table}[h!]
\centering
\begin{tabular}{|c|c|c|} 
\hline
 Код & Словарь & Выходной поток 
\hline

 & [] & 
\\ \hline
0'б' & [, б] & б
\\ \hline
0'е' & [, б, е] & е
\\ \hline
0'р' & [, б, е, р] & р
\\ \hline
2'т' & [, б, е, р, ет] & ет
\\ \hline
0' ' & [, б, е, р, ет,  ] &  
\\ \hline
1'е' & [, б, е, р, ет,  , бе] & бе
\\ \hline
3'е' & [, б, е, р, ет,  , бе, ре] & ре
\\ \hline
0'г' & [, б, е, р, ет,  , бе, ре, г] & г
\\ \hline
5'б' & [, б, е, р, ет,  , бе, ре, г,  б] &  б
\\ \hline
7'г' & [, б, е, р, ет,  , бе, ре, г,  б, рег] & рег
\\ \hline
\end{tabular}
\end{table}

Результат: берет берег брег
\pagebreak
\subsection{Вариант №20}
\paragraph{Задание 1. Блочный хаффман \\}

Строка РРРРАААААА, размер блока: 3
\begin{center}
 \begin{tabular}{ |c|c|l| } 
  \hline
     Буква & Вероятность & Код\\ \hline
А & 0.60 & 1\\\hline
Р & 0.40 & 0
\\ \hline \end{tabular}
\end{center}
Энтропия алфавита: 0.9710
\begin{center}
 \begin{tabular}{ |c|c|l| } 
  \hline
     Блок & Вероятность & Код\\ \hline
ААА & 0.22 & 01\\\hline
РАА & 0.14 & 100\\\hline
АРА & 0.14 & 101\\\hline
ААР & 0.14 & 110\\\hline
РРА & 0.10 & 001\\\hline
РАР & 0.10 & 1111\\\hline
АРР & 0.10 & 000\\\hline
РРР & 0.06 & 1110
\\ \hline \end{tabular}
\end{center}
Бит на символ при посимвольном кодировании: 1.0000, при блочном: 0.9813


\pagebreak
\paragraph{Задание 2. Сжать адаптивным хаффманом\\}

Строка: 
КУИРЕИИККК\\
Результат: 'К' 0'У' 00'И' 100'Р' 000'Е' 01 10 101 111 10










\pagebreak

\paragraph{Задание 3.3}

Закодировать сообщение методом LZ78\\
Строка:КАКТУС\_ТУСА\_ТУЗ\_УСА\\
\begin{table}[h!]
\centering
\begin{tabular}{|c|c|c|} 
\hline
 Входная фраза (в словарь) & Код & Позиция словаря \\ \hline

 &  & 0 \\ \hline
К & 0'К' & 1 \\ \hline
А & 0'А' & 2 \\ \hline
КТ & 1'Т' & 3 \\ \hline
У & 0'У' & 4 \\ \hline
С & 0'С' & 5 \\ \hline
\_ & 0'\_' & 6 \\ \hline
Т & 0'Т' & 7 \\ \hline
УС & 4'С' & 8 \\ \hline
А\_ & 2'\_' & 9 \\ \hline
ТУ & 7'У' & 10 \\ \hline
З & 0'З' & 11 \\ \hline
\_У & 6'У' & 12 \\ \hline
СА & 5'А' & 13 \\ \hline
\end{tabular}
\end{table}

Результат: 0'К' 0'А' 1'Т' 0'У' 0'С' 0'\_' 0'Т' 4'С' 2'\_' 7'У' 0'З' 6'У' 5'А'\\
\pagebreak
\paragraph{Задание 4. Арифметическое кодирование\\}

Исходная строка: КУИРЕИИККК\
\begin{center}
 \begin{tabular}{ |c|c| } 
  \hline
     Буква & Вероятность \\ \hline
К & 0.40\\\hline
И & 0.30\\\hline
Р & 0.10\\\hline
У & 0.10\\\hline
Е & 0.10
\\ \hline \end{tabular}
\end{center}
\begin{center}
 \begin{tabular}{ |c|c|c| } 
  \hline
     Буква & Начало & Конец \\ \hline
К & 0.00 & 0.40\\\hline
И & 0.40 & 0.70\\\hline
Р & 0.70 & 0.80\\\hline
У & 0.80 & 0.90\\\hline
Е & 0.90 & 1.00
\\ \hline \end{tabular}
\end{center}
\begin{center}
 \begin{tabular}{ |c|c|c|c| } 
  \hline
     Буква & delta & min & max \\ \hline
К & 0.4000000000 & 0.0000000000 & 0.4000000000\\\hline
У & 0.0400000000 & 0.3200000000 & 0.3600000000\\\hline
И & 0.0120000000 & 0.3360000000 & 0.3480000000\\\hline
Р & 0.0012000000 & 0.3444000000 & 0.3456000000\\\hline
Е & 0.0001200000 & 0.3454800000 & 0.3456000000\\\hline
И & 0.0000360000 & 0.3455280000 & 0.3455640000\\\hline
И & 0.0000108000 & 0.3455424000 & 0.3455532000\\\hline
К & 0.0000043200 & 0.3455424000 & 0.3455467200\\\hline
К & 0.0000017280 & 0.3455424000 & 0.3455441280\\\hline
К & 0.0000006912 & 0.3455424000 & 0.3455430912
\\ \hline \end{tabular}
\end{center}
Результат: 345543
\pagebreak
\paragraph{Задание 5.1}

\\ 

Декодировать сообщение методом адаптивного хаффмана \\
Строка: 
'P'0'O'0100'K'000'M'110110110111110\\
Результат: POOKMMMOMOO











\paragraph{Задание 5.3 Декодировать строку(LZ78)\\}

Исходная строка: [0'в'] [0'а'] [0'р'] [1'а'] [3' '] [3'в'] [2'н'] [0'ь'] [0' '] [4'н'] [0'н'] [0'а']\\
\begin{table}[h!]
\centering
\begin{tabular}{|c|c|c|} 
\hline
 Код & Словарь & Выходной поток 
\hline

 & [] & 
\\ \hline
0'в' & [, в] & в
\\ \hline
0'а' & [, в, а] & а
\\ \hline
0'р' & [, в, а, р] & р
\\ \hline
1'а' & [, в, а, р, ва] & ва
\\ \hline
3' ' & [, в, а, р, ва, р ] & р 
\\ \hline
3'в' & [, в, а, р, ва, р , рв] & рв
\\ \hline
2'н' & [, в, а, р, ва, р , рв, ан] & ан
\\ \hline
0'ь' & [, в, а, р, ва, р , рв, ан, ь] & ь
\\ \hline
0' ' & [, в, а, р, ва, р , рв, ан, ь,  ] &  
\\ \hline
4'н' & [, в, а, р, ва, р , рв, ан, ь,  , ван] & ван
\\ \hline
0'н' & [, в, а, р, ва, р , рв, ан, ь,  , ван, н] & н
\\ \hline
0'а' & [, в, а, р, ва, р , рв, ан, ь,  , ван, н, а] & а
\\ \hline
\end{tabular}
\end{table}

Результат: варвар рвань ванна
\pagebreak
\subsection{Вариант №21}
\paragraph{Задание 1. Блочный хаффман \\}

Строка ЛЕЛЕЛЕЕЕЕЕ, размер блока: 3
\begin{center}
 \begin{tabular}{ |c|c|l| } 
  \hline
     Буква & Вероятность & Код\\ \hline
Е & 0.70 & 1\\\hline
Л & 0.30 & 0
\\ \hline \end{tabular}
\end{center}
Энтропия алфавита: 0.8813
\begin{center}
 \begin{tabular}{ |c|c|l| } 
  \hline
     Блок & Вероятность & Код\\ \hline
ЕЕЕ & 0.34 & 11\\\hline
ЕЛЕ & 0.15 & 101\\\hline
ЛЕЕ & 0.15 & 00\\\hline
ЕЕЛ & 0.15 & 100\\\hline
ЕЛЛ & 0.06 & 0101\\\hline
ЛЛЕ & 0.06 & 0110\\\hline
ЛЕЛ & 0.06 & 0111\\\hline
ЛЛЛ & 0.03 & 0100
\\ \hline \end{tabular}
\end{center}
Бит на символ при посимвольном кодировании: 1.0000, при блочном: 0.9087


\pagebreak
\paragraph{Задание 2. Сжать адаптивным хаффманом\\}

Строка: 
СВИВТРИИИИ\\
Результат: 'С' 0'В' 00'И' 11 100'Т' 1100'Р' 01 01 11 0










\pagebreak
\paragraph{Задание 3.1}

Закодировать сообщение методом LZ77\\
Строка:ЛОДКА\_ЛОДОЧКА\_ОЧКИ\\
Результат: <0,0,Л> <0,0,О> <0,0,Д> <0,0,К> <0,0,А> <0,0,\_> <4,3,О> <0,0,Ч> <2,3,О> <5,2,И>\\
\begin{table}[h!]
\centering
\begin{tabular}{|c|c|c|c|c|c|c|c|c|c|c|c|c|c|c|c|c|} 
\hline
\multicolumn{10}{|c|}{Cловарь} & \multicolumn{6}{c|}{Буфер} & Код  \\ \hline
  &   &   &   &   &   &   &   &   &   & \cellcolor[HTML]{8CE4F6} Л & О & Д & К & А &   & <0,0,Л>
\\ \hline
  &   &   &   &   &   &   &   &   & Л & \cellcolor[HTML]{8CE4F6} О & Д & К & А &   & Л & <0,0,О>
\\ \hline
  &   &   &   &   &   &   &   & Л & О & \cellcolor[HTML]{8CE4F6} Д & К & А &   & Л & О & <0,0,Д>
\\ \hline
  &   &   &   &   &   &   & Л & О & Д & \cellcolor[HTML]{8CE4F6} К & А &   & Л & О & Д & <0,0,К>
\\ \hline
  &   &   &   &   &   & Л & О & Д & К & \cellcolor[HTML]{8CE4F6} А &   & Л & О & Д & О & <0,0,А>
\\ \hline
  &   &   &   &   & Л & О & Д & К & А & \cellcolor[HTML]{8CE4F6}   & Л & О & Д & О & Ч & <0,0,\_>
\\ \hline
  &   &   &   & \cellcolor[HTML]{FFFF00} Л & \cellcolor[HTML]{FFFF00} О & \cellcolor[HTML]{FFFF00} Д & К & А &   & \cellcolor[HTML]{FFFF00} Л & \cellcolor[HTML]{FFFF00} О & \cellcolor[HTML]{FFFF00} Д & \cellcolor[HTML]{8CE4F6} О & Ч & К & <4,3,О>
\\ \hline
Л & О & Д & К & А &   & Л & О & Д & О & \cellcolor[HTML]{8CE4F6} Ч & К & А &   & О & Ч & <0,0,Ч>
\\ \hline
О & Д & \cellcolor[HTML]{FFFF00} К & \cellcolor[HTML]{FFFF00} А & \cellcolor[HTML]{FFFF00}   & Л & О & Д & О & Ч & \cellcolor[HTML]{FFFF00} К & \cellcolor[HTML]{FFFF00} А & \cellcolor[HTML]{FFFF00}   & \cellcolor[HTML]{8CE4F6} О & Ч & К & <2,3,О>
\\ \hline
  & Л & О & Д & О & \cellcolor[HTML]{FFFF00} Ч & \cellcolor[HTML]{FFFF00} К & А &   & О & \cellcolor[HTML]{FFFF00} Ч & \cellcolor[HTML]{FFFF00} К & \cellcolor[HTML]{8CE4F6} И &   &   &   & <5,2,И>
\\ \hline
\end{tabular}
\end{table}

\paragraph{Задание 3.3}

Закодировать сообщение методом LZ78\\
Строка:ЛОДКА\_ЛОДОЧКА\_ОЧКИ\\
\begin{table}[h!]
\centering
\begin{tabular}{|c|c|c|} 
\hline
 Входная фраза (в словарь) & Код & Позиция словаря \\ \hline

 &  & 0 \\ \hline
Л & 0'Л' & 1 \\ \hline
О & 0'О' & 2 \\ \hline
Д & 0'Д' & 3 \\ \hline
К & 0'К' & 4 \\ \hline
А & 0'А' & 5 \\ \hline
\_ & 0'\_' & 6 \\ \hline
ЛО & 1'О' & 7 \\ \hline
ДО & 3'О' & 8 \\ \hline
Ч & 0'Ч' & 9 \\ \hline
КА & 4'А' & 10 \\ \hline
\_О & 6'О' & 11 \\ \hline
ЧК & 9'К' & 12 \\ \hline
И & 0'И' & 13 \\ \hline
\end{tabular}
\end{table}

Результат: 0'Л' 0'О' 0'Д' 0'К' 0'А' 0'\_' 1'О' 3'О' 0'Ч' 4'А' 6'О' 9'К' 0'И'\\
\pagebreak
\paragraph{Задание 4. Арифметическое кодирование\\}

Исходная строка: СВИВТРИИИИ\
\begin{center}
 \begin{tabular}{ |c|c| } 
  \hline
     Буква & Вероятность \\ \hline
И & 0.50\\\hline
В & 0.20\\\hline
Р & 0.10\\\hline
С & 0.10\\\hline
Т & 0.10
\\ \hline \end{tabular}
\end{center}
\begin{center}
 \begin{tabular}{ |c|c|c| } 
  \hline
     Буква & Начало & Конец \\ \hline
И & 0.00 & 0.50\\\hline
В & 0.50 & 0.70\\\hline
Р & 0.70 & 0.80\\\hline
С & 0.80 & 0.90\\\hline
Т & 0.90 & 1.00
\\ \hline \end{tabular}
\end{center}
\begin{center}
 \begin{tabular}{ |c|c|c|c| } 
  \hline
     Буква & delta & min & max \\ \hline
С & 0.1000000000 & 0.8000000000 & 0.9000000000\\\hline
В & 0.0200000000 & 0.8500000000 & 0.8700000000\\\hline
И & 0.0100000000 & 0.8500000000 & 0.8600000000\\\hline
В & 0.0020000000 & 0.8550000000 & 0.8570000000\\\hline
Т & 0.0002000000 & 0.8568000000 & 0.8570000000\\\hline
Р & 0.0000200000 & 0.8569400000 & 0.8569600000\\\hline
И & 0.0000100000 & 0.8569400000 & 0.8569500000\\\hline
И & 0.0000050000 & 0.8569400000 & 0.8569450000\\\hline
И & 0.0000025000 & 0.8569400000 & 0.8569425000\\\hline
И & 0.0000012500 & 0.8569400000 & 0.8569412500
\\ \hline \end{tabular}
\end{center}
Результат: 85694
\pagebreak
\paragraph{Задание 5.1}

\\ 

Декодировать сообщение методом адаптивного хаффмана \\
Строка: 
'C'0'X'0100'V'001100'R'10010111111\\
Результат: CXXVVRRRRV










\paragraph{Задание 5.3 Декодировать строку(LZ78)\\}

Исходная строка: [0'н'] [0'о'] [0'с'] [2'к'] [0' '] [2'с'] [4'а'] [5'с'] [4'о'] [0'л']\\
\begin{table}[h!]
\centering
\begin{tabular}{|c|c|c|} 
\hline
 Код & Словарь & Выходной поток 
\hline

 & [] & 
\\ \hline
0'н' & [, н] & н
\\ \hline
0'о' & [, н, о] & о
\\ \hline
0'с' & [, н, о, с] & с
\\ \hline
2'к' & [, н, о, с, ок] & ок
\\ \hline
0' ' & [, н, о, с, ок,  ] &  
\\ \hline
2'с' & [, н, о, с, ок,  , ос] & ос
\\ \hline
4'а' & [, н, о, с, ок,  , ос, ока] & ока
\\ \hline
5'с' & [, н, о, с, ок,  , ос, ока,  с] &  с
\\ \hline
4'о' & [, н, о, с, ок,  , ос, ока,  с, око] & око
\\ \hline
0'л' & [, н, о, с, ок,  , ос, ока,  с, око, л] & л
\\ \hline
\end{tabular}
\end{table}

Результат: носок осока сокол
\pagebreak
\subsection{Вариант №22}
\paragraph{Задание 1. Блочный хаффман \\}

Строка КЛЛЛККККОО, размер блока: 2
\begin{center}
 \begin{tabular}{ |c|c|l| } 
  \hline
     Буква & Вероятность & Код\\ \hline
К & 0.50 & 0\\\hline
Л & 0.30 & 11\\\hline
О & 0.20 & 10
\\ \hline \end{tabular}
\end{center}
Энтропия алфавита: 1.4855
\begin{center}
 \begin{tabular}{ |c|c|l| } 
  \hline
     Блок & Вероятность & Код\\ \hline
КК & 0.25 & 01\\\hline
КЛ & 0.15 & 101\\\hline
ЛК & 0.15 & 110\\\hline
КО & 0.10 & 000\\\hline
ОК & 0.10 & 001\\\hline
ЛЛ & 0.09 & 1111\\\hline
ЛО & 0.06 & 1001\\\hline
ОЛ & 0.06 & 1110\\\hline
ОО & 0.04 & 1000
\\ \hline \end{tabular}
\end{center}
Бит на символ при посимвольном кодировании: 1.5000, при блочном: 1.5000


\pagebreak
\paragraph{Задание 2. Сжать адаптивным хаффманом\\}

Строка: 
ДЕИМЕИДДДД\\
Результат: 'Д' 0'Е' 00'И' 100'М' 11 01 01 01 11 0










\pagebreak
\paragraph{Задание 3.1}

Закодировать сообщение методом LZ77\\
Строка:КЛУБ\_КЛУБОК\_КЛУБНИ\\
Результат: <0,0,К> <0,0,Л> <0,0,У> <0,0,Б> <0,0,\_> <5,4,О> <0,1,\_> <3,4,Н> <0,0,И>\\
\begin{table}[h!]
\centering
\begin{tabular}{|c|c|c|c|c|c|c|c|c|c|c|c|c|c|c|c|c|} 
\hline
\multicolumn{10}{|c|}{Cловарь} & \multicolumn{6}{c|}{Буфер} & Код  \\ \hline
  &   &   &   &   &   &   &   &   &   & \cellcolor[HTML]{8CE4F6} К & Л & У & Б &   & К & <0,0,К>
\\ \hline
  &   &   &   &   &   &   &   &   & К & \cellcolor[HTML]{8CE4F6} Л & У & Б &   & К & Л & <0,0,Л>
\\ \hline
  &   &   &   &   &   &   &   & К & Л & \cellcolor[HTML]{8CE4F6} У & Б &   & К & Л & У & <0,0,У>
\\ \hline
  &   &   &   &   &   &   & К & Л & У & \cellcolor[HTML]{8CE4F6} Б &   & К & Л & У & Б & <0,0,Б>
\\ \hline
  &   &   &   &   &   & К & Л & У & Б & \cellcolor[HTML]{8CE4F6}   & К & Л & У & Б & О & <0,0,\_>
\\ \hline
  &   &   &   &   & \cellcolor[HTML]{FFFF00} К & \cellcolor[HTML]{FFFF00} Л & \cellcolor[HTML]{FFFF00} У & \cellcolor[HTML]{FFFF00} Б &   & \cellcolor[HTML]{FFFF00} К & \cellcolor[HTML]{FFFF00} Л & \cellcolor[HTML]{FFFF00} У & \cellcolor[HTML]{FFFF00} Б & \cellcolor[HTML]{8CE4F6} О & К & <5,4,О>
\\ \hline
\cellcolor[HTML]{FFFF00} К & Л & У & Б &   & К & Л & У & Б & О & \cellcolor[HTML]{FFFF00} К & \cellcolor[HTML]{8CE4F6}   & К & Л & У & Б & <0,1,\_>
\\ \hline
У & Б &   & \cellcolor[HTML]{FFFF00} К & \cellcolor[HTML]{FFFF00} Л & \cellcolor[HTML]{FFFF00} У & \cellcolor[HTML]{FFFF00} Б & О & К &   & \cellcolor[HTML]{FFFF00} К & \cellcolor[HTML]{FFFF00} Л & \cellcolor[HTML]{FFFF00} У & \cellcolor[HTML]{FFFF00} Б & \cellcolor[HTML]{8CE4F6} Н & И & <3,4,Н>
\\ \hline
У & Б & О & К &   & К & Л & У & Б & Н & \cellcolor[HTML]{8CE4F6} И &   &   &   &   &   & <0,0,И>
\\ \hline
\end{tabular}
\end{table}

\paragraph{Задание 3.3}

Закодировать сообщение методом LZ78\\
Строка:КЛУБ\_КЛУБОК\_КЛУБНИ\\
\begin{table}[h!]
\centering
\begin{tabular}{|c|c|c|} 
\hline
 Входная фраза (в словарь) & Код & Позиция словаря \\ \hline

 &  & 0 \\ \hline
К & 0'К' & 1 \\ \hline
Л & 0'Л' & 2 \\ \hline
У & 0'У' & 3 \\ \hline
Б & 0'Б' & 4 \\ \hline
\_ & 0'\_' & 5 \\ \hline
КЛ & 1'Л' & 6 \\ \hline
УБ & 3'Б' & 7 \\ \hline
О & 0'О' & 8 \\ \hline
К\_ & 1'\_' & 9 \\ \hline
КЛУ & 6'У' & 10 \\ \hline
БН & 4'Н' & 11 \\ \hline
И & 0'И' & 12 \\ \hline
\end{tabular}
\end{table}

Результат: 0'К' 0'Л' 0'У' 0'Б' 0'\_' 1'Л' 3'Б' 0'О' 1'\_' 6'У' 4'Н' 0'И'\\
\pagebreak
\paragraph{Задание 4. Арифметическое кодирование\\}

Исходная строка: ДЕИМЕИДДДД\
\begin{center}
 \begin{tabular}{ |c|c| } 
  \hline
     Буква & Вероятность \\ \hline
Д & 0.50\\\hline
Е & 0.20\\\hline
И & 0.20\\\hline
М & 0.10
\\ \hline \end{tabular}
\end{center}
\begin{center}
 \begin{tabular}{ |c|c|c| } 
  \hline
     Буква & Начало & Конец \\ \hline
Д & 0.00 & 0.50\\\hline
Е & 0.50 & 0.70\\\hline
И & 0.70 & 0.90\\\hline
М & 0.90 & 1.00
\\ \hline \end{tabular}
\end{center}
\begin{center}
 \begin{tabular}{ |c|c|c|c| } 
  \hline
     Буква & delta & min & max \\ \hline
Д & 0.5000000000 & 0.0000000000 & 0.5000000000\\\hline
Е & 0.1000000000 & 0.2500000000 & 0.3500000000\\\hline
И & 0.0200000000 & 0.3200000000 & 0.3400000000\\\hline
М & 0.0020000000 & 0.3380000000 & 0.3400000000\\\hline
Е & 0.0004000000 & 0.3390000000 & 0.3394000000\\\hline
И & 0.0000800000 & 0.3392800000 & 0.3393600000\\\hline
Д & 0.0000400000 & 0.3392800000 & 0.3393200000\\\hline
Д & 0.0000200000 & 0.3392800000 & 0.3393000000\\\hline
Д & 0.0000100000 & 0.3392800000 & 0.3392900000\\\hline
Д & 0.0000050000 & 0.3392800000 & 0.3392850000
\\ \hline \end{tabular}
\end{center}
Результат: 33928
\pagebreak
\paragraph{Задание 5.1}

\\ 

Декодировать сообщение методом адаптивного хаффмана \\
Строка: 
Ошибка декодирования\\
Результат: Ошибка декодирования
\paragraph{Задание 5.3 Декодировать строку(LZ78)\\}

Исходная строка: [0'к'] [0'о'] [0'л'] [2'б'] [2'к'] [0' '] [0'б'] [5' '] [7'о'] [1'а'] [0'л']\\
\begin{table}[h!]
\centering
\begin{tabular}{|c|c|c|} 
\hline
 Код & Словарь & Выходной поток 
\hline

 & [] & 
\\ \hline
0'к' & [, к] & к
\\ \hline
0'о' & [, к, о] & о
\\ \hline
0'л' & [, к, о, л] & л
\\ \hline
2'б' & [, к, о, л, об] & об
\\ \hline
2'к' & [, к, о, л, об, ок] & ок
\\ \hline
0' ' & [, к, о, л, об, ок,  ] &  
\\ \hline
0'б' & [, к, о, л, об, ок,  , б] & б
\\ \hline
5' ' & [, к, о, л, об, ок,  , б, ок ] & ок 
\\ \hline
7'о' & [, к, о, л, об, ок,  , б, ок , бо] & бо
\\ \hline
1'а' & [, к, о, л, об, ок,  , б, ок , бо, ка] & ка
\\ \hline
0'л' & [, к, о, л, об, ок,  , б, ок , бо, ка, л] & л
\\ \hline
\end{tabular}
\end{table}

Результат: колобок бок бокал
\pagebreak
\subsection{Вариант №23}
\paragraph{Задание 1. Блочный хаффман \\}

Строка РРРООРТТТР, размер блока: 2
\begin{center}
 \begin{tabular}{ |c|c|l| } 
  \hline
     Буква & Вероятность & Код\\ \hline
Р & 0.50 & 0\\\hline
Т & 0.30 & 11\\\hline
О & 0.20 & 10
\\ \hline \end{tabular}
\end{center}
Энтропия алфавита: 1.4855
\begin{center}
 \begin{tabular}{ |c|c|l| } 
  \hline
     Блок & Вероятность & Код\\ \hline
РР & 0.25 & 01\\\hline
РТ & 0.15 & 101\\\hline
ТР & 0.15 & 110\\\hline
ОР & 0.10 & 000\\\hline
РО & 0.10 & 001\\\hline
ТТ & 0.09 & 1111\\\hline
ОТ & 0.06 & 1001\\\hline
ТО & 0.06 & 1110\\\hline
ОО & 0.04 & 1000
\\ \hline \end{tabular}
\end{center}
Бит на символ при посимвольном кодировании: 1.5000, при блочном: 1.5000


\pagebreak
\paragraph{Задание 2. Сжать адаптивным хаффманом\\}

Строка: 
НЕЕИИННЕАА\\
Результат: 'Н' 0'Е' 01 00'И' 001 101 101 10 100'А' 1001










\pagebreak
\paragraph{Задание 3.1}

Закодировать сообщение методом LZ77\\
Строка:БОЛОТО\_БОЛТ\_БОЛЬ\_ОЛЯ\\
Результат: <0,0,Б> <0,0,О> <0,0,Л> <8,1,Т> <6,1,\_> <3,3,Т> <5,4,Ь> <0,1,О> <1,1,Я>\\
\begin{table}[h!]
\centering
\begin{tabular}{|c|c|c|c|c|c|c|c|c|c|c|c|c|c|c|c|c|} 
\hline
\multicolumn{10}{|c|}{Cловарь} & \multicolumn{6}{c|}{Буфер} & Код  \\ \hline
  &   &   &   &   &   &   &   &   &   & \cellcolor[HTML]{8CE4F6} Б & О & Л & О & Т & О & <0,0,Б>
\\ \hline
  &   &   &   &   &   &   &   &   & Б & \cellcolor[HTML]{8CE4F6} О & Л & О & Т & О &   & <0,0,О>
\\ \hline
  &   &   &   &   &   &   &   & Б & О & \cellcolor[HTML]{8CE4F6} Л & О & Т & О &   & Б & <0,0,Л>
\\ \hline
  &   &   &   &   &   &   & Б & \cellcolor[HTML]{FFFF00} О & Л & \cellcolor[HTML]{FFFF00} О & \cellcolor[HTML]{8CE4F6} Т & О &   & Б & О & <8,1,Т>
\\ \hline
  &   &   &   &   & Б & \cellcolor[HTML]{FFFF00} О & Л & О & Т & \cellcolor[HTML]{FFFF00} О & \cellcolor[HTML]{8CE4F6}   & Б & О & Л & Т & <6,1,\_>
\\ \hline
  &   &   & \cellcolor[HTML]{FFFF00} Б & \cellcolor[HTML]{FFFF00} О & \cellcolor[HTML]{FFFF00} Л & О & Т & О &   & \cellcolor[HTML]{FFFF00} Б & \cellcolor[HTML]{FFFF00} О & \cellcolor[HTML]{FFFF00} Л & \cellcolor[HTML]{8CE4F6} Т &   & Б & <3,3,Т>
\\ \hline
О & Л & О & Т & О & \cellcolor[HTML]{FFFF00}   & \cellcolor[HTML]{FFFF00} Б & \cellcolor[HTML]{FFFF00} О & \cellcolor[HTML]{FFFF00} Л & Т & \cellcolor[HTML]{FFFF00}   & \cellcolor[HTML]{FFFF00} Б & \cellcolor[HTML]{FFFF00} О & \cellcolor[HTML]{FFFF00} Л & \cellcolor[HTML]{8CE4F6} Ь &   & <5,4,Ь>
\\ \hline
\cellcolor[HTML]{FFFF00}   & Б & О & Л & Т &   & Б & О & Л & Ь & \cellcolor[HTML]{FFFF00}   & \cellcolor[HTML]{8CE4F6} О & Л & Я &   &   & <0,1,О>
\\ \hline
О & \cellcolor[HTML]{FFFF00} Л & Т &   & Б & О & Л & Ь &   & О & \cellcolor[HTML]{FFFF00} Л & \cellcolor[HTML]{8CE4F6} Я &   &   &   &   & <1,1,Я>
\\ \hline
\end{tabular}
\end{table}

\paragraph{Задание 3.3}

Закодировать сообщение методом LZ78\\
Строка:БОЛОТО\_БОЛТ\_БОЛЬ\_ОЛЯ\\
\begin{table}[h!]
\centering
\begin{tabular}{|c|c|c|} 
\hline
 Входная фраза (в словарь) & Код & Позиция словаря \\ \hline

 &  & 0 \\ \hline
Б & 0'Б' & 1 \\ \hline
О & 0'О' & 2 \\ \hline
Л & 0'Л' & 3 \\ \hline
ОТ & 2'Т' & 4 \\ \hline
О\_ & 2'\_' & 5 \\ \hline
БО & 1'О' & 6 \\ \hline
ЛТ & 3'Т' & 7 \\ \hline
\_ & 0'\_' & 8 \\ \hline
БОЛ & 6'Л' & 9 \\ \hline
Ь & 0'Ь' & 10 \\ \hline
\_О & 8'О' & 11 \\ \hline
ЛЯ & 3'Я' & 12 \\ \hline
\end{tabular}
\end{table}

Результат: 0'Б' 0'О' 0'Л' 2'Т' 2'\_' 1'О' 3'Т' 0'\_' 6'Л' 0'Ь' 8'О' 3'Я'\\
\pagebreak
\paragraph{Задание 4. Арифметическое кодирование\\}

Исходная строка: НЕЕИИННЕАА\
\begin{center}
 \begin{tabular}{ |c|c| } 
  \hline
     Буква & Вероятность \\ \hline
Е & 0.30\\\hline
Н & 0.30\\\hline
А & 0.20\\\hline
И & 0.20
\\ \hline \end{tabular}
\end{center}
\begin{center}
 \begin{tabular}{ |c|c|c| } 
  \hline
     Буква & Начало & Конец \\ \hline
Е & 0.00 & 0.30\\\hline
Н & 0.30 & 0.60\\\hline
А & 0.60 & 0.80\\\hline
И & 0.80 & 1.00
\\ \hline \end{tabular}
\end{center}
\begin{center}
 \begin{tabular}{ |c|c|c|c| } 
  \hline
     Буква & delta & min & max \\ \hline
Н & 0.3000000000 & 0.3000000000 & 0.6000000000\\\hline
Е & 0.0900000000 & 0.3000000000 & 0.3900000000\\\hline
Е & 0.0270000000 & 0.3000000000 & 0.3270000000\\\hline
И & 0.0054000000 & 0.3216000000 & 0.3270000000\\\hline
И & 0.0010800000 & 0.3259200000 & 0.3270000000\\\hline
Н & 0.0003240000 & 0.3262440000 & 0.3265680000\\\hline
Н & 0.0000972000 & 0.3263412000 & 0.3264384000\\\hline
Е & 0.0000291600 & 0.3263412000 & 0.3263703600\\\hline
А & 0.0000058320 & 0.3263586960 & 0.3263645280\\\hline
А & 0.0000011664 & 0.3263621952 & 0.3263633616
\\ \hline \end{tabular}
\end{center}
Результат: 326363
\pagebreak
\paragraph{Задание 5.1}

\\ 

Декодировать сообщение методом адаптивного хаффмана \\
Строка: 
'P'0'O'0100'U'0011110110111100'K'\\
Результат: POOUUUPPPK










\paragraph{Задание 5.3 Декодировать строку(LZ78)\\}

Исходная строка: [0'к'] [0'л'] [0'у'] [0'б'] [0' '] [1'л'] [3'б'] [0'о'] [1' '] [4'о'] [0'к']\\
\begin{table}[h!]
\centering
\begin{tabular}{|c|c|c|} 
\hline
 Код & Словарь & Выходной поток 
\hline

 & [] & 
\\ \hline
0'к' & [, к] & к
\\ \hline
0'л' & [, к, л] & л
\\ \hline
0'у' & [, к, л, у] & у
\\ \hline
0'б' & [, к, л, у, б] & б
\\ \hline
0' ' & [, к, л, у, б,  ] &  
\\ \hline
1'л' & [, к, л, у, б,  , кл] & кл
\\ \hline
3'б' & [, к, л, у, б,  , кл, уб] & уб
\\ \hline
0'о' & [, к, л, у, б,  , кл, уб, о] & о
\\ \hline
1' ' & [, к, л, у, б,  , кл, уб, о, к ] & к 
\\ \hline
4'о' & [, к, л, у, б,  , кл, уб, о, к , бо] & бо
\\ \hline
0'к' & [, к, л, у, б,  , кл, уб, о, к , бо, к] & к
\\ \hline
\end{tabular}
\end{table}

Результат: клуб клубок бок
\pagebreak
\subsection{Вариант №24}
\paragraph{Задание 1. Блочный хаффман \\}

Строка ККЛКЮВВВВ, размер блока: 2
\begin{center}
 \begin{tabular}{ |c|c|l| } 
  \hline
     Буква & Вероятность & Код\\ \hline
В & 0.44 & 0\\\hline
К & 0.33 & 11\\\hline
Л & 0.11 & 100\\\hline
Ю & 0.11 & 101
\\ \hline \end{tabular}
\end{center}
Энтропия алфавита: 1.7527
\begin{center}
 \begin{tabular}{ |c|c|l| } 
  \hline
     Блок & Вероятность & Код\\ \hline
ВВ & 0.20 & 00\\\hline
ВК & 0.15 & 101\\\hline
КВ & 0.15 & 110\\\hline
КК & 0.11 & 011\\\hline
ЮВ & 0.05 & 11110\\\hline
ВЛ & 0.05 & 11111\\\hline
ЛВ & 0.05 & 0100\\\hline
ВЮ & 0.05 & 0101\\\hline
КЛ & 0.04 & 10010\\\hline
ЮК & 0.04 & 10011\\\hline
КЮ & 0.04 & 11100\\\hline
ЛК & 0.04 & 11101\\\hline
ЛЛ & 0.01 & 100000\\\hline
ЮЮ & 0.01 & 100001\\\hline
ЮЛ & 0.01 & 100010\\\hline
ЛЮ & 0.01 & 100011
\\ \hline \end{tabular}
\end{center}
Бит на символ при посимвольном кодировании: 1.7778, при блочном: 1.7716


\pagebreak
\paragraph{Задание 2. Сжать адаптивным хаффманом\\}

Строка: 
ЕАКАКККРАА\\
Результат: 'Е' 0'А' 00'К' 11 101 11 0 000'Р' 01 11










\pagebreak
\paragraph{Задание 3.1}

Закодировать сообщение методом LZ77\\
Строка:ЛАПКИ\_ЛАПЫ\_ЛАПИТАЛЬ\\
Результат: <0,0,Л> <0,0,А> <0,0,П> <0,0,К> <0,0,И> <0,0,\_> <4,3,Ы> <5,4,И> <0,0,Т> <1,1,Л> <0,0,Ь>\\
\begin{table}[h!]
\centering
\begin{tabular}{|c|c|c|c|c|c|c|c|c|c|c|c|c|c|c|c|c|} 
\hline
\multicolumn{10}{|c|}{Cловарь} & \multicolumn{6}{c|}{Буфер} & Код  \\ \hline
  &   &   &   &   &   &   &   &   &   & \cellcolor[HTML]{8CE4F6} Л & А & П & К & И &   & <0,0,Л>
\\ \hline
  &   &   &   &   &   &   &   &   & Л & \cellcolor[HTML]{8CE4F6} А & П & К & И &   & Л & <0,0,А>
\\ \hline
  &   &   &   &   &   &   &   & Л & А & \cellcolor[HTML]{8CE4F6} П & К & И &   & Л & А & <0,0,П>
\\ \hline
  &   &   &   &   &   &   & Л & А & П & \cellcolor[HTML]{8CE4F6} К & И &   & Л & А & П & <0,0,К>
\\ \hline
  &   &   &   &   &   & Л & А & П & К & \cellcolor[HTML]{8CE4F6} И &   & Л & А & П & Ы & <0,0,И>
\\ \hline
  &   &   &   &   & Л & А & П & К & И & \cellcolor[HTML]{8CE4F6}   & Л & А & П & Ы &   & <0,0,\_>
\\ \hline
  &   &   &   & \cellcolor[HTML]{FFFF00} Л & \cellcolor[HTML]{FFFF00} А & \cellcolor[HTML]{FFFF00} П & К & И &   & \cellcolor[HTML]{FFFF00} Л & \cellcolor[HTML]{FFFF00} А & \cellcolor[HTML]{FFFF00} П & \cellcolor[HTML]{8CE4F6} Ы &   & Л & <4,3,Ы>
\\ \hline
Л & А & П & К & И & \cellcolor[HTML]{FFFF00}   & \cellcolor[HTML]{FFFF00} Л & \cellcolor[HTML]{FFFF00} А & \cellcolor[HTML]{FFFF00} П & Ы & \cellcolor[HTML]{FFFF00}   & \cellcolor[HTML]{FFFF00} Л & \cellcolor[HTML]{FFFF00} А & \cellcolor[HTML]{FFFF00} П & \cellcolor[HTML]{8CE4F6} И & Т & <5,4,И>
\\ \hline
  & Л & А & П & Ы &   & Л & А & П & И & \cellcolor[HTML]{8CE4F6} Т & А & Л & Ь &   &   & <0,0,Т>
\\ \hline
Л & \cellcolor[HTML]{FFFF00} А & П & Ы &   & Л & А & П & И & Т & \cellcolor[HTML]{FFFF00} А & \cellcolor[HTML]{8CE4F6} Л & Ь &   &   &   & <1,1,Л>
\\ \hline
П & Ы &   & Л & А & П & И & Т & А & Л & \cellcolor[HTML]{8CE4F6} Ь &   &   &   &   &   & <0,0,Ь>
\\ \hline
\end{tabular}
\end{table}

\paragraph{Задание 3.3}

Закодировать сообщение методом LZ78\\
Строка:ЛАПКИ\_ЛАПЫ\_ЛАПИТАЛЬ\\
\begin{table}[h!]
\centering
\begin{tabular}{|c|c|c|} 
\hline
 Входная фраза (в словарь) & Код & Позиция словаря \\ \hline

 &  & 0 \\ \hline
Л & 0'Л' & 1 \\ \hline
А & 0'А' & 2 \\ \hline
П & 0'П' & 3 \\ \hline
К & 0'К' & 4 \\ \hline
И & 0'И' & 5 \\ \hline
\_ & 0'\_' & 6 \\ \hline
ЛА & 1'А' & 7 \\ \hline
ПЫ & 3'Ы' & 8 \\ \hline
\_Л & 6'Л' & 9 \\ \hline
АП & 2'П' & 10 \\ \hline
ИТ & 5'Т' & 11 \\ \hline
АЛ & 2'Л' & 12 \\ \hline
Ь & 0'Ь' & 13 \\ \hline
\end{tabular}
\end{table}

Результат: 0'Л' 0'А' 0'П' 0'К' 0'И' 0'\_' 1'А' 3'Ы' 6'Л' 2'П' 5'Т' 2'Л' 0'Ь'\\
\pagebreak
\paragraph{Задание 4. Арифметическое кодирование\\}

Исходная строка: ЕАКАКККРАА\
\begin{center}
 \begin{tabular}{ |c|c| } 
  \hline
     Буква & Вероятность \\ \hline
А & 0.40\\\hline
К & 0.40\\\hline
Р & 0.10\\\hline
Е & 0.10
\\ \hline \end{tabular}
\end{center}
\begin{center}
 \begin{tabular}{ |c|c|c| } 
  \hline
     Буква & Начало & Конец \\ \hline
А & 0.00 & 0.40\\\hline
К & 0.40 & 0.80\\\hline
Р & 0.80 & 0.90\\\hline
Е & 0.90 & 1.00
\\ \hline \end{tabular}
\end{center}
\begin{center}
 \begin{tabular}{ |c|c|c|c| } 
  \hline
     Буква & delta & min & max \\ \hline
Е & 0.1000000000 & 0.9000000000 & 1.0000000000\\\hline
А & 0.0400000000 & 0.9000000000 & 0.9400000000\\\hline
К & 0.0160000000 & 0.9160000000 & 0.9320000000\\\hline
А & 0.0064000000 & 0.9160000000 & 0.9224000000\\\hline
К & 0.0025600000 & 0.9185600000 & 0.9211200000\\\hline
К & 0.0010240000 & 0.9195840000 & 0.9206080000\\\hline
К & 0.0004096000 & 0.9199936000 & 0.9204032000\\\hline
Р & 0.0000409600 & 0.9203212800 & 0.9203622400\\\hline
А & 0.0000163840 & 0.9203212800 & 0.9203376640\\\hline
А & 0.0000065536 & 0.9203212800 & 0.9203278336
\\ \hline \end{tabular}
\end{center}
Результат: 920322
\pagebreak
\paragraph{Задание 5.1}

\\ 

Декодировать сообщение методом адаптивного хаффмана \\
Строка: 
'A'0'B'00'C'10110111100'S'100111110\\
Результат: ABCCAACAASABAAAA
















\paragraph{Задание 5.3 Декодировать строку(LZ78)\\}

Исходная строка: [0'к'] [0'и'] [1'и'] [0'м'] [0'о'] [0'р'] [0'а'] [0' '] [4'о'] [6' '] [9'р'] [0'ж']\\
\begin{table}[h!]
\centering
\begin{tabular}{|c|c|c|} 
\hline
 Код & Словарь & Выходной поток 
\hline

 & [] & 
\\ \hline
0'к' & [, к] & к
\\ \hline
0'и' & [, к, и] & и
\\ \hline
1'и' & [, к, и, ки] & ки
\\ \hline
0'м' & [, к, и, ки, м] & м
\\ \hline
0'о' & [, к, и, ки, м, о] & о
\\ \hline
0'р' & [, к, и, ки, м, о, р] & р
\\ \hline
0'а' & [, к, и, ки, м, о, р, а] & а
\\ \hline
0' ' & [, к, и, ки, м, о, р, а,  ] &  
\\ \hline
4'о' & [, к, и, ки, м, о, р, а,  , мо] & мо
\\ \hline
6' ' & [, к, и, ки, м, о, р, а,  , мо, р ] & р 
\\ \hline
9'р' & [, к, и, ки, м, о, р, а,  , мо, р , мор] & мор
\\ \hline
0'ж' & [, к, и, ки, м, о, р, а,  , мо, р , мор, ж] & ж
\\ \hline
\end{tabular}
\end{table}

Результат: кикимора мор морж
\pagebreak
\subsection{Вариант №25}
\paragraph{Задание 1. Блочный хаффман \\}

Строка ЛЛИМЛЛЛМИИ, размер блока: 2
\begin{center}
 \begin{tabular}{ |c|c|l| } 
  \hline
     Буква & Вероятность & Код\\ \hline
Л & 0.50 & 0\\\hline
И & 0.30 & 11\\\hline
М & 0.20 & 10
\\ \hline \end{tabular}
\end{center}
Энтропия алфавита: 1.4855
\begin{center}
 \begin{tabular}{ |c|c|l| } 
  \hline
     Блок & Вероятность & Код\\ \hline
ЛЛ & 0.25 & 01\\\hline
ИЛ & 0.15 & 101\\\hline
ЛИ & 0.15 & 110\\\hline
ЛМ & 0.10 & 000\\\hline
МЛ & 0.10 & 001\\\hline
ИИ & 0.09 & 1111\\\hline
ИМ & 0.06 & 1001\\\hline
МИ & 0.06 & 1110\\\hline
ММ & 0.04 & 1000
\\ \hline \end{tabular}
\end{center}
Бит на символ при посимвольном кодировании: 1.5000, при блочном: 1.5000


\pagebreak
\paragraph{Задание 2. Сжать адаптивным хаффманом\\}

Строка: 
ГОРОНПОРРР\\
Результат: 'Г' 0'О' 00'Р' 11 100'Н' 1100'П' 11 111 111 10










\pagebreak

\paragraph{Задание 3.3}

Закодировать сообщение методом LZ78\\
Строка:КУКУРУЗА\_УРЮК\_КРЮК\\
\begin{table}[h!]
\centering
\begin{tabular}{|c|c|c|} 
\hline
 Входная фраза (в словарь) & Код & Позиция словаря \\ \hline

 &  & 0 \\ \hline
К & 0'К' & 1 \\ \hline
У & 0'У' & 2 \\ \hline
КУ & 1'У' & 3 \\ \hline
Р & 0'Р' & 4 \\ \hline
УЗ & 2'З' & 5 \\ \hline
А & 0'А' & 6 \\ \hline
\_ & 0'\_' & 7 \\ \hline
УР & 2'Р' & 8 \\ \hline
Ю & 0'Ю' & 9 \\ \hline
К\_ & 1'\_' & 10 \\ \hline
КР & 1'Р' & 11 \\ \hline
ЮК & 9'К' & 12 \\ \hline
\end{tabular}
\end{table}

Результат: 0'К' 0'У' 1'У' 0'Р' 2'З' 0'А' 0'\_' 2'Р' 0'Ю' 1'\_' 1'Р' 9'К'\\
\pagebreak
\paragraph{Задание 4. Арифметическое кодирование\\}

Исходная строка: ГОРОНПОРРР\
\begin{center}
 \begin{tabular}{ |c|c| } 
  \hline
     Буква & Вероятность \\ \hline
Р & 0.40\\\hline
О & 0.30\\\hline
Г & 0.10\\\hline
Н & 0.10\\\hline
П & 0.10
\\ \hline \end{tabular}
\end{center}
\begin{center}
 \begin{tabular}{ |c|c|c| } 
  \hline
     Буква & Начало & Конец \\ \hline
Р & 0.00 & 0.40\\\hline
О & 0.40 & 0.70\\\hline
Г & 0.70 & 0.80\\\hline
Н & 0.80 & 0.90\\\hline
П & 0.90 & 1.00
\\ \hline \end{tabular}
\end{center}
\begin{center}
 \begin{tabular}{ |c|c|c|c| } 
  \hline
     Буква & delta & min & max \\ \hline
Г & 0.1000000000 & 0.7000000000 & 0.8000000000\\\hline
О & 0.0300000000 & 0.7400000000 & 0.7700000000\\\hline
Р & 0.0120000000 & 0.7400000000 & 0.7520000000\\\hline
О & 0.0036000000 & 0.7448000000 & 0.7484000000\\\hline
Н & 0.0003600000 & 0.7476800000 & 0.7480400000\\\hline
П & 0.0000360000 & 0.7480040000 & 0.7480400000\\\hline
О & 0.0000108000 & 0.7480184000 & 0.7480292000\\\hline
Р & 0.0000043200 & 0.7480184000 & 0.7480227200\\\hline
Р & 0.0000017280 & 0.7480184000 & 0.7480201280\\\hline
Р & 0.0000006912 & 0.7480184000 & 0.7480190912
\\ \hline \end{tabular}
\end{center}
Результат: 748019
\pagebreak
\paragraph{Задание 5.1}

\\ 

Декодировать сообщение методом адаптивного хаффмана \\
Строка: 
'K'0'L'0100'M'000'N'110110110111110\\
Результат: KLLMNNNLNLL











\paragraph{Задание 5.3 Декодировать строку(LZ78)\\}

Исходная строка: [0'м'] [0'а'] [1'а'] [0' '] [0'р'] [2'м'] [2' '] [5'а'] [0'к'] [0'и']\\
\begin{table}[h!]
\centering
\begin{tabular}{|c|c|c|} 
\hline
 Код & Словарь & Выходной поток 
\hline

 & [] & 
\\ \hline
0'м' & [, м] & м
\\ \hline
0'а' & [, м, а] & а
\\ \hline
1'а' & [, м, а, ма] & ма
\\ \hline
0' ' & [, м, а, ма,  ] &  
\\ \hline
0'р' & [, м, а, ма,  , р] & р
\\ \hline
2'м' & [, м, а, ма,  , р, ам] & ам
\\ \hline
2' ' & [, м, а, ма,  , р, ам, а ] & а 
\\ \hline
5'а' & [, м, а, ма,  , р, ам, а , ра] & ра
\\ \hline
0'к' & [, м, а, ма,  , р, ам, а , ра, к] & к
\\ \hline
0'и' & [, м, а, ма,  , р, ам, а , ра, к, и] & и
\\ \hline
\end{tabular}
\end{table}

Результат: мама рама раки
\pagebreak
\subsection{Вариант №26}
\paragraph{Задание 1. Блочный хаффман \\}

Строка БРББРРРБББ, размер блока: 3
\begin{center}
 \begin{tabular}{ |c|c|l| } 
  \hline
     Буква & Вероятность & Код\\ \hline
Б & 0.60 & 1\\\hline
Р & 0.40 & 0
\\ \hline \end{tabular}
\end{center}
Энтропия алфавита: 0.9710
\begin{center}
 \begin{tabular}{ |c|c|l| } 
  \hline
     Блок & Вероятность & Код\\ \hline
БББ & 0.22 & 01\\\hline
БРБ & 0.14 & 100\\\hline
РББ & 0.14 & 101\\\hline
ББР & 0.14 & 110\\\hline
РРБ & 0.10 & 001\\\hline
РБР & 0.10 & 1111\\\hline
БРР & 0.10 & 000\\\hline
РРР & 0.06 & 1110
\\ \hline \end{tabular}
\end{center}
Бит на символ при посимвольном кодировании: 1.0000, при блочном: 0.9813


\pagebreak
\paragraph{Задание 2. Сжать адаптивным хаффманом\\}

Строка: 
ВУАКУВАМММ\\
Результат: 'В' 0'У' 00'А' 100'К' 11 10 01 000'М' 0001 111










\pagebreak
\paragraph{Задание 3.1}

Закодировать сообщение методом LZ77\\
Строка:ДОДО\_ДОМ\_ДОМИК\_МИГ\\
Результат: <0,0,Д> <0,0,О> <8,2,\_> <5,2,М> <6,4,И> <0,0,К> <0,1,М> <6,1,Г>\\
\begin{table}[h!]
\centering
\begin{tabular}{|c|c|c|c|c|c|c|c|c|c|c|c|c|c|c|c|c|} 
\hline
\multicolumn{10}{|c|}{Cловарь} & \multicolumn{6}{c|}{Буфер} & Код  \\ \hline
  &   &   &   &   &   &   &   &   &   & \cellcolor[HTML]{8CE4F6} Д & О & Д & О &   & Д & <0,0,Д>
\\ \hline
  &   &   &   &   &   &   &   &   & Д & \cellcolor[HTML]{8CE4F6} О & Д & О &   & Д & О & <0,0,О>
\\ \hline
  &   &   &   &   &   &   &   & \cellcolor[HTML]{FFFF00} Д & \cellcolor[HTML]{FFFF00} О & \cellcolor[HTML]{FFFF00} Д & \cellcolor[HTML]{FFFF00} О & \cellcolor[HTML]{8CE4F6}   & Д & О & М & <8,2,\_>
\\ \hline
  &   &   &   &   & \cellcolor[HTML]{FFFF00} Д & \cellcolor[HTML]{FFFF00} О & Д & О &   & \cellcolor[HTML]{FFFF00} Д & \cellcolor[HTML]{FFFF00} О & \cellcolor[HTML]{8CE4F6} М &   & Д & О & <5,2,М>
\\ \hline
  &   & Д & О & Д & О & \cellcolor[HTML]{FFFF00}   & \cellcolor[HTML]{FFFF00} Д & \cellcolor[HTML]{FFFF00} О & \cellcolor[HTML]{FFFF00} М & \cellcolor[HTML]{FFFF00}   & \cellcolor[HTML]{FFFF00} Д & \cellcolor[HTML]{FFFF00} О & \cellcolor[HTML]{FFFF00} М & \cellcolor[HTML]{8CE4F6} И & К & <6,4,И>
\\ \hline
О &   & Д & О & М &   & Д & О & М & И & \cellcolor[HTML]{8CE4F6} К &   & М & И & Г &   & <0,0,К>
\\ \hline
\cellcolor[HTML]{FFFF00}   & Д & О & М &   & Д & О & М & И & К & \cellcolor[HTML]{FFFF00}   & \cellcolor[HTML]{8CE4F6} М & И & Г &   &   & <0,1,М>
\\ \hline
О & М &   & Д & О & М & \cellcolor[HTML]{FFFF00} И & К &   & М & \cellcolor[HTML]{FFFF00} И & \cellcolor[HTML]{8CE4F6} Г &   &   &   &   & <6,1,Г>
\\ \hline
\end{tabular}
\end{table}

\paragraph{Задание 3.3}

Закодировать сообщение методом LZ78\\
Строка:ДОДО\_ДОМ\_ДОМИК\_МИГ\\
\begin{table}[h!]
\centering
\begin{tabular}{|c|c|c|} 
\hline
 Входная фраза (в словарь) & Код & Позиция словаря \\ \hline

 &  & 0 \\ \hline
Д & 0'Д' & 1 \\ \hline
О & 0'О' & 2 \\ \hline
ДО & 1'О' & 3 \\ \hline
\_ & 0'\_' & 4 \\ \hline
ДОМ & 3'М' & 5 \\ \hline
\_Д & 4'Д' & 6 \\ \hline
ОМ & 2'М' & 7 \\ \hline
И & 0'И' & 8 \\ \hline
К & 0'К' & 9 \\ \hline
\_М & 4'М' & 10 \\ \hline
ИГ & 8'Г' & 11 \\ \hline
\end{tabular}
\end{table}

Результат: 0'Д' 0'О' 1'О' 0'\_' 3'М' 4'Д' 2'М' 0'И' 0'К' 4'М' 8'Г'\\
\pagebreak
\paragraph{Задание 4. Арифметическое кодирование\\}

Исходная строка: ВУАКУВАМММ\
\begin{center}
 \begin{tabular}{ |c|c| } 
  \hline
     Буква & Вероятность \\ \hline
М & 0.30\\\hline
А & 0.20\\\hline
В & 0.20\\\hline
У & 0.20\\\hline
К & 0.10
\\ \hline \end{tabular}
\end{center}
\begin{center}
 \begin{tabular}{ |c|c|c| } 
  \hline
     Буква & Начало & Конец \\ \hline
М & 0.00 & 0.30\\\hline
А & 0.30 & 0.50\\\hline
В & 0.50 & 0.70\\\hline
У & 0.70 & 0.90\\\hline
К & 0.90 & 1.00
\\ \hline \end{tabular}
\end{center}
\begin{center}
 \begin{tabular}{ |c|c|c|c| } 
  \hline
     Буква & delta & min & max \\ \hline
В & 0.2000000000 & 0.5000000000 & 0.7000000000\\\hline
У & 0.0400000000 & 0.6400000000 & 0.6800000000\\\hline
А & 0.0080000000 & 0.6520000000 & 0.6600000000\\\hline
К & 0.0008000000 & 0.6592000000 & 0.6600000000\\\hline
У & 0.0001600000 & 0.6597600000 & 0.6599200000\\\hline
В & 0.0000320000 & 0.6598400000 & 0.6598720000\\\hline
А & 0.0000064000 & 0.6598496000 & 0.6598560000\\\hline
М & 0.0000019200 & 0.6598496000 & 0.6598515200\\\hline
М & 0.0000005760 & 0.6598496000 & 0.6598501760\\\hline
М & 0.0000001728 & 0.6598496000 & 0.6598497728
\\ \hline \end{tabular}
\end{center}
Результат: 6598496
\pagebreak
\paragraph{Задание 5.1}

\\ 

Декодировать сообщение методом адаптивного хаффмана \\
Строка: 
'H'0'B'00'V'100'N'0011111011111101001\\
Результат: HBVNNNBVVBH











\paragraph{Задание 5.3 Декодировать строку(LZ78)\\}

Исходная строка: [0'л'] [0'о'] [0'г'] [2'в'] [2' '] [0'в'] [2'л'] [0' '] [6'о'] [1'к']\\
\begin{table}[h!]
\centering
\begin{tabular}{|c|c|c|} 
\hline
 Код & Словарь & Выходной поток 
\hline

 & [] & 
\\ \hline
0'л' & [, л] & л
\\ \hline
0'о' & [, л, о] & о
\\ \hline
0'г' & [, л, о, г] & г
\\ \hline
2'в' & [, л, о, г, ов] & ов
\\ \hline
2' ' & [, л, о, г, ов, о ] & о 
\\ \hline
0'в' & [, л, о, г, ов, о , в] & в
\\ \hline
2'л' & [, л, о, г, ов, о , в, ол] & ол
\\ \hline
0' ' & [, л, о, г, ов, о , в, ол,  ] &  
\\ \hline
6'о' & [, л, о, г, ов, о , в, ол,  , во] & во
\\ \hline
1'к' & [, л, о, г, ов, о , в, ол,  , во, лк] & лк
\\ \hline
\end{tabular}
\end{table}

Результат: логово вол волк
\pagebreak
\subsection{Вариант №27}
\paragraph{Задание 1. Блочный хаффман \\}

Строка КВКККВВВВВ, размер блока: 3
\begin{center}
 \begin{tabular}{ |c|c|l| } 
  \hline
     Буква & Вероятность & Код\\ \hline
В & 0.60 & 1\\\hline
К & 0.40 & 0
\\ \hline \end{tabular}
\end{center}
Энтропия алфавита: 0.9710
\begin{center}
 \begin{tabular}{ |c|c|l| } 
  \hline
     Блок & Вероятность & Код\\ \hline
ВВВ & 0.22 & 01\\\hline
ВВК & 0.14 & 100\\\hline
ВКВ & 0.14 & 101\\\hline
КВВ & 0.14 & 110\\\hline
ККВ & 0.10 & 001\\\hline
ВКК & 0.10 & 1111\\\hline
КВК & 0.10 & 000\\\hline
ККК & 0.06 & 1110
\\ \hline \end{tabular}
\end{center}
Бит на символ при посимвольном кодировании: 1.0000, при блочном: 0.9813


\pagebreak
\paragraph{Задание 2. Сжать адаптивным хаффманом\\}

Строка: 
УЧЧРККЧУУУ\\
Результат: 'У' 0'Ч' 01 00'Р' 000'К' 1101 11 110 111 10










\pagebreak
\paragraph{Задание 3.1}

Закодировать сообщение методом LZ77\\
Строка:ЗИГЗАГ\_ЗАЗОР\_ЗОРКИЙ\\
Результат: <0,0,З> <0,0,И> <0,0,Г> <7,1,А> <7,1,\_> <6,2,З> <0,0,О> <0,0,Р> <4,2,О> <6,1,К> <0,0,И> <0,0,Й>\\
\begin{table}[h!]
\centering
\begin{tabular}{|c|c|c|c|c|c|c|c|c|c|c|c|c|c|c|c|c|} 
\hline
\multicolumn{10}{|c|}{Cловарь} & \multicolumn{6}{c|}{Буфер} & Код  \\ \hline
  &   &   &   &   &   &   &   &   &   & \cellcolor[HTML]{8CE4F6} З & И & Г & З & А & Г & <0,0,З>
\\ \hline
  &   &   &   &   &   &   &   &   & З & \cellcolor[HTML]{8CE4F6} И & Г & З & А & Г &   & <0,0,И>
\\ \hline
  &   &   &   &   &   &   &   & З & И & \cellcolor[HTML]{8CE4F6} Г & З & А & Г &   & З & <0,0,Г>
\\ \hline
  &   &   &   &   &   &   & \cellcolor[HTML]{FFFF00} З & И & Г & \cellcolor[HTML]{FFFF00} З & \cellcolor[HTML]{8CE4F6} А & Г &   & З & А & <7,1,А>
\\ \hline
  &   &   &   &   & З & И & \cellcolor[HTML]{FFFF00} Г & З & А & \cellcolor[HTML]{FFFF00} Г & \cellcolor[HTML]{8CE4F6}   & З & А & З & О & <7,1,\_>
\\ \hline
  &   &   & З & И & Г & \cellcolor[HTML]{FFFF00} З & \cellcolor[HTML]{FFFF00} А & Г &   & \cellcolor[HTML]{FFFF00} З & \cellcolor[HTML]{FFFF00} А & \cellcolor[HTML]{8CE4F6} З & О & Р &   & <6,2,З>
\\ \hline
З & И & Г & З & А & Г &   & З & А & З & \cellcolor[HTML]{8CE4F6} О & Р &   & З & О & Р & <0,0,О>
\\ \hline
И & Г & З & А & Г &   & З & А & З & О & \cellcolor[HTML]{8CE4F6} Р &   & З & О & Р & К & <0,0,Р>
\\ \hline
Г & З & А & Г & \cellcolor[HTML]{FFFF00}   & \cellcolor[HTML]{FFFF00} З & А & З & О & Р & \cellcolor[HTML]{FFFF00}   & \cellcolor[HTML]{FFFF00} З & \cellcolor[HTML]{8CE4F6} О & Р & К & И & <4,2,О>
\\ \hline
Г &   & З & А & З & О & \cellcolor[HTML]{FFFF00} Р &   & З & О & \cellcolor[HTML]{FFFF00} Р & \cellcolor[HTML]{8CE4F6} К & И & Й &   &   & <6,1,К>
\\ \hline
З & А & З & О & Р &   & З & О & Р & К & \cellcolor[HTML]{8CE4F6} И & Й &   &   &   &   & <0,0,И>
\\ \hline
А & З & О & Р &   & З & О & Р & К & И & \cellcolor[HTML]{8CE4F6} Й &   &   &   &   &   & <0,0,Й>
\\ \hline
\end{tabular}
\end{table}

\paragraph{Задание 3.3}

Закодировать сообщение методом LZ78\\
Строка:ЗИГЗАГ\_ЗАЗОР\_ЗОРКИЙ\\
\begin{table}[h!]
\centering
\begin{tabular}{|c|c|c|} 
\hline
 Входная фраза (в словарь) & Код & Позиция словаря \\ \hline

 &  & 0 \\ \hline
З & 0'З' & 1 \\ \hline
И & 0'И' & 2 \\ \hline
Г & 0'Г' & 3 \\ \hline
ЗА & 1'А' & 4 \\ \hline
Г\_ & 3'\_' & 5 \\ \hline
ЗАЗ & 4'З' & 6 \\ \hline
О & 0'О' & 7 \\ \hline
Р & 0'Р' & 8 \\ \hline
\_ & 0'\_' & 9 \\ \hline
ЗО & 1'О' & 10 \\ \hline
РК & 8'К' & 11 \\ \hline
ИЙ & 2'Й' & 12 \\ \hline
\end{tabular}
\end{table}

Результат: 0'З' 0'И' 0'Г' 1'А' 3'\_' 4'З' 0'О' 0'Р' 0'\_' 1'О' 8'К' 2'Й'\\
\pagebreak
\paragraph{Задание 4. Арифметическое кодирование\\}

Исходная строка: УЧЧРККЧУУУ\
\begin{center}
 \begin{tabular}{ |c|c| } 
  \hline
     Буква & Вероятность \\ \hline
У & 0.40\\\hline
Ч & 0.30\\\hline
К & 0.20\\\hline
Р & 0.10
\\ \hline \end{tabular}
\end{center}
\begin{center}
 \begin{tabular}{ |c|c|c| } 
  \hline
     Буква & Начало & Конец \\ \hline
У & 0.00 & 0.40\\\hline
Ч & 0.40 & 0.70\\\hline
К & 0.70 & 0.90\\\hline
Р & 0.90 & 1.00
\\ \hline \end{tabular}
\end{center}
\begin{center}
 \begin{tabular}{ |c|c|c|c| } 
  \hline
     Буква & delta & min & max \\ \hline
У & 0.4000000000 & 0.0000000000 & 0.4000000000\\\hline
Ч & 0.1200000000 & 0.1600000000 & 0.2800000000\\\hline
Ч & 0.0360000000 & 0.2080000000 & 0.2440000000\\\hline
Р & 0.0036000000 & 0.2404000000 & 0.2440000000\\\hline
К & 0.0007200000 & 0.2429200000 & 0.2436400000\\\hline
К & 0.0001440000 & 0.2434240000 & 0.2435680000\\\hline
Ч & 0.0000432000 & 0.2434816000 & 0.2435248000\\\hline
У & 0.0000172800 & 0.2434816000 & 0.2434988800\\\hline
У & 0.0000069120 & 0.2434816000 & 0.2434885120\\\hline
У & 0.0000027648 & 0.2434816000 & 0.2434843648
\\ \hline \end{tabular}
\end{center}
Результат: 243482
\pagebreak
\paragraph{Задание 5.1}

\\ 

Декодировать сообщение методом адаптивного хаффмана \\
Строка: 
'D'0'F'00'C'100'S'010011010011100'H'01\\
Результат: DFCSCSSDDHC











\paragraph{Задание 5.3 Декодировать строку(LZ78)\\}

Исходная строка: [0'т'] [0'и'] [0'н'] [0'а'] [0' '] [1'и'] [0'к'] [5'н'] [2'т'] [2' '] [3'и'] [1'к'] [0'и']\\
\begin{table}[h!]
\centering
\begin{tabular}{|c|c|c|} 
\hline
 Код & Словарь & Выходной поток 
\hline

 & [] & 
\\ \hline
0'т' & [, т] & т
\\ \hline
0'и' & [, т, и] & и
\\ \hline
0'н' & [, т, и, н] & н
\\ \hline
0'а' & [, т, и, н, а] & а
\\ \hline
0' ' & [, т, и, н, а,  ] &  
\\ \hline
1'и' & [, т, и, н, а,  , ти] & ти
\\ \hline
0'к' & [, т, и, н, а,  , ти, к] & к
\\ \hline
5'н' & [, т, и, н, а,  , ти, к,  н] &  н
\\ \hline
2'т' & [, т, и, н, а,  , ти, к,  н, ит] & ит
\\ \hline
2' ' & [, т, и, н, а,  , ти, к,  н, ит, и ] & и 
\\ \hline
3'и' & [, т, и, н, а,  , ти, к,  н, ит, и , ни] & ни
\\ \hline
1'к' & [, т, и, н, а,  , ти, к,  н, ит, и , ни, тк] & тк
\\ \hline
0'и' & [, т, и, н, а,  , ти, к,  н, ит, и , ни, тк, и] & и
\\ \hline
\end{tabular}
\end{table}

Результат: тина тик нити нитки
\pagebreak
\subsection{Вариант №28}
\paragraph{Задание 1. Блочный хаффман \\}

Строка УККУУККККК, размер блока: 3
\begin{center}
 \begin{tabular}{ |c|c|l| } 
  \hline
     Буква & Вероятность & Код\\ \hline
К & 0.70 & 1\\\hline
У & 0.30 & 0
\\ \hline \end{tabular}
\end{center}
Энтропия алфавита: 0.8813
\begin{center}
 \begin{tabular}{ |c|c|l| } 
  \hline
     Блок & Вероятность & Код\\ \hline
ККК & 0.34 & 11\\\hline
УКК & 0.15 & 101\\\hline
КУК & 0.15 & 00\\\hline
ККУ & 0.15 & 100\\\hline
КУУ & 0.06 & 0101\\\hline
УУК & 0.06 & 0110\\\hline
УКУ & 0.06 & 0111\\\hline
УУУ & 0.03 & 0100
\\ \hline \end{tabular}
\end{center}
Бит на символ при посимвольном кодировании: 1.0000, при блочном: 0.9087


\pagebreak
\paragraph{Задание 2. Сжать адаптивным хаффманом\\}

Строка: 
КЛЮЧЧИИИИК\\
Результат: 'К' 0'Л' 00'Ю' 100'Ч' 001 100'И' 1001 01 11 101










\pagebreak
\paragraph{Задание 3.1}

Закодировать сообщение методом LZ77\\
Строка:ТИКТАК\_ТИК\_ТАК\_ТАКСА\\
Результат: <0,0,Т> <0,0,И> <0,0,К> <7,1,А> <7,1,\_> <3,3,\_> <2,5,А> <2,1,С> <0,0,А>\\
\begin{table}[h!]
\centering
\begin{tabular}{|c|c|c|c|c|c|c|c|c|c|c|c|c|c|c|c|c|} 
\hline
\multicolumn{10}{|c|}{Cловарь} & \multicolumn{6}{c|}{Буфер} & Код  \\ \hline
  &   &   &   &   &   &   &   &   &   & \cellcolor[HTML]{8CE4F6} Т & И & К & Т & А & К & <0,0,Т>
\\ \hline
  &   &   &   &   &   &   &   &   & Т & \cellcolor[HTML]{8CE4F6} И & К & Т & А & К &   & <0,0,И>
\\ \hline
  &   &   &   &   &   &   &   & Т & И & \cellcolor[HTML]{8CE4F6} К & Т & А & К &   & Т & <0,0,К>
\\ \hline
  &   &   &   &   &   &   & \cellcolor[HTML]{FFFF00} Т & И & К & \cellcolor[HTML]{FFFF00} Т & \cellcolor[HTML]{8CE4F6} А & К &   & Т & И & <7,1,А>
\\ \hline
  &   &   &   &   & Т & И & \cellcolor[HTML]{FFFF00} К & Т & А & \cellcolor[HTML]{FFFF00} К & \cellcolor[HTML]{8CE4F6}   & Т & И & К &   & <7,1,\_>
\\ \hline
  &   &   & \cellcolor[HTML]{FFFF00} Т & \cellcolor[HTML]{FFFF00} И & \cellcolor[HTML]{FFFF00} К & Т & А & К &   & \cellcolor[HTML]{FFFF00} Т & \cellcolor[HTML]{FFFF00} И & \cellcolor[HTML]{FFFF00} К & \cellcolor[HTML]{8CE4F6}   & Т & А & <3,3,\_>
\\ \hline
И & К & \cellcolor[HTML]{FFFF00} Т & \cellcolor[HTML]{FFFF00} А & \cellcolor[HTML]{FFFF00} К & \cellcolor[HTML]{FFFF00}   & \cellcolor[HTML]{FFFF00} Т & И & К &   & \cellcolor[HTML]{FFFF00} Т & \cellcolor[HTML]{FFFF00} А & \cellcolor[HTML]{FFFF00} К & \cellcolor[HTML]{FFFF00}   & \cellcolor[HTML]{FFFF00} Т & \cellcolor[HTML]{8CE4F6} А & <2,5,А>
\\ \hline
Т & И & \cellcolor[HTML]{FFFF00} К &   & Т & А & К &   & Т & А & \cellcolor[HTML]{FFFF00} К & \cellcolor[HTML]{8CE4F6} С & А &   &   &   & <2,1,С>
\\ \hline
К &   & Т & А & К &   & Т & А & К & С & \cellcolor[HTML]{8CE4F6} А &   &   &   &   &   & <0,0,А>
\\ \hline
\end{tabular}
\end{table}

\paragraph{Задание 3.3}

Закодировать сообщение методом LZ78\\
Строка:ТИКТАК\_ТИК\_ТАК\_ТАКСА\\
\begin{table}[h!]
\centering
\begin{tabular}{|c|c|c|} 
\hline
 Входная фраза (в словарь) & Код & Позиция словаря \\ \hline

 &  & 0 \\ \hline
Т & 0'Т' & 1 \\ \hline
И & 0'И' & 2 \\ \hline
К & 0'К' & 3 \\ \hline
ТА & 1'А' & 4 \\ \hline
К\_ & 3'\_' & 5 \\ \hline
ТИ & 1'И' & 6 \\ \hline
К\_Т & 5'Т' & 7 \\ \hline
А & 0'А' & 8 \\ \hline
К\_ТА & 7'А' & 9 \\ \hline
КС & 3'С' & 10 \\ \hline
\end{tabular}
\end{table}

Результат: 0'Т' 0'И' 0'К' 1'А' 3'\_' 1'И' 5'Т' 0'А' 7'А' 3'С'\\
\pagebreak
\paragraph{Задание 4. Арифметическое кодирование\\}

Исходная строка: КЛЮЧЧИИИИК\
\begin{center}
 \begin{tabular}{ |c|c| } 
  \hline
     Буква & Вероятность \\ \hline
И & 0.40\\\hline
Ч & 0.20\\\hline
К & 0.20\\\hline
Л & 0.10\\\hline
Ю & 0.10
\\ \hline \end{tabular}
\end{center}
\begin{center}
 \begin{tabular}{ |c|c|c| } 
  \hline
     Буква & Начало & Конец \\ \hline
И & 0.00 & 0.40\\\hline
Ч & 0.40 & 0.60\\\hline
К & 0.60 & 0.80\\\hline
Л & 0.80 & 0.90\\\hline
Ю & 0.90 & 1.00
\\ \hline \end{tabular}
\end{center}
\begin{center}
 \begin{tabular}{ |c|c|c|c| } 
  \hline
     Буква & delta & min & max \\ \hline
К & 0.2000000000 & 0.6000000000 & 0.8000000000\\\hline
Л & 0.0200000000 & 0.7600000000 & 0.7800000000\\\hline
Ю & 0.0020000000 & 0.7780000000 & 0.7800000000\\\hline
Ч & 0.0004000000 & 0.7788000000 & 0.7792000000\\\hline
Ч & 0.0000800000 & 0.7789600000 & 0.7790400000\\\hline
И & 0.0000320000 & 0.7789600000 & 0.7789920000\\\hline
И & 0.0000128000 & 0.7789600000 & 0.7789728000\\\hline
И & 0.0000051200 & 0.7789600000 & 0.7789651200\\\hline
И & 0.0000020480 & 0.7789600000 & 0.7789620480\\\hline
К & 0.0000004096 & 0.7789612288 & 0.7789616384
\\ \hline \end{tabular}
\end{center}
Результат: 7789613
\pagebreak
\paragraph{Задание 5.1}

\\ 

Декодировать сообщение методом адаптивного хаффмана \\
Строка: 
'K'0'L'0100'F'0111100'V'10011101001\\
Результат: KLLFKKVVFF










\paragraph{Задание 5.3 Декодировать строку(LZ78)\\}

Исходная строка: [0'к'] [0'а'] [0'б'] [2'н'] [0' '] [3'а'] [0'н'] [1'а'] [5'б'] [2'к'] [0'е'] [0'н']\\
\begin{table}[h!]
\centering
\begin{tabular}{|c|c|c|} 
\hline
 Код & Словарь & Выходной поток 
\hline

 & [] & 
\\ \hline
0'к' & [, к] & к
\\ \hline
0'а' & [, к, а] & а
\\ \hline
0'б' & [, к, а, б] & б
\\ \hline
2'н' & [, к, а, б, ан] & ан
\\ \hline
0' ' & [, к, а, б, ан,  ] &  
\\ \hline
3'а' & [, к, а, б, ан,  , ба] & ба
\\ \hline
0'н' & [, к, а, б, ан,  , ба, н] & н
\\ \hline
1'а' & [, к, а, б, ан,  , ба, н, ка] & ка
\\ \hline
5'б' & [, к, а, б, ан,  , ба, н, ка,  б] &  б
\\ \hline
2'к' & [, к, а, б, ан,  , ба, н, ка,  б, ак] & ак
\\ \hline
0'е' & [, к, а, б, ан,  , ба, н, ка,  б, ак, е] & е
\\ \hline
0'н' & [, к, а, б, ан,  , ба, н, ка,  б, ак, е, н] & н
\\ \hline
\end{tabular}
\end{table}

Результат: кабан банка бакен
\pagebreak
\subsection{Вариант №29}
\paragraph{Задание 1. Блочный хаффман \\}

Строка ИИММИИИРРР, размер блока: 2
\begin{center}
 \begin{tabular}{ |c|c|l| } 
  \hline
     Буква & Вероятность & Код\\ \hline
И & 0.50 & 0\\\hline
Р & 0.30 & 11\\\hline
М & 0.20 & 10
\\ \hline \end{tabular}
\end{center}
Энтропия алфавита: 1.4855
\begin{center}
 \begin{tabular}{ |c|c|l| } 
  \hline
     Блок & Вероятность & Код\\ \hline
ИИ & 0.25 & 01\\\hline
РИ & 0.15 & 101\\\hline
ИР & 0.15 & 110\\\hline
ИМ & 0.10 & 000\\\hline
МИ & 0.10 & 001\\\hline
РР & 0.09 & 1111\\\hline
МР & 0.06 & 1001\\\hline
РМ & 0.06 & 1110\\\hline
ММ & 0.04 & 1000
\\ \hline \end{tabular}
\end{center}
Бит на символ при посимвольном кодировании: 1.5000, при блочном: 1.5000


\pagebreak
\paragraph{Задание 2. Сжать адаптивным хаффманом\\}

Строка: 
БАЗАААРРРР\\
Результат: 'Б' 0'А' 00'З' 11 0 1 000'Р' 0101 00 11










\pagebreak
\paragraph{Задание 3.1}

Закодировать сообщение методом LZ77\\
Строка:КУРКУЛЬ\_КУЛЕК\_ЛЕКАЛО\\
Результат: <0,0,К> <0,0,У> <0,0,Р> <7,2,Л> <0,0,Ь> <0,0,\_> <5,3,Е> <1,1,\_> <6,3,А> <2,1,О>\\
\begin{table}[h!]
\centering
\begin{tabular}{|c|c|c|c|c|c|c|c|c|c|c|c|c|c|c|c|c|} 
\hline
\multicolumn{10}{|c|}{Cловарь} & \multicolumn{6}{c|}{Буфер} & Код  \\ \hline
  &   &   &   &   &   &   &   &   &   & \cellcolor[HTML]{8CE4F6} К & У & Р & К & У & Л & <0,0,К>
\\ \hline
  &   &   &   &   &   &   &   &   & К & \cellcolor[HTML]{8CE4F6} У & Р & К & У & Л & Ь & <0,0,У>
\\ \hline
  &   &   &   &   &   &   &   & К & У & \cellcolor[HTML]{8CE4F6} Р & К & У & Л & Ь &   & <0,0,Р>
\\ \hline
  &   &   &   &   &   &   & \cellcolor[HTML]{FFFF00} К & \cellcolor[HTML]{FFFF00} У & Р & \cellcolor[HTML]{FFFF00} К & \cellcolor[HTML]{FFFF00} У & \cellcolor[HTML]{8CE4F6} Л & Ь &   & К & <7,2,Л>
\\ \hline
  &   &   &   & К & У & Р & К & У & Л & \cellcolor[HTML]{8CE4F6} Ь &   & К & У & Л & Е & <0,0,Ь>
\\ \hline
  &   &   & К & У & Р & К & У & Л & Ь & \cellcolor[HTML]{8CE4F6}   & К & У & Л & Е & К & <0,0,\_>
\\ \hline
  &   & К & У & Р & \cellcolor[HTML]{FFFF00} К & \cellcolor[HTML]{FFFF00} У & \cellcolor[HTML]{FFFF00} Л & Ь &   & \cellcolor[HTML]{FFFF00} К & \cellcolor[HTML]{FFFF00} У & \cellcolor[HTML]{FFFF00} Л & \cellcolor[HTML]{8CE4F6} Е & К &   & <5,3,Е>
\\ \hline
Р & \cellcolor[HTML]{FFFF00} К & У & Л & Ь &   & К & У & Л & Е & \cellcolor[HTML]{FFFF00} К & \cellcolor[HTML]{8CE4F6}   & Л & Е & К & А & <1,1,\_>
\\ \hline
У & Л & Ь &   & К & У & \cellcolor[HTML]{FFFF00} Л & \cellcolor[HTML]{FFFF00} Е & \cellcolor[HTML]{FFFF00} К &   & \cellcolor[HTML]{FFFF00} Л & \cellcolor[HTML]{FFFF00} Е & \cellcolor[HTML]{FFFF00} К & \cellcolor[HTML]{8CE4F6} А & Л & О & <6,3,А>
\\ \hline
К & У & \cellcolor[HTML]{FFFF00} Л & Е & К &   & Л & Е & К & А & \cellcolor[HTML]{FFFF00} Л & \cellcolor[HTML]{8CE4F6} О &   &   &   &   & <2,1,О>
\\ \hline
\end{tabular}
\end{table}

\paragraph{Задание 3.3}

Закодировать сообщение методом LZ78\\
Строка:КУРКУЛЬ\_КУЛЕК\_ЛЕКАЛО\\
\begin{table}[h!]
\centering
\begin{tabular}{|c|c|c|} 
\hline
 Входная фраза (в словарь) & Код & Позиция словаря \\ \hline

 &  & 0 \\ \hline
К & 0'К' & 1 \\ \hline
У & 0'У' & 2 \\ \hline
Р & 0'Р' & 3 \\ \hline
КУ & 1'У' & 4 \\ \hline
Л & 0'Л' & 5 \\ \hline
Ь & 0'Ь' & 6 \\ \hline
\_ & 0'\_' & 7 \\ \hline
КУЛ & 4'Л' & 8 \\ \hline
Е & 0'Е' & 9 \\ \hline
К\_ & 1'\_' & 10 \\ \hline
ЛЕ & 5'Е' & 11 \\ \hline
КА & 1'А' & 12 \\ \hline
ЛО & 5'О' & 13 \\ \hline
\end{tabular}
\end{table}

Результат: 0'К' 0'У' 0'Р' 1'У' 0'Л' 0'Ь' 0'\_' 4'Л' 0'Е' 1'\_' 5'Е' 1'А' 5'О'\\
\pagebreak
\paragraph{Задание 4. Арифметическое кодирование\\}

Исходная строка: БАЗАААРРРР\
\begin{center}
 \begin{tabular}{ |c|c| } 
  \hline
     Буква & Вероятность \\ \hline
А & 0.40\\\hline
Р & 0.40\\\hline
Б & 0.10\\\hline
З & 0.10
\\ \hline \end{tabular}
\end{center}
\begin{center}
 \begin{tabular}{ |c|c|c| } 
  \hline
     Буква & Начало & Конец \\ \hline
А & 0.00 & 0.40\\\hline
Р & 0.40 & 0.80\\\hline
Б & 0.80 & 0.90\\\hline
З & 0.90 & 1.00
\\ \hline \end{tabular}
\end{center}
\begin{center}
 \begin{tabular}{ |c|c|c|c| } 
  \hline
     Буква & delta & min & max \\ \hline
Б & 0.1000000000 & 0.8000000000 & 0.9000000000\\\hline
А & 0.0400000000 & 0.8000000000 & 0.8400000000\\\hline
З & 0.0040000000 & 0.8360000000 & 0.8400000000\\\hline
А & 0.0016000000 & 0.8360000000 & 0.8376000000\\\hline
А & 0.0006400000 & 0.8360000000 & 0.8366400000\\\hline
А & 0.0002560000 & 0.8360000000 & 0.8362560000\\\hline
Р & 0.0001024000 & 0.8361024000 & 0.8362048000\\\hline
Р & 0.0000409600 & 0.8361433600 & 0.8361843200\\\hline
Р & 0.0000163840 & 0.8361597440 & 0.8361761280\\\hline
Р & 0.0000065536 & 0.8361662976 & 0.8361728512
\\ \hline \end{tabular}
\end{center}
Результат: 83617
\pagebreak
\paragraph{Задание 5.1}

\\ 

Декодировать сообщение методом адаптивного хаффмана \\
Строка: 
Ошибка декодирования\\
Результат: Ошибка декодирования
\paragraph{Задание 5.3 Декодировать строку(LZ78)\\}

Исходная строка: [0'з'] [0'а'] [0'р'] [0'я'] [0' '] [1'а'] [3'я'] [0'д'] [0'к'] [2' '] [7'д']\\
\begin{table}[h!]
\centering
\begin{tabular}{|c|c|c|} 
\hline
 Код & Словарь & Выходной поток 
\hline

 & [] & 
\\ \hline
0'з' & [, з] & з
\\ \hline
0'а' & [, з, а] & а
\\ \hline
0'р' & [, з, а, р] & р
\\ \hline
0'я' & [, з, а, р, я] & я
\\ \hline
0' ' & [, з, а, р, я,  ] &  
\\ \hline
1'а' & [, з, а, р, я,  , за] & за
\\ \hline
3'я' & [, з, а, р, я,  , за, ря] & ря
\\ \hline
0'д' & [, з, а, р, я,  , за, ря, д] & д
\\ \hline
0'к' & [, з, а, р, я,  , за, ря, д, к] & к
\\ \hline
2' ' & [, з, а, р, я,  , за, ря, д, к, а ] & а 
\\ \hline
7'д' & [, з, а, р, я,  , за, ря, д, к, а , ряд] & ряд
\\ \hline
\end{tabular}
\end{table}

Результат: заря зарядка ряд
\pagebreak
\subsection{Вариант №30}
\paragraph{Задание 1. Блочный хаффман \\}

Строка ОККОЛТКККК, размер блока: 2
\begin{center}
 \begin{tabular}{ |c|c|l| } 
  \hline
     Буква & Вероятность & Код\\ \hline
К & 0.60 & 1\\\hline
О & 0.20 & 00\\\hline
Т & 0.10 & 010\\\hline
Л & 0.10 & 011
\\ \hline \end{tabular}
\end{center}
Энтропия алфавита: 1.5710
\begin{center}
 \begin{tabular}{ |c|c|l| } 
  \hline
     Блок & Вероятность & Код\\ \hline
КК & 0.36 & 11\\\hline
КО & 0.12 & 010\\\hline
ОК & 0.12 & 011\\\hline
КЛ & 0.06 & 1000\\\hline
КТ & 0.06 & 1001\\\hline
ТК & 0.06 & 1010\\\hline
ЛК & 0.06 & 1011\\\hline
ОО & 0.04 & 0000\\\hline
ЛО & 0.02 & 00010\\\hline
ОТ & 0.02 & 00011\\\hline
ТО & 0.02 & 00100\\\hline
ОЛ & 0.02 & 00101\\\hline
ТТ & 0.01 & 001100\\\hline
ЛЛ & 0.01 & 001101\\\hline
ЛТ & 0.01 & 001110\\\hline
ТЛ & 0.01 & 001111
\\ \hline \end{tabular}
\end{center}
Бит на символ при посимвольном кодировании: 1.6000, при блочном: 1.6000


\pagebreak
\paragraph{Задание 2. Сжать адаптивным хаффманом\\}

Строка: 
КРЫЛЛЛЛЫРР\\
Результат: 'К' 0'Р' 00'Ы' 100'Л' 001 11 0 011 010 111










\pagebreak
\paragraph{Задание 3.1}

Закодировать сообщение методом LZ77\\
Строка:СКЛАД\_КЛАД\_КЛАДЕЗЬ\\
Результат: <0,0,С> <0,0,К> <0,0,Л> <0,0,А> <0,0,Д> <0,0,\_> <5,5,К> <0,3,Е> <0,0,З> <0,0,Ь>\\
\begin{table}[h!]
\centering
\begin{tabular}{|c|c|c|c|c|c|c|c|c|c|c|c|c|c|c|c|c|} 
\hline
\multicolumn{10}{|c|}{Cловарь} & \multicolumn{6}{c|}{Буфер} & Код  \\ \hline
  &   &   &   &   &   &   &   &   &   & \cellcolor[HTML]{8CE4F6} С & К & Л & А & Д &   & <0,0,С>
\\ \hline
  &   &   &   &   &   &   &   &   & С & \cellcolor[HTML]{8CE4F6} К & Л & А & Д &   & К & <0,0,К>
\\ \hline
  &   &   &   &   &   &   &   & С & К & \cellcolor[HTML]{8CE4F6} Л & А & Д &   & К & Л & <0,0,Л>
\\ \hline
  &   &   &   &   &   &   & С & К & Л & \cellcolor[HTML]{8CE4F6} А & Д &   & К & Л & А & <0,0,А>
\\ \hline
  &   &   &   &   &   & С & К & Л & А & \cellcolor[HTML]{8CE4F6} Д &   & К & Л & А & Д & <0,0,Д>
\\ \hline
  &   &   &   &   & С & К & Л & А & Д & \cellcolor[HTML]{8CE4F6}   & К & Л & А & Д &   & <0,0,\_>
\\ \hline
  &   &   &   & С & \cellcolor[HTML]{FFFF00} К & \cellcolor[HTML]{FFFF00} Л & \cellcolor[HTML]{FFFF00} А & \cellcolor[HTML]{FFFF00} Д & \cellcolor[HTML]{FFFF00}   & \cellcolor[HTML]{FFFF00} К & \cellcolor[HTML]{FFFF00} Л & \cellcolor[HTML]{FFFF00} А & \cellcolor[HTML]{FFFF00} Д & \cellcolor[HTML]{FFFF00}   & \cellcolor[HTML]{8CE4F6} К & <5,5,К>
\\ \hline
\cellcolor[HTML]{FFFF00} Л & \cellcolor[HTML]{FFFF00} А & \cellcolor[HTML]{FFFF00} Д &   & К & Л & А & Д &   & К & \cellcolor[HTML]{FFFF00} Л & \cellcolor[HTML]{FFFF00} А & \cellcolor[HTML]{FFFF00} Д & \cellcolor[HTML]{8CE4F6} Е & З & Ь & <0,3,Е>
\\ \hline
К & Л & А & Д &   & К & Л & А & Д & Е & \cellcolor[HTML]{8CE4F6} З & Ь &   &   &   &   & <0,0,З>
\\ \hline
Л & А & Д &   & К & Л & А & Д & Е & З & \cellcolor[HTML]{8CE4F6} Ь &   &   &   &   &   & <0,0,Ь>
\\ \hline
\end{tabular}
\end{table}

\paragraph{Задание 3.3}

Закодировать сообщение методом LZ78\\
Строка:СКЛАД\_КЛАД\_КЛАДЕЗЬ\\
\begin{table}[h!]
\centering
\begin{tabular}{|c|c|c|} 
\hline
 Входная фраза (в словарь) & Код & Позиция словаря \\ \hline

 &  & 0 \\ \hline
С & 0'С' & 1 \\ \hline
К & 0'К' & 2 \\ \hline
Л & 0'Л' & 3 \\ \hline
А & 0'А' & 4 \\ \hline
Д & 0'Д' & 5 \\ \hline
\_ & 0'\_' & 6 \\ \hline
КЛ & 2'Л' & 7 \\ \hline
АД & 4'Д' & 8 \\ \hline
\_К & 6'К' & 9 \\ \hline
ЛА & 3'А' & 10 \\ \hline
ДЕ & 5'Е' & 11 \\ \hline
З & 0'З' & 12 \\ \hline
Ь & 0'Ь' & 13 \\ \hline
\end{tabular}
\end{table}

Результат: 0'С' 0'К' 0'Л' 0'А' 0'Д' 0'\_' 2'Л' 4'Д' 6'К' 3'А' 5'Е' 0'З' 0'Ь'\\
\pagebreak
\paragraph{Задание 4. Арифметическое кодирование\\}

Исходная строка: КРЫЛЛЛЛЫРР\
\begin{center}
 \begin{tabular}{ |c|c| } 
  \hline
     Буква & Вероятность \\ \hline
Л & 0.40\\\hline
Р & 0.30\\\hline
Ы & 0.20\\\hline
К & 0.10
\\ \hline \end{tabular}
\end{center}
\begin{center}
 \begin{tabular}{ |c|c|c| } 
  \hline
     Буква & Начало & Конец \\ \hline
Л & 0.00 & 0.40\\\hline
Р & 0.40 & 0.70\\\hline
Ы & 0.70 & 0.90\\\hline
К & 0.90 & 1.00
\\ \hline \end{tabular}
\end{center}
\begin{center}
 \begin{tabular}{ |c|c|c|c| } 
  \hline
     Буква & delta & min & max \\ \hline
К & 0.1000000000 & 0.9000000000 & 1.0000000000\\\hline
Р & 0.0300000000 & 0.9400000000 & 0.9700000000\\\hline
Ы & 0.0060000000 & 0.9610000000 & 0.9670000000\\\hline
Л & 0.0024000000 & 0.9610000000 & 0.9634000000\\\hline
Л & 0.0009600000 & 0.9610000000 & 0.9619600000\\\hline
Л & 0.0003840000 & 0.9610000000 & 0.9613840000\\\hline
Л & 0.0001536000 & 0.9610000000 & 0.9611536000\\\hline
Ы & 0.0000307200 & 0.9611075200 & 0.9611382400\\\hline
Р & 0.0000092160 & 0.9611198080 & 0.9611290240\\\hline
Р & 0.0000027648 & 0.9611234944 & 0.9611262592
\\ \hline \end{tabular}
\end{center}
Результат: 961124
\pagebreak
\paragraph{Задание 5.1}

\\ 

Декодировать сообщение методом адаптивного хаффмана \\
Строка: 
'S'0'K'00'T'100'R'10111101110111110\\
Результат: SKTRSSRKKSS











\paragraph{Задание 5.3 Декодировать строку(LZ78)\\}

Исходная строка: [0'к'] [0'л'] [0'а'] [0'д'] [0' '] [0'с'] [1'л'] [3'д'] [5'л'] [8' '] [2'а'] [4'ь'] [0'я']\\
\begin{table}[h!]
\centering
\begin{tabular}{|c|c|c|} 
\hline
 Код & Словарь & Выходной поток 
\hline

 & [] & 
\\ \hline
0'к' & [, к] & к
\\ \hline
0'л' & [, к, л] & л
\\ \hline
0'а' & [, к, л, а] & а
\\ \hline
0'д' & [, к, л, а, д] & д
\\ \hline
0' ' & [, к, л, а, д,  ] &  
\\ \hline
0'с' & [, к, л, а, д,  , с] & с
\\ \hline
1'л' & [, к, л, а, д,  , с, кл] & кл
\\ \hline
3'д' & [, к, л, а, д,  , с, кл, ад] & ад
\\ \hline
5'л' & [, к, л, а, д,  , с, кл, ад,  л] &  л
\\ \hline
8' ' & [, к, л, а, д,  , с, кл, ад,  л, ад ] & ад 
\\ \hline
2'а' & [, к, л, а, д,  , с, кл, ад,  л, ад , ла] & ла
\\ \hline
4'ь' & [, к, л, а, д,  , с, кл, ад,  л, ад , ла, дь] & дь
\\ \hline
0'я' & [, к, л, а, д,  , с, кл, ад,  л, ад , ла, дь, я] & я
\\ \hline
\end{tabular}
\end{table}

Результат: клад склад лад ладья
\pagebreak
\subsection{Вариант №0}
\paragraph{Задание 1. Блочный хаффман \\}

Строка ОККОЛТКККК, размер блока: 2
\begin{center}
 \begin{tabular}{ |c|c|l| } 
  \hline
     Буква & Вероятность & Код\\ \hline
К & 0.60 & 1\\\hline
О & 0.20 & 00\\\hline
Т & 0.10 & 010\\\hline
Л & 0.10 & 011
\\ \hline \end{tabular}
\end{center}
Энтропия алфавита: 1.5710
\begin{center}
 \begin{tabular}{ |c|c|l| } 
  \hline
     Блок & Вероятность & Код\\ \hline
КК & 0.36 & 11\\\hline
КО & 0.12 & 010\\\hline
ОК & 0.12 & 011\\\hline
КЛ & 0.06 & 1000\\\hline
КТ & 0.06 & 1001\\\hline
ТК & 0.06 & 1010\\\hline
ЛК & 0.06 & 1011\\\hline
ОО & 0.04 & 0000\\\hline
ЛО & 0.02 & 00010\\\hline
ОТ & 0.02 & 00011\\\hline
ТО & 0.02 & 00100\\\hline
ОЛ & 0.02 & 00101\\\hline
ТТ & 0.01 & 001100\\\hline
ЛЛ & 0.01 & 001101\\\hline
ЛТ & 0.01 & 001110\\\hline
ТЛ & 0.01 & 001111
\\ \hline \end{tabular}
\end{center}
Бит на символ при посимвольном кодировании: 1.6000, при блочном: 1.6000


\pagebreak
\paragraph{Задание 2. Сжать адаптивным хаффманом\\}

Строка: 
ABCCDDDDBB\\
Результат: 'A' 0'B' 00'C' 101 110'D' 1101 10 0 1011 111










\pagebreak
\paragraph{Задание 3.1}

Закодировать сообщение методом LZ77\\
Строка:СКЛАД\_КЛАД\_КЛАДЕЗЬ\\
Результат: <0,0,С> <0,0,К> <0,0,Л> <0,0,А> <0,0,Д> <0,0,\_> <5,5,К> <0,3,Е> <0,0,З> <0,0,Ь>\\
\begin{table}[h!]
\centering
\begin{tabular}{|c|c|c|c|c|c|c|c|c|c|c|c|c|c|c|c|c|} 
\hline
\multicolumn{10}{|c|}{Cловарь} & \multicolumn{6}{c|}{Буфер} & Код  \\ \hline
  &   &   &   &   &   &   &   &   &   & \cellcolor[HTML]{8CE4F6} С & К & Л & А & Д &   & <0,0,С>
\\ \hline
  &   &   &   &   &   &   &   &   & С & \cellcolor[HTML]{8CE4F6} К & Л & А & Д &   & К & <0,0,К>
\\ \hline
  &   &   &   &   &   &   &   & С & К & \cellcolor[HTML]{8CE4F6} Л & А & Д &   & К & Л & <0,0,Л>
\\ \hline
  &   &   &   &   &   &   & С & К & Л & \cellcolor[HTML]{8CE4F6} А & Д &   & К & Л & А & <0,0,А>
\\ \hline
  &   &   &   &   &   & С & К & Л & А & \cellcolor[HTML]{8CE4F6} Д &   & К & Л & А & Д & <0,0,Д>
\\ \hline
  &   &   &   &   & С & К & Л & А & Д & \cellcolor[HTML]{8CE4F6}   & К & Л & А & Д &   & <0,0,\_>
\\ \hline
  &   &   &   & С & \cellcolor[HTML]{FFFF00} К & \cellcolor[HTML]{FFFF00} Л & \cellcolor[HTML]{FFFF00} А & \cellcolor[HTML]{FFFF00} Д & \cellcolor[HTML]{FFFF00}   & \cellcolor[HTML]{FFFF00} К & \cellcolor[HTML]{FFFF00} Л & \cellcolor[HTML]{FFFF00} А & \cellcolor[HTML]{FFFF00} Д & \cellcolor[HTML]{FFFF00}   & \cellcolor[HTML]{8CE4F6} К & <5,5,К>
\\ \hline
\cellcolor[HTML]{FFFF00} Л & \cellcolor[HTML]{FFFF00} А & \cellcolor[HTML]{FFFF00} Д &   & К & Л & А & Д &   & К & \cellcolor[HTML]{FFFF00} Л & \cellcolor[HTML]{FFFF00} А & \cellcolor[HTML]{FFFF00} Д & \cellcolor[HTML]{8CE4F6} Е & З & Ь & <0,3,Е>
\\ \hline
К & Л & А & Д &   & К & Л & А & Д & Е & \cellcolor[HTML]{8CE4F6} З & Ь &   &   &   &   & <0,0,З>
\\ \hline
Л & А & Д &   & К & Л & А & Д & Е & З & \cellcolor[HTML]{8CE4F6} Ь &   &   &   &   &   & <0,0,Ь>
\\ \hline
\end{tabular}
\end{table}

\paragraph{Задание 3.3}

Закодировать сообщение методом LZ78\\
Строка:СКЛАД\_КЛАД\_КЛАДЕЗЬ\\
\begin{table}[h!]
\centering
\begin{tabular}{|c|c|c|} 
\hline
 Входная фраза (в словарь) & Код & Позиция словаря \\ \hline

 &  & 0 \\ \hline
С & 0'С' & 1 \\ \hline
К & 0'К' & 2 \\ \hline
Л & 0'Л' & 3 \\ \hline
А & 0'А' & 4 \\ \hline
Д & 0'Д' & 5 \\ \hline
\_ & 0'\_' & 6 \\ \hline
КЛ & 2'Л' & 7 \\ \hline
АД & 4'Д' & 8 \\ \hline
\_К & 6'К' & 9 \\ \hline
ЛА & 3'А' & 10 \\ \hline
ДЕ & 5'Е' & 11 \\ \hline
З & 0'З' & 12 \\ \hline
Ь & 0'Ь' & 13 \\ \hline
\end{tabular}
\end{table}

Результат: 0'С' 0'К' 0'Л' 0'А' 0'Д' 0'\_' 2'Л' 4'Д' 6'К' 3'А' 5'Е' 0'З' 0'Ь'\\
\pagebreak
\paragraph{Задание 4. Арифметическое кодирование\\}

Исходная строка: ABCCDDDDBB\
\begin{center}
 \begin{tabular}{ |c|c| } 
  \hline
     Буква & Вероятность \\ \hline
D & 0.40\\\hline
B & 0.30\\\hline
C & 0.20\\\hline
A & 0.10
\\ \hline \end{tabular}
\end{center}
\begin{center}
 \begin{tabular}{ |c|c|c| } 
  \hline
     Буква & Начало & Конец \\ \hline
D & 0.00 & 0.40\\\hline
B & 0.40 & 0.70\\\hline
C & 0.70 & 0.90\\\hline
A & 0.90 & 1.00
\\ \hline \end{tabular}
\end{center}
\begin{center}
 \begin{tabular}{ |c|c|c|c| } 
  \hline
     Буква & delta & min & max \\ \hline
A & 0.1000000000 & 0.9000000000 & 1.0000000000\\\hline
B & 0.0300000000 & 0.9400000000 & 0.9700000000\\\hline
C & 0.0060000000 & 0.9610000000 & 0.9670000000\\\hline
C & 0.0012000000 & 0.9652000000 & 0.9664000000\\\hline
D & 0.0004800000 & 0.9652000000 & 0.9656800000\\\hline
D & 0.0001920000 & 0.9652000000 & 0.9653920000\\\hline
D & 0.0000768000 & 0.9652000000 & 0.9652768000\\\hline
D & 0.0000307200 & 0.9652000000 & 0.9652307200\\\hline
B & 0.0000092160 & 0.9652122880 & 0.9652215040\\\hline
B & 0.0000027648 & 0.9652159744 & 0.9652187392
\\ \hline \end{tabular}
\end{center}
Результат: 965216
\pagebreak
\paragraph{Задание 5.1}

\\ 

Декодировать сообщение методом адаптивного хаффмана \\
Строка: 
'S'0'K'00'T'100'R'10111101110111110\\
Результат: SKTRSSRKKSS











\paragraph{Задание 5.3 Декодировать строку(LZ78)\\}

Исходная строка: [0'к'] [0'л'] [0'а'] [0'д'] [0' '] [0'с'] [1'л'] [3'д'] [5'л'] [8' '] [2'а'] [4'ь'] [0'я']\\
\begin{table}[h!]
\centering
\begin{tabular}{|c|c|c|} 
\hline
 Код & Словарь & Выходной поток 
\hline

 & [] & 
\\ \hline
0'к' & [, к] & к
\\ \hline
0'л' & [, к, л] & л
\\ \hline
0'а' & [, к, л, а] & а
\\ \hline
0'д' & [, к, л, а, д] & д
\\ \hline
0' ' & [, к, л, а, д,  ] &  
\\ \hline
0'с' & [, к, л, а, д,  , с] & с
\\ \hline
1'л' & [, к, л, а, д,  , с, кл] & кл
\\ \hline
3'д' & [, к, л, а, д,  , с, кл, ад] & ад
\\ \hline
5'л' & [, к, л, а, д,  , с, кл, ад,  л] &  л
\\ \hline
8' ' & [, к, л, а, д,  , с, кл, ад,  л, ад ] & ад 
\\ \hline
2'а' & [, к, л, а, д,  , с, кл, ад,  л, ад , ла] & ла
\\ \hline
4'ь' & [, к, л, а, д,  , с, кл, ад,  л, ад , ла, дь] & дь
\\ \hline
0'я' & [, к, л, а, д,  , с, кл, ад,  л, ад , ла, дь, я] & я
\\ \hline
\end{tabular}
\end{table}

Результат: клад склад лад ладья
\pagebreak
\end{document}
