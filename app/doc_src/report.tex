
\documentclass[a4paper, 12pt]{article}
\usepackage[left=3cm,right=1.5cm,top=2cm,bottom=2cm]{geometry}
\usepackage[T2A]{fontenc}
\usepackage{graphicx}
\usepackage[table]{xcolor}

%Hyphenation rules
%--------------------------------------
\usepackage{hyphenat}
\hyphenation{ма-те-ма-ти-ка вос-ста-нав-ли-вать}
%--------------------------------------
\usepackage[english, russian]{babel}
\begin{document}
 
\tableofcontents

\begin{abstract}
  Это вводный абзац в начале документа.
\end{abstract}
 
\section{Задание}
\begin{enumerate}
\item Составить таблицу кодов блоков для метода Хаффмана с блокированием. Вероятности букв считать по фрагменту сообщения в задании. Длина блока указана. Вычислить EX, ML(X), ML(Xбл). Здесь EX – энтропия алфавита из букв сообщения, ML(X) – среднее количество элементарных символов на букву при сжатии методом Хаффмана, ML(Xбл) – среднее количество элементарных символов на букву при сжатии методом Хаффмана с блокированием. 
\item Сжать сообщение адаптивным методом Хаффмана. 
\item Сжать сообщение методами LZ77, LZSS, LZ78  Для методов LZ77, LZSS размер словаря – 10 символов, буфера – 6 символов. Для метода LZ78 размер словаря 32 записи. 
\item Сжать сообщение из задания №2 арифметическим методом. 
\item Распаковать сообщения, сжатые адаптивным методом Хаффмана, методами LZ77, LZSS, LZ78 и арифметическим методом. Для методов LZ77, LZSS размер словаря – 10 символов. Для метода LZ78 размер словаря – 16 записей. При декодировании таблица состоит из следующих столбцов: «Код», «Словарь» и «Выходной поток».
\end{enumerate}
\pagebreak
\section{Решение}
\subsection{Вариант №1}
\paragraph{Задание 3.1}

Закодировать сообщение методом LZ77\\
Строка:КУКУКУ\_КУКУШКА\_КУКИШ\\
Результат: <0,0,К> <0,0,У> <8,4,\_> <3,4,Ш> <0,1,А> <2,4,И> <0,0,Ш>\\
\begin{table}[h!]
\centering
\begin{tabular}{|c|c|c|c|c|c|c|c|c|c|c|c|c|c|c|c|c|} 
\hline
\multicolumn{10}{|c|}{Cловарь} & \multicolumn{6}{c|}{Буфер} & Код  \\ \hline
  &   &   &   &   &   &   &   &   &   & \cellcolor[HTML]{8CE4F6} К & У & К & У & К & У & <0,0,К>
\\ \hline
  &   &   &   &   &   &   &   &   & К & \cellcolor[HTML]{8CE4F6} У & К & У & К & У &   & <0,0,У>
\\ \hline
  &   &   &   &   &   &   &   & \cellcolor[HTML]{FFFF00} К & \cellcolor[HTML]{FFFF00} У & \cellcolor[HTML]{FFFF00} К & \cellcolor[HTML]{FFFF00} У & \cellcolor[HTML]{FFFF00} К & \cellcolor[HTML]{FFFF00} У & \cellcolor[HTML]{8CE4F6}   & К & <8,4,\_>
\\ \hline
  &   &   & \cellcolor[HTML]{FFFF00} К & \cellcolor[HTML]{FFFF00} У & \cellcolor[HTML]{FFFF00} К & \cellcolor[HTML]{FFFF00} У & К & У &   & \cellcolor[HTML]{FFFF00} К & \cellcolor[HTML]{FFFF00} У & \cellcolor[HTML]{FFFF00} К & \cellcolor[HTML]{FFFF00} У & \cellcolor[HTML]{8CE4F6} Ш & К & <3,4,Ш>
\\ \hline
\cellcolor[HTML]{FFFF00} К & У & К & У &   & К & У & К & У & Ш & \cellcolor[HTML]{FFFF00} К & \cellcolor[HTML]{8CE4F6} А &   & К & У & К & <0,1,А>
\\ \hline
К & У & \cellcolor[HTML]{FFFF00}   & \cellcolor[HTML]{FFFF00} К & \cellcolor[HTML]{FFFF00} У & \cellcolor[HTML]{FFFF00} К & У & Ш & К & А & \cellcolor[HTML]{FFFF00}   & \cellcolor[HTML]{FFFF00} К & \cellcolor[HTML]{FFFF00} У & \cellcolor[HTML]{FFFF00} К & \cellcolor[HTML]{8CE4F6} И & Ш & <2,4,И>
\\ \hline
К & У & Ш & К & А &   & К & У & К & И & \cellcolor[HTML]{8CE4F6} Ш &   &   &   &   &   & <0,0,Ш>
\\ \hline
\end{tabular}
\end{table}

\pagebreak
\subsection{Вариант №2}

\pagebreak
\subsection{Вариант №3}
\paragraph{Задание 3.1}

Закодировать сообщение методом LZ77\\
Строка:ТАРАРА\_ТАРТАР\_ТАРТ\_ТАРА\\
Результат: <0,0,Т> <0,0,А> <0,0,Р> <8,3,\_> <3,3,Т> <0,2,\_> <3,4,\_> <1,3,А>\\
\begin{table}[h!]
\centering
\begin{tabular}{|c|c|c|c|c|c|c|c|c|c|c|c|c|c|c|c|c|} 
\hline
\multicolumn{10}{|c|}{Cловарь} & \multicolumn{6}{c|}{Буфер} & Код  \\ \hline
  &   &   &   &   &   &   &   &   &   & \cellcolor[HTML]{8CE4F6} Т & А & Р & А & Р & А & <0,0,Т>
\\ \hline
  &   &   &   &   &   &   &   &   & Т & \cellcolor[HTML]{8CE4F6} А & Р & А & Р & А &   & <0,0,А>
\\ \hline
  &   &   &   &   &   &   &   & Т & А & \cellcolor[HTML]{8CE4F6} Р & А & Р & А &   & Т & <0,0,Р>
\\ \hline
  &   &   &   &   &   &   & Т & \cellcolor[HTML]{FFFF00} А & \cellcolor[HTML]{FFFF00} Р & \cellcolor[HTML]{FFFF00} А & \cellcolor[HTML]{FFFF00} Р & \cellcolor[HTML]{FFFF00} А & \cellcolor[HTML]{8CE4F6}   & Т & А & <8,3,\_>
\\ \hline
  &   &   & \cellcolor[HTML]{FFFF00} Т & \cellcolor[HTML]{FFFF00} А & \cellcolor[HTML]{FFFF00} Р & А & Р & А &   & \cellcolor[HTML]{FFFF00} Т & \cellcolor[HTML]{FFFF00} А & \cellcolor[HTML]{FFFF00} Р & \cellcolor[HTML]{8CE4F6} Т & А & Р & <3,3,Т>
\\ \hline
\cellcolor[HTML]{FFFF00} А & \cellcolor[HTML]{FFFF00} Р & А & Р & А &   & Т & А & Р & Т & \cellcolor[HTML]{FFFF00} А & \cellcolor[HTML]{FFFF00} Р & \cellcolor[HTML]{8CE4F6}   & Т & А & Р & <0,2,\_>
\\ \hline
Р & А &   & \cellcolor[HTML]{FFFF00} Т & \cellcolor[HTML]{FFFF00} А & \cellcolor[HTML]{FFFF00} Р & \cellcolor[HTML]{FFFF00} Т & А & Р &   & \cellcolor[HTML]{FFFF00} Т & \cellcolor[HTML]{FFFF00} А & \cellcolor[HTML]{FFFF00} Р & \cellcolor[HTML]{FFFF00} Т & \cellcolor[HTML]{8CE4F6}   & Т & <3,4,\_>
\\ \hline
Р & \cellcolor[HTML]{FFFF00} Т & \cellcolor[HTML]{FFFF00} А & \cellcolor[HTML]{FFFF00} Р &   & Т & А & Р & Т &   & \cellcolor[HTML]{FFFF00} Т & \cellcolor[HTML]{FFFF00} А & \cellcolor[HTML]{FFFF00} Р & \cellcolor[HTML]{8CE4F6} А &   &   & <1,3,А>
\\ \hline
\end{tabular}
\end{table}

\pagebreak
\subsection{Вариант №4}
\paragraph{Задание 3.1}

Закодировать сообщение методом LZ77\\
Строка:СЫР\_СЫН\_СЫРОК\_СЫНОК\\
Результат: <0,0,С> <0,0,Ы> <0,0,Р> <0,0,\_> <6,2,Н> <6,3,Р> <0,0,О> <0,0,К> <0,4,О> <0,0,К>\\
\begin{table}[h!]
\centering
\begin{tabular}{|c|c|c|c|c|c|c|c|c|c|c|c|c|c|c|c|c|} 
\hline
\multicolumn{10}{|c|}{Cловарь} & \multicolumn{6}{c|}{Буфер} & Код  \\ \hline
  &   &   &   &   &   &   &   &   &   & \cellcolor[HTML]{8CE4F6} С & Ы & Р &   & С & Ы & <0,0,С>
\\ \hline
  &   &   &   &   &   &   &   &   & С & \cellcolor[HTML]{8CE4F6} Ы & Р &   & С & Ы & Н & <0,0,Ы>
\\ \hline
  &   &   &   &   &   &   &   & С & Ы & \cellcolor[HTML]{8CE4F6} Р &   & С & Ы & Н &   & <0,0,Р>
\\ \hline
  &   &   &   &   &   &   & С & Ы & Р & \cellcolor[HTML]{8CE4F6}   & С & Ы & Н &   & С & <0,0,\_>
\\ \hline
  &   &   &   &   &   & \cellcolor[HTML]{FFFF00} С & \cellcolor[HTML]{FFFF00} Ы & Р &   & \cellcolor[HTML]{FFFF00} С & \cellcolor[HTML]{FFFF00} Ы & \cellcolor[HTML]{8CE4F6} Н &   & С & Ы & <6,2,Н>
\\ \hline
  &   &   & С & Ы & Р & \cellcolor[HTML]{FFFF00}   & \cellcolor[HTML]{FFFF00} С & \cellcolor[HTML]{FFFF00} Ы & Н & \cellcolor[HTML]{FFFF00}   & \cellcolor[HTML]{FFFF00} С & \cellcolor[HTML]{FFFF00} Ы & \cellcolor[HTML]{8CE4F6} Р & О & К & <6,3,Р>
\\ \hline
Ы & Р &   & С & Ы & Н &   & С & Ы & Р & \cellcolor[HTML]{8CE4F6} О & К &   & С & Ы & Н & <0,0,О>
\\ \hline
Р &   & С & Ы & Н &   & С & Ы & Р & О & \cellcolor[HTML]{8CE4F6} К &   & С & Ы & Н & О & <0,0,К>
\\ \hline
\cellcolor[HTML]{FFFF00}   & \cellcolor[HTML]{FFFF00} С & \cellcolor[HTML]{FFFF00} Ы & \cellcolor[HTML]{FFFF00} Н &   & С & Ы & Р & О & К & \cellcolor[HTML]{FFFF00}   & \cellcolor[HTML]{FFFF00} С & \cellcolor[HTML]{FFFF00} Ы & \cellcolor[HTML]{FFFF00} Н & \cellcolor[HTML]{8CE4F6} О & К & <0,4,О>
\\ \hline
С & Ы & Р & О & К &   & С & Ы & Н & О & \cellcolor[HTML]{8CE4F6} К &   &   &   &   &   & <0,0,К>
\\ \hline
\end{tabular}
\end{table}

\pagebreak
\subsection{Вариант №5}

\pagebreak
\subsection{Вариант №6}

\pagebreak
\subsection{Вариант №7}
\paragraph{Задание 3.1}

Закодировать сообщение методом LZ77\\
Строка:РОЗА\_РОЗАРИЙ\_ЗАРЯДКА\\
Результат: <0,0,Р> <0,0,О> <0,0,З> <0,0,А> <0,0,\_> <5,4,Р> <0,0,И> <0,0,Й> <2,1,З> <4,2,Я> <0,0,Д> <0,0,К> <0,0,А>\\
\begin{table}[h!]
\centering
\begin{tabular}{|c|c|c|c|c|c|c|c|c|c|c|c|c|c|c|c|c|} 
\hline
\multicolumn{10}{|c|}{Cловарь} & \multicolumn{6}{c|}{Буфер} & Код  \\ \hline
  &   &   &   &   &   &   &   &   &   & \cellcolor[HTML]{8CE4F6} Р & О & З & А &   & Р & <0,0,Р>
\\ \hline
  &   &   &   &   &   &   &   &   & Р & \cellcolor[HTML]{8CE4F6} О & З & А &   & Р & О & <0,0,О>
\\ \hline
  &   &   &   &   &   &   &   & Р & О & \cellcolor[HTML]{8CE4F6} З & А &   & Р & О & З & <0,0,З>
\\ \hline
  &   &   &   &   &   &   & Р & О & З & \cellcolor[HTML]{8CE4F6} А &   & Р & О & З & А & <0,0,А>
\\ \hline
  &   &   &   &   &   & Р & О & З & А & \cellcolor[HTML]{8CE4F6}   & Р & О & З & А & Р & <0,0,\_>
\\ \hline
  &   &   &   &   & \cellcolor[HTML]{FFFF00} Р & \cellcolor[HTML]{FFFF00} О & \cellcolor[HTML]{FFFF00} З & \cellcolor[HTML]{FFFF00} А &   & \cellcolor[HTML]{FFFF00} Р & \cellcolor[HTML]{FFFF00} О & \cellcolor[HTML]{FFFF00} З & \cellcolor[HTML]{FFFF00} А & \cellcolor[HTML]{8CE4F6} Р & И & <5,4,Р>
\\ \hline
Р & О & З & А &   & Р & О & З & А & Р & \cellcolor[HTML]{8CE4F6} И & Й &   & З & А & Р & <0,0,И>
\\ \hline
О & З & А &   & Р & О & З & А & Р & И & \cellcolor[HTML]{8CE4F6} Й &   & З & А & Р & Я & <0,0,Й>
\\ \hline
З & А & \cellcolor[HTML]{FFFF00}   & Р & О & З & А & Р & И & Й & \cellcolor[HTML]{FFFF00}   & \cellcolor[HTML]{8CE4F6} З & А & Р & Я & Д & <2,1,З>
\\ \hline
  & Р & О & З & \cellcolor[HTML]{FFFF00} А & \cellcolor[HTML]{FFFF00} Р & И & Й &   & З & \cellcolor[HTML]{FFFF00} А & \cellcolor[HTML]{FFFF00} Р & \cellcolor[HTML]{8CE4F6} Я & Д & К & А & <4,2,Я>
\\ \hline
З & А & Р & И & Й &   & З & А & Р & Я & \cellcolor[HTML]{8CE4F6} Д & К & А &   &   &   & <0,0,Д>
\\ \hline
А & Р & И & Й &   & З & А & Р & Я & Д & \cellcolor[HTML]{8CE4F6} К & А &   &   &   &   & <0,0,К>
\\ \hline
Р & И & Й &   & З & А & Р & Я & Д & К & \cellcolor[HTML]{8CE4F6} А &   &   &   &   &   & <0,0,А>
\\ \hline
\end{tabular}
\end{table}

\pagebreak
\subsection{Вариант №8}
\paragraph{Задание 3.1}

Закодировать сообщение методом LZ77\\
Строка:ПОЛ\_ПОЛОВНИК\_ПОЛОВЕЦ\\
Результат: <0,0,П> <0,0,О> <0,0,Л> <0,0,\_> <6,3,О> <0,0,В> <0,0,Н> <0,0,И> <0,0,К> <1,5,В> <0,0,Е> <0,0,Ц>\\
\begin{table}[h!]
\centering
\begin{tabular}{|c|c|c|c|c|c|c|c|c|c|c|c|c|c|c|c|c|} 
\hline
\multicolumn{10}{|c|}{Cловарь} & \multicolumn{6}{c|}{Буфер} & Код  \\ \hline
  &   &   &   &   &   &   &   &   &   & \cellcolor[HTML]{8CE4F6} П & О & Л &   & П & О & <0,0,П>
\\ \hline
  &   &   &   &   &   &   &   &   & П & \cellcolor[HTML]{8CE4F6} О & Л &   & П & О & Л & <0,0,О>
\\ \hline
  &   &   &   &   &   &   &   & П & О & \cellcolor[HTML]{8CE4F6} Л &   & П & О & Л & О & <0,0,Л>
\\ \hline
  &   &   &   &   &   &   & П & О & Л & \cellcolor[HTML]{8CE4F6}   & П & О & Л & О & В & <0,0,\_>
\\ \hline
  &   &   &   &   &   & \cellcolor[HTML]{FFFF00} П & \cellcolor[HTML]{FFFF00} О & \cellcolor[HTML]{FFFF00} Л &   & \cellcolor[HTML]{FFFF00} П & \cellcolor[HTML]{FFFF00} О & \cellcolor[HTML]{FFFF00} Л & \cellcolor[HTML]{8CE4F6} О & В & Н & <6,3,О>
\\ \hline
  &   & П & О & Л &   & П & О & Л & О & \cellcolor[HTML]{8CE4F6} В & Н & И & К &   & П & <0,0,В>
\\ \hline
  & П & О & Л &   & П & О & Л & О & В & \cellcolor[HTML]{8CE4F6} Н & И & К &   & П & О & <0,0,Н>
\\ \hline
П & О & Л &   & П & О & Л & О & В & Н & \cellcolor[HTML]{8CE4F6} И & К &   & П & О & Л & <0,0,И>
\\ \hline
О & Л &   & П & О & Л & О & В & Н & И & \cellcolor[HTML]{8CE4F6} К &   & П & О & Л & О & <0,0,К>
\\ \hline
Л & \cellcolor[HTML]{FFFF00}   & \cellcolor[HTML]{FFFF00} П & \cellcolor[HTML]{FFFF00} О & \cellcolor[HTML]{FFFF00} Л & \cellcolor[HTML]{FFFF00} О & В & Н & И & К & \cellcolor[HTML]{FFFF00}   & \cellcolor[HTML]{FFFF00} П & \cellcolor[HTML]{FFFF00} О & \cellcolor[HTML]{FFFF00} Л & \cellcolor[HTML]{FFFF00} О & \cellcolor[HTML]{8CE4F6} В & <1,5,В>
\\ \hline
В & Н & И & К &   & П & О & Л & О & В & \cellcolor[HTML]{8CE4F6} Е & Ц &   &   &   &   & <0,0,Е>
\\ \hline
Н & И & К &   & П & О & Л & О & В & Е & \cellcolor[HTML]{8CE4F6} Ц &   &   &   &   &   & <0,0,Ц>
\\ \hline
\end{tabular}
\end{table}

\pagebreak
\subsection{Вариант №9}

\pagebreak
\subsection{Вариант №10}
\paragraph{Задание 3.1}

Закодировать сообщение методом LZ77\\
Строка:КОК\_КОКЛЮШ\_КЛУБ\_КЛУБОК\\
Результат: <0,0,К> <0,0,О> <8,1,\_> <6,3,Л> <0,0,Ю> <0,0,Ш> <3,2,Л> <0,0,У> <0,0,Б> <5,5,О> <0,0,К>\\
\begin{table}[h!]
\centering
\begin{tabular}{|c|c|c|c|c|c|c|c|c|c|c|c|c|c|c|c|c|} 
\hline
\multicolumn{10}{|c|}{Cловарь} & \multicolumn{6}{c|}{Буфер} & Код  \\ \hline
  &   &   &   &   &   &   &   &   &   & \cellcolor[HTML]{8CE4F6} К & О & К &   & К & О & <0,0,К>
\\ \hline
  &   &   &   &   &   &   &   &   & К & \cellcolor[HTML]{8CE4F6} О & К &   & К & О & К & <0,0,О>
\\ \hline
  &   &   &   &   &   &   &   & \cellcolor[HTML]{FFFF00} К & О & \cellcolor[HTML]{FFFF00} К & \cellcolor[HTML]{8CE4F6}   & К & О & К & Л & <8,1,\_>
\\ \hline
  &   &   &   &   &   & \cellcolor[HTML]{FFFF00} К & \cellcolor[HTML]{FFFF00} О & \cellcolor[HTML]{FFFF00} К &   & \cellcolor[HTML]{FFFF00} К & \cellcolor[HTML]{FFFF00} О & \cellcolor[HTML]{FFFF00} К & \cellcolor[HTML]{8CE4F6} Л & Ю & Ш & <6,3,Л>
\\ \hline
  &   & К & О & К &   & К & О & К & Л & \cellcolor[HTML]{8CE4F6} Ю & Ш &   & К & Л & У & <0,0,Ю>
\\ \hline
  & К & О & К &   & К & О & К & Л & Ю & \cellcolor[HTML]{8CE4F6} Ш &   & К & Л & У & Б & <0,0,Ш>
\\ \hline
К & О & К & \cellcolor[HTML]{FFFF00}   & \cellcolor[HTML]{FFFF00} К & О & К & Л & Ю & Ш & \cellcolor[HTML]{FFFF00}   & \cellcolor[HTML]{FFFF00} К & \cellcolor[HTML]{8CE4F6} Л & У & Б &   & <3,2,Л>
\\ \hline
  & К & О & К & Л & Ю & Ш &   & К & Л & \cellcolor[HTML]{8CE4F6} У & Б &   & К & Л & У & <0,0,У>
\\ \hline
К & О & К & Л & Ю & Ш &   & К & Л & У & \cellcolor[HTML]{8CE4F6} Б &   & К & Л & У & Б & <0,0,Б>
\\ \hline
О & К & Л & Ю & Ш & \cellcolor[HTML]{FFFF00}   & \cellcolor[HTML]{FFFF00} К & \cellcolor[HTML]{FFFF00} Л & \cellcolor[HTML]{FFFF00} У & \cellcolor[HTML]{FFFF00} Б & \cellcolor[HTML]{FFFF00}   & \cellcolor[HTML]{FFFF00} К & \cellcolor[HTML]{FFFF00} Л & \cellcolor[HTML]{FFFF00} У & \cellcolor[HTML]{FFFF00} Б & \cellcolor[HTML]{8CE4F6} О & <5,5,О>
\\ \hline
К & Л & У & Б &   & К & Л & У & Б & О & \cellcolor[HTML]{8CE4F6} К &   &   &   &   &   & <0,0,К>
\\ \hline
\end{tabular}
\end{table}

\pagebreak
\subsection{Вариант №11}
\paragraph{Задание 3.1}

Закодировать сообщение методом LZ77\\
Строка:ВАРВАР\_ВАРИТ\_ВАРЕНЬЕ\\
Результат: <0,0,В> <0,0,А> <0,0,Р> <7,3,\_> <3,3,И> <0,0,Т> <4,4,Е> <0,0,Н> <0,0,Ь> <0,0,Е>\\
\begin{table}[h!]
\centering
\begin{tabular}{|c|c|c|c|c|c|c|c|c|c|c|c|c|c|c|c|c|} 
\hline
\multicolumn{10}{|c|}{Cловарь} & \multicolumn{6}{c|}{Буфер} & Код  \\ \hline
  &   &   &   &   &   &   &   &   &   & \cellcolor[HTML]{8CE4F6} В & А & Р & В & А & Р & <0,0,В>
\\ \hline
  &   &   &   &   &   &   &   &   & В & \cellcolor[HTML]{8CE4F6} А & Р & В & А & Р &   & <0,0,А>
\\ \hline
  &   &   &   &   &   &   &   & В & А & \cellcolor[HTML]{8CE4F6} Р & В & А & Р &   & В & <0,0,Р>
\\ \hline
  &   &   &   &   &   &   & \cellcolor[HTML]{FFFF00} В & \cellcolor[HTML]{FFFF00} А & \cellcolor[HTML]{FFFF00} Р & \cellcolor[HTML]{FFFF00} В & \cellcolor[HTML]{FFFF00} А & \cellcolor[HTML]{FFFF00} Р & \cellcolor[HTML]{8CE4F6}   & В & А & <7,3,\_>
\\ \hline
  &   &   & \cellcolor[HTML]{FFFF00} В & \cellcolor[HTML]{FFFF00} А & \cellcolor[HTML]{FFFF00} Р & В & А & Р &   & \cellcolor[HTML]{FFFF00} В & \cellcolor[HTML]{FFFF00} А & \cellcolor[HTML]{FFFF00} Р & \cellcolor[HTML]{8CE4F6} И & Т &   & <3,3,И>
\\ \hline
А & Р & В & А & Р &   & В & А & Р & И & \cellcolor[HTML]{8CE4F6} Т &   & В & А & Р & Е & <0,0,Т>
\\ \hline
Р & В & А & Р & \cellcolor[HTML]{FFFF00}   & \cellcolor[HTML]{FFFF00} В & \cellcolor[HTML]{FFFF00} А & \cellcolor[HTML]{FFFF00} Р & И & Т & \cellcolor[HTML]{FFFF00}   & \cellcolor[HTML]{FFFF00} В & \cellcolor[HTML]{FFFF00} А & \cellcolor[HTML]{FFFF00} Р & \cellcolor[HTML]{8CE4F6} Е & Н & <4,4,Е>
\\ \hline
В & А & Р & И & Т &   & В & А & Р & Е & \cellcolor[HTML]{8CE4F6} Н & Ь & Е &   &   &   & <0,0,Н>
\\ \hline
А & Р & И & Т &   & В & А & Р & Е & Н & \cellcolor[HTML]{8CE4F6} Ь & Е &   &   &   &   & <0,0,Ь>
\\ \hline
Р & И & Т &   & В & А & Р & Е & Н & Ь & \cellcolor[HTML]{8CE4F6} Е &   &   &   &   &   & <0,0,Е>
\\ \hline
\end{tabular}
\end{table}

\pagebreak
\subsection{Вариант №12}
\paragraph{Задание 3.1}

Закодировать сообщение методом LZ77\\
Строка:СОКОЛ\_СОК\_КОЛ\_КОЛОСОК\\
Результат: <0,0,С> <0,0,О> <0,0,К> <8,1,Л> <0,0,\_> <4,3,\_> <2,4,К> <6,2,О> <0,0,С> <2,1,К>\\
\begin{table}[h!]
\centering
\begin{tabular}{|c|c|c|c|c|c|c|c|c|c|c|c|c|c|c|c|c|} 
\hline
\multicolumn{10}{|c|}{Cловарь} & \multicolumn{6}{c|}{Буфер} & Код  \\ \hline
  &   &   &   &   &   &   &   &   &   & \cellcolor[HTML]{8CE4F6} С & О & К & О & Л &   & <0,0,С>
\\ \hline
  &   &   &   &   &   &   &   &   & С & \cellcolor[HTML]{8CE4F6} О & К & О & Л &   & С & <0,0,О>
\\ \hline
  &   &   &   &   &   &   &   & С & О & \cellcolor[HTML]{8CE4F6} К & О & Л &   & С & О & <0,0,К>
\\ \hline
  &   &   &   &   &   &   & С & \cellcolor[HTML]{FFFF00} О & К & \cellcolor[HTML]{FFFF00} О & \cellcolor[HTML]{8CE4F6} Л &   & С & О & К & <8,1,Л>
\\ \hline
  &   &   &   &   & С & О & К & О & Л & \cellcolor[HTML]{8CE4F6}   & С & О & К &   & К & <0,0,\_>
\\ \hline
  &   &   &   & \cellcolor[HTML]{FFFF00} С & \cellcolor[HTML]{FFFF00} О & \cellcolor[HTML]{FFFF00} К & О & Л &   & \cellcolor[HTML]{FFFF00} С & \cellcolor[HTML]{FFFF00} О & \cellcolor[HTML]{FFFF00} К & \cellcolor[HTML]{8CE4F6}   & К & О & <4,3,\_>
\\ \hline
С & О & \cellcolor[HTML]{FFFF00} К & \cellcolor[HTML]{FFFF00} О & \cellcolor[HTML]{FFFF00} Л & \cellcolor[HTML]{FFFF00}   & С & О & К &   & \cellcolor[HTML]{FFFF00} К & \cellcolor[HTML]{FFFF00} О & \cellcolor[HTML]{FFFF00} Л & \cellcolor[HTML]{FFFF00}   & \cellcolor[HTML]{8CE4F6} К & О & <2,4,К>
\\ \hline
  & С & О & К &   & К & \cellcolor[HTML]{FFFF00} О & \cellcolor[HTML]{FFFF00} Л &   & К & \cellcolor[HTML]{FFFF00} О & \cellcolor[HTML]{FFFF00} Л & \cellcolor[HTML]{8CE4F6} О & С & О & К & <6,2,О>
\\ \hline
К &   & К & О & Л &   & К & О & Л & О & \cellcolor[HTML]{8CE4F6} С & О & К &   &   &   & <0,0,С>
\\ \hline
  & К & \cellcolor[HTML]{FFFF00} О & Л &   & К & О & Л & О & С & \cellcolor[HTML]{FFFF00} О & \cellcolor[HTML]{8CE4F6} К &   &   &   &   & <2,1,К>
\\ \hline
\end{tabular}
\end{table}

\pagebreak
\subsection{Вариант №13}

\pagebreak
\subsection{Вариант №14}

\pagebreak
\subsection{Вариант №15}

\pagebreak
\subsection{Вариант №16}
\paragraph{Задание 3.1}

Закодировать сообщение методом LZ77\\
Строка:УКУС\_КУСКУС\_УКСУС\_КСИ\\
Результат: <0,0,У> <0,0,К> <8,1,С> <0,0,\_> <6,3,К> <3,3,У> <2,1,С> <4,3,К> <1,1,И>\\
\begin{table}[h!]
\centering
\begin{tabular}{|c|c|c|c|c|c|c|c|c|c|c|c|c|c|c|c|c|} 
\hline
\multicolumn{10}{|c|}{Cловарь} & \multicolumn{6}{c|}{Буфер} & Код  \\ \hline
  &   &   &   &   &   &   &   &   &   & \cellcolor[HTML]{8CE4F6} У & К & У & С &   & К & <0,0,У>
\\ \hline
  &   &   &   &   &   &   &   &   & У & \cellcolor[HTML]{8CE4F6} К & У & С &   & К & У & <0,0,К>
\\ \hline
  &   &   &   &   &   &   &   & \cellcolor[HTML]{FFFF00} У & К & \cellcolor[HTML]{FFFF00} У & \cellcolor[HTML]{8CE4F6} С &   & К & У & С & <8,1,С>
\\ \hline
  &   &   &   &   &   & У & К & У & С & \cellcolor[HTML]{8CE4F6}   & К & У & С & К & У & <0,0,\_>
\\ \hline
  &   &   &   &   & У & \cellcolor[HTML]{FFFF00} К & \cellcolor[HTML]{FFFF00} У & \cellcolor[HTML]{FFFF00} С &   & \cellcolor[HTML]{FFFF00} К & \cellcolor[HTML]{FFFF00} У & \cellcolor[HTML]{FFFF00} С & \cellcolor[HTML]{8CE4F6} К & У & С & <6,3,К>
\\ \hline
  & У & К & \cellcolor[HTML]{FFFF00} У & \cellcolor[HTML]{FFFF00} С & \cellcolor[HTML]{FFFF00}   & К & У & С & К & \cellcolor[HTML]{FFFF00} У & \cellcolor[HTML]{FFFF00} С & \cellcolor[HTML]{FFFF00}   & \cellcolor[HTML]{8CE4F6} У & К & С & <3,3,У>
\\ \hline
С &   & \cellcolor[HTML]{FFFF00} К & У & С & К & У & С &   & У & \cellcolor[HTML]{FFFF00} К & \cellcolor[HTML]{8CE4F6} С & У & С &   & К & <2,1,С>
\\ \hline
К & У & С & К & \cellcolor[HTML]{FFFF00} У & \cellcolor[HTML]{FFFF00} С & \cellcolor[HTML]{FFFF00}   & У & К & С & \cellcolor[HTML]{FFFF00} У & \cellcolor[HTML]{FFFF00} С & \cellcolor[HTML]{FFFF00}   & \cellcolor[HTML]{8CE4F6} К & С & И & <4,3,К>
\\ \hline
У & \cellcolor[HTML]{FFFF00} С &   & У & К & С & У & С &   & К & \cellcolor[HTML]{FFFF00} С & \cellcolor[HTML]{8CE4F6} И &   &   &   &   & <1,1,И>
\\ \hline
\end{tabular}
\end{table}

\pagebreak
\subsection{Вариант №17}
\paragraph{Задание 3.1}

Закодировать сообщение методом LZ77\\
Строка:ДОМ\_ДОМИК\_ОМИК\_МИР\\
Результат: <0,0,Д> <0,0,О> <0,0,М> <0,0,\_> <6,3,И> <0,0,К> <4,1,О> <5,4,М> <1,1,Р>\\
\begin{table}[h!]
\centering
\begin{tabular}{|c|c|c|c|c|c|c|c|c|c|c|c|c|c|c|c|c|} 
\hline
\multicolumn{10}{|c|}{Cловарь} & \multicolumn{6}{c|}{Буфер} & Код  \\ \hline
  &   &   &   &   &   &   &   &   &   & \cellcolor[HTML]{8CE4F6} Д & О & М &   & Д & О & <0,0,Д>
\\ \hline
  &   &   &   &   &   &   &   &   & Д & \cellcolor[HTML]{8CE4F6} О & М &   & Д & О & М & <0,0,О>
\\ \hline
  &   &   &   &   &   &   &   & Д & О & \cellcolor[HTML]{8CE4F6} М &   & Д & О & М & И & <0,0,М>
\\ \hline
  &   &   &   &   &   &   & Д & О & М & \cellcolor[HTML]{8CE4F6}   & Д & О & М & И & К & <0,0,\_>
\\ \hline
  &   &   &   &   &   & \cellcolor[HTML]{FFFF00} Д & \cellcolor[HTML]{FFFF00} О & \cellcolor[HTML]{FFFF00} М &   & \cellcolor[HTML]{FFFF00} Д & \cellcolor[HTML]{FFFF00} О & \cellcolor[HTML]{FFFF00} М & \cellcolor[HTML]{8CE4F6} И & К &   & <6,3,И>
\\ \hline
  &   & Д & О & М &   & Д & О & М & И & \cellcolor[HTML]{8CE4F6} К &   & О & М & И & К & <0,0,К>
\\ \hline
  & Д & О & М & \cellcolor[HTML]{FFFF00}   & Д & О & М & И & К & \cellcolor[HTML]{FFFF00}   & \cellcolor[HTML]{8CE4F6} О & М & И & К &   & <4,1,О>
\\ \hline
О & М &   & Д & О & \cellcolor[HTML]{FFFF00} М & \cellcolor[HTML]{FFFF00} И & \cellcolor[HTML]{FFFF00} К & \cellcolor[HTML]{FFFF00}   & О & \cellcolor[HTML]{FFFF00} М & \cellcolor[HTML]{FFFF00} И & \cellcolor[HTML]{FFFF00} К & \cellcolor[HTML]{FFFF00}   & \cellcolor[HTML]{8CE4F6} М & И & <5,4,М>
\\ \hline
М & \cellcolor[HTML]{FFFF00} И & К &   & О & М & И & К &   & М & \cellcolor[HTML]{FFFF00} И & \cellcolor[HTML]{8CE4F6} Р &   &   &   &   & <1,1,Р>
\\ \hline
\end{tabular}
\end{table}

\pagebreak
\subsection{Вариант №18}
\paragraph{Задание 3.1}

Закодировать сообщение методом LZ77\\
Строка:РИМ\_РОМ\_МУРОМ\_МУРКА\\
Результат: <0,0,Р> <0,0,И> <0,0,М> <0,0,\_> <6,1,О> <6,2,М> <0,0,У> <4,5,У> <4,1,К> <0,0,А>\\
\begin{table}[h!]
\centering
\begin{tabular}{|c|c|c|c|c|c|c|c|c|c|c|c|c|c|c|c|c|} 
\hline
\multicolumn{10}{|c|}{Cловарь} & \multicolumn{6}{c|}{Буфер} & Код  \\ \hline
  &   &   &   &   &   &   &   &   &   & \cellcolor[HTML]{8CE4F6} Р & И & М &   & Р & О & <0,0,Р>
\\ \hline
  &   &   &   &   &   &   &   &   & Р & \cellcolor[HTML]{8CE4F6} И & М &   & Р & О & М & <0,0,И>
\\ \hline
  &   &   &   &   &   &   &   & Р & И & \cellcolor[HTML]{8CE4F6} М &   & Р & О & М &   & <0,0,М>
\\ \hline
  &   &   &   &   &   &   & Р & И & М & \cellcolor[HTML]{8CE4F6}   & Р & О & М &   & М & <0,0,\_>
\\ \hline
  &   &   &   &   &   & \cellcolor[HTML]{FFFF00} Р & И & М &   & \cellcolor[HTML]{FFFF00} Р & \cellcolor[HTML]{8CE4F6} О & М &   & М & У & <6,1,О>
\\ \hline
  &   &   &   & Р & И & \cellcolor[HTML]{FFFF00} М & \cellcolor[HTML]{FFFF00}   & Р & О & \cellcolor[HTML]{FFFF00} М & \cellcolor[HTML]{FFFF00}   & \cellcolor[HTML]{8CE4F6} М & У & Р & О & <6,2,М>
\\ \hline
  & Р & И & М &   & Р & О & М &   & М & \cellcolor[HTML]{8CE4F6} У & Р & О & М &   & М & <0,0,У>
\\ \hline
Р & И & М &   & \cellcolor[HTML]{FFFF00} Р & \cellcolor[HTML]{FFFF00} О & \cellcolor[HTML]{FFFF00} М & \cellcolor[HTML]{FFFF00}   & \cellcolor[HTML]{FFFF00} М & У & \cellcolor[HTML]{FFFF00} Р & \cellcolor[HTML]{FFFF00} О & \cellcolor[HTML]{FFFF00} М & \cellcolor[HTML]{FFFF00}   & \cellcolor[HTML]{FFFF00} М & \cellcolor[HTML]{8CE4F6} У & <4,5,У>
\\ \hline
М &   & М & У & \cellcolor[HTML]{FFFF00} Р & О & М &   & М & У & \cellcolor[HTML]{FFFF00} Р & \cellcolor[HTML]{8CE4F6} К & А &   &   &   & <4,1,К>
\\ \hline
М & У & Р & О & М &   & М & У & Р & К & \cellcolor[HTML]{8CE4F6} А &   &   &   &   &   & <0,0,А>
\\ \hline
\end{tabular}
\end{table}

\pagebreak
\subsection{Вариант №19}
\paragraph{Задание 3.1}

Закодировать сообщение методом LZ77\\
Строка:ОЛОВО\_ЛОВЕЦ\_ОВЦА\_ЦАП\\
Результат: <0,0,О> <0,0,Л> <8,1,В> <6,1,\_> <5,3,Е> <0,0,Ц> <4,1,О> <0,1,Ц> <0,0,А> <5,1,Ц> <7,1,П>\\
\begin{table}[h!]
\centering
\begin{tabular}{|c|c|c|c|c|c|c|c|c|c|c|c|c|c|c|c|c|} 
\hline
\multicolumn{10}{|c|}{Cловарь} & \multicolumn{6}{c|}{Буфер} & Код  \\ \hline
  &   &   &   &   &   &   &   &   &   & \cellcolor[HTML]{8CE4F6} О & Л & О & В & О &   & <0,0,О>
\\ \hline
  &   &   &   &   &   &   &   &   & О & \cellcolor[HTML]{8CE4F6} Л & О & В & О &   & Л & <0,0,Л>
\\ \hline
  &   &   &   &   &   &   &   & \cellcolor[HTML]{FFFF00} О & Л & \cellcolor[HTML]{FFFF00} О & \cellcolor[HTML]{8CE4F6} В & О &   & Л & О & <8,1,В>
\\ \hline
  &   &   &   &   &   & \cellcolor[HTML]{FFFF00} О & Л & О & В & \cellcolor[HTML]{FFFF00} О & \cellcolor[HTML]{8CE4F6}   & Л & О & В & Е & <6,1,\_>
\\ \hline
  &   &   &   & О & \cellcolor[HTML]{FFFF00} Л & \cellcolor[HTML]{FFFF00} О & \cellcolor[HTML]{FFFF00} В & О &   & \cellcolor[HTML]{FFFF00} Л & \cellcolor[HTML]{FFFF00} О & \cellcolor[HTML]{FFFF00} В & \cellcolor[HTML]{8CE4F6} Е & Ц &   & <5,3,Е>
\\ \hline
О & Л & О & В & О &   & Л & О & В & Е & \cellcolor[HTML]{8CE4F6} Ц &   & О & В & Ц & А & <0,0,Ц>
\\ \hline
Л & О & В & О & \cellcolor[HTML]{FFFF00}   & Л & О & В & Е & Ц & \cellcolor[HTML]{FFFF00}   & \cellcolor[HTML]{8CE4F6} О & В & Ц & А &   & <4,1,О>
\\ \hline
\cellcolor[HTML]{FFFF00} В & О &   & Л & О & В & Е & Ц &   & О & \cellcolor[HTML]{FFFF00} В & \cellcolor[HTML]{8CE4F6} Ц & А &   & Ц & А & <0,1,Ц>
\\ \hline
  & Л & О & В & Е & Ц &   & О & В & Ц & \cellcolor[HTML]{8CE4F6} А &   & Ц & А & П &   & <0,0,А>
\\ \hline
Л & О & В & Е & Ц & \cellcolor[HTML]{FFFF00}   & О & В & Ц & А & \cellcolor[HTML]{FFFF00}   & \cellcolor[HTML]{8CE4F6} Ц & А & П &   &   & <5,1,Ц>
\\ \hline
В & Е & Ц &   & О & В & Ц & \cellcolor[HTML]{FFFF00} А &   & Ц & \cellcolor[HTML]{FFFF00} А & \cellcolor[HTML]{8CE4F6} П &   &   &   &   & <7,1,П>
\\ \hline
\end{tabular}
\end{table}

\pagebreak
\subsection{Вариант №20}

\pagebreak
\subsection{Вариант №21}
\paragraph{Задание 3.1}

Закодировать сообщение методом LZ77\\
Строка:ЛОДКА\_ЛОДОЧКА\_ОЧКИ\\
Результат: <0,0,Л> <0,0,О> <0,0,Д> <0,0,К> <0,0,А> <0,0,\_> <4,3,О> <0,0,Ч> <2,3,О> <5,2,И>\\
\begin{table}[h!]
\centering
\begin{tabular}{|c|c|c|c|c|c|c|c|c|c|c|c|c|c|c|c|c|} 
\hline
\multicolumn{10}{|c|}{Cловарь} & \multicolumn{6}{c|}{Буфер} & Код  \\ \hline
  &   &   &   &   &   &   &   &   &   & \cellcolor[HTML]{8CE4F6} Л & О & Д & К & А &   & <0,0,Л>
\\ \hline
  &   &   &   &   &   &   &   &   & Л & \cellcolor[HTML]{8CE4F6} О & Д & К & А &   & Л & <0,0,О>
\\ \hline
  &   &   &   &   &   &   &   & Л & О & \cellcolor[HTML]{8CE4F6} Д & К & А &   & Л & О & <0,0,Д>
\\ \hline
  &   &   &   &   &   &   & Л & О & Д & \cellcolor[HTML]{8CE4F6} К & А &   & Л & О & Д & <0,0,К>
\\ \hline
  &   &   &   &   &   & Л & О & Д & К & \cellcolor[HTML]{8CE4F6} А &   & Л & О & Д & О & <0,0,А>
\\ \hline
  &   &   &   &   & Л & О & Д & К & А & \cellcolor[HTML]{8CE4F6}   & Л & О & Д & О & Ч & <0,0,\_>
\\ \hline
  &   &   &   & \cellcolor[HTML]{FFFF00} Л & \cellcolor[HTML]{FFFF00} О & \cellcolor[HTML]{FFFF00} Д & К & А &   & \cellcolor[HTML]{FFFF00} Л & \cellcolor[HTML]{FFFF00} О & \cellcolor[HTML]{FFFF00} Д & \cellcolor[HTML]{8CE4F6} О & Ч & К & <4,3,О>
\\ \hline
Л & О & Д & К & А &   & Л & О & Д & О & \cellcolor[HTML]{8CE4F6} Ч & К & А &   & О & Ч & <0,0,Ч>
\\ \hline
О & Д & \cellcolor[HTML]{FFFF00} К & \cellcolor[HTML]{FFFF00} А & \cellcolor[HTML]{FFFF00}   & Л & О & Д & О & Ч & \cellcolor[HTML]{FFFF00} К & \cellcolor[HTML]{FFFF00} А & \cellcolor[HTML]{FFFF00}   & \cellcolor[HTML]{8CE4F6} О & Ч & К & <2,3,О>
\\ \hline
  & Л & О & Д & О & \cellcolor[HTML]{FFFF00} Ч & \cellcolor[HTML]{FFFF00} К & А &   & О & \cellcolor[HTML]{FFFF00} Ч & \cellcolor[HTML]{FFFF00} К & \cellcolor[HTML]{8CE4F6} И &   &   &   & <5,2,И>
\\ \hline
\end{tabular}
\end{table}

\pagebreak
\subsection{Вариант №22}
\paragraph{Задание 3.1}

Закодировать сообщение методом LZ77\\
Строка:КЛУБ\_КЛУБОК\_КЛУБНИ\\
Результат: <0,0,К> <0,0,Л> <0,0,У> <0,0,Б> <0,0,\_> <5,4,О> <0,1,\_> <3,4,Н> <0,0,И>\\
\begin{table}[h!]
\centering
\begin{tabular}{|c|c|c|c|c|c|c|c|c|c|c|c|c|c|c|c|c|} 
\hline
\multicolumn{10}{|c|}{Cловарь} & \multicolumn{6}{c|}{Буфер} & Код  \\ \hline
  &   &   &   &   &   &   &   &   &   & \cellcolor[HTML]{8CE4F6} К & Л & У & Б &   & К & <0,0,К>
\\ \hline
  &   &   &   &   &   &   &   &   & К & \cellcolor[HTML]{8CE4F6} Л & У & Б &   & К & Л & <0,0,Л>
\\ \hline
  &   &   &   &   &   &   &   & К & Л & \cellcolor[HTML]{8CE4F6} У & Б &   & К & Л & У & <0,0,У>
\\ \hline
  &   &   &   &   &   &   & К & Л & У & \cellcolor[HTML]{8CE4F6} Б &   & К & Л & У & Б & <0,0,Б>
\\ \hline
  &   &   &   &   &   & К & Л & У & Б & \cellcolor[HTML]{8CE4F6}   & К & Л & У & Б & О & <0,0,\_>
\\ \hline
  &   &   &   &   & \cellcolor[HTML]{FFFF00} К & \cellcolor[HTML]{FFFF00} Л & \cellcolor[HTML]{FFFF00} У & \cellcolor[HTML]{FFFF00} Б &   & \cellcolor[HTML]{FFFF00} К & \cellcolor[HTML]{FFFF00} Л & \cellcolor[HTML]{FFFF00} У & \cellcolor[HTML]{FFFF00} Б & \cellcolor[HTML]{8CE4F6} О & К & <5,4,О>
\\ \hline
\cellcolor[HTML]{FFFF00} К & Л & У & Б &   & К & Л & У & Б & О & \cellcolor[HTML]{FFFF00} К & \cellcolor[HTML]{8CE4F6}   & К & Л & У & Б & <0,1,\_>
\\ \hline
У & Б &   & \cellcolor[HTML]{FFFF00} К & \cellcolor[HTML]{FFFF00} Л & \cellcolor[HTML]{FFFF00} У & \cellcolor[HTML]{FFFF00} Б & О & К &   & \cellcolor[HTML]{FFFF00} К & \cellcolor[HTML]{FFFF00} Л & \cellcolor[HTML]{FFFF00} У & \cellcolor[HTML]{FFFF00} Б & \cellcolor[HTML]{8CE4F6} Н & И & <3,4,Н>
\\ \hline
У & Б & О & К &   & К & Л & У & Б & Н & \cellcolor[HTML]{8CE4F6} И &   &   &   &   &   & <0,0,И>
\\ \hline
\end{tabular}
\end{table}

\pagebreak
\subsection{Вариант №23}
\paragraph{Задание 3.1}

Закодировать сообщение методом LZ77\\
Строка:БОЛОТО\_БОЛТ\_БОЛЬ\_ОЛЯ\\
Результат: <0,0,Б> <0,0,О> <0,0,Л> <8,1,Т> <6,1,\_> <3,3,Т> <5,4,Ь> <0,1,О> <1,1,Я>\\
\begin{table}[h!]
\centering
\begin{tabular}{|c|c|c|c|c|c|c|c|c|c|c|c|c|c|c|c|c|} 
\hline
\multicolumn{10}{|c|}{Cловарь} & \multicolumn{6}{c|}{Буфер} & Код  \\ \hline
  &   &   &   &   &   &   &   &   &   & \cellcolor[HTML]{8CE4F6} Б & О & Л & О & Т & О & <0,0,Б>
\\ \hline
  &   &   &   &   &   &   &   &   & Б & \cellcolor[HTML]{8CE4F6} О & Л & О & Т & О &   & <0,0,О>
\\ \hline
  &   &   &   &   &   &   &   & Б & О & \cellcolor[HTML]{8CE4F6} Л & О & Т & О &   & Б & <0,0,Л>
\\ \hline
  &   &   &   &   &   &   & Б & \cellcolor[HTML]{FFFF00} О & Л & \cellcolor[HTML]{FFFF00} О & \cellcolor[HTML]{8CE4F6} Т & О &   & Б & О & <8,1,Т>
\\ \hline
  &   &   &   &   & Б & \cellcolor[HTML]{FFFF00} О & Л & О & Т & \cellcolor[HTML]{FFFF00} О & \cellcolor[HTML]{8CE4F6}   & Б & О & Л & Т & <6,1,\_>
\\ \hline
  &   &   & \cellcolor[HTML]{FFFF00} Б & \cellcolor[HTML]{FFFF00} О & \cellcolor[HTML]{FFFF00} Л & О & Т & О &   & \cellcolor[HTML]{FFFF00} Б & \cellcolor[HTML]{FFFF00} О & \cellcolor[HTML]{FFFF00} Л & \cellcolor[HTML]{8CE4F6} Т &   & Б & <3,3,Т>
\\ \hline
О & Л & О & Т & О & \cellcolor[HTML]{FFFF00}   & \cellcolor[HTML]{FFFF00} Б & \cellcolor[HTML]{FFFF00} О & \cellcolor[HTML]{FFFF00} Л & Т & \cellcolor[HTML]{FFFF00}   & \cellcolor[HTML]{FFFF00} Б & \cellcolor[HTML]{FFFF00} О & \cellcolor[HTML]{FFFF00} Л & \cellcolor[HTML]{8CE4F6} Ь &   & <5,4,Ь>
\\ \hline
\cellcolor[HTML]{FFFF00}   & Б & О & Л & Т &   & Б & О & Л & Ь & \cellcolor[HTML]{FFFF00}   & \cellcolor[HTML]{8CE4F6} О & Л & Я &   &   & <0,1,О>
\\ \hline
О & \cellcolor[HTML]{FFFF00} Л & Т &   & Б & О & Л & Ь &   & О & \cellcolor[HTML]{FFFF00} Л & \cellcolor[HTML]{8CE4F6} Я &   &   &   &   & <1,1,Я>
\\ \hline
\end{tabular}
\end{table}

\pagebreak
\subsection{Вариант №24}
\paragraph{Задание 3.1}

Закодировать сообщение методом LZ77\\
Строка:ЛАПКИ\_ЛАПЫ\_ЛАПИТАЛЬ\\
Результат: <0,0,Л> <0,0,А> <0,0,П> <0,0,К> <0,0,И> <0,0,\_> <4,3,Ы> <5,4,И> <0,0,Т> <1,1,Л> <0,0,Ь>\\
\begin{table}[h!]
\centering
\begin{tabular}{|c|c|c|c|c|c|c|c|c|c|c|c|c|c|c|c|c|} 
\hline
\multicolumn{10}{|c|}{Cловарь} & \multicolumn{6}{c|}{Буфер} & Код  \\ \hline
  &   &   &   &   &   &   &   &   &   & \cellcolor[HTML]{8CE4F6} Л & А & П & К & И &   & <0,0,Л>
\\ \hline
  &   &   &   &   &   &   &   &   & Л & \cellcolor[HTML]{8CE4F6} А & П & К & И &   & Л & <0,0,А>
\\ \hline
  &   &   &   &   &   &   &   & Л & А & \cellcolor[HTML]{8CE4F6} П & К & И &   & Л & А & <0,0,П>
\\ \hline
  &   &   &   &   &   &   & Л & А & П & \cellcolor[HTML]{8CE4F6} К & И &   & Л & А & П & <0,0,К>
\\ \hline
  &   &   &   &   &   & Л & А & П & К & \cellcolor[HTML]{8CE4F6} И &   & Л & А & П & Ы & <0,0,И>
\\ \hline
  &   &   &   &   & Л & А & П & К & И & \cellcolor[HTML]{8CE4F6}   & Л & А & П & Ы &   & <0,0,\_>
\\ \hline
  &   &   &   & \cellcolor[HTML]{FFFF00} Л & \cellcolor[HTML]{FFFF00} А & \cellcolor[HTML]{FFFF00} П & К & И &   & \cellcolor[HTML]{FFFF00} Л & \cellcolor[HTML]{FFFF00} А & \cellcolor[HTML]{FFFF00} П & \cellcolor[HTML]{8CE4F6} Ы &   & Л & <4,3,Ы>
\\ \hline
Л & А & П & К & И & \cellcolor[HTML]{FFFF00}   & \cellcolor[HTML]{FFFF00} Л & \cellcolor[HTML]{FFFF00} А & \cellcolor[HTML]{FFFF00} П & Ы & \cellcolor[HTML]{FFFF00}   & \cellcolor[HTML]{FFFF00} Л & \cellcolor[HTML]{FFFF00} А & \cellcolor[HTML]{FFFF00} П & \cellcolor[HTML]{8CE4F6} И & Т & <5,4,И>
\\ \hline
  & Л & А & П & Ы &   & Л & А & П & И & \cellcolor[HTML]{8CE4F6} Т & А & Л & Ь &   &   & <0,0,Т>
\\ \hline
Л & \cellcolor[HTML]{FFFF00} А & П & Ы &   & Л & А & П & И & Т & \cellcolor[HTML]{FFFF00} А & \cellcolor[HTML]{8CE4F6} Л & Ь &   &   &   & <1,1,Л>
\\ \hline
П & Ы &   & Л & А & П & И & Т & А & Л & \cellcolor[HTML]{8CE4F6} Ь &   &   &   &   &   & <0,0,Ь>
\\ \hline
\end{tabular}
\end{table}

\pagebreak
\subsection{Вариант №25}

\pagebreak
\subsection{Вариант №26}
\paragraph{Задание 3.1}

Закодировать сообщение методом LZ77\\
Строка:ДОДО\_ДОМ\_ДОМИК\_МИГ\\
Результат: <0,0,Д> <0,0,О> <8,2,\_> <5,2,М> <6,4,И> <0,0,К> <0,1,М> <6,1,Г>\\
\begin{table}[h!]
\centering
\begin{tabular}{|c|c|c|c|c|c|c|c|c|c|c|c|c|c|c|c|c|} 
\hline
\multicolumn{10}{|c|}{Cловарь} & \multicolumn{6}{c|}{Буфер} & Код  \\ \hline
  &   &   &   &   &   &   &   &   &   & \cellcolor[HTML]{8CE4F6} Д & О & Д & О &   & Д & <0,0,Д>
\\ \hline
  &   &   &   &   &   &   &   &   & Д & \cellcolor[HTML]{8CE4F6} О & Д & О &   & Д & О & <0,0,О>
\\ \hline
  &   &   &   &   &   &   &   & \cellcolor[HTML]{FFFF00} Д & \cellcolor[HTML]{FFFF00} О & \cellcolor[HTML]{FFFF00} Д & \cellcolor[HTML]{FFFF00} О & \cellcolor[HTML]{8CE4F6}   & Д & О & М & <8,2,\_>
\\ \hline
  &   &   &   &   & \cellcolor[HTML]{FFFF00} Д & \cellcolor[HTML]{FFFF00} О & Д & О &   & \cellcolor[HTML]{FFFF00} Д & \cellcolor[HTML]{FFFF00} О & \cellcolor[HTML]{8CE4F6} М &   & Д & О & <5,2,М>
\\ \hline
  &   & Д & О & Д & О & \cellcolor[HTML]{FFFF00}   & \cellcolor[HTML]{FFFF00} Д & \cellcolor[HTML]{FFFF00} О & \cellcolor[HTML]{FFFF00} М & \cellcolor[HTML]{FFFF00}   & \cellcolor[HTML]{FFFF00} Д & \cellcolor[HTML]{FFFF00} О & \cellcolor[HTML]{FFFF00} М & \cellcolor[HTML]{8CE4F6} И & К & <6,4,И>
\\ \hline
О &   & Д & О & М &   & Д & О & М & И & \cellcolor[HTML]{8CE4F6} К &   & М & И & Г &   & <0,0,К>
\\ \hline
\cellcolor[HTML]{FFFF00}   & Д & О & М &   & Д & О & М & И & К & \cellcolor[HTML]{FFFF00}   & \cellcolor[HTML]{8CE4F6} М & И & Г &   &   & <0,1,М>
\\ \hline
О & М &   & Д & О & М & \cellcolor[HTML]{FFFF00} И & К &   & М & \cellcolor[HTML]{FFFF00} И & \cellcolor[HTML]{8CE4F6} Г &   &   &   &   & <6,1,Г>
\\ \hline
\end{tabular}
\end{table}

\pagebreak
\subsection{Вариант №27}
\paragraph{Задание 3.1}

Закодировать сообщение методом LZ77\\
Строка:ЗИГЗАГ\_ЗАЗОР\_ЗОРКИЙ\\
Результат: <0,0,З> <0,0,И> <0,0,Г> <7,1,А> <7,1,\_> <6,2,З> <0,0,О> <0,0,Р> <4,2,О> <6,1,К> <0,0,И> <0,0,Й>\\
\begin{table}[h!]
\centering
\begin{tabular}{|c|c|c|c|c|c|c|c|c|c|c|c|c|c|c|c|c|} 
\hline
\multicolumn{10}{|c|}{Cловарь} & \multicolumn{6}{c|}{Буфер} & Код  \\ \hline
  &   &   &   &   &   &   &   &   &   & \cellcolor[HTML]{8CE4F6} З & И & Г & З & А & Г & <0,0,З>
\\ \hline
  &   &   &   &   &   &   &   &   & З & \cellcolor[HTML]{8CE4F6} И & Г & З & А & Г &   & <0,0,И>
\\ \hline
  &   &   &   &   &   &   &   & З & И & \cellcolor[HTML]{8CE4F6} Г & З & А & Г &   & З & <0,0,Г>
\\ \hline
  &   &   &   &   &   &   & \cellcolor[HTML]{FFFF00} З & И & Г & \cellcolor[HTML]{FFFF00} З & \cellcolor[HTML]{8CE4F6} А & Г &   & З & А & <7,1,А>
\\ \hline
  &   &   &   &   & З & И & \cellcolor[HTML]{FFFF00} Г & З & А & \cellcolor[HTML]{FFFF00} Г & \cellcolor[HTML]{8CE4F6}   & З & А & З & О & <7,1,\_>
\\ \hline
  &   &   & З & И & Г & \cellcolor[HTML]{FFFF00} З & \cellcolor[HTML]{FFFF00} А & Г &   & \cellcolor[HTML]{FFFF00} З & \cellcolor[HTML]{FFFF00} А & \cellcolor[HTML]{8CE4F6} З & О & Р &   & <6,2,З>
\\ \hline
З & И & Г & З & А & Г &   & З & А & З & \cellcolor[HTML]{8CE4F6} О & Р &   & З & О & Р & <0,0,О>
\\ \hline
И & Г & З & А & Г &   & З & А & З & О & \cellcolor[HTML]{8CE4F6} Р &   & З & О & Р & К & <0,0,Р>
\\ \hline
Г & З & А & Г & \cellcolor[HTML]{FFFF00}   & \cellcolor[HTML]{FFFF00} З & А & З & О & Р & \cellcolor[HTML]{FFFF00}   & \cellcolor[HTML]{FFFF00} З & \cellcolor[HTML]{8CE4F6} О & Р & К & И & <4,2,О>
\\ \hline
Г &   & З & А & З & О & \cellcolor[HTML]{FFFF00} Р &   & З & О & \cellcolor[HTML]{FFFF00} Р & \cellcolor[HTML]{8CE4F6} К & И & Й &   &   & <6,1,К>
\\ \hline
З & А & З & О & Р &   & З & О & Р & К & \cellcolor[HTML]{8CE4F6} И & Й &   &   &   &   & <0,0,И>
\\ \hline
А & З & О & Р &   & З & О & Р & К & И & \cellcolor[HTML]{8CE4F6} Й &   &   &   &   &   & <0,0,Й>
\\ \hline
\end{tabular}
\end{table}

\pagebreak
\subsection{Вариант №28}
\paragraph{Задание 3.1}

Закодировать сообщение методом LZ77\\
Строка:ТИКТАК\_ТИК\_ТАК\_ТАКСА\\
Результат: <0,0,Т> <0,0,И> <0,0,К> <7,1,А> <7,1,\_> <3,3,\_> <2,5,А> <2,1,С> <0,0,А>\\
\begin{table}[h!]
\centering
\begin{tabular}{|c|c|c|c|c|c|c|c|c|c|c|c|c|c|c|c|c|} 
\hline
\multicolumn{10}{|c|}{Cловарь} & \multicolumn{6}{c|}{Буфер} & Код  \\ \hline
  &   &   &   &   &   &   &   &   &   & \cellcolor[HTML]{8CE4F6} Т & И & К & Т & А & К & <0,0,Т>
\\ \hline
  &   &   &   &   &   &   &   &   & Т & \cellcolor[HTML]{8CE4F6} И & К & Т & А & К &   & <0,0,И>
\\ \hline
  &   &   &   &   &   &   &   & Т & И & \cellcolor[HTML]{8CE4F6} К & Т & А & К &   & Т & <0,0,К>
\\ \hline
  &   &   &   &   &   &   & \cellcolor[HTML]{FFFF00} Т & И & К & \cellcolor[HTML]{FFFF00} Т & \cellcolor[HTML]{8CE4F6} А & К &   & Т & И & <7,1,А>
\\ \hline
  &   &   &   &   & Т & И & \cellcolor[HTML]{FFFF00} К & Т & А & \cellcolor[HTML]{FFFF00} К & \cellcolor[HTML]{8CE4F6}   & Т & И & К &   & <7,1,\_>
\\ \hline
  &   &   & \cellcolor[HTML]{FFFF00} Т & \cellcolor[HTML]{FFFF00} И & \cellcolor[HTML]{FFFF00} К & Т & А & К &   & \cellcolor[HTML]{FFFF00} Т & \cellcolor[HTML]{FFFF00} И & \cellcolor[HTML]{FFFF00} К & \cellcolor[HTML]{8CE4F6}   & Т & А & <3,3,\_>
\\ \hline
И & К & \cellcolor[HTML]{FFFF00} Т & \cellcolor[HTML]{FFFF00} А & \cellcolor[HTML]{FFFF00} К & \cellcolor[HTML]{FFFF00}   & \cellcolor[HTML]{FFFF00} Т & И & К &   & \cellcolor[HTML]{FFFF00} Т & \cellcolor[HTML]{FFFF00} А & \cellcolor[HTML]{FFFF00} К & \cellcolor[HTML]{FFFF00}   & \cellcolor[HTML]{FFFF00} Т & \cellcolor[HTML]{8CE4F6} А & <2,5,А>
\\ \hline
Т & И & \cellcolor[HTML]{FFFF00} К &   & Т & А & К &   & Т & А & \cellcolor[HTML]{FFFF00} К & \cellcolor[HTML]{8CE4F6} С & А &   &   &   & <2,1,С>
\\ \hline
К &   & Т & А & К &   & Т & А & К & С & \cellcolor[HTML]{8CE4F6} А &   &   &   &   &   & <0,0,А>
\\ \hline
\end{tabular}
\end{table}

\pagebreak
\subsection{Вариант №29}
\paragraph{Задание 3.1}

Закодировать сообщение методом LZ77\\
Строка:КУРКУЛЬ\_КУЛЕК\_ЛЕКАЛО\\
Результат: <0,0,К> <0,0,У> <0,0,Р> <7,2,Л> <0,0,Ь> <0,0,\_> <5,3,Е> <1,1,\_> <6,3,А> <2,1,О>\\
\begin{table}[h!]
\centering
\begin{tabular}{|c|c|c|c|c|c|c|c|c|c|c|c|c|c|c|c|c|} 
\hline
\multicolumn{10}{|c|}{Cловарь} & \multicolumn{6}{c|}{Буфер} & Код  \\ \hline
  &   &   &   &   &   &   &   &   &   & \cellcolor[HTML]{8CE4F6} К & У & Р & К & У & Л & <0,0,К>
\\ \hline
  &   &   &   &   &   &   &   &   & К & \cellcolor[HTML]{8CE4F6} У & Р & К & У & Л & Ь & <0,0,У>
\\ \hline
  &   &   &   &   &   &   &   & К & У & \cellcolor[HTML]{8CE4F6} Р & К & У & Л & Ь &   & <0,0,Р>
\\ \hline
  &   &   &   &   &   &   & \cellcolor[HTML]{FFFF00} К & \cellcolor[HTML]{FFFF00} У & Р & \cellcolor[HTML]{FFFF00} К & \cellcolor[HTML]{FFFF00} У & \cellcolor[HTML]{8CE4F6} Л & Ь &   & К & <7,2,Л>
\\ \hline
  &   &   &   & К & У & Р & К & У & Л & \cellcolor[HTML]{8CE4F6} Ь &   & К & У & Л & Е & <0,0,Ь>
\\ \hline
  &   &   & К & У & Р & К & У & Л & Ь & \cellcolor[HTML]{8CE4F6}   & К & У & Л & Е & К & <0,0,\_>
\\ \hline
  &   & К & У & Р & \cellcolor[HTML]{FFFF00} К & \cellcolor[HTML]{FFFF00} У & \cellcolor[HTML]{FFFF00} Л & Ь &   & \cellcolor[HTML]{FFFF00} К & \cellcolor[HTML]{FFFF00} У & \cellcolor[HTML]{FFFF00} Л & \cellcolor[HTML]{8CE4F6} Е & К &   & <5,3,Е>
\\ \hline
Р & \cellcolor[HTML]{FFFF00} К & У & Л & Ь &   & К & У & Л & Е & \cellcolor[HTML]{FFFF00} К & \cellcolor[HTML]{8CE4F6}   & Л & Е & К & А & <1,1,\_>
\\ \hline
У & Л & Ь &   & К & У & \cellcolor[HTML]{FFFF00} Л & \cellcolor[HTML]{FFFF00} Е & \cellcolor[HTML]{FFFF00} К &   & \cellcolor[HTML]{FFFF00} Л & \cellcolor[HTML]{FFFF00} Е & \cellcolor[HTML]{FFFF00} К & \cellcolor[HTML]{8CE4F6} А & Л & О & <6,3,А>
\\ \hline
К & У & \cellcolor[HTML]{FFFF00} Л & Е & К &   & Л & Е & К & А & \cellcolor[HTML]{FFFF00} Л & \cellcolor[HTML]{8CE4F6} О &   &   &   &   & <2,1,О>
\\ \hline
\end{tabular}
\end{table}

\pagebreak
\subsection{Вариант №30}
\paragraph{Задание 3.1}

Закодировать сообщение методом LZ77\\
Строка:СКЛАД\_КЛАД\_КЛАДЕЗЬ\\
Результат: <0,0,С> <0,0,К> <0,0,Л> <0,0,А> <0,0,Д> <0,0,\_> <5,5,К> <0,3,Е> <0,0,З> <0,0,Ь>\\
\begin{table}[h!]
\centering
\begin{tabular}{|c|c|c|c|c|c|c|c|c|c|c|c|c|c|c|c|c|} 
\hline
\multicolumn{10}{|c|}{Cловарь} & \multicolumn{6}{c|}{Буфер} & Код  \\ \hline
  &   &   &   &   &   &   &   &   &   & \cellcolor[HTML]{8CE4F6} С & К & Л & А & Д &   & <0,0,С>
\\ \hline
  &   &   &   &   &   &   &   &   & С & \cellcolor[HTML]{8CE4F6} К & Л & А & Д &   & К & <0,0,К>
\\ \hline
  &   &   &   &   &   &   &   & С & К & \cellcolor[HTML]{8CE4F6} Л & А & Д &   & К & Л & <0,0,Л>
\\ \hline
  &   &   &   &   &   &   & С & К & Л & \cellcolor[HTML]{8CE4F6} А & Д &   & К & Л & А & <0,0,А>
\\ \hline
  &   &   &   &   &   & С & К & Л & А & \cellcolor[HTML]{8CE4F6} Д &   & К & Л & А & Д & <0,0,Д>
\\ \hline
  &   &   &   &   & С & К & Л & А & Д & \cellcolor[HTML]{8CE4F6}   & К & Л & А & Д &   & <0,0,\_>
\\ \hline
  &   &   &   & С & \cellcolor[HTML]{FFFF00} К & \cellcolor[HTML]{FFFF00} Л & \cellcolor[HTML]{FFFF00} А & \cellcolor[HTML]{FFFF00} Д & \cellcolor[HTML]{FFFF00}   & \cellcolor[HTML]{FFFF00} К & \cellcolor[HTML]{FFFF00} Л & \cellcolor[HTML]{FFFF00} А & \cellcolor[HTML]{FFFF00} Д & \cellcolor[HTML]{FFFF00}   & \cellcolor[HTML]{8CE4F6} К & <5,5,К>
\\ \hline
\cellcolor[HTML]{FFFF00} Л & \cellcolor[HTML]{FFFF00} А & \cellcolor[HTML]{FFFF00} Д &   & К & Л & А & Д &   & К & \cellcolor[HTML]{FFFF00} Л & \cellcolor[HTML]{FFFF00} А & \cellcolor[HTML]{FFFF00} Д & \cellcolor[HTML]{8CE4F6} Е & З & Ь & <0,3,Е>
\\ \hline
К & Л & А & Д &   & К & Л & А & Д & Е & \cellcolor[HTML]{8CE4F6} З & Ь &   &   &   &   & <0,0,З>
\\ \hline
Л & А & Д &   & К & Л & А & Д & Е & З & \cellcolor[HTML]{8CE4F6} Ь &   &   &   &   &   & <0,0,Ь>
\\ \hline
\end{tabular}
\end{table}

\pagebreak
\subsection{Вариант №0}
\paragraph{Задание 3.1}

Закодировать сообщение методом LZ77\\
Строка:СКЛАД\_КЛАД\_КЛАДЕЗЬ\\
Результат: <0,0,С> <0,0,К> <0,0,Л> <0,0,А> <0,0,Д> <0,0,\_> <5,5,К> <0,3,Е> <0,0,З> <0,0,Ь>\\
\begin{table}[h!]
\centering
\begin{tabular}{|c|c|c|c|c|c|c|c|c|c|c|c|c|c|c|c|c|} 
\hline
\multicolumn{10}{|c|}{Cловарь} & \multicolumn{6}{c|}{Буфер} & Код  \\ \hline
  &   &   &   &   &   &   &   &   &   & \cellcolor[HTML]{8CE4F6} С & К & Л & А & Д &   & <0,0,С>
\\ \hline
  &   &   &   &   &   &   &   &   & С & \cellcolor[HTML]{8CE4F6} К & Л & А & Д &   & К & <0,0,К>
\\ \hline
  &   &   &   &   &   &   &   & С & К & \cellcolor[HTML]{8CE4F6} Л & А & Д &   & К & Л & <0,0,Л>
\\ \hline
  &   &   &   &   &   &   & С & К & Л & \cellcolor[HTML]{8CE4F6} А & Д &   & К & Л & А & <0,0,А>
\\ \hline
  &   &   &   &   &   & С & К & Л & А & \cellcolor[HTML]{8CE4F6} Д &   & К & Л & А & Д & <0,0,Д>
\\ \hline
  &   &   &   &   & С & К & Л & А & Д & \cellcolor[HTML]{8CE4F6}   & К & Л & А & Д &   & <0,0,\_>
\\ \hline
  &   &   &   & С & \cellcolor[HTML]{FFFF00} К & \cellcolor[HTML]{FFFF00} Л & \cellcolor[HTML]{FFFF00} А & \cellcolor[HTML]{FFFF00} Д & \cellcolor[HTML]{FFFF00}   & \cellcolor[HTML]{FFFF00} К & \cellcolor[HTML]{FFFF00} Л & \cellcolor[HTML]{FFFF00} А & \cellcolor[HTML]{FFFF00} Д & \cellcolor[HTML]{FFFF00}   & \cellcolor[HTML]{8CE4F6} К & <5,5,К>
\\ \hline
\cellcolor[HTML]{FFFF00} Л & \cellcolor[HTML]{FFFF00} А & \cellcolor[HTML]{FFFF00} Д &   & К & Л & А & Д &   & К & \cellcolor[HTML]{FFFF00} Л & \cellcolor[HTML]{FFFF00} А & \cellcolor[HTML]{FFFF00} Д & \cellcolor[HTML]{8CE4F6} Е & З & Ь & <0,3,Е>
\\ \hline
К & Л & А & Д &   & К & Л & А & Д & Е & \cellcolor[HTML]{8CE4F6} З & Ь &   &   &   &   & <0,0,З>
\\ \hline
Л & А & Д &   & К & Л & А & Д & Е & З & \cellcolor[HTML]{8CE4F6} Ь &   &   &   &   &   & <0,0,Ь>
\\ \hline
\end{tabular}
\end{table}

\pagebreak
\end{document}