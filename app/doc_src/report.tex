
\documentclass[a4paper, 12pt]{article}
\usepackage[left=3cm,right=1.5cm,top=2cm,bottom=2cm]{geometry}
\usepackage[T2A]{fontenc}
\usepackage{graphicx}

%Hyphenation rules
%--------------------------------------
\usepackage{hyphenat}
\hyphenation{ма-те-ма-ти-ка вос-ста-нав-ли-вать}
%--------------------------------------
\usepackage[english, russian]{babel}
\begin{document}
 
\tableofcontents

\begin{abstract}
  Это вводный абзац в начале документа.
\end{abstract}
 
\section{Задание}
\begin{enumerate}
\item Составить таблицу кодов блоков для метода Хаффмана с блокированием. Вероятности букв считать по фрагменту сообщения в задании. Длина блока указана. Вычислить EX, ML(X), ML(Xбл). Здесь EX – энтропия алфавита из букв сообщения, ML(X) – среднее количество элементарных символов на букву при сжатии методом Хаффмана, ML(Xбл) – среднее количество элементарных символов на букву при сжатии методом Хаффмана с блокированием. 
\item Сжать сообщение адаптивным методом Хаффмана. 
\item Сжать сообщение методами LZ77, LZSS, LZ78  Для методов LZ77, LZSS размер словаря – 10 символов, буфера – 6 символов. Для метода LZ78 размер словаря 32 записи. 
\item Сжать сообщение из задания №2 арифметическим методом. 
\item Распаковать сообщения, сжатые адаптивным методом Хаффмана, методами LZ77, LZSS, LZ78 и арифметическим методом. Для методов LZ77, LZSS размер словаря – 10 символов. Для метода LZ78 размер словаря – 16 записей. При декодировании таблица состоит из следующих столбцов: «Код», «Словарь» и «Выходной поток».
\end{enumerate}
\pagebreak
\section{Решение}
\subsection{Вариант №9}
\paragraph{Задание 1}

Строка СОКККККООО, размер блока: 2
\begin{center}
 \begin{tabular}{ |c|c|l| } 
  \hline
     Буква & Вероятность & Код\\ \hline
К & 0.50 & 0\\\hline
О & 0.40 & 11\\\hline
С & 0.10 & 10
\\ \hline \end{tabular}
\end{center}
Энтропия алфавита: 1.36
\begin{center}
 \begin{tabular}{ |c|c|l| } 
  \hline
     Блок & Вероятность & Код\\ \hline
КК & 0.25 & 10\\\hline
КО & 0.20 & 00\\\hline
ОК & 0.20 & 01\\\hline
ОО & 0.16 & 110\\\hline
КС & 0.05 & 11101\\\hline
СК & 0.05 & 11110\\\hline
ОС & 0.04 & 111111\\\hline
СО & 0.04 & 11100\\\hline
СС & 0.01 & 111110
\\ \hline \end{tabular}
\end{center}
Бит на символ при посимвольном кодировании: 1.50, при блочном: 1.39

\includegraphics[width=0.5\linewidth]{/home/fizlrock/data/files/backup/code_backup/hobby/algoritms/LabExecutor/app/./doc_src/images/1945050250.jpg}

\includegraphics[width=0.9\linewidth]{/home/fizlrock/data/files/backup/code_backup/hobby/algoritms/LabExecutor/app/./doc_src/images/1265374033.jpg}
\paragraph{Задание 4}


Исходная строка: РОРНРПОООО

\begin{center}
 \begin{tabular}{ |c|c| } 
  \hline
     Буква & Вероятность \\ \hline
О & 0.50\\\hline
Р & 0.30\\\hline
Н & 0.10\\\hline
П & 0.10
\\ \hline \end{tabular}
\end{center}
\begin{center}
 \begin{tabular}{ |c|c|c| } 
  \hline
     Буква & Начало & Конец \\ \hline
О & 0.00 & 0.50\\\hline
Р & 0.50 & 0.80\\\hline
Н & 0.80 & 0.90\\\hline
П & 0.90 & 1.00
\\ \hline \end{tabular}
\end{center}
\begin{center}
 \begin{tabular}{ |c|c|c|c| } 
  \hline
     Буква & delta & min & max \\ \hline
Р & 1.00000 & 0.500000 & 0.800000\\\hline
О & 0.300000 & 0.500000 & 0.650000\\\hline
Р & 0.150000 & 0.575000 & 0.620000\\\hline
Н & 0.0450000 & 0.611000 & 0.615500\\\hline
Р & 0.00450000 & 0.613250 & 0.614600\\\hline
П & 0.00135000 & 0.614465 & 0.614600\\\hline
О & 0.000135000 & 0.614465 & 0.614532\\\hline
О & 6.75000E-05 & 0.614465 & 0.614499\\\hline
О & 3.37500E-05 & 0.614465 & 0.614482\\\hline
О & 1.68750E-05 & 0.614465 & 0.614473
\\ \hline \end{tabular}
\end{center}
Результат: 6144699999999996
\end{document}