
\documentclass[a4paper, 12pt]{article}
\usepackage[left=3cm,right=1.5cm,top=2cm,bottom=2cm]{geometry}
\usepackage[T2A]{fontenc}
\usepackage{graphicx}
\usepackage[table]{xcolor}

%Hyphenation rules
%--------------------------------------
\usepackage{hyphenat}
\hyphenation{ма-те-ма-ти-ка вос-ста-нав-ли-вать}
%--------------------------------------
\usepackage[english, russian]{babel}
\begin{document}
 
\tableofcontents

\pagebreak

\begin{abstract}
  Это вводный абзац в начале документа.
\end{abstract}
 
\section{Задание}
\begin{enumerate}
\item Составить таблицу кодов блоков для метода Хаффмана с блокированием. Вероятности букв считать по фрагменту сообщения в задании. Длина блока указана. Вычислить EX, ML(X), ML(Xбл). Здесь EX – энтропия алфавита из букв сообщения, ML(X) – среднее количество элементарных символов на букву при сжатии методом Хаффмана, ML(Xбл) – среднее количество элементарных символов на букву при сжатии методом Хаффмана с блокированием. 
\item Сжать сообщение адаптивным методом Хаффмана. 
\item Сжать сообщение методами LZ77, LZSS, LZ78  Для методов LZ77, LZSS размер словаря – 10 символов, буфера – 6 символов. Для метода LZ78 размер словаря 32 записи. 
\item Сжать сообщение из задания №2 арифметическим методом. 
\item Распаковать сообщения, сжатые адаптивным методом Хаффмана, методами LZ77, LZSS, LZ78 и арифметическим методом. Для методов LZ77, LZSS размер словаря – 10 символов. Для метода LZ78 размер словаря – 16 записей. При декодировании таблица состоит из следующих столбцов: «Код», «Словарь» и «Выходной поток».
\end{enumerate}
\pagebreak
\section{Решение}
\subsection{Вариант №1}
\paragraph{Задание 1. Блочный хаффман \\}

Строка ББААББББББ, размер блока: 3
\begin{center}
 \begin{tabular}{ |c|c|l| } 
  \hline
     Буква & Вероятность & Код\\ \hline
Б & 0.80 & 1\\\hline
А & 0.20 & 0
\\ \hline \end{tabular}
\end{center}
Энтропия алфавита: 0.7219
\begin{center}
 \begin{tabular}{ |c|c|l| } 
  \hline
     Блок & Вероятность & Код\\ \hline
БББ & 0.51 & 1\\\hline
БАБ & 0.13 & 001\\\hline
АББ & 0.13 & 010\\\hline
ББА & 0.13 & 011\\\hline
АБА & 0.03 & 00001\\\hline
ААБ & 0.03 & 00010\\\hline
БАА & 0.03 & 00011\\\hline
ААА & 0.01 & 00000
\\ \hline \end{tabular}
\end{center}
Бит на символ при посимвольном кодировании: 1.0000, при блочном: 0.7280

\includegraphics[width=0.5\linewidth]{/home/fizlrock/data/files/backup/code_backup/hobby/algoritms/LabExecutor/app/./doc_src/images/43809709.jpg}

\includegraphics[width=0.9\linewidth]{/home/fizlrock/data/files/backup/code_backup/hobby/algoritms/LabExecutor/app/./doc_src/images/918409523.jpg}
\pagebreak
\paragraph{Задание 2. Сжать адаптивным хаффманом\\}

Строка: 
КЕЕЕНООННН\\
Результат: 'К' 0'Е' 01 1 00'Н' 000'О' 0101 1111 111 10

\includegraphics[width=0.8\linewidth]{/home/fizlrock/data/files/backup/code_backup/hobby/algoritms/LabExecutor/app/./doc_src/images/2023502411.jpg}

\includegraphics[width=0.8\linewidth]{/home/fizlrock/data/files/backup/code_backup/hobby/algoritms/LabExecutor/app/./doc_src/images/494551219.jpg}

\includegraphics[width=0.8\linewidth]{/home/fizlrock/data/files/backup/code_backup/hobby/algoritms/LabExecutor/app/./doc_src/images/611255853.jpg}

\includegraphics[width=0.8\linewidth]{/home/fizlrock/data/files/backup/code_backup/hobby/algoritms/LabExecutor/app/./doc_src/images/1882467246.jpg}

\includegraphics[width=0.8\linewidth]{/home/fizlrock/data/files/backup/code_backup/hobby/algoritms/LabExecutor/app/./doc_src/images/1541961967.jpg}

\includegraphics[width=0.8\linewidth]{/home/fizlrock/data/files/backup/code_backup/hobby/algoritms/LabExecutor/app/./doc_src/images/761136446.jpg}

\includegraphics[width=0.8\linewidth]{/home/fizlrock/data/files/backup/code_backup/hobby/algoritms/LabExecutor/app/./doc_src/images/1432296739.jpg}

\includegraphics[width=0.8\linewidth]{/home/fizlrock/data/files/backup/code_backup/hobby/algoritms/LabExecutor/app/./doc_src/images/1770294668.jpg}

\includegraphics[width=0.8\linewidth]{/home/fizlrock/data/files/backup/code_backup/hobby/algoritms/LabExecutor/app/./doc_src/images/1083671640.jpg}

\includegraphics[width=0.8\linewidth]{/home/fizlrock/data/files/backup/code_backup/hobby/algoritms/LabExecutor/app/./doc_src/images/1314094524.jpg}
\pagebreak
\paragraph{Задание 3.1}

Закодировать сообщение методом LZ77\\
Строка:КУКУКУ\_КУКУШКА\_КУКИШ\\
Результат: <0,0,К> <0,0,У> <8,4,\_> <3,4,Ш> <0,1,А> <2,4,И> <0,0,Ш>\\
\begin{table}[h!]
\centering
\begin{tabular}{|c|c|c|c|c|c|c|c|c|c|c|c|c|c|c|c|c|} 
\hline
\multicolumn{10}{|c|}{Cловарь} & \multicolumn{6}{c|}{Буфер} & Код  \\ \hline
  &   &   &   &   &   &   &   &   &   & \cellcolor[HTML]{8CE4F6} К & У & К & У & К & У & <0,0,К>
\\ \hline
  &   &   &   &   &   &   &   &   & К & \cellcolor[HTML]{8CE4F6} У & К & У & К & У &   & <0,0,У>
\\ \hline
  &   &   &   &   &   &   &   & \cellcolor[HTML]{FFFF00} К & \cellcolor[HTML]{FFFF00} У & \cellcolor[HTML]{FFFF00} К & \cellcolor[HTML]{FFFF00} У & \cellcolor[HTML]{FFFF00} К & \cellcolor[HTML]{FFFF00} У & \cellcolor[HTML]{8CE4F6}   & К & <8,4,\_>
\\ \hline
  &   &   & \cellcolor[HTML]{FFFF00} К & \cellcolor[HTML]{FFFF00} У & \cellcolor[HTML]{FFFF00} К & \cellcolor[HTML]{FFFF00} У & К & У &   & \cellcolor[HTML]{FFFF00} К & \cellcolor[HTML]{FFFF00} У & \cellcolor[HTML]{FFFF00} К & \cellcolor[HTML]{FFFF00} У & \cellcolor[HTML]{8CE4F6} Ш & К & <3,4,Ш>
\\ \hline
\cellcolor[HTML]{FFFF00} К & У & К & У &   & К & У & К & У & Ш & \cellcolor[HTML]{FFFF00} К & \cellcolor[HTML]{8CE4F6} А &   & К & У & К & <0,1,А>
\\ \hline
К & У & \cellcolor[HTML]{FFFF00}   & \cellcolor[HTML]{FFFF00} К & \cellcolor[HTML]{FFFF00} У & \cellcolor[HTML]{FFFF00} К & У & Ш & К & А & \cellcolor[HTML]{FFFF00}   & \cellcolor[HTML]{FFFF00} К & \cellcolor[HTML]{FFFF00} У & \cellcolor[HTML]{FFFF00} К & \cellcolor[HTML]{8CE4F6} И & Ш & <2,4,И>
\\ \hline
К & У & Ш & К & А &   & К & У & К & И & \cellcolor[HTML]{8CE4F6} Ш &   &   &   &   &   & <0,0,Ш>
\\ \hline
\end{tabular}
\end{table}

\paragraph{Задание 3.2}

Закодировать сообщение методом LZSS\\
Строка:КУКУКУ\_КУКУШКА\_КУКИШ\\
Результат: 0'К' 0'У' 1<8,2> 1<6,2> 0'\_' 1<3,4> 0'Ш' 1<0,1> 0'А' 1<2,4> 0'И' 1<2,1>\\
\begin{table}[h!]
\centering
\begin{tabular}{|c|c|c|c|c|c|c|c|c|c|c|c|c|c|c|c|c|}
\hline
\multicolumn{10}{|c|}{Cловарь} & \multicolumn{6}{c|}{Буфер} & Код  \\ \hline
  &   &   &   &   &   &   &   &   &   & К & У & К & У & К & У & 0'К'\\ \hline
  &   &   &   &   &   &   &   &   & К & У & К & У & К & У & \_ & 0'У'\\ \hline
  &   &   &   &   &   &   &   & \cellcolor[HTML]{FFFF00} К & \cellcolor[HTML]{FFFF00} У & \cellcolor[HTML]{FFFF00} К & \cellcolor[HTML]{FFFF00} У & К & У & \_ & К & 1<8,2>\\ \hline
  &   &   &   &   &   & \cellcolor[HTML]{FFFF00} К & \cellcolor[HTML]{FFFF00} У & К & У & \cellcolor[HTML]{FFFF00} К & \cellcolor[HTML]{FFFF00} У & \_ & К & У & К & 1<6,2>\\ \hline
  &   &   &   & К & У & К & У & К & У & \_ & К & У & К & У & Ш & 0'\_'\\ \hline
  &   &   & \cellcolor[HTML]{FFFF00} К & \cellcolor[HTML]{FFFF00} У & \cellcolor[HTML]{FFFF00} К & \cellcolor[HTML]{FFFF00} У & К & У & \_ & \cellcolor[HTML]{FFFF00} К & \cellcolor[HTML]{FFFF00} У & \cellcolor[HTML]{FFFF00} К & \cellcolor[HTML]{FFFF00} У & Ш & К & 1<3,4>\\ \hline
У & К & У & К & У & \_ & К & У & К & У & Ш & К & А & \_ & К & У & 0'Ш'\\ \hline
\cellcolor[HTML]{FFFF00} К & У & К & У & \_ & К & У & К & У & Ш & \cellcolor[HTML]{FFFF00} К & А & \_ & К & У & К & 1<0,1>\\ \hline
У & К & У & \_ & К & У & К & У & Ш & К & А & \_ & К & У & К & И & 0'А'\\ \hline
К & У & \cellcolor[HTML]{FFFF00} \_ & \cellcolor[HTML]{FFFF00} К & \cellcolor[HTML]{FFFF00} У & \cellcolor[HTML]{FFFF00} К & У & Ш & К & А & \cellcolor[HTML]{FFFF00} \_ & \cellcolor[HTML]{FFFF00} К & \cellcolor[HTML]{FFFF00} У & \cellcolor[HTML]{FFFF00} К & И & Ш & 1<2,4>\\ \hline
У & К & У & Ш & К & А & \_ & К & У & К & И & Ш &   &   &   &   & 0'И'\\ \hline
К & У & \cellcolor[HTML]{FFFF00} Ш & К & А & \_ & К & У & К & И & \cellcolor[HTML]{FFFF00} Ш &   &   &   &   &   & 1<2,1>\\ \hline
\end{tabular}
\end{table}

\paragraph{Задание 3.3}

Закодировать сообщение методом LZ78\\
Строка:КУКУКУ\_КУКУШКА\_КУКИШ\\
\begin{table}[h!]
\centering
\begin{tabular}{|c|c|c|} 
\hline
 Входная фраза (в словарь) & Код & Позиция словаря \\ \hline

 &  & 0 \\ \hline
К & 0'К' & 1 \\ \hline
У & 0'У' & 2 \\ \hline
КУ & 1'У' & 3 \\ \hline
КУ\_ & 3'\_' & 4 \\ \hline
КУК & 3'К' & 5 \\ \hline
УШ & 2'Ш' & 6 \\ \hline
КА & 1'А' & 7 \\ \hline
\_ & 0'\_' & 8 \\ \hline
КУКИ & 5'И' & 9 \\ \hline
Ш & 0'Ш' & 10 \\ \hline
\end{tabular}
\end{table}

Результат: 0'К' 0'У' 1'У' 3'\_' 3'К' 2'Ш' 1'А' 0'\_' 5'И' 0'Ш'\\
\pagebreak
\paragraph{Задание 4. Арифметическое кодирование\\}

Исходная строка: КЕЕЕНООННН\
\begin{center}
 \begin{tabular}{ |c|c| } 
  \hline
     Буква & Вероятность \\ \hline
Н & 0.40\\\hline
Е & 0.30\\\hline
О & 0.20\\\hline
К & 0.10
\\ \hline \end{tabular}
\end{center}
\begin{center}
 \begin{tabular}{ |c|c|c| } 
  \hline
     Буква & Начало & Конец \\ \hline
Н & 0.00 & 0.40\\\hline
Е & 0.40 & 0.70\\\hline
О & 0.70 & 0.90\\\hline
К & 0.90 & 1.00
\\ \hline \end{tabular}
\end{center}
\begin{center}
 \begin{tabular}{ |c|c|c|c| } 
  \hline
     Буква & delta & min & max \\ \hline
К & 0.1000000000 & 0.9000000000 & 1.0000000000\\\hline
Е & 0.0300000000 & 0.9400000000 & 0.9700000000\\\hline
Е & 0.0090000000 & 0.9520000000 & 0.9610000000\\\hline
Е & 0.0027000000 & 0.9556000000 & 0.9583000000\\\hline
Н & 0.0010800000 & 0.9556000000 & 0.9566800000\\\hline
О & 0.0002160000 & 0.9563560000 & 0.9565720000\\\hline
О & 0.0000432000 & 0.9565072000 & 0.9565504000\\\hline
Н & 0.0000172800 & 0.9565072000 & 0.9565244800\\\hline
Н & 0.0000069120 & 0.9565072000 & 0.9565141120\\\hline
Н & 0.0000027648 & 0.9565072000 & 0.9565099648
\\ \hline \end{tabular}
\end{center}
Результат: 956508
\pagebreak
\paragraph{Задание 5.1}

\\ 

Декодировать сообщение методом адаптивного хаффмана \\
Строка: 
'О'0'Р'00'П'100'Н'11011001001111\\
Результат: ОРПНРПППНН

\includegraphics[width=0.8\linewidth]{/home/fizlrock/data/files/backup/code_backup/hobby/algoritms/LabExecutor/app/./doc_src/images/779085812.jpg}

\includegraphics[width=0.8\linewidth]{/home/fizlrock/data/files/backup/code_backup/hobby/algoritms/LabExecutor/app/./doc_src/images/1780206968.jpg}

\includegraphics[width=0.8\linewidth]{/home/fizlrock/data/files/backup/code_backup/hobby/algoritms/LabExecutor/app/./doc_src/images/712906222.jpg}

\includegraphics[width=0.8\linewidth]{/home/fizlrock/data/files/backup/code_backup/hobby/algoritms/LabExecutor/app/./doc_src/images/992471388.jpg}

\includegraphics[width=0.8\linewidth]{/home/fizlrock/data/files/backup/code_backup/hobby/algoritms/LabExecutor/app/./doc_src/images/570650730.jpg}

\includegraphics[width=0.8\linewidth]{/home/fizlrock/data/files/backup/code_backup/hobby/algoritms/LabExecutor/app/./doc_src/images/604895193.jpg}

\includegraphics[width=0.8\linewidth]{/home/fizlrock/data/files/backup/code_backup/hobby/algoritms/LabExecutor/app/./doc_src/images/532708288.jpg}

\includegraphics[width=0.8\linewidth]{/home/fizlrock/data/files/backup/code_backup/hobby/algoritms/LabExecutor/app/./doc_src/images/747049354.jpg}

\includegraphics[width=0.8\linewidth]{/home/fizlrock/data/files/backup/code_backup/hobby/algoritms/LabExecutor/app/./doc_src/images/675645095.jpg}

\includegraphics[width=0.8\linewidth]{/home/fizlrock/data/files/backup/code_backup/hobby/algoritms/LabExecutor/app/./doc_src/images/320111325.jpg}
\pagebreak
\paragraph{Задание 5.3 Декодировать строку(LZSS)\\}

Исходная строка: [0'л'] [0'о'] [0'т'] [1<8,1>] [0'к'] [0' '] [1<6,4>] [1<0,3>] [1<6,4>][1<0,1>] [0'к']\\
\begin{table}[h!]
\centering
\begin{tabular}{|c|c|c|}
\hline
 Cловарь & Буфер & Код  \\ \hline
0'л' & [ ,  ,  ,  ,  ,  ,  ,  ,  , л] & л
\\ \hline
0'о' & [ ,  ,  ,  ,  ,  ,  ,  , л, о] & о
\\ \hline
0'т' & [ ,  ,  ,  ,  ,  ,  , л, о, т] & т
\\ \hline
1<8,1> & [ ,  ,  ,  ,  ,  , л, о, т, о] & о
\\ \hline
0'к' & [ ,  ,  ,  ,  , л, о, т, о, к] & к
\\ \hline
0' ' & [ ,  ,  ,  , л, о, т, о, к,  ] &  
\\ \hline
1<6,4> & [л, о, т, о, к,  , т, о, к,  ] & ток 
\\ \hline
1<0,3> & [о, к,  , т, о, к,  , л, о, т] & лот
\\ \hline
1<6,4> & [о, к,  , л, о, т,  , л, о, т] &  лот
\\ \hline
1<0,1> & [к,  , л, о, т,  , л, о, т, о] & о
\\ \hline
0'к' & [ , л, о, т,  , л, о, т, о, к] & к
\\ \hline
\end{tabular}
\end{table}

Результат: лоток ток лот лоток
\pagebreak
\paragraph{Задание 5.4 Декодировать строку(LZ78)\\}

Исходная строка: [0'д'] [0'о'] [0'р'] [2'г'] [2' '] [3'о'] [0'г'] [0'а'] [0' '] [7'о'] [3'а'] [9'р'] [2'г']\\
\begin{table}[h!]
\centering
\begin{tabular}{|c|c|c|}
\hline
 Cловарь & Буфер & Код  \\ \hline
 & [] & 
\\ \hline
0'д' & [, д] & д
\\ \hline
0'о' & [, д, о] & о
\\ \hline
0'р' & [, д, о, р] & р
\\ \hline
2'г' & [, д, о, р, ог] & ог
\\ \hline
2' ' & [, д, о, р, ог, о ] & о 
\\ \hline
3'о' & [, д, о, р, ог, о , ро] & ро
\\ \hline
0'г' & [, д, о, р, ог, о , ро, г] & г
\\ \hline
0'а' & [, д, о, р, ог, о , ро, г, а] & а
\\ \hline
0' ' & [, д, о, р, ог, о , ро, г, а,  ] &  
\\ \hline
7'о' & [, д, о, р, ог, о , ро, г, а,  , го] & го
\\ \hline
3'а' & [, д, о, р, ог, о , ро, г, а,  , го, ра] & ра
\\ \hline
9'р' & [, д, о, р, ог, о , ро, г, а,  , го, ра,  р] &  р
\\ \hline
2'г' & [, д, о, р, ог, о , ро, г, а,  , го, ра,  р, ог] & ог
\\ \hline
\end{tabular}
\end{table}

Результат: дорого рога гора рог
\pagebreak
\subsection{Вариант №2}
\paragraph{Задание 1. Блочный хаффман \\}

Строка САСВВАВВВС, размер блока: 2
\begin{center}
 \begin{tabular}{ |c|c|l| } 
  \hline
     Буква & Вероятность & Код\\ \hline
В & 0.50 & 0\\\hline
С & 0.30 & 11\\\hline
А & 0.20 & 10
\\ \hline \end{tabular}
\end{center}
Энтропия алфавита: 1.4855
\begin{center}
 \begin{tabular}{ |c|c|l| } 
  \hline
     Блок & Вероятность & Код\\ \hline
ВВ & 0.25 & 01\\\hline
СВ & 0.15 & 101\\\hline
ВС & 0.15 & 110\\\hline
АВ & 0.10 & 000\\\hline
ВА & 0.10 & 001\\\hline
СС & 0.09 & 1111\\\hline
АС & 0.06 & 1001\\\hline
СА & 0.06 & 1110\\\hline
АА & 0.04 & 1000
\\ \hline \end{tabular}
\end{center}
Бит на символ при посимвольном кодировании: 1.5000, при блочном: 1.5000

\includegraphics[width=0.5\linewidth]{/home/fizlrock/data/files/backup/code_backup/hobby/algoritms/LabExecutor/app/./doc_src/images/1820585941.jpg}

\includegraphics[width=0.9\linewidth]{/home/fizlrock/data/files/backup/code_backup/hobby/algoritms/LabExecutor/app/./doc_src/images/849322140.jpg}
\pagebreak
\paragraph{Задание 2. Сжать адаптивным хаффманом\\}

Строка: 
АББААСКААС\\
Результат: 'А' 0'Б' 01 01 01 00'С' 000'К' 0 0 001

\includegraphics[width=0.8\linewidth]{/home/fizlrock/data/files/backup/code_backup/hobby/algoritms/LabExecutor/app/./doc_src/images/1162644101.jpg}

\includegraphics[width=0.8\linewidth]{/home/fizlrock/data/files/backup/code_backup/hobby/algoritms/LabExecutor/app/./doc_src/images/1250130964.jpg}

\includegraphics[width=0.8\linewidth]{/home/fizlrock/data/files/backup/code_backup/hobby/algoritms/LabExecutor/app/./doc_src/images/220986441.jpg}

\includegraphics[width=0.8\linewidth]{/home/fizlrock/data/files/backup/code_backup/hobby/algoritms/LabExecutor/app/./doc_src/images/1650428030.jpg}

\includegraphics[width=0.8\linewidth]{/home/fizlrock/data/files/backup/code_backup/hobby/algoritms/LabExecutor/app/./doc_src/images/784751604.jpg}

\includegraphics[width=0.8\linewidth]{/home/fizlrock/data/files/backup/code_backup/hobby/algoritms/LabExecutor/app/./doc_src/images/1190015776.jpg}

\includegraphics[width=0.8\linewidth]{/home/fizlrock/data/files/backup/code_backup/hobby/algoritms/LabExecutor/app/./doc_src/images/540407485.jpg}

\includegraphics[width=0.8\linewidth]{/home/fizlrock/data/files/backup/code_backup/hobby/algoritms/LabExecutor/app/./doc_src/images/2096395001.jpg}

\includegraphics[width=0.8\linewidth]{/home/fizlrock/data/files/backup/code_backup/hobby/algoritms/LabExecutor/app/./doc_src/images/510044010.jpg}

\includegraphics[width=0.8\linewidth]{/home/fizlrock/data/files/backup/code_backup/hobby/algoritms/LabExecutor/app/./doc_src/images/714370652.jpg}
\pagebreak

\paragraph{Задание 3.2}

Закодировать сообщение методом LZSS\\
Строка:ЛЯЛЯЛЯ\_ЛЯЛЯ\_ЯЛИК\_МЯЛ\\
Результат: 0'Л' 0'Я' 1<8,2> 1<6,2> 0'\_' 1<5,5> 1<1,2> 0'И' 0'К' 1<0,1> 0'М' 1<0,2>\\
\begin{table}[h!]
\centering
\begin{tabular}{|c|c|c|c|c|c|c|c|c|c|c|c|c|c|c|c|c|}
\hline
\multicolumn{10}{|c|}{Cловарь} & \multicolumn{6}{c|}{Буфер} & Код  \\ \hline
  &   &   &   &   &   &   &   &   &   & Л & Я & Л & Я & Л & Я & 0'Л'\\ \hline
  &   &   &   &   &   &   &   &   & Л & Я & Л & Я & Л & Я & \_ & 0'Я'\\ \hline
  &   &   &   &   &   &   &   & \cellcolor[HTML]{FFFF00} Л & \cellcolor[HTML]{FFFF00} Я & \cellcolor[HTML]{FFFF00} Л & \cellcolor[HTML]{FFFF00} Я & Л & Я & \_ & Л & 1<8,2>\\ \hline
  &   &   &   &   &   & \cellcolor[HTML]{FFFF00} Л & \cellcolor[HTML]{FFFF00} Я & Л & Я & \cellcolor[HTML]{FFFF00} Л & \cellcolor[HTML]{FFFF00} Я & \_ & Л & Я & Л & 1<6,2>\\ \hline
  &   &   &   & Л & Я & Л & Я & Л & Я & \_ & Л & Я & Л & Я & \_ & 0'\_'\\ \hline
  &   &   & Л & Я & \cellcolor[HTML]{FFFF00} Л & \cellcolor[HTML]{FFFF00} Я & \cellcolor[HTML]{FFFF00} Л & \cellcolor[HTML]{FFFF00} Я & \cellcolor[HTML]{FFFF00} \_ & \cellcolor[HTML]{FFFF00} Л & \cellcolor[HTML]{FFFF00} Я & \cellcolor[HTML]{FFFF00} Л & \cellcolor[HTML]{FFFF00} Я & \cellcolor[HTML]{FFFF00} \_ & Я & 1<5,5>\\ \hline
Л & \cellcolor[HTML]{FFFF00} Я & \cellcolor[HTML]{FFFF00} Л & Я & \_ & Л & Я & Л & Я & \_ & \cellcolor[HTML]{FFFF00} Я & \cellcolor[HTML]{FFFF00} Л & И & К & \_ & М & 1<1,2>\\ \hline
Л & Я & \_ & Л & Я & Л & Я & \_ & Я & Л & И & К & \_ & М & Я & Л & 0'И'\\ \hline
Я & \_ & Л & Я & Л & Я & \_ & Я & Л & И & К & \_ & М & Я & Л &   & 0'К'\\ \hline
\cellcolor[HTML]{FFFF00} \_ & Л & Я & Л & Я & \_ & Я & Л & И & К & \cellcolor[HTML]{FFFF00} \_ & М & Я & Л &   &   & 1<0,1>\\ \hline
Л & Я & Л & Я & \_ & Я & Л & И & К & \_ & М & Я & Л &   &   &   & 0'М'\\ \hline
\cellcolor[HTML]{FFFF00} Я & \cellcolor[HTML]{FFFF00} Л & Я & \_ & Я & Л & И & К & \_ & М & \cellcolor[HTML]{FFFF00} Я & \cellcolor[HTML]{FFFF00} Л &   &   &   &   & 1<0,2>\\ \hline
\end{tabular}
\end{table}

\paragraph{Задание 3.3}

Закодировать сообщение методом LZ78\\
Строка:ЛЯЛЯЛЯ\_ЛЯЛЯ\_ЯЛИК\_МЯЛ\\
\begin{table}[h!]
\centering
\begin{tabular}{|c|c|c|} 
\hline
 Входная фраза (в словарь) & Код & Позиция словаря \\ \hline

 &  & 0 \\ \hline
Л & 0'Л' & 1 \\ \hline
Я & 0'Я' & 2 \\ \hline
ЛЯ & 1'Я' & 3 \\ \hline
ЛЯ\_ & 3'\_' & 4 \\ \hline
ЛЯЛ & 3'Л' & 5 \\ \hline
Я\_ & 2'\_' & 6 \\ \hline
ЯЛ & 2'Л' & 7 \\ \hline
И & 0'И' & 8 \\ \hline
К & 0'К' & 9 \\ \hline
\_ & 0'\_' & 10 \\ \hline
М & 0'М' & 11 \\ \hline
\end{tabular}
\end{table}

Результат: 0'Л' 0'Я' 1'Я' 3'\_' 3'Л' 2'\_' 2'Л' 0'И' 0'К' 0'\_' 0'М'\\
\pagebreak
\paragraph{Задание 4. Арифметическое кодирование\\}

Исходная строка: АББААСКААС\
\begin{center}
 \begin{tabular}{ |c|c| } 
  \hline
     Буква & Вероятность \\ \hline
А & 0.50\\\hline
Б & 0.20\\\hline
С & 0.20\\\hline
К & 0.10
\\ \hline \end{tabular}
\end{center}
\begin{center}
 \begin{tabular}{ |c|c|c| } 
  \hline
     Буква & Начало & Конец \\ \hline
А & 0.00 & 0.50\\\hline
Б & 0.50 & 0.70\\\hline
С & 0.70 & 0.90\\\hline
К & 0.90 & 1.00
\\ \hline \end{tabular}
\end{center}
\begin{center}
 \begin{tabular}{ |c|c|c|c| } 
  \hline
     Буква & delta & min & max \\ \hline
А & 0.5000000000 & 0.0000000000 & 0.5000000000\\\hline
Б & 0.1000000000 & 0.2500000000 & 0.3500000000\\\hline
Б & 0.0200000000 & 0.3000000000 & 0.3200000000\\\hline
А & 0.0100000000 & 0.3000000000 & 0.3100000000\\\hline
А & 0.0050000000 & 0.3000000000 & 0.3050000000\\\hline
С & 0.0010000000 & 0.3035000000 & 0.3045000000\\\hline
К & 0.0001000000 & 0.3044000000 & 0.3045000000\\\hline
А & 0.0000500000 & 0.3044000000 & 0.3044500000\\\hline
А & 0.0000250000 & 0.3044000000 & 0.3044250000\\\hline
С & 0.0000050000 & 0.3044175000 & 0.3044225000
\\ \hline \end{tabular}
\end{center}
Результат: 30442
\pagebreak
\paragraph{Задание 5.1}

\\ 

Декодировать сообщение методом адаптивного хаффмана \\
Строка: 
'R'0'T'01100'N'010111100'D'1001\\
Результат: RTTTNRRRDD

\includegraphics[width=0.8\linewidth]{/home/fizlrock/data/files/backup/code_backup/hobby/algoritms/LabExecutor/app/./doc_src/images/721193084.jpg}

\includegraphics[width=0.8\linewidth]{/home/fizlrock/data/files/backup/code_backup/hobby/algoritms/LabExecutor/app/./doc_src/images/1554607697.jpg}

\includegraphics[width=0.8\linewidth]{/home/fizlrock/data/files/backup/code_backup/hobby/algoritms/LabExecutor/app/./doc_src/images/1176723365.jpg}

\includegraphics[width=0.8\linewidth]{/home/fizlrock/data/files/backup/code_backup/hobby/algoritms/LabExecutor/app/./doc_src/images/1635101540.jpg}

\includegraphics[width=0.8\linewidth]{/home/fizlrock/data/files/backup/code_backup/hobby/algoritms/LabExecutor/app/./doc_src/images/1232876425.jpg}

\includegraphics[width=0.8\linewidth]{/home/fizlrock/data/files/backup/code_backup/hobby/algoritms/LabExecutor/app/./doc_src/images/388413155.jpg}

\includegraphics[width=0.8\linewidth]{/home/fizlrock/data/files/backup/code_backup/hobby/algoritms/LabExecutor/app/./doc_src/images/393407227.jpg}

\includegraphics[width=0.8\linewidth]{/home/fizlrock/data/files/backup/code_backup/hobby/algoritms/LabExecutor/app/./doc_src/images/1006438626.jpg}

\includegraphics[width=0.8\linewidth]{/home/fizlrock/data/files/backup/code_backup/hobby/algoritms/LabExecutor/app/./doc_src/images/1184371447.jpg}

\includegraphics[width=0.8\linewidth]{/home/fizlrock/data/files/backup/code_backup/hobby/algoritms/LabExecutor/app/./doc_src/images/1811242998.jpg}
\pagebreak
\paragraph{Задание 5.3 Декодировать строку(LZSS)\\}

Исходная строка: [0'к'] [0'и'] [0'з'] [1<8,1>] [0'л'] [0' '] [1<6,2>] [0'м'] [0'а'] [1<5,1>][1<7,2>] [0'к'] [1<6,4>] [0'е'] [0'т']\\
\begin{table}[h!]
\centering
\begin{tabular}{|c|c|c|}
\hline
 Cловарь & Буфер & Код  \\ \hline
0'к' & [ ,  ,  ,  ,  ,  ,  ,  ,  , к] & к
\\ \hline
0'и' & [ ,  ,  ,  ,  ,  ,  ,  , к, и] & и
\\ \hline
0'з' & [ ,  ,  ,  ,  ,  ,  , к, и, з] & з
\\ \hline
1<8,1> & [ ,  ,  ,  ,  ,  , к, и, з, и] & и
\\ \hline
0'л' & [ ,  ,  ,  ,  , к, и, з, и, л] & л
\\ \hline
0' ' & [ ,  ,  ,  , к, и, з, и, л,  ] &  
\\ \hline
1<6,2> & [ ,  , к, и, з, и, л,  , з, и] & зи
\\ \hline
0'м' & [ , к, и, з, и, л,  , з, и, м] & м
\\ \hline
0'а' & [к, и, з, и, л,  , з, и, м, а] & а
\\ \hline
1<5,1> & [и, з, и, л,  , з, и, м, а,  ] &  
\\ \hline
1<7,2> & [и, л,  , з, и, м, а,  , м, а] & ма
\\ \hline
0'к' & [л,  , з, и, м, а,  , м, а, к] & к
\\ \hline
1<6,4> & [м, а,  , м, а, к,  , м, а, к] &  мак
\\ \hline
0'е' & [а,  , м, а, к,  , м, а, к, е] & е
\\ \hline
0'т' & [ , м, а, к,  , м, а, к, е, т] & т
\\ \hline
\end{tabular}
\end{table}

Результат: кизил зима мак макет
\pagebreak
\paragraph{Задание 5.4 Декодировать строку(LZ78)\\}

Исходная строка: [0'м'] [0'и'] [0'р'] [0' '] [0'п'] [2'р'] [4'т'] [6' '] [0'т'] [2'г'] [0'р']\\
\begin{table}[h!]
\centering
\begin{tabular}{|c|c|c|}
\hline
 Cловарь & Буфер & Код  \\ \hline
 & [] & 
\\ \hline
0'м' & [, м] & м
\\ \hline
0'и' & [, м, и] & и
\\ \hline
0'р' & [, м, и, р] & р
\\ \hline
0' ' & [, м, и, р,  ] &  
\\ \hline
0'п' & [, м, и, р,  , п] & п
\\ \hline
2'р' & [, м, и, р,  , п, ир] & ир
\\ \hline
4'т' & [, м, и, р,  , п, ир,  т] &  т
\\ \hline
6' ' & [, м, и, р,  , п, ир,  т, ир ] & ир 
\\ \hline
0'т' & [, м, и, р,  , п, ир,  т, ир , т] & т
\\ \hline
2'г' & [, м, и, р,  , п, ир,  т, ир , т, иг] & иг
\\ \hline
0'р' & [, м, и, р,  , п, ир,  т, ир , т, иг, р] & р
\\ \hline
\end{tabular}
\end{table}

Результат: мир пир тир тигр
\pagebreak
\subsection{Вариант №3}
\paragraph{Задание 1. Блочный хаффман \\}

Строка ТИИИИККККК, размер блока: 2
\begin{center}
 \begin{tabular}{ |c|c|l| } 
  \hline
     Буква & Вероятность & Код\\ \hline
К & 0.50 & 0\\\hline
И & 0.40 & 11\\\hline
Т & 0.10 & 10
\\ \hline \end{tabular}
\end{center}
Энтропия алфавита: 1.3610
\begin{center}
 \begin{tabular}{ |c|c|l| } 
  \hline
     Блок & Вероятность & Код\\ \hline
КК & 0.25 & 10\\\hline
ИК & 0.20 & 00\\\hline
КИ & 0.20 & 01\\\hline
ИИ & 0.16 & 110\\\hline
ТК & 0.05 & 11101\\\hline
КТ & 0.05 & 11110\\\hline
ТИ & 0.04 & 111111\\\hline
ИТ & 0.04 & 11100\\\hline
ТТ & 0.01 & 111110
\\ \hline \end{tabular}
\end{center}
Бит на символ при посимвольном кодировании: 1.5000, при блочном: 1.3900

\includegraphics[width=0.5\linewidth]{/home/fizlrock/data/files/backup/code_backup/hobby/algoritms/LabExecutor/app/./doc_src/images/1648609484.jpg}

\includegraphics[width=0.9\linewidth]{/home/fizlrock/data/files/backup/code_backup/hobby/algoritms/LabExecutor/app/./doc_src/images/1777246626.jpg}
\pagebreak
\paragraph{Задание 2. Сжать адаптивным хаффманом\\}

Строка: 
ПРОВППРРРО\\
Результат: 'П' 0'Р' 00'О' 100'В' 10 11 10 10 11 101

\includegraphics[width=0.8\linewidth]{/home/fizlrock/data/files/backup/code_backup/hobby/algoritms/LabExecutor/app/./doc_src/images/254642975.jpg}

\includegraphics[width=0.8\linewidth]{/home/fizlrock/data/files/backup/code_backup/hobby/algoritms/LabExecutor/app/./doc_src/images/589081034.jpg}

\includegraphics[width=0.8\linewidth]{/home/fizlrock/data/files/backup/code_backup/hobby/algoritms/LabExecutor/app/./doc_src/images/490859978.jpg}

\includegraphics[width=0.8\linewidth]{/home/fizlrock/data/files/backup/code_backup/hobby/algoritms/LabExecutor/app/./doc_src/images/132270689.jpg}

\includegraphics[width=0.8\linewidth]{/home/fizlrock/data/files/backup/code_backup/hobby/algoritms/LabExecutor/app/./doc_src/images/1786098800.jpg}

\includegraphics[width=0.8\linewidth]{/home/fizlrock/data/files/backup/code_backup/hobby/algoritms/LabExecutor/app/./doc_src/images/1535888112.jpg}

\includegraphics[width=0.8\linewidth]{/home/fizlrock/data/files/backup/code_backup/hobby/algoritms/LabExecutor/app/./doc_src/images/844628614.jpg}

\includegraphics[width=0.8\linewidth]{/home/fizlrock/data/files/backup/code_backup/hobby/algoritms/LabExecutor/app/./doc_src/images/327093148.jpg}

\includegraphics[width=0.8\linewidth]{/home/fizlrock/data/files/backup/code_backup/hobby/algoritms/LabExecutor/app/./doc_src/images/296708007.jpg}

\includegraphics[width=0.8\linewidth]{/home/fizlrock/data/files/backup/code_backup/hobby/algoritms/LabExecutor/app/./doc_src/images/2036050284.jpg}
\pagebreak
\paragraph{Задание 3.1}

Закодировать сообщение методом LZ77\\
Строка:ТАРАРА\_ТАРТАР\_ТАРТ\_ТАРА\\
Результат: <0,0,Т> <0,0,А> <0,0,Р> <8,3,\_> <3,3,Т> <0,2,\_> <3,4,\_> <1,3,А>\\
\begin{table}[h!]
\centering
\begin{tabular}{|c|c|c|c|c|c|c|c|c|c|c|c|c|c|c|c|c|} 
\hline
\multicolumn{10}{|c|}{Cловарь} & \multicolumn{6}{c|}{Буфер} & Код  \\ \hline
  &   &   &   &   &   &   &   &   &   & \cellcolor[HTML]{8CE4F6} Т & А & Р & А & Р & А & <0,0,Т>
\\ \hline
  &   &   &   &   &   &   &   &   & Т & \cellcolor[HTML]{8CE4F6} А & Р & А & Р & А &   & <0,0,А>
\\ \hline
  &   &   &   &   &   &   &   & Т & А & \cellcolor[HTML]{8CE4F6} Р & А & Р & А &   & Т & <0,0,Р>
\\ \hline
  &   &   &   &   &   &   & Т & \cellcolor[HTML]{FFFF00} А & \cellcolor[HTML]{FFFF00} Р & \cellcolor[HTML]{FFFF00} А & \cellcolor[HTML]{FFFF00} Р & \cellcolor[HTML]{FFFF00} А & \cellcolor[HTML]{8CE4F6}   & Т & А & <8,3,\_>
\\ \hline
  &   &   & \cellcolor[HTML]{FFFF00} Т & \cellcolor[HTML]{FFFF00} А & \cellcolor[HTML]{FFFF00} Р & А & Р & А &   & \cellcolor[HTML]{FFFF00} Т & \cellcolor[HTML]{FFFF00} А & \cellcolor[HTML]{FFFF00} Р & \cellcolor[HTML]{8CE4F6} Т & А & Р & <3,3,Т>
\\ \hline
\cellcolor[HTML]{FFFF00} А & \cellcolor[HTML]{FFFF00} Р & А & Р & А &   & Т & А & Р & Т & \cellcolor[HTML]{FFFF00} А & \cellcolor[HTML]{FFFF00} Р & \cellcolor[HTML]{8CE4F6}   & Т & А & Р & <0,2,\_>
\\ \hline
Р & А &   & \cellcolor[HTML]{FFFF00} Т & \cellcolor[HTML]{FFFF00} А & \cellcolor[HTML]{FFFF00} Р & \cellcolor[HTML]{FFFF00} Т & А & Р &   & \cellcolor[HTML]{FFFF00} Т & \cellcolor[HTML]{FFFF00} А & \cellcolor[HTML]{FFFF00} Р & \cellcolor[HTML]{FFFF00} Т & \cellcolor[HTML]{8CE4F6}   & Т & <3,4,\_>
\\ \hline
Р & \cellcolor[HTML]{FFFF00} Т & \cellcolor[HTML]{FFFF00} А & \cellcolor[HTML]{FFFF00} Р &   & Т & А & Р & Т &   & \cellcolor[HTML]{FFFF00} Т & \cellcolor[HTML]{FFFF00} А & \cellcolor[HTML]{FFFF00} Р & \cellcolor[HTML]{8CE4F6} А &   &   & <1,3,А>
\\ \hline
\end{tabular}
\end{table}

\paragraph{Задание 3.2}

Закодировать сообщение методом LZSS\\
Строка:ТАРАРА\_ТАРТАР\_ТАРТ\_ТАРА\\
Результат: 0'Т' 0'А' 0'Р' 1<8,2> 1<6,1> 0'\_' 1<3,3> 1<0,3> 1<3,5> 1<5,4> 1<3,1>\\
\begin{table}[h!]
\centering
\begin{tabular}{|c|c|c|c|c|c|c|c|c|c|c|c|c|c|c|c|c|}
\hline
\multicolumn{10}{|c|}{Cловарь} & \multicolumn{6}{c|}{Буфер} & Код  \\ \hline
  &   &   &   &   &   &   &   &   &   & Т & А & Р & А & Р & А & 0'Т'\\ \hline
  &   &   &   &   &   &   &   &   & Т & А & Р & А & Р & А & \_ & 0'А'\\ \hline
  &   &   &   &   &   &   &   & Т & А & Р & А & Р & А & \_ & Т & 0'Р'\\ \hline
  &   &   &   &   &   &   & Т & \cellcolor[HTML]{FFFF00} А & \cellcolor[HTML]{FFFF00} Р & \cellcolor[HTML]{FFFF00} А & \cellcolor[HTML]{FFFF00} Р & А & \_ & Т & А & 1<8,2>\\ \hline
  &   &   &   &   & Т & \cellcolor[HTML]{FFFF00} А & Р & А & Р & \cellcolor[HTML]{FFFF00} А & \_ & Т & А & Р & Т & 1<6,1>\\ \hline
  &   &   &   & Т & А & Р & А & Р & А & \_ & Т & А & Р & Т & А & 0'\_'\\ \hline
  &   &   & \cellcolor[HTML]{FFFF00} Т & \cellcolor[HTML]{FFFF00} А & \cellcolor[HTML]{FFFF00} Р & А & Р & А & \_ & \cellcolor[HTML]{FFFF00} Т & \cellcolor[HTML]{FFFF00} А & \cellcolor[HTML]{FFFF00} Р & Т & А & Р & 1<3,3>\\ \hline
\cellcolor[HTML]{FFFF00} Т & \cellcolor[HTML]{FFFF00} А & \cellcolor[HTML]{FFFF00} Р & А & Р & А & \_ & Т & А & Р & \cellcolor[HTML]{FFFF00} Т & \cellcolor[HTML]{FFFF00} А & \cellcolor[HTML]{FFFF00} Р & \_ & Т & А & 1<0,3>\\ \hline
А & Р & А & \cellcolor[HTML]{FFFF00} \_ & \cellcolor[HTML]{FFFF00} Т & \cellcolor[HTML]{FFFF00} А & \cellcolor[HTML]{FFFF00} Р & \cellcolor[HTML]{FFFF00} Т & А & Р & \cellcolor[HTML]{FFFF00} \_ & \cellcolor[HTML]{FFFF00} Т & \cellcolor[HTML]{FFFF00} А & \cellcolor[HTML]{FFFF00} Р & \cellcolor[HTML]{FFFF00} Т & \_ & 1<3,5>\\ \hline
А & Р & Т & А & Р & \cellcolor[HTML]{FFFF00} \_ & \cellcolor[HTML]{FFFF00} Т & \cellcolor[HTML]{FFFF00} А & \cellcolor[HTML]{FFFF00} Р & Т & \cellcolor[HTML]{FFFF00} \_ & \cellcolor[HTML]{FFFF00} Т & \cellcolor[HTML]{FFFF00} А & \cellcolor[HTML]{FFFF00} Р & А &   & 1<5,4>\\ \hline
Р & \_ & Т & \cellcolor[HTML]{FFFF00} А & Р & Т & \_ & Т & А & Р & \cellcolor[HTML]{FFFF00} А &   &   &   &   &   & 1<3,1>\\ \hline
\end{tabular}
\end{table}

\paragraph{Задание 3.3}

Закодировать сообщение методом LZ78\\
Строка:ТАРАРА\_ТАРТАР\_ТАРТ\_ТАРА\\
\begin{table}[h!]
\centering
\begin{tabular}{|c|c|c|} 
\hline
 Входная фраза (в словарь) & Код & Позиция словаря \\ \hline

 &  & 0 \\ \hline
Т & 0'Т' & 1 \\ \hline
А & 0'А' & 2 \\ \hline
Р & 0'Р' & 3 \\ \hline
АР & 2'Р' & 4 \\ \hline
А\_ & 2'\_' & 5 \\ \hline
ТА & 1'А' & 6 \\ \hline
РТ & 3'Т' & 7 \\ \hline
АР\_ & 4'\_' & 8 \\ \hline
ТАР & 6'Р' & 9 \\ \hline
Т\_ & 1'\_' & 10 \\ \hline
ТАРА & 9'А' & 11 \\ \hline
\end{tabular}
\end{table}

Результат: 0'Т' 0'А' 0'Р' 2'Р' 2'\_' 1'А' 3'Т' 4'\_' 6'Р' 1'\_' 9'А'\\
\pagebreak
\paragraph{Задание 4. Арифметическое кодирование\\}

Исходная строка: ПРОВППРРРО\
\begin{center}
 \begin{tabular}{ |c|c| } 
  \hline
     Буква & Вероятность \\ \hline
Р & 0.40\\\hline
П & 0.30\\\hline
О & 0.20\\\hline
В & 0.10
\\ \hline \end{tabular}
\end{center}
\begin{center}
 \begin{tabular}{ |c|c|c| } 
  \hline
     Буква & Начало & Конец \\ \hline
Р & 0.00 & 0.40\\\hline
П & 0.40 & 0.70\\\hline
О & 0.70 & 0.90\\\hline
В & 0.90 & 1.00
\\ \hline \end{tabular}
\end{center}
\begin{center}
 \begin{tabular}{ |c|c|c|c| } 
  \hline
     Буква & delta & min & max \\ \hline
П & 0.3000000000 & 0.4000000000 & 0.7000000000\\\hline
Р & 0.1200000000 & 0.4000000000 & 0.5200000000\\\hline
О & 0.0240000000 & 0.4840000000 & 0.5080000000\\\hline
В & 0.0024000000 & 0.5056000000 & 0.5080000000\\\hline
П & 0.0007200000 & 0.5065600000 & 0.5072800000\\\hline
П & 0.0002160000 & 0.5068480000 & 0.5070640000\\\hline
Р & 0.0000864000 & 0.5068480000 & 0.5069344000\\\hline
Р & 0.0000345600 & 0.5068480000 & 0.5068825600\\\hline
Р & 0.0000138240 & 0.5068480000 & 0.5068618240\\\hline
О & 0.0000027648 & 0.5068576768 & 0.5068604416
\\ \hline \end{tabular}
\end{center}
Результат: 50686
\pagebreak
\paragraph{Задание 5.1}

\\ 

Декодировать сообщение методом адаптивного хаффмана \\
Строка: 
'S'0'D'00'A'1101000'R'011001001\\
Результат: SDADDDRAAR

\includegraphics[width=0.8\linewidth]{/home/fizlrock/data/files/backup/code_backup/hobby/algoritms/LabExecutor/app/./doc_src/images/1580748119.jpg}

\includegraphics[width=0.8\linewidth]{/home/fizlrock/data/files/backup/code_backup/hobby/algoritms/LabExecutor/app/./doc_src/images/1949500031.jpg}

\includegraphics[width=0.8\linewidth]{/home/fizlrock/data/files/backup/code_backup/hobby/algoritms/LabExecutor/app/./doc_src/images/1353796591.jpg}

\includegraphics[width=0.8\linewidth]{/home/fizlrock/data/files/backup/code_backup/hobby/algoritms/LabExecutor/app/./doc_src/images/2063790246.jpg}

\includegraphics[width=0.8\linewidth]{/home/fizlrock/data/files/backup/code_backup/hobby/algoritms/LabExecutor/app/./doc_src/images/560867418.jpg}

\includegraphics[width=0.8\linewidth]{/home/fizlrock/data/files/backup/code_backup/hobby/algoritms/LabExecutor/app/./doc_src/images/1022802198.jpg}

\includegraphics[width=0.8\linewidth]{/home/fizlrock/data/files/backup/code_backup/hobby/algoritms/LabExecutor/app/./doc_src/images/1870698351.jpg}

\includegraphics[width=0.8\linewidth]{/home/fizlrock/data/files/backup/code_backup/hobby/algoritms/LabExecutor/app/./doc_src/images/1778774045.jpg}

\includegraphics[width=0.8\linewidth]{/home/fizlrock/data/files/backup/code_backup/hobby/algoritms/LabExecutor/app/./doc_src/images/250192003.jpg}

\includegraphics[width=0.8\linewidth]{/home/fizlrock/data/files/backup/code_backup/hobby/algoritms/LabExecutor/app/./doc_src/images/1682399966.jpg}
\pagebreak
\paragraph{Задание 5.3 Декодировать строку(LZSS)\\}

Исходная строка: [0'к'] [0'а'] [0'н'] [0'в'] [1<7,1>] [0' '] [1<7,2>] [0'т'] [1<5,5>] [0'н'] [0'и'][0'к'] [1<3,4>] [0'а']\\
\begin{table}[h!]
\centering
\begin{tabular}{|c|c|c|}
\hline
 Cловарь & Буфер & Код  \\ \hline
0'к' & [ ,  ,  ,  ,  ,  ,  ,  ,  , к] & к
\\ \hline
0'а' & [ ,  ,  ,  ,  ,  ,  ,  , к, а] & а
\\ \hline
0'н' & [ ,  ,  ,  ,  ,  ,  , к, а, н] & н
\\ \hline
0'в' & [ ,  ,  ,  ,  ,  , к, а, н, в] & в
\\ \hline
1<7,1> & [ ,  ,  ,  ,  , к, а, н, в, а] & а
\\ \hline
0' ' & [ ,  ,  ,  , к, а, н, в, а,  ] &  
\\ \hline
1<7,2> & [ ,  , к, а, н, в, а,  , в, а] & ва
\\ \hline
0'т' & [ , к, а, н, в, а,  , в, а, т] & т
\\ \hline
1<5,5> & [а,  , в, а, т, а,  , в, а, т] & а ват
\\ \hline
0'н' & [ , в, а, т, а,  , в, а, т, н] & н
\\ \hline
0'и' & [в, а, т, а,  , в, а, т, н, и] & и
\\ \hline
0'к' & [а, т, а,  , в, а, т, н, и, к] & к
\\ \hline
1<3,4> & [в, а, т, н, и, к,  , в, а, т] &  ват
\\ \hline
0'а' & [а, т, н, и, к,  , в, а, т, а] & а
\\ \hline
\end{tabular}
\end{table}

Результат: канва вата ватник вата
\pagebreak
\paragraph{Задание 5.4 Декодировать строку(LZ78)\\}

Исходная строка: [0'г'] [0'о'] [0'р'] [2'д'] [0' '] [1'о'] [3'а'] [5'р'] [4' '] [3'о'] [0'г']\\
\begin{table}[h!]
\centering
\begin{tabular}{|c|c|c|}
\hline
 Cловарь & Буфер & Код  \\ \hline
 & [] & 
\\ \hline
0'г' & [, г] & г
\\ \hline
0'о' & [, г, о] & о
\\ \hline
0'р' & [, г, о, р] & р
\\ \hline
2'д' & [, г, о, р, од] & од
\\ \hline
0' ' & [, г, о, р, од,  ] &  
\\ \hline
1'о' & [, г, о, р, од,  , го] & го
\\ \hline
3'а' & [, г, о, р, од,  , го, ра] & ра
\\ \hline
5'р' & [, г, о, р, од,  , го, ра,  р] &  р
\\ \hline
4' ' & [, г, о, р, од,  , го, ра,  р, од ] & од 
\\ \hline
3'о' & [, г, о, р, од,  , го, ра,  р, од , ро] & ро
\\ \hline
0'г' & [, г, о, р, од,  , го, ра,  р, од , ро, г] & г
\\ \hline
\end{tabular}
\end{table}

Результат: город гора род рог
\pagebreak
\subsection{Вариант №4}
\paragraph{Задание 1. Блочный хаффман \\}

Строка ДДУДУУУУУУ, размер блока: 3
\begin{center}
 \begin{tabular}{ |c|c|l| } 
  \hline
     Буква & Вероятность & Код\\ \hline
У & 0.70 & 1\\\hline
Д & 0.30 & 0
\\ \hline \end{tabular}
\end{center}
Энтропия алфавита: 0.8813
\begin{center}
 \begin{tabular}{ |c|c|l| } 
  \hline
     Блок & Вероятность & Код\\ \hline
УУУ & 0.34 & 11\\\hline
УДУ & 0.15 & 101\\\hline
ДУУ & 0.15 & 00\\\hline
УУД & 0.15 & 100\\\hline
УДД & 0.06 & 0101\\\hline
ДУД & 0.06 & 0110\\\hline
ДДУ & 0.06 & 0111\\\hline
ДДД & 0.03 & 0100
\\ \hline \end{tabular}
\end{center}
Бит на символ при посимвольном кодировании: 1.0000, при блочном: 0.9087

\includegraphics[width=0.5\linewidth]{/home/fizlrock/data/files/backup/code_backup/hobby/algoritms/LabExecutor/app/./doc_src/images/9071951.jpg}

\includegraphics[width=0.9\linewidth]{/home/fizlrock/data/files/backup/code_backup/hobby/algoritms/LabExecutor/app/./doc_src/images/732036341.jpg}
\pagebreak
\paragraph{Задание 2. Сжать адаптивным хаффманом\\}

Строка: 
АППРОПММММ\\
Результат: 'А' 0'П' 01 00'Р' 000'О' 0 1100'М' 1001 111 10

\includegraphics[width=0.8\linewidth]{/home/fizlrock/data/files/backup/code_backup/hobby/algoritms/LabExecutor/app/./doc_src/images/1386273745.jpg}

\includegraphics[width=0.8\linewidth]{/home/fizlrock/data/files/backup/code_backup/hobby/algoritms/LabExecutor/app/./doc_src/images/964202529.jpg}

\includegraphics[width=0.8\linewidth]{/home/fizlrock/data/files/backup/code_backup/hobby/algoritms/LabExecutor/app/./doc_src/images/2077370905.jpg}

\includegraphics[width=0.8\linewidth]{/home/fizlrock/data/files/backup/code_backup/hobby/algoritms/LabExecutor/app/./doc_src/images/949843867.jpg}

\includegraphics[width=0.8\linewidth]{/home/fizlrock/data/files/backup/code_backup/hobby/algoritms/LabExecutor/app/./doc_src/images/1483660544.jpg}

\includegraphics[width=0.8\linewidth]{/home/fizlrock/data/files/backup/code_backup/hobby/algoritms/LabExecutor/app/./doc_src/images/36474485.jpg}

\includegraphics[width=0.8\linewidth]{/home/fizlrock/data/files/backup/code_backup/hobby/algoritms/LabExecutor/app/./doc_src/images/828478805.jpg}

\includegraphics[width=0.8\linewidth]{/home/fizlrock/data/files/backup/code_backup/hobby/algoritms/LabExecutor/app/./doc_src/images/1035274517.jpg}

\includegraphics[width=0.8\linewidth]{/home/fizlrock/data/files/backup/code_backup/hobby/algoritms/LabExecutor/app/./doc_src/images/1290384726.jpg}

\includegraphics[width=0.8\linewidth]{/home/fizlrock/data/files/backup/code_backup/hobby/algoritms/LabExecutor/app/./doc_src/images/918134702.jpg}
\pagebreak
\paragraph{Задание 3.1}

Закодировать сообщение методом LZ77\\
Строка:СЫР\_СЫН\_СЫРОК\_СЫНОК\\
Результат: <0,0,С> <0,0,Ы> <0,0,Р> <0,0,\_> <6,2,Н> <6,3,Р> <0,0,О> <0,0,К> <0,4,О> <0,0,К>\\
\begin{table}[h!]
\centering
\begin{tabular}{|c|c|c|c|c|c|c|c|c|c|c|c|c|c|c|c|c|} 
\hline
\multicolumn{10}{|c|}{Cловарь} & \multicolumn{6}{c|}{Буфер} & Код  \\ \hline
  &   &   &   &   &   &   &   &   &   & \cellcolor[HTML]{8CE4F6} С & Ы & Р &   & С & Ы & <0,0,С>
\\ \hline
  &   &   &   &   &   &   &   &   & С & \cellcolor[HTML]{8CE4F6} Ы & Р &   & С & Ы & Н & <0,0,Ы>
\\ \hline
  &   &   &   &   &   &   &   & С & Ы & \cellcolor[HTML]{8CE4F6} Р &   & С & Ы & Н &   & <0,0,Р>
\\ \hline
  &   &   &   &   &   &   & С & Ы & Р & \cellcolor[HTML]{8CE4F6}   & С & Ы & Н &   & С & <0,0,\_>
\\ \hline
  &   &   &   &   &   & \cellcolor[HTML]{FFFF00} С & \cellcolor[HTML]{FFFF00} Ы & Р &   & \cellcolor[HTML]{FFFF00} С & \cellcolor[HTML]{FFFF00} Ы & \cellcolor[HTML]{8CE4F6} Н &   & С & Ы & <6,2,Н>
\\ \hline
  &   &   & С & Ы & Р & \cellcolor[HTML]{FFFF00}   & \cellcolor[HTML]{FFFF00} С & \cellcolor[HTML]{FFFF00} Ы & Н & \cellcolor[HTML]{FFFF00}   & \cellcolor[HTML]{FFFF00} С & \cellcolor[HTML]{FFFF00} Ы & \cellcolor[HTML]{8CE4F6} Р & О & К & <6,3,Р>
\\ \hline
Ы & Р &   & С & Ы & Н &   & С & Ы & Р & \cellcolor[HTML]{8CE4F6} О & К &   & С & Ы & Н & <0,0,О>
\\ \hline
Р &   & С & Ы & Н &   & С & Ы & Р & О & \cellcolor[HTML]{8CE4F6} К &   & С & Ы & Н & О & <0,0,К>
\\ \hline
\cellcolor[HTML]{FFFF00}   & \cellcolor[HTML]{FFFF00} С & \cellcolor[HTML]{FFFF00} Ы & \cellcolor[HTML]{FFFF00} Н &   & С & Ы & Р & О & К & \cellcolor[HTML]{FFFF00}   & \cellcolor[HTML]{FFFF00} С & \cellcolor[HTML]{FFFF00} Ы & \cellcolor[HTML]{FFFF00} Н & \cellcolor[HTML]{8CE4F6} О & К & <0,4,О>
\\ \hline
С & Ы & Р & О & К &   & С & Ы & Н & О & \cellcolor[HTML]{8CE4F6} К &   &   &   &   &   & <0,0,К>
\\ \hline
\end{tabular}
\end{table}

\paragraph{Задание 3.2}

Закодировать сообщение методом LZSS\\
Строка:СЫР\_СЫН\_СЫРОК\_СЫНОК\\
Результат: 0'С' 0'Ы' 0'Р' 0'\_' 1<6,2> 0'Н' 1<6,3> 1<2,1> 0'О' 0'К' 1<0,4> 1<4,2>\\
\begin{table}[h!]
\centering
\begin{tabular}{|c|c|c|c|c|c|c|c|c|c|c|c|c|c|c|c|c|}
\hline
\multicolumn{10}{|c|}{Cловарь} & \multicolumn{6}{c|}{Буфер} & Код  \\ \hline
  &   &   &   &   &   &   &   &   &   & С & Ы & Р & \_ & С & Ы & 0'С'\\ \hline
  &   &   &   &   &   &   &   &   & С & Ы & Р & \_ & С & Ы & Н & 0'Ы'\\ \hline
  &   &   &   &   &   &   &   & С & Ы & Р & \_ & С & Ы & Н & \_ & 0'Р'\\ \hline
  &   &   &   &   &   &   & С & Ы & Р & \_ & С & Ы & Н & \_ & С & 0'\_'\\ \hline
  &   &   &   &   &   & \cellcolor[HTML]{FFFF00} С & \cellcolor[HTML]{FFFF00} Ы & Р & \_ & \cellcolor[HTML]{FFFF00} С & \cellcolor[HTML]{FFFF00} Ы & Н & \_ & С & Ы & 1<6,2>\\ \hline
  &   &   &   & С & Ы & Р & \_ & С & Ы & Н & \_ & С & Ы & Р & О & 0'Н'\\ \hline
  &   &   & С & Ы & Р & \cellcolor[HTML]{FFFF00} \_ & \cellcolor[HTML]{FFFF00} С & \cellcolor[HTML]{FFFF00} Ы & Н & \cellcolor[HTML]{FFFF00} \_ & \cellcolor[HTML]{FFFF00} С & \cellcolor[HTML]{FFFF00} Ы & Р & О & К & 1<6,3>\\ \hline
С & Ы & \cellcolor[HTML]{FFFF00} Р & \_ & С & Ы & Н & \_ & С & Ы & \cellcolor[HTML]{FFFF00} Р & О & К & \_ & С & Ы & 1<2,1>\\ \hline
Ы & Р & \_ & С & Ы & Н & \_ & С & Ы & Р & О & К & \_ & С & Ы & Н & 0'О'\\ \hline
Р & \_ & С & Ы & Н & \_ & С & Ы & Р & О & К & \_ & С & Ы & Н & О & 0'К'\\ \hline
\cellcolor[HTML]{FFFF00} \_ & \cellcolor[HTML]{FFFF00} С & \cellcolor[HTML]{FFFF00} Ы & \cellcolor[HTML]{FFFF00} Н & \_ & С & Ы & Р & О & К & \cellcolor[HTML]{FFFF00} \_ & \cellcolor[HTML]{FFFF00} С & \cellcolor[HTML]{FFFF00} Ы & \cellcolor[HTML]{FFFF00} Н & О & К & 1<0,4>\\ \hline
\_ & С & Ы & Р & \cellcolor[HTML]{FFFF00} О & \cellcolor[HTML]{FFFF00} К & \_ & С & Ы & Н & \cellcolor[HTML]{FFFF00} О & \cellcolor[HTML]{FFFF00} К &   &   &   &   & 1<4,2>\\ \hline
\end{tabular}
\end{table}

\paragraph{Задание 3.3}

Закодировать сообщение методом LZ78\\
Строка:СЫР\_СЫН\_СЫРОК\_СЫНОК\\
\begin{table}[h!]
\centering
\begin{tabular}{|c|c|c|} 
\hline
 Входная фраза (в словарь) & Код & Позиция словаря \\ \hline

 &  & 0 \\ \hline
С & 0'С' & 1 \\ \hline
Ы & 0'Ы' & 2 \\ \hline
Р & 0'Р' & 3 \\ \hline
\_ & 0'\_' & 4 \\ \hline
СЫ & 1'Ы' & 5 \\ \hline
Н & 0'Н' & 6 \\ \hline
\_С & 4'С' & 7 \\ \hline
ЫР & 2'Р' & 8 \\ \hline
О & 0'О' & 9 \\ \hline
К & 0'К' & 10 \\ \hline
\_СЫ & 7'Ы' & 11 \\ \hline
НО & 6'О' & 12 \\ \hline
\end{tabular}
\end{table}

Результат: 0'С' 0'Ы' 0'Р' 0'\_' 1'Ы' 0'Н' 4'С' 2'Р' 0'О' 0'К' 7'Ы' 6'О'\\
\pagebreak
\paragraph{Задание 4. Арифметическое кодирование\\}

Исходная строка: АППРОПММММ\
\begin{center}
 \begin{tabular}{ |c|c| } 
  \hline
     Буква & Вероятность \\ \hline
М & 0.40\\\hline
П & 0.30\\\hline
А & 0.10\\\hline
Р & 0.10\\\hline
О & 0.10
\\ \hline \end{tabular}
\end{center}
\begin{center}
 \begin{tabular}{ |c|c|c| } 
  \hline
     Буква & Начало & Конец \\ \hline
М & 0.00 & 0.40\\\hline
П & 0.40 & 0.70\\\hline
А & 0.70 & 0.80\\\hline
Р & 0.80 & 0.90\\\hline
О & 0.90 & 1.00
\\ \hline \end{tabular}
\end{center}
\begin{center}
 \begin{tabular}{ |c|c|c|c| } 
  \hline
     Буква & delta & min & max \\ \hline
А & 0.1000000000 & 0.7000000000 & 0.8000000000\\\hline
П & 0.0300000000 & 0.7400000000 & 0.7700000000\\\hline
П & 0.0090000000 & 0.7520000000 & 0.7610000000\\\hline
Р & 0.0009000000 & 0.7592000000 & 0.7601000000\\\hline
О & 0.0000900000 & 0.7600100000 & 0.7601000000\\\hline
П & 0.0000270000 & 0.7600460000 & 0.7600730000\\\hline
М & 0.0000108000 & 0.7600460000 & 0.7600568000\\\hline
М & 0.0000043200 & 0.7600460000 & 0.7600503200\\\hline
М & 0.0000017280 & 0.7600460000 & 0.7600477280\\\hline
М & 0.0000006912 & 0.7600460000 & 0.7600466912
\\ \hline \end{tabular}
\end{center}
Результат: 760046
\pagebreak
\paragraph{Задание 5.1}

\\ 

Декодировать сообщение методом адаптивного хаффмана \\
Строка: 
'H'0'K'00'N'11100'F'1111110111\\
Результат: HKNKFNKNN

\includegraphics[width=0.8\linewidth]{/home/fizlrock/data/files/backup/code_backup/hobby/algoritms/LabExecutor/app/./doc_src/images/436900012.jpg}

\includegraphics[width=0.8\linewidth]{/home/fizlrock/data/files/backup/code_backup/hobby/algoritms/LabExecutor/app/./doc_src/images/1825749952.jpg}

\includegraphics[width=0.8\linewidth]{/home/fizlrock/data/files/backup/code_backup/hobby/algoritms/LabExecutor/app/./doc_src/images/1520069440.jpg}

\includegraphics[width=0.8\linewidth]{/home/fizlrock/data/files/backup/code_backup/hobby/algoritms/LabExecutor/app/./doc_src/images/1859359828.jpg}

\includegraphics[width=0.8\linewidth]{/home/fizlrock/data/files/backup/code_backup/hobby/algoritms/LabExecutor/app/./doc_src/images/711066480.jpg}

\includegraphics[width=0.8\linewidth]{/home/fizlrock/data/files/backup/code_backup/hobby/algoritms/LabExecutor/app/./doc_src/images/1687550094.jpg}

\includegraphics[width=0.8\linewidth]{/home/fizlrock/data/files/backup/code_backup/hobby/algoritms/LabExecutor/app/./doc_src/images/1338816759.jpg}

\includegraphics[width=0.8\linewidth]{/home/fizlrock/data/files/backup/code_backup/hobby/algoritms/LabExecutor/app/./doc_src/images/1495694219.jpg}

\includegraphics[width=0.8\linewidth]{/home/fizlrock/data/files/backup/code_backup/hobby/algoritms/LabExecutor/app/./doc_src/images/51380492.jpg}
\pagebreak
\paragraph{Задание 5.3 Декодировать строку(LZSS)\\}

Исходная строка: [0'л'] [0'и'] [1<8,2>] [0'я'] [0' '] [1<4,2>] [0'с'] [1<6,4>] [0'а'] [1<1,4>][0'т']\\
\begin{table}[h!]
\centering
\begin{tabular}{|c|c|c|}
\hline
 Cловарь & Буфер & Код  \\ \hline
0'л' & [ ,  ,  ,  ,  ,  ,  ,  ,  , л] & л
\\ \hline
0'и' & [ ,  ,  ,  ,  ,  ,  ,  , л, и] & и
\\ \hline
1<8,2> & [ ,  ,  ,  ,  ,  , л, и, л, и] & ли
\\ \hline
0'я' & [ ,  ,  ,  ,  , л, и, л, и, я] & я
\\ \hline
0' ' & [ ,  ,  ,  , л, и, л, и, я,  ] &  
\\ \hline
1<4,2> & [ ,  , л, и, л, и, я,  , л, и] & ли
\\ \hline
0'с' & [ , л, и, л, и, я,  , л, и, с] & с
\\ \hline
1<6,4> & [и, я,  , л, и, с,  , л, и, с] &  лис
\\ \hline
0'а' & [я,  , л, и, с,  , л, и, с, а] & а
\\ \hline
1<1,4> & [с,  , л, и, с, а,  , л, и, с] &  лис
\\ \hline
0'т' & [ , л, и, с, а,  , л, и, с, т] & т
\\ \hline
\end{tabular}
\end{table}

Результат: лилия лис лиса лист
\pagebreak
\paragraph{Задание 5.4 Декодировать строку(LZ78)\\}

Исходная строка: [0'л'] [0'о'] [0'с'] [2'с'] [0'ь'] [0' '] [1'о'] [3'ь'] [6'о'] [8' '] [4'а']\\
\begin{table}[h!]
\centering
\begin{tabular}{|c|c|c|}
\hline
 Cловарь & Буфер & Код  \\ \hline
 & [] & 
\\ \hline
0'л' & [, л] & л
\\ \hline
0'о' & [, л, о] & о
\\ \hline
0'с' & [, л, о, с] & с
\\ \hline
2'с' & [, л, о, с, ос] & ос
\\ \hline
0'ь' & [, л, о, с, ос, ь] & ь
\\ \hline
0' ' & [, л, о, с, ос, ь,  ] &  
\\ \hline
1'о' & [, л, о, с, ос, ь,  , ло] & ло
\\ \hline
3'ь' & [, л, о, с, ос, ь,  , ло, сь] & сь
\\ \hline
6'о' & [, л, о, с, ос, ь,  , ло, сь,  о] &  о
\\ \hline
8' ' & [, л, о, с, ос, ь,  , ло, сь,  о, сь ] & сь 
\\ \hline
4'а' & [, л, о, с, ос, ь,  , ло, сь,  о, сь , оса] & оса
\\ \hline
\end{tabular}
\end{table}

Результат: лосось лось ось оса
\pagebreak
\subsection{Вариант №5}
\paragraph{Задание 1. Блочный хаффман \\}

Строка ТОКООКККТК, размер блока: 2
\begin{center}
 \begin{tabular}{ |c|c|l| } 
  \hline
     Буква & Вероятность & Код\\ \hline
К & 0.50 & 0\\\hline
О & 0.30 & 11\\\hline
Т & 0.20 & 10
\\ \hline \end{tabular}
\end{center}
Энтропия алфавита: 1.4855
\begin{center}
 \begin{tabular}{ |c|c|l| } 
  \hline
     Блок & Вероятность & Код\\ \hline
КК & 0.25 & 01\\\hline
КО & 0.15 & 101\\\hline
ОК & 0.15 & 110\\\hline
ТК & 0.10 & 000\\\hline
КТ & 0.10 & 001\\\hline
ОО & 0.09 & 1111\\\hline
ОТ & 0.06 & 1001\\\hline
ТО & 0.06 & 1110\\\hline
ТТ & 0.04 & 1000
\\ \hline \end{tabular}
\end{center}
Бит на символ при посимвольном кодировании: 1.5000, при блочном: 1.5000

\includegraphics[width=0.5\linewidth]{/home/fizlrock/data/files/backup/code_backup/hobby/algoritms/LabExecutor/app/./doc_src/images/825197170.jpg}

\includegraphics[width=0.9\linewidth]{/home/fizlrock/data/files/backup/code_backup/hobby/algoritms/LabExecutor/app/./doc_src/images/1734134071.jpg}
\pagebreak
\paragraph{Задание 2. Сжать адаптивным хаффманом\\}

Строка: 
РККЕАРРООО\\
Результат: 'Р' 0'К' 01 00'Е' 000'А' 10 10 1100'О' 11101 00

\includegraphics[width=0.8\linewidth]{/home/fizlrock/data/files/backup/code_backup/hobby/algoritms/LabExecutor/app/./doc_src/images/1551014741.jpg}

\includegraphics[width=0.8\linewidth]{/home/fizlrock/data/files/backup/code_backup/hobby/algoritms/LabExecutor/app/./doc_src/images/1056432200.jpg}

\includegraphics[width=0.8\linewidth]{/home/fizlrock/data/files/backup/code_backup/hobby/algoritms/LabExecutor/app/./doc_src/images/1235922462.jpg}

\includegraphics[width=0.8\linewidth]{/home/fizlrock/data/files/backup/code_backup/hobby/algoritms/LabExecutor/app/./doc_src/images/1924728261.jpg}

\includegraphics[width=0.8\linewidth]{/home/fizlrock/data/files/backup/code_backup/hobby/algoritms/LabExecutor/app/./doc_src/images/1891417173.jpg}

\includegraphics[width=0.8\linewidth]{/home/fizlrock/data/files/backup/code_backup/hobby/algoritms/LabExecutor/app/./doc_src/images/1533934047.jpg}

\includegraphics[width=0.8\linewidth]{/home/fizlrock/data/files/backup/code_backup/hobby/algoritms/LabExecutor/app/./doc_src/images/1517940554.jpg}

\includegraphics[width=0.8\linewidth]{/home/fizlrock/data/files/backup/code_backup/hobby/algoritms/LabExecutor/app/./doc_src/images/1999906634.jpg}

\includegraphics[width=0.8\linewidth]{/home/fizlrock/data/files/backup/code_backup/hobby/algoritms/LabExecutor/app/./doc_src/images/658208360.jpg}

\includegraphics[width=0.8\linewidth]{/home/fizlrock/data/files/backup/code_backup/hobby/algoritms/LabExecutor/app/./doc_src/images/1223167200.jpg}
\pagebreak

\paragraph{Задание 3.2}

Закодировать сообщение методом LZSS\\
Строка:ОСЫ\_ОСЫ\_СЫПЬ\_НАСЫПЬ\\
Результат: 0'О' 0'С' 0'Ы' 0'\_' 1<6,4> 1<3,2> 0'П' 0'Ь' 1<1,1> 0'Н' 0'А' 1<3,4>\\
\begin{table}[h!]
\centering
\begin{tabular}{|c|c|c|c|c|c|c|c|c|c|c|c|c|c|c|c|c|}
\hline
\multicolumn{10}{|c|}{Cловарь} & \multicolumn{6}{c|}{Буфер} & Код  \\ \hline
  &   &   &   &   &   &   &   &   &   & О & С & Ы & \_ & О & С & 0'О'\\ \hline
  &   &   &   &   &   &   &   &   & О & С & Ы & \_ & О & С & Ы & 0'С'\\ \hline
  &   &   &   &   &   &   &   & О & С & Ы & \_ & О & С & Ы & \_ & 0'Ы'\\ \hline
  &   &   &   &   &   &   & О & С & Ы & \_ & О & С & Ы & \_ & С & 0'\_'\\ \hline
  &   &   &   &   &   & \cellcolor[HTML]{FFFF00} О & \cellcolor[HTML]{FFFF00} С & \cellcolor[HTML]{FFFF00} Ы & \cellcolor[HTML]{FFFF00} \_ & \cellcolor[HTML]{FFFF00} О & \cellcolor[HTML]{FFFF00} С & \cellcolor[HTML]{FFFF00} Ы & \cellcolor[HTML]{FFFF00} \_ & С & Ы & 1<6,4>\\ \hline
  &   & О & \cellcolor[HTML]{FFFF00} С & \cellcolor[HTML]{FFFF00} Ы & \_ & О & С & Ы & \_ & \cellcolor[HTML]{FFFF00} С & \cellcolor[HTML]{FFFF00} Ы & П & Ь & \_ & Н & 1<3,2>\\ \hline
О & С & Ы & \_ & О & С & Ы & \_ & С & Ы & П & Ь & \_ & Н & А & С & 0'П'\\ \hline
С & Ы & \_ & О & С & Ы & \_ & С & Ы & П & Ь & \_ & Н & А & С & Ы & 0'Ь'\\ \hline
Ы & \cellcolor[HTML]{FFFF00} \_ & О & С & Ы & \_ & С & Ы & П & Ь & \cellcolor[HTML]{FFFF00} \_ & Н & А & С & Ы & П & 1<1,1>\\ \hline
\_ & О & С & Ы & \_ & С & Ы & П & Ь & \_ & Н & А & С & Ы & П & Ь & 0'Н'\\ \hline
О & С & Ы & \_ & С & Ы & П & Ь & \_ & Н & А & С & Ы & П & Ь &   & 0'А'\\ \hline
С & Ы & \_ & \cellcolor[HTML]{FFFF00} С & \cellcolor[HTML]{FFFF00} Ы & \cellcolor[HTML]{FFFF00} П & \cellcolor[HTML]{FFFF00} Ь & \_ & Н & А & \cellcolor[HTML]{FFFF00} С & \cellcolor[HTML]{FFFF00} Ы & \cellcolor[HTML]{FFFF00} П & \cellcolor[HTML]{FFFF00} Ь &   &   & 1<3,4>\\ \hline
\end{tabular}
\end{table}

\paragraph{Задание 3.3}

Закодировать сообщение методом LZ78\\
Строка:ОСЫ\_ОСЫ\_СЫПЬ\_НАСЫПЬ\\
\begin{table}[h!]
\centering
\begin{tabular}{|c|c|c|} 
\hline
 Входная фраза (в словарь) & Код & Позиция словаря \\ \hline

 &  & 0 \\ \hline
О & 0'О' & 1 \\ \hline
С & 0'С' & 2 \\ \hline
Ы & 0'Ы' & 3 \\ \hline
\_ & 0'\_' & 4 \\ \hline
ОС & 1'С' & 5 \\ \hline
Ы\_ & 3'\_' & 6 \\ \hline
СЫ & 2'Ы' & 7 \\ \hline
П & 0'П' & 8 \\ \hline
Ь & 0'Ь' & 9 \\ \hline
\_Н & 4'Н' & 10 \\ \hline
А & 0'А' & 11 \\ \hline
СЫП & 7'П' & 12 \\ \hline
\end{tabular}
\end{table}

Результат: 0'О' 0'С' 0'Ы' 0'\_' 1'С' 3'\_' 2'Ы' 0'П' 0'Ь' 4'Н' 0'А' 7'П'\\
\pagebreak
\paragraph{Задание 4. Арифметическое кодирование\\}

Исходная строка: РККЕАРРООО\
\begin{center}
 \begin{tabular}{ |c|c| } 
  \hline
     Буква & Вероятность \\ \hline
Р & 0.30\\\hline
О & 0.30\\\hline
К & 0.20\\\hline
А & 0.10\\\hline
Е & 0.10
\\ \hline \end{tabular}
\end{center}
\begin{center}
 \begin{tabular}{ |c|c|c| } 
  \hline
     Буква & Начало & Конец \\ \hline
Р & 0.00 & 0.30\\\hline
О & 0.30 & 0.60\\\hline
К & 0.60 & 0.80\\\hline
А & 0.80 & 0.90\\\hline
Е & 0.90 & 1.00
\\ \hline \end{tabular}
\end{center}
\begin{center}
 \begin{tabular}{ |c|c|c|c| } 
  \hline
     Буква & delta & min & max \\ \hline
Р & 0.3000000000 & 0.0000000000 & 0.3000000000\\\hline
К & 0.0600000000 & 0.1800000000 & 0.2400000000\\\hline
К & 0.0120000000 & 0.2160000000 & 0.2280000000\\\hline
Е & 0.0012000000 & 0.2268000000 & 0.2280000000\\\hline
А & 0.0001200000 & 0.2277600000 & 0.2278800000\\\hline
Р & 0.0000360000 & 0.2277600000 & 0.2277960000\\\hline
Р & 0.0000108000 & 0.2277600000 & 0.2277708000\\\hline
О & 0.0000032400 & 0.2277632400 & 0.2277664800\\\hline
О & 0.0000009720 & 0.2277642120 & 0.2277651840\\\hline
О & 0.0000002916 & 0.2277645036 & 0.2277647952
\\ \hline \end{tabular}
\end{center}
Результат: 2277646
\pagebreak
\paragraph{Задание 5.1}

\\ 

Декодировать сообщение методом адаптивного хаффмана \\
Строка: 
'D'0'C'00'B'101100'F'11011011011101001\\
Результат: DCBBDBFDBDBDBDCBB

\includegraphics[width=0.8\linewidth]{/home/fizlrock/data/files/backup/code_backup/hobby/algoritms/LabExecutor/app/./doc_src/images/1237884354.jpg}

\includegraphics[width=0.8\linewidth]{/home/fizlrock/data/files/backup/code_backup/hobby/algoritms/LabExecutor/app/./doc_src/images/1474476860.jpg}

\includegraphics[width=0.8\linewidth]{/home/fizlrock/data/files/backup/code_backup/hobby/algoritms/LabExecutor/app/./doc_src/images/1756327583.jpg}

\includegraphics[width=0.8\linewidth]{/home/fizlrock/data/files/backup/code_backup/hobby/algoritms/LabExecutor/app/./doc_src/images/485887412.jpg}

\includegraphics[width=0.8\linewidth]{/home/fizlrock/data/files/backup/code_backup/hobby/algoritms/LabExecutor/app/./doc_src/images/1909547607.jpg}

\includegraphics[width=0.8\linewidth]{/home/fizlrock/data/files/backup/code_backup/hobby/algoritms/LabExecutor/app/./doc_src/images/914931609.jpg}

\includegraphics[width=0.8\linewidth]{/home/fizlrock/data/files/backup/code_backup/hobby/algoritms/LabExecutor/app/./doc_src/images/1887889174.jpg}

\includegraphics[width=0.8\linewidth]{/home/fizlrock/data/files/backup/code_backup/hobby/algoritms/LabExecutor/app/./doc_src/images/1709251259.jpg}

\includegraphics[width=0.8\linewidth]{/home/fizlrock/data/files/backup/code_backup/hobby/algoritms/LabExecutor/app/./doc_src/images/1870616064.jpg}

\includegraphics[width=0.8\linewidth]{/home/fizlrock/data/files/backup/code_backup/hobby/algoritms/LabExecutor/app/./doc_src/images/1983616638.jpg}

\includegraphics[width=0.8\linewidth]{/home/fizlrock/data/files/backup/code_backup/hobby/algoritms/LabExecutor/app/./doc_src/images/1821308649.jpg}

\includegraphics[width=0.8\linewidth]{/home/fizlrock/data/files/backup/code_backup/hobby/algoritms/LabExecutor/app/./doc_src/images/1527014055.jpg}

\includegraphics[width=0.8\linewidth]{/home/fizlrock/data/files/backup/code_backup/hobby/algoritms/LabExecutor/app/./doc_src/images/950468321.jpg}

\includegraphics[width=0.8\linewidth]{/home/fizlrock/data/files/backup/code_backup/hobby/algoritms/LabExecutor/app/./doc_src/images/304505162.jpg}

\includegraphics[width=0.8\linewidth]{/home/fizlrock/data/files/backup/code_backup/hobby/algoritms/LabExecutor/app/./doc_src/images/1495252669.jpg}

\includegraphics[width=0.8\linewidth]{/home/fizlrock/data/files/backup/code_backup/hobby/algoritms/LabExecutor/app/./doc_src/images/1821045054.jpg}

\includegraphics[width=0.8\linewidth]{/home/fizlrock/data/files/backup/code_backup/hobby/algoritms/LabExecutor/app/./doc_src/images/1402197105.jpg}
\pagebreak
\paragraph{Задание 5.3 Декодировать строку(LZSS)\\}

Исходная строка: [0'с'] [0'и'] [0'л'] [0'а'] [0' '] [1<5,1>] [0'е'] [1<5,1>] [0'о'] [1<5,1>][1<7,2>] [1<3,1>] [0'ь'] [1<0,1>] [1<6,2>] [0'а']\\
\begin{table}[h!]
\centering
\begin{tabular}{|c|c|c|}
\hline
 Cловарь & Буфер & Код  \\ \hline
0'с' & [ ,  ,  ,  ,  ,  ,  ,  ,  , с] & с
\\ \hline
0'и' & [ ,  ,  ,  ,  ,  ,  ,  , с, и] & и
\\ \hline
0'л' & [ ,  ,  ,  ,  ,  ,  , с, и, л] & л
\\ \hline
0'а' & [ ,  ,  ,  ,  ,  , с, и, л, а] & а
\\ \hline
0' ' & [ ,  ,  ,  ,  , с, и, л, а,  ] &  
\\ \hline
1<5,1> & [ ,  ,  ,  , с, и, л, а,  , с] & с
\\ \hline
0'е' & [ ,  ,  , с, и, л, а,  , с, е] & е
\\ \hline
1<5,1> & [ ,  , с, и, л, а,  , с, е, л] & л
\\ \hline
0'о' & [ , с, и, л, а,  , с, е, л, о] & о
\\ \hline
1<5,1> & [с, и, л, а,  , с, е, л, о,  ] &  
\\ \hline
1<7,2> & [л, а,  , с, е, л, о,  , л, о] & ло
\\ \hline
1<3,1> & [а,  , с, е, л, о,  , л, о, с] & с
\\ \hline
0'ь' & [ , с, е, л, о,  , л, о, с, ь] & ь
\\ \hline
1<0,1> & [с, е, л, о,  , л, о, с, ь,  ] &  
\\ \hline
1<6,2> & [л, о,  , л, о, с, ь,  , о, с] & ос
\\ \hline
0'а' & [о,  , л, о, с, ь,  , о, с, а] & а
\\ \hline
\end{tabular}
\end{table}

Результат: сила село лось оса
\pagebreak
\paragraph{Задание 5.4 Декодировать строку(LZ78)\\}

Исходная строка: [0'л'] [0'е'] [0'с'] [0' '] [1'е'] [3'а'] [4'л'] [2'с'] [0'к'] [0'а'] [7'е'] [3'о'] [0'к']\\
\begin{table}[h!]
\centering
\begin{tabular}{|c|c|c|}
\hline
 Cловарь & Буфер & Код  \\ \hline
 & [] & 
\\ \hline
0'л' & [, л] & л
\\ \hline
0'е' & [, л, е] & е
\\ \hline
0'с' & [, л, е, с] & с
\\ \hline
0' ' & [, л, е, с,  ] &  
\\ \hline
1'е' & [, л, е, с,  , ле] & ле
\\ \hline
3'а' & [, л, е, с,  , ле, са] & са
\\ \hline
4'л' & [, л, е, с,  , ле, са,  л] &  л
\\ \hline
2'с' & [, л, е, с,  , ле, са,  л, ес] & ес
\\ \hline
0'к' & [, л, е, с,  , ле, са,  л, ес, к] & к
\\ \hline
0'а' & [, л, е, с,  , ле, са,  л, ес, к, а] & а
\\ \hline
7'е' & [, л, е, с,  , ле, са,  л, ес, к, а,  ле] &  ле
\\ \hline
3'о' & [, л, е, с,  , ле, са,  л, ес, к, а,  ле, со] & со
\\ \hline
0'к' & [, л, е, с,  , ле, са,  л, ес, к, а,  ле, со, к] & к
\\ \hline
\end{tabular}
\end{table}

Результат: лес леса леска лесок
\pagebreak
\subsection{Вариант №6}
\paragraph{Задание 1. Блочный хаффман \\}

Строка КООКЛЛЛЛЛЛ, размер блока: 2
\begin{center}
 \begin{tabular}{ |c|c|l| } 
  \hline
     Буква & Вероятность & Код\\ \hline
Л & 0.60 & 1\\\hline
К & 0.20 & 00\\\hline
О & 0.20 & 01
\\ \hline \end{tabular}
\end{center}
Энтропия алфавита: 1.3710
\begin{center}
 \begin{tabular}{ |c|c|l| } 
  \hline
     Блок & Вероятность & Код\\ \hline
ЛЛ & 0.36 & 11\\\hline
КЛ & 0.12 & 010\\\hline
ЛО & 0.12 & 011\\\hline
ОЛ & 0.12 & 100\\\hline
ЛК & 0.12 & 101\\\hline
КК & 0.04 & 0000\\\hline
ОО & 0.04 & 0001\\\hline
КО & 0.04 & 0010\\\hline
ОК & 0.04 & 0011
\\ \hline \end{tabular}
\end{center}
Бит на символ при посимвольном кодировании: 1.4000, при блочном: 1.4000

\includegraphics[width=0.5\linewidth]{/home/fizlrock/data/files/backup/code_backup/hobby/algoritms/LabExecutor/app/./doc_src/images/1428285273.jpg}

\includegraphics[width=0.9\linewidth]{/home/fizlrock/data/files/backup/code_backup/hobby/algoritms/LabExecutor/app/./doc_src/images/121103270.jpg}
\pagebreak
\paragraph{Задание 2. Сжать адаптивным хаффманом\\}

Строка: 
СРОССКРРРР\\
Результат: 'С' 0'Р' 00'О' 0 0 000'К' 00 10 11 0

\includegraphics[width=0.8\linewidth]{/home/fizlrock/data/files/backup/code_backup/hobby/algoritms/LabExecutor/app/./doc_src/images/1757159978.jpg}

\includegraphics[width=0.8\linewidth]{/home/fizlrock/data/files/backup/code_backup/hobby/algoritms/LabExecutor/app/./doc_src/images/939309388.jpg}

\includegraphics[width=0.8\linewidth]{/home/fizlrock/data/files/backup/code_backup/hobby/algoritms/LabExecutor/app/./doc_src/images/1214568008.jpg}

\includegraphics[width=0.8\linewidth]{/home/fizlrock/data/files/backup/code_backup/hobby/algoritms/LabExecutor/app/./doc_src/images/2099656114.jpg}

\includegraphics[width=0.8\linewidth]{/home/fizlrock/data/files/backup/code_backup/hobby/algoritms/LabExecutor/app/./doc_src/images/1859982613.jpg}

\includegraphics[width=0.8\linewidth]{/home/fizlrock/data/files/backup/code_backup/hobby/algoritms/LabExecutor/app/./doc_src/images/1803660820.jpg}

\includegraphics[width=0.8\linewidth]{/home/fizlrock/data/files/backup/code_backup/hobby/algoritms/LabExecutor/app/./doc_src/images/1605648271.jpg}

\includegraphics[width=0.8\linewidth]{/home/fizlrock/data/files/backup/code_backup/hobby/algoritms/LabExecutor/app/./doc_src/images/1740062570.jpg}

\includegraphics[width=0.8\linewidth]{/home/fizlrock/data/files/backup/code_backup/hobby/algoritms/LabExecutor/app/./doc_src/images/1987125884.jpg}

\includegraphics[width=0.8\linewidth]{/home/fizlrock/data/files/backup/code_backup/hobby/algoritms/LabExecutor/app/./doc_src/images/160654923.jpg}
\pagebreak

\paragraph{Задание 3.2}

Закодировать сообщение методом LZSS\\
Строка:КУСКУС\_КУСАКА\_СОБАКА\\
Результат: 0'К' 0'У' 0'С' 1<7,3> 0'\_' 1<3,3> 0'А' 1<2,1> 1<8,1> 1<3,1> 1<1,1> 0'О' 0'Б' 1<3,3>\\
\begin{table}[h!]
\centering
\begin{tabular}{|c|c|c|c|c|c|c|c|c|c|c|c|c|c|c|c|c|}
\hline
\multicolumn{10}{|c|}{Cловарь} & \multicolumn{6}{c|}{Буфер} & Код  \\ \hline
  &   &   &   &   &   &   &   &   &   & К & У & С & К & У & С & 0'К'\\ \hline
  &   &   &   &   &   &   &   &   & К & У & С & К & У & С & \_ & 0'У'\\ \hline
  &   &   &   &   &   &   &   & К & У & С & К & У & С & \_ & К & 0'С'\\ \hline
  &   &   &   &   &   &   & \cellcolor[HTML]{FFFF00} К & \cellcolor[HTML]{FFFF00} У & \cellcolor[HTML]{FFFF00} С & \cellcolor[HTML]{FFFF00} К & \cellcolor[HTML]{FFFF00} У & \cellcolor[HTML]{FFFF00} С & \_ & К & У & 1<7,3>\\ \hline
  &   &   &   & К & У & С & К & У & С & \_ & К & У & С & А & К & 0'\_'\\ \hline
  &   &   & \cellcolor[HTML]{FFFF00} К & \cellcolor[HTML]{FFFF00} У & \cellcolor[HTML]{FFFF00} С & К & У & С & \_ & \cellcolor[HTML]{FFFF00} К & \cellcolor[HTML]{FFFF00} У & \cellcolor[HTML]{FFFF00} С & А & К & А & 1<3,3>\\ \hline
К & У & С & К & У & С & \_ & К & У & С & А & К & А & \_ & С & О & 0'А'\\ \hline
У & С & \cellcolor[HTML]{FFFF00} К & У & С & \_ & К & У & С & А & \cellcolor[HTML]{FFFF00} К & А & \_ & С & О & Б & 1<2,1>\\ \hline
С & К & У & С & \_ & К & У & С & \cellcolor[HTML]{FFFF00} А & К & \cellcolor[HTML]{FFFF00} А & \_ & С & О & Б & А & 1<8,1>\\ \hline
К & У & С & \cellcolor[HTML]{FFFF00} \_ & К & У & С & А & К & А & \cellcolor[HTML]{FFFF00} \_ & С & О & Б & А & К & 1<3,1>\\ \hline
У & \cellcolor[HTML]{FFFF00} С & \_ & К & У & С & А & К & А & \_ & \cellcolor[HTML]{FFFF00} С & О & Б & А & К & А & 1<1,1>\\ \hline
С & \_ & К & У & С & А & К & А & \_ & С & О & Б & А & К & А &   & 0'О'\\ \hline
\_ & К & У & С & А & К & А & \_ & С & О & Б & А & К & А &   &   & 0'Б'\\ \hline
К & У & С & \cellcolor[HTML]{FFFF00} А & \cellcolor[HTML]{FFFF00} К & \cellcolor[HTML]{FFFF00} А & \_ & С & О & Б & \cellcolor[HTML]{FFFF00} А & \cellcolor[HTML]{FFFF00} К & \cellcolor[HTML]{FFFF00} А &   &   &   & 1<3,3>\\ \hline
\end{tabular}
\end{table}

\paragraph{Задание 3.3}

Закодировать сообщение методом LZ78\\
Строка:КУСКУС\_КУСАКА\_СОБАКА\\
\begin{table}[h!]
\centering
\begin{tabular}{|c|c|c|} 
\hline
 Входная фраза (в словарь) & Код & Позиция словаря \\ \hline

 &  & 0 \\ \hline
К & 0'К' & 1 \\ \hline
У & 0'У' & 2 \\ \hline
С & 0'С' & 3 \\ \hline
КУ & 1'У' & 4 \\ \hline
С\_ & 3'\_' & 5 \\ \hline
КУС & 4'С' & 6 \\ \hline
А & 0'А' & 7 \\ \hline
КА & 1'А' & 8 \\ \hline
\_ & 0'\_' & 9 \\ \hline
СО & 3'О' & 10 \\ \hline
Б & 0'Б' & 11 \\ \hline
АК & 7'К' & 12 \\ \hline
\end{tabular}
\end{table}

Результат: 0'К' 0'У' 0'С' 1'У' 3'\_' 4'С' 0'А' 1'А' 0'\_' 3'О' 0'Б' 7'К'\\
\pagebreak
\paragraph{Задание 4. Арифметическое кодирование\\}

Исходная строка: СРОССКРРРР\
\begin{center}
 \begin{tabular}{ |c|c| } 
  \hline
     Буква & Вероятность \\ \hline
Р & 0.50\\\hline
С & 0.30\\\hline
К & 0.10\\\hline
О & 0.10
\\ \hline \end{tabular}
\end{center}
\begin{center}
 \begin{tabular}{ |c|c|c| } 
  \hline
     Буква & Начало & Конец \\ \hline
Р & 0.00 & 0.50\\\hline
С & 0.50 & 0.80\\\hline
К & 0.80 & 0.90\\\hline
О & 0.90 & 1.00
\\ \hline \end{tabular}
\end{center}
\begin{center}
 \begin{tabular}{ |c|c|c|c| } 
  \hline
     Буква & delta & min & max \\ \hline
С & 0.3000000000 & 0.5000000000 & 0.8000000000\\\hline
Р & 0.1500000000 & 0.5000000000 & 0.6500000000\\\hline
О & 0.0150000000 & 0.6350000000 & 0.6500000000\\\hline
С & 0.0045000000 & 0.6425000000 & 0.6470000000\\\hline
С & 0.0013500000 & 0.6447500000 & 0.6461000000\\\hline
К & 0.0001350000 & 0.6458300000 & 0.6459650000\\\hline
Р & 0.0000675000 & 0.6458300000 & 0.6458975000\\\hline
Р & 0.0000337500 & 0.6458300000 & 0.6458637500\\\hline
Р & 0.0000168750 & 0.6458300000 & 0.6458468750\\\hline
Р & 0.0000084375 & 0.6458300000 & 0.6458384375
\\ \hline \end{tabular}
\end{center}
Результат: 64583
\pagebreak
\paragraph{Задание 5.1}

\\ 

Декодировать сообщение методом адаптивного хаффмана \\
Строка: 
'G'0'H'00'F'100'D'000'C'100110100100\\
Результат: GHFDCGFGCC

\includegraphics[width=0.8\linewidth]{/home/fizlrock/data/files/backup/code_backup/hobby/algoritms/LabExecutor/app/./doc_src/images/947643929.jpg}

\includegraphics[width=0.8\linewidth]{/home/fizlrock/data/files/backup/code_backup/hobby/algoritms/LabExecutor/app/./doc_src/images/1599495176.jpg}

\includegraphics[width=0.8\linewidth]{/home/fizlrock/data/files/backup/code_backup/hobby/algoritms/LabExecutor/app/./doc_src/images/1230493783.jpg}

\includegraphics[width=0.8\linewidth]{/home/fizlrock/data/files/backup/code_backup/hobby/algoritms/LabExecutor/app/./doc_src/images/2088238748.jpg}

\includegraphics[width=0.8\linewidth]{/home/fizlrock/data/files/backup/code_backup/hobby/algoritms/LabExecutor/app/./doc_src/images/1834294690.jpg}

\includegraphics[width=0.8\linewidth]{/home/fizlrock/data/files/backup/code_backup/hobby/algoritms/LabExecutor/app/./doc_src/images/2052380028.jpg}

\includegraphics[width=0.8\linewidth]{/home/fizlrock/data/files/backup/code_backup/hobby/algoritms/LabExecutor/app/./doc_src/images/230879808.jpg}

\includegraphics[width=0.8\linewidth]{/home/fizlrock/data/files/backup/code_backup/hobby/algoritms/LabExecutor/app/./doc_src/images/820801023.jpg}

\includegraphics[width=0.8\linewidth]{/home/fizlrock/data/files/backup/code_backup/hobby/algoritms/LabExecutor/app/./doc_src/images/1091206044.jpg}

\includegraphics[width=0.8\linewidth]{/home/fizlrock/data/files/backup/code_backup/hobby/algoritms/LabExecutor/app/./doc_src/images/789184648.jpg}
\pagebreak
\paragraph{Задание 5.3 Декодировать строку(LZSS)\\}

Исходная строка: [0'л'] [0'е'] [0'с'] [0' '] [1<6,3>] [0'о'] [0'к'] [1<4,1>] [1<6,4>] [1<2,3>][1<0,1>] [0'л']\\
\begin{table}[h!]
\centering
\begin{tabular}{|c|c|c|}
\hline
 Cловарь & Буфер & Код  \\ \hline
0'л' & [ ,  ,  ,  ,  ,  ,  ,  ,  , л] & л
\\ \hline
0'е' & [ ,  ,  ,  ,  ,  ,  ,  , л, е] & е
\\ \hline
0'с' & [ ,  ,  ,  ,  ,  ,  , л, е, с] & с
\\ \hline
0' ' & [ ,  ,  ,  ,  ,  , л, е, с,  ] &  
\\ \hline
1<6,3> & [ ,  ,  , л, е, с,  , л, е, с] & лес
\\ \hline
0'о' & [ ,  , л, е, с,  , л, е, с, о] & о
\\ \hline
0'к' & [ , л, е, с,  , л, е, с, о, к] & к
\\ \hline
1<4,1> & [л, е, с,  , л, е, с, о, к,  ] &  
\\ \hline
1<6,4> & [л, е, с, о, к,  , с, о, к,  ] & сок 
\\ \hline
1<2,3> & [о, к,  , с, о, к,  , с, о, к] & сок
\\ \hline
1<0,1> & [к,  , с, о, к,  , с, о, к, о] & о
\\ \hline
0'л' & [ , с, о, к,  , с, о, к, о, л] & л
\\ \hline
\end{tabular}
\end{table}

Результат: лес лесок сок сокол
\pagebreak
\paragraph{Задание 5.4 Декодировать строку(LZ78)\\}

Исходная строка: [0'с'] [0'о'] [1'у'] [0'д'] [0' '] [3'д'] [0'н'] [2' '] [6' '] [4'н'] [0'о']\\
\begin{table}[h!]
\centering
\begin{tabular}{|c|c|c|}
\hline
 Cловарь & Буфер & Код  \\ \hline
 & [] & 
\\ \hline
0'с' & [, с] & с
\\ \hline
0'о' & [, с, о] & о
\\ \hline
1'у' & [, с, о, су] & су
\\ \hline
0'д' & [, с, о, су, д] & д
\\ \hline
0' ' & [, с, о, су, д,  ] &  
\\ \hline
3'д' & [, с, о, су, д,  , суд] & суд
\\ \hline
0'н' & [, с, о, су, д,  , суд, н] & н
\\ \hline
2' ' & [, с, о, су, д,  , суд, н, о ] & о 
\\ \hline
6' ' & [, с, о, су, д,  , суд, н, о , суд ] & суд 
\\ \hline
4'н' & [, с, о, су, д,  , суд, н, о , суд , дн] & дн
\\ \hline
0'о' & [, с, о, су, д,  , суд, н, о , суд , дн, о] & о
\\ \hline
\end{tabular}
\end{table}

Результат: сосуд судно суд дно
\pagebreak
\subsection{Вариант №7}
\paragraph{Задание 1. Блочный хаффман \\}

Строка ТТУТТТТТТТ, размер блока: 3
\begin{center}
 \begin{tabular}{ |c|c|l| } 
  \hline
     Буква & Вероятность & Код\\ \hline
Т & 0.90 & 1\\\hline
У & 0.10 & 0
\\ \hline \end{tabular}
\end{center}
Энтропия алфавита: 0.4690
\begin{center}
 \begin{tabular}{ |c|c|l| } 
  \hline
     Блок & Вероятность & Код\\ \hline
ТТТ & 0.73 & 1\\\hline
ТУТ & 0.08 & 001\\\hline
ТТУ & 0.08 & 010\\\hline
УТТ & 0.08 & 011\\\hline
УУТ & 0.01 & 00011\\\hline
ТУУ & 0.01 & 00001\\\hline
УТУ & 0.01 & 00010\\\hline
УУУ & 0.00 & 00000
\\ \hline \end{tabular}
\end{center}
Бит на символ при посимвольном кодировании: 1.0000, при блочном: 0.5327

\includegraphics[width=0.5\linewidth]{/home/fizlrock/data/files/backup/code_backup/hobby/algoritms/LabExecutor/app/./doc_src/images/1057538162.jpg}

\includegraphics[width=0.9\linewidth]{/home/fizlrock/data/files/backup/code_backup/hobby/algoritms/LabExecutor/app/./doc_src/images/873107585.jpg}
\pagebreak
\paragraph{Задание 2. Сжать адаптивным хаффманом\\}

Строка: 
ОРОПАВРРРР\\
Результат: 'О' 0'Р' 1 00'П' 000'А' 1100'В' 00 01 11 0

\includegraphics[width=0.8\linewidth]{/home/fizlrock/data/files/backup/code_backup/hobby/algoritms/LabExecutor/app/./doc_src/images/2060315250.jpg}

\includegraphics[width=0.8\linewidth]{/home/fizlrock/data/files/backup/code_backup/hobby/algoritms/LabExecutor/app/./doc_src/images/567880466.jpg}

\includegraphics[width=0.8\linewidth]{/home/fizlrock/data/files/backup/code_backup/hobby/algoritms/LabExecutor/app/./doc_src/images/103698698.jpg}

\includegraphics[width=0.8\linewidth]{/home/fizlrock/data/files/backup/code_backup/hobby/algoritms/LabExecutor/app/./doc_src/images/1220951346.jpg}

\includegraphics[width=0.8\linewidth]{/home/fizlrock/data/files/backup/code_backup/hobby/algoritms/LabExecutor/app/./doc_src/images/1370384361.jpg}

\includegraphics[width=0.8\linewidth]{/home/fizlrock/data/files/backup/code_backup/hobby/algoritms/LabExecutor/app/./doc_src/images/1163201439.jpg}

\includegraphics[width=0.8\linewidth]{/home/fizlrock/data/files/backup/code_backup/hobby/algoritms/LabExecutor/app/./doc_src/images/249200535.jpg}

\includegraphics[width=0.8\linewidth]{/home/fizlrock/data/files/backup/code_backup/hobby/algoritms/LabExecutor/app/./doc_src/images/1525994912.jpg}

\includegraphics[width=0.8\linewidth]{/home/fizlrock/data/files/backup/code_backup/hobby/algoritms/LabExecutor/app/./doc_src/images/2019962342.jpg}

\includegraphics[width=0.8\linewidth]{/home/fizlrock/data/files/backup/code_backup/hobby/algoritms/LabExecutor/app/./doc_src/images/1634585429.jpg}
\pagebreak
\paragraph{Задание 3.1}

Закодировать сообщение методом LZ77\\
Строка:РОЗА\_РОЗАРИЙ\_ЗАРЯДКА\\
Результат: <0,0,Р> <0,0,О> <0,0,З> <0,0,А> <0,0,\_> <5,4,Р> <0,0,И> <0,0,Й> <2,1,З> <4,2,Я> <0,0,Д> <0,0,К> <0,0,А>\\
\begin{table}[h!]
\centering
\begin{tabular}{|c|c|c|c|c|c|c|c|c|c|c|c|c|c|c|c|c|} 
\hline
\multicolumn{10}{|c|}{Cловарь} & \multicolumn{6}{c|}{Буфер} & Код  \\ \hline
  &   &   &   &   &   &   &   &   &   & \cellcolor[HTML]{8CE4F6} Р & О & З & А &   & Р & <0,0,Р>
\\ \hline
  &   &   &   &   &   &   &   &   & Р & \cellcolor[HTML]{8CE4F6} О & З & А &   & Р & О & <0,0,О>
\\ \hline
  &   &   &   &   &   &   &   & Р & О & \cellcolor[HTML]{8CE4F6} З & А &   & Р & О & З & <0,0,З>
\\ \hline
  &   &   &   &   &   &   & Р & О & З & \cellcolor[HTML]{8CE4F6} А &   & Р & О & З & А & <0,0,А>
\\ \hline
  &   &   &   &   &   & Р & О & З & А & \cellcolor[HTML]{8CE4F6}   & Р & О & З & А & Р & <0,0,\_>
\\ \hline
  &   &   &   &   & \cellcolor[HTML]{FFFF00} Р & \cellcolor[HTML]{FFFF00} О & \cellcolor[HTML]{FFFF00} З & \cellcolor[HTML]{FFFF00} А &   & \cellcolor[HTML]{FFFF00} Р & \cellcolor[HTML]{FFFF00} О & \cellcolor[HTML]{FFFF00} З & \cellcolor[HTML]{FFFF00} А & \cellcolor[HTML]{8CE4F6} Р & И & <5,4,Р>
\\ \hline
Р & О & З & А &   & Р & О & З & А & Р & \cellcolor[HTML]{8CE4F6} И & Й &   & З & А & Р & <0,0,И>
\\ \hline
О & З & А &   & Р & О & З & А & Р & И & \cellcolor[HTML]{8CE4F6} Й &   & З & А & Р & Я & <0,0,Й>
\\ \hline
З & А & \cellcolor[HTML]{FFFF00}   & Р & О & З & А & Р & И & Й & \cellcolor[HTML]{FFFF00}   & \cellcolor[HTML]{8CE4F6} З & А & Р & Я & Д & <2,1,З>
\\ \hline
  & Р & О & З & \cellcolor[HTML]{FFFF00} А & \cellcolor[HTML]{FFFF00} Р & И & Й &   & З & \cellcolor[HTML]{FFFF00} А & \cellcolor[HTML]{FFFF00} Р & \cellcolor[HTML]{8CE4F6} Я & Д & К & А & <4,2,Я>
\\ \hline
З & А & Р & И & Й &   & З & А & Р & Я & \cellcolor[HTML]{8CE4F6} Д & К & А &   &   &   & <0,0,Д>
\\ \hline
А & Р & И & Й &   & З & А & Р & Я & Д & \cellcolor[HTML]{8CE4F6} К & А &   &   &   &   & <0,0,К>
\\ \hline
Р & И & Й &   & З & А & Р & Я & Д & К & \cellcolor[HTML]{8CE4F6} А &   &   &   &   &   & <0,0,А>
\\ \hline
\end{tabular}
\end{table}

\paragraph{Задание 3.2}

Закодировать сообщение методом LZSS\\
Строка:РОЗА\_РОЗАРИЙ\_ЗАРЯДКА\\
Результат: 0'Р' 0'О' 0'З' 0'А' 0'\_' 1<5,4> 1<1,1> 0'И' 0'Й' 1<2,1> 1<4,3> 0'Я' 0'Д' 0'К' 1<5,1>\\
\begin{table}[h!]
\centering
\begin{tabular}{|c|c|c|c|c|c|c|c|c|c|c|c|c|c|c|c|c|}
\hline
\multicolumn{10}{|c|}{Cловарь} & \multicolumn{6}{c|}{Буфер} & Код  \\ \hline
  &   &   &   &   &   &   &   &   &   & Р & О & З & А & \_ & Р & 0'Р'\\ \hline
  &   &   &   &   &   &   &   &   & Р & О & З & А & \_ & Р & О & 0'О'\\ \hline
  &   &   &   &   &   &   &   & Р & О & З & А & \_ & Р & О & З & 0'З'\\ \hline
  &   &   &   &   &   &   & Р & О & З & А & \_ & Р & О & З & А & 0'А'\\ \hline
  &   &   &   &   &   & Р & О & З & А & \_ & Р & О & З & А & Р & 0'\_'\\ \hline
  &   &   &   &   & \cellcolor[HTML]{FFFF00} Р & \cellcolor[HTML]{FFFF00} О & \cellcolor[HTML]{FFFF00} З & \cellcolor[HTML]{FFFF00} А & \_ & \cellcolor[HTML]{FFFF00} Р & \cellcolor[HTML]{FFFF00} О & \cellcolor[HTML]{FFFF00} З & \cellcolor[HTML]{FFFF00} А & Р & И & 1<5,4>\\ \hline
  & \cellcolor[HTML]{FFFF00} Р & О & З & А & \_ & Р & О & З & А & \cellcolor[HTML]{FFFF00} Р & И & Й & \_ & З & А & 1<1,1>\\ \hline
Р & О & З & А & \_ & Р & О & З & А & Р & И & Й & \_ & З & А & Р & 0'И'\\ \hline
О & З & А & \_ & Р & О & З & А & Р & И & Й & \_ & З & А & Р & Я & 0'Й'\\ \hline
З & А & \cellcolor[HTML]{FFFF00} \_ & Р & О & З & А & Р & И & Й & \cellcolor[HTML]{FFFF00} \_ & З & А & Р & Я & Д & 1<2,1>\\ \hline
А & \_ & Р & О & \cellcolor[HTML]{FFFF00} З & \cellcolor[HTML]{FFFF00} А & \cellcolor[HTML]{FFFF00} Р & И & Й & \_ & \cellcolor[HTML]{FFFF00} З & \cellcolor[HTML]{FFFF00} А & \cellcolor[HTML]{FFFF00} Р & Я & Д & К & 1<4,3>\\ \hline
О & З & А & Р & И & Й & \_ & З & А & Р & Я & Д & К & А &   &   & 0'Я'\\ \hline
З & А & Р & И & Й & \_ & З & А & Р & Я & Д & К & А &   &   &   & 0'Д'\\ \hline
А & Р & И & Й & \_ & З & А & Р & Я & Д & К & А &   &   &   &   & 0'К'\\ \hline
Р & И & Й & \_ & З & \cellcolor[HTML]{FFFF00} А & Р & Я & Д & К & \cellcolor[HTML]{FFFF00} А &   &   &   &   &   & 1<5,1>\\ \hline
\end{tabular}
\end{table}

\paragraph{Задание 3.3}

Закодировать сообщение методом LZ78\\
Строка:РОЗА\_РОЗАРИЙ\_ЗАРЯДКА\\
\begin{table}[h!]
\centering
\begin{tabular}{|c|c|c|} 
\hline
 Входная фраза (в словарь) & Код & Позиция словаря \\ \hline

 &  & 0 \\ \hline
Р & 0'Р' & 1 \\ \hline
О & 0'О' & 2 \\ \hline
З & 0'З' & 3 \\ \hline
А & 0'А' & 4 \\ \hline
\_ & 0'\_' & 5 \\ \hline
РО & 1'О' & 6 \\ \hline
ЗА & 3'А' & 7 \\ \hline
РИ & 1'И' & 8 \\ \hline
Й & 0'Й' & 9 \\ \hline
\_З & 5'З' & 10 \\ \hline
АР & 4'Р' & 11 \\ \hline
Я & 0'Я' & 12 \\ \hline
Д & 0'Д' & 13 \\ \hline
К & 0'К' & 14 \\ \hline
\end{tabular}
\end{table}

Результат: 0'Р' 0'О' 0'З' 0'А' 0'\_' 1'О' 3'А' 1'И' 0'Й' 5'З' 4'Р' 0'Я' 0'Д' 0'К'\\
\pagebreak
\paragraph{Задание 4. Арифметическое кодирование\\}

Исходная строка: ОРОПАВРРРР\
\begin{center}
 \begin{tabular}{ |c|c| } 
  \hline
     Буква & Вероятность \\ \hline
Р & 0.50\\\hline
О & 0.20\\\hline
А & 0.10\\\hline
В & 0.10\\\hline
П & 0.10
\\ \hline \end{tabular}
\end{center}
\begin{center}
 \begin{tabular}{ |c|c|c| } 
  \hline
     Буква & Начало & Конец \\ \hline
Р & 0.00 & 0.50\\\hline
О & 0.50 & 0.70\\\hline
А & 0.70 & 0.80\\\hline
В & 0.80 & 0.90\\\hline
П & 0.90 & 1.00
\\ \hline \end{tabular}
\end{center}
\begin{center}
 \begin{tabular}{ |c|c|c|c| } 
  \hline
     Буква & delta & min & max \\ \hline
О & 0.2000000000 & 0.5000000000 & 0.7000000000\\\hline
Р & 0.1000000000 & 0.5000000000 & 0.6000000000\\\hline
О & 0.0200000000 & 0.5500000000 & 0.5700000000\\\hline
П & 0.0020000000 & 0.5680000000 & 0.5700000000\\\hline
А & 0.0002000000 & 0.5694000000 & 0.5696000000\\\hline
В & 0.0000200000 & 0.5695600000 & 0.5695800000\\\hline
Р & 0.0000100000 & 0.5695600000 & 0.5695700000\\\hline
Р & 0.0000050000 & 0.5695600000 & 0.5695650000\\\hline
Р & 0.0000025000 & 0.5695600000 & 0.5695625000\\\hline
Р & 0.0000012500 & 0.5695600000 & 0.5695612500
\\ \hline \end{tabular}
\end{center}
Результат: 56956
\pagebreak
\paragraph{Задание 5.1}

\\ 

Декодировать сообщение методом адаптивного хаффмана \\
Строка: 
'V'0'B'00'C'100'N'11000'F'00001101\\
Результат: VBCNBFVCBV

\includegraphics[width=0.8\linewidth]{/home/fizlrock/data/files/backup/code_backup/hobby/algoritms/LabExecutor/app/./doc_src/images/257595123.jpg}

\includegraphics[width=0.8\linewidth]{/home/fizlrock/data/files/backup/code_backup/hobby/algoritms/LabExecutor/app/./doc_src/images/1886886132.jpg}

\includegraphics[width=0.8\linewidth]{/home/fizlrock/data/files/backup/code_backup/hobby/algoritms/LabExecutor/app/./doc_src/images/1919504548.jpg}

\includegraphics[width=0.8\linewidth]{/home/fizlrock/data/files/backup/code_backup/hobby/algoritms/LabExecutor/app/./doc_src/images/65218532.jpg}

\includegraphics[width=0.8\linewidth]{/home/fizlrock/data/files/backup/code_backup/hobby/algoritms/LabExecutor/app/./doc_src/images/1541043278.jpg}

\includegraphics[width=0.8\linewidth]{/home/fizlrock/data/files/backup/code_backup/hobby/algoritms/LabExecutor/app/./doc_src/images/1981603597.jpg}

\includegraphics[width=0.8\linewidth]{/home/fizlrock/data/files/backup/code_backup/hobby/algoritms/LabExecutor/app/./doc_src/images/774974336.jpg}

\includegraphics[width=0.8\linewidth]{/home/fizlrock/data/files/backup/code_backup/hobby/algoritms/LabExecutor/app/./doc_src/images/2097092182.jpg}

\includegraphics[width=0.8\linewidth]{/home/fizlrock/data/files/backup/code_backup/hobby/algoritms/LabExecutor/app/./doc_src/images/322170953.jpg}

\includegraphics[width=0.8\linewidth]{/home/fizlrock/data/files/backup/code_backup/hobby/algoritms/LabExecutor/app/./doc_src/images/486208275.jpg}
\pagebreak
\paragraph{Задание 5.3 Декодировать строку(LZSS)\\}

Исходная строка: [0'л'] [0'о'] [0'к'] [1<8,1>] [0'н'] [0' '] [1<6,4>] [1<2,3>] [0'ь'] [1<1,3>] [1<3,2>] [0'н']\\
\begin{table}[h!]
\centering
\begin{tabular}{|c|c|c|}
\hline
 Cловарь & Буфер & Код  \\ \hline
0'л' & [ ,  ,  ,  ,  ,  ,  ,  ,  , л] & л
\\ \hline
0'о' & [ ,  ,  ,  ,  ,  ,  ,  , л, о] & о
\\ \hline
0'к' & [ ,  ,  ,  ,  ,  ,  , л, о, к] & к
\\ \hline
1<8,1> & [ ,  ,  ,  ,  ,  , л, о, к, о] & о
\\ \hline
0'н' & [ ,  ,  ,  ,  , л, о, к, о, н] & н
\\ \hline
0' ' & [ ,  ,  ,  , л, о, к, о, н,  ] &  
\\ \hline
1<6,4> & [л, о, к, о, н,  , к, о, н,  ] & кон 
\\ \hline
1<2,3> & [о, н,  , к, о, н,  , к, о, н] & кон
\\ \hline
0'ь' & [н,  , к, о, н,  , к, о, н, ь] & ь
\\ \hline
1<1,3> & [о, н,  , к, о, н, ь,  , к, о] &  ко
\\ \hline
1<3,2> & [ , к, о, н, ь,  , к, о, к, о] & ко
\\ \hline
0'н' & [к, о, н, ь,  , к, о, к, о, н] & н
\\ \hline
\end{tabular}
\end{table}

Результат: локон кон конь кокон
\pagebreak
\paragraph{Задание 5.4 Декодировать строку(LZ78)\\}

Исходная строка: [0'б'] [0'а'] [0'з'] [2'р'] [0' '] [1'а'] [0'р'] [5'з'] [4'я'] [5'а'] [0'м'] [6'р']\\
\begin{table}[h!]
\centering
\begin{tabular}{|c|c|c|}
\hline
 Cловарь & Буфер & Код  \\ \hline
 & [] & 
\\ \hline
0'б' & [, б] & б
\\ \hline
0'а' & [, б, а] & а
\\ \hline
0'з' & [, б, а, з] & з
\\ \hline
2'р' & [, б, а, з, ар] & ар
\\ \hline
0' ' & [, б, а, з, ар,  ] &  
\\ \hline
1'а' & [, б, а, з, ар,  , ба] & ба
\\ \hline
0'р' & [, б, а, з, ар,  , ба, р] & р
\\ \hline
5'з' & [, б, а, з, ар,  , ба, р,  з] &  з
\\ \hline
4'я' & [, б, а, з, ар,  , ба, р,  з, аря] & аря
\\ \hline
5'а' & [, б, а, з, ар,  , ба, р,  з, аря,  а] &  а
\\ \hline
0'м' & [, б, а, з, ар,  , ба, р,  з, аря,  а, м] & м
\\ \hline
6'р' & [, б, а, з, ар,  , ба, р,  з, аря,  а, м, бар] & бар
\\ \hline
\end{tabular}
\end{table}

Результат: базар бар заря амбар
\pagebreak
\subsection{Вариант №8}
\paragraph{Задание 1. Блочный хаффман \\}

Строка TОООTTTTTО, размер блока: 3
\begin{center}
 \begin{tabular}{ |c|c|l| } 
  \hline
     Буква & Вероятность & Код\\ \hline
T & 0.60 & 1\\\hline
О & 0.40 & 0
\\ \hline \end{tabular}
\end{center}
Энтропия алфавита: 0.9710
\begin{center}
 \begin{tabular}{ |c|c|l| } 
  \hline
     Блок & Вероятность & Код\\ \hline
TTT & 0.22 & 01\\\hline
ОTT & 0.14 & 100\\\hline
TОT & 0.14 & 101\\\hline
TTО & 0.14 & 110\\\hline
ООT & 0.10 & 001\\\hline
TОО & 0.10 & 1111\\\hline
ОTО & 0.10 & 000\\\hline
ООО & 0.06 & 1110
\\ \hline \end{tabular}
\end{center}
Бит на символ при посимвольном кодировании: 1.0000, при блочном: 0.9813

\includegraphics[width=0.5\linewidth]{/home/fizlrock/data/files/backup/code_backup/hobby/algoritms/LabExecutor/app/./doc_src/images/100016131.jpg}

\includegraphics[width=0.9\linewidth]{/home/fizlrock/data/files/backup/code_backup/hobby/algoritms/LabExecutor/app/./doc_src/images/2096031103.jpg}
\pagebreak
\paragraph{Задание 2. Сжать адаптивным хаффманом\\}

Строка: 
РОПВПАРВВВ\\
Результат: 'Р' 0'О' 00'П' 100'В' 01 000'А' 00 101 00 11

\includegraphics[width=0.8\linewidth]{/home/fizlrock/data/files/backup/code_backup/hobby/algoritms/LabExecutor/app/./doc_src/images/329682355.jpg}

\includegraphics[width=0.8\linewidth]{/home/fizlrock/data/files/backup/code_backup/hobby/algoritms/LabExecutor/app/./doc_src/images/1343425262.jpg}

\includegraphics[width=0.8\linewidth]{/home/fizlrock/data/files/backup/code_backup/hobby/algoritms/LabExecutor/app/./doc_src/images/72366520.jpg}

\includegraphics[width=0.8\linewidth]{/home/fizlrock/data/files/backup/code_backup/hobby/algoritms/LabExecutor/app/./doc_src/images/336734146.jpg}

\includegraphics[width=0.8\linewidth]{/home/fizlrock/data/files/backup/code_backup/hobby/algoritms/LabExecutor/app/./doc_src/images/1522633724.jpg}

\includegraphics[width=0.8\linewidth]{/home/fizlrock/data/files/backup/code_backup/hobby/algoritms/LabExecutor/app/./doc_src/images/1351365374.jpg}

\includegraphics[width=0.8\linewidth]{/home/fizlrock/data/files/backup/code_backup/hobby/algoritms/LabExecutor/app/./doc_src/images/794620102.jpg}

\includegraphics[width=0.8\linewidth]{/home/fizlrock/data/files/backup/code_backup/hobby/algoritms/LabExecutor/app/./doc_src/images/730560934.jpg}

\includegraphics[width=0.8\linewidth]{/home/fizlrock/data/files/backup/code_backup/hobby/algoritms/LabExecutor/app/./doc_src/images/1380100366.jpg}

\includegraphics[width=0.8\linewidth]{/home/fizlrock/data/files/backup/code_backup/hobby/algoritms/LabExecutor/app/./doc_src/images/1590060507.jpg}
\pagebreak
\paragraph{Задание 3.1}

Закодировать сообщение методом LZ77\\
Строка:ПОЛ\_ПОЛОВНИК\_ПОЛОВЕЦ\\
Результат: <0,0,П> <0,0,О> <0,0,Л> <0,0,\_> <6,3,О> <0,0,В> <0,0,Н> <0,0,И> <0,0,К> <1,5,В> <0,0,Е> <0,0,Ц>\\
\begin{table}[h!]
\centering
\begin{tabular}{|c|c|c|c|c|c|c|c|c|c|c|c|c|c|c|c|c|} 
\hline
\multicolumn{10}{|c|}{Cловарь} & \multicolumn{6}{c|}{Буфер} & Код  \\ \hline
  &   &   &   &   &   &   &   &   &   & \cellcolor[HTML]{8CE4F6} П & О & Л &   & П & О & <0,0,П>
\\ \hline
  &   &   &   &   &   &   &   &   & П & \cellcolor[HTML]{8CE4F6} О & Л &   & П & О & Л & <0,0,О>
\\ \hline
  &   &   &   &   &   &   &   & П & О & \cellcolor[HTML]{8CE4F6} Л &   & П & О & Л & О & <0,0,Л>
\\ \hline
  &   &   &   &   &   &   & П & О & Л & \cellcolor[HTML]{8CE4F6}   & П & О & Л & О & В & <0,0,\_>
\\ \hline
  &   &   &   &   &   & \cellcolor[HTML]{FFFF00} П & \cellcolor[HTML]{FFFF00} О & \cellcolor[HTML]{FFFF00} Л &   & \cellcolor[HTML]{FFFF00} П & \cellcolor[HTML]{FFFF00} О & \cellcolor[HTML]{FFFF00} Л & \cellcolor[HTML]{8CE4F6} О & В & Н & <6,3,О>
\\ \hline
  &   & П & О & Л &   & П & О & Л & О & \cellcolor[HTML]{8CE4F6} В & Н & И & К &   & П & <0,0,В>
\\ \hline
  & П & О & Л &   & П & О & Л & О & В & \cellcolor[HTML]{8CE4F6} Н & И & К &   & П & О & <0,0,Н>
\\ \hline
П & О & Л &   & П & О & Л & О & В & Н & \cellcolor[HTML]{8CE4F6} И & К &   & П & О & Л & <0,0,И>
\\ \hline
О & Л &   & П & О & Л & О & В & Н & И & \cellcolor[HTML]{8CE4F6} К &   & П & О & Л & О & <0,0,К>
\\ \hline
Л & \cellcolor[HTML]{FFFF00}   & \cellcolor[HTML]{FFFF00} П & \cellcolor[HTML]{FFFF00} О & \cellcolor[HTML]{FFFF00} Л & \cellcolor[HTML]{FFFF00} О & В & Н & И & К & \cellcolor[HTML]{FFFF00}   & \cellcolor[HTML]{FFFF00} П & \cellcolor[HTML]{FFFF00} О & \cellcolor[HTML]{FFFF00} Л & \cellcolor[HTML]{FFFF00} О & \cellcolor[HTML]{8CE4F6} В & <1,5,В>
\\ \hline
В & Н & И & К &   & П & О & Л & О & В & \cellcolor[HTML]{8CE4F6} Е & Ц &   &   &   &   & <0,0,Е>
\\ \hline
Н & И & К &   & П & О & Л & О & В & Е & \cellcolor[HTML]{8CE4F6} Ц &   &   &   &   &   & <0,0,Ц>
\\ \hline
\end{tabular}
\end{table}

\paragraph{Задание 3.2}

Закодировать сообщение методом LZSS\\
Строка:ПОЛ\_ПОЛОВНИК\_ПОЛОВЕЦ\\
Результат: 0'П' 0'О' 0'Л' 0'\_' 1<6,3> 1<4,1> 0'В' 0'Н' 0'И' 0'К' 1<1,6> 0'Е' 0'Ц'\\
\begin{table}[h!]
\centering
\begin{tabular}{|c|c|c|c|c|c|c|c|c|c|c|c|c|c|c|c|c|}
\hline
\multicolumn{10}{|c|}{Cловарь} & \multicolumn{6}{c|}{Буфер} & Код  \\ \hline
  &   &   &   &   &   &   &   &   &   & П & О & Л & \_ & П & О & 0'П'\\ \hline
  &   &   &   &   &   &   &   &   & П & О & Л & \_ & П & О & Л & 0'О'\\ \hline
  &   &   &   &   &   &   &   & П & О & Л & \_ & П & О & Л & О & 0'Л'\\ \hline
  &   &   &   &   &   &   & П & О & Л & \_ & П & О & Л & О & В & 0'\_'\\ \hline
  &   &   &   &   &   & \cellcolor[HTML]{FFFF00} П & \cellcolor[HTML]{FFFF00} О & \cellcolor[HTML]{FFFF00} Л & \_ & \cellcolor[HTML]{FFFF00} П & \cellcolor[HTML]{FFFF00} О & \cellcolor[HTML]{FFFF00} Л & О & В & Н & 1<6,3>\\ \hline
  &   &   & П & \cellcolor[HTML]{FFFF00} О & Л & \_ & П & О & Л & \cellcolor[HTML]{FFFF00} О & В & Н & И & К & \_ & 1<4,1>\\ \hline
  &   & П & О & Л & \_ & П & О & Л & О & В & Н & И & К & \_ & П & 0'В'\\ \hline
  & П & О & Л & \_ & П & О & Л & О & В & Н & И & К & \_ & П & О & 0'Н'\\ \hline
П & О & Л & \_ & П & О & Л & О & В & Н & И & К & \_ & П & О & Л & 0'И'\\ \hline
О & Л & \_ & П & О & Л & О & В & Н & И & К & \_ & П & О & Л & О & 0'К'\\ \hline
Л & \cellcolor[HTML]{FFFF00} \_ & \cellcolor[HTML]{FFFF00} П & \cellcolor[HTML]{FFFF00} О & \cellcolor[HTML]{FFFF00} Л & \cellcolor[HTML]{FFFF00} О & \cellcolor[HTML]{FFFF00} В & Н & И & К & \cellcolor[HTML]{FFFF00} \_ & \cellcolor[HTML]{FFFF00} П & \cellcolor[HTML]{FFFF00} О & \cellcolor[HTML]{FFFF00} Л & \cellcolor[HTML]{FFFF00} О & \cellcolor[HTML]{FFFF00} В & 1<1,6>\\ \hline
В & Н & И & К & \_ & П & О & Л & О & В & Е & Ц &   &   &   &   & 0'Е'\\ \hline
Н & И & К & \_ & П & О & Л & О & В & Е & Ц &   &   &   &   &   & 0'Ц'\\ \hline
\end{tabular}
\end{table}

\paragraph{Задание 3.3}

Закодировать сообщение методом LZ78\\
Строка:ПОЛ\_ПОЛОВНИК\_ПОЛОВЕЦ\\
\begin{table}[h!]
\centering
\begin{tabular}{|c|c|c|} 
\hline
 Входная фраза (в словарь) & Код & Позиция словаря \\ \hline

 &  & 0 \\ \hline
П & 0'П' & 1 \\ \hline
О & 0'О' & 2 \\ \hline
Л & 0'Л' & 3 \\ \hline
\_ & 0'\_' & 4 \\ \hline
ПО & 1'О' & 5 \\ \hline
ЛО & 3'О' & 6 \\ \hline
В & 0'В' & 7 \\ \hline
Н & 0'Н' & 8 \\ \hline
И & 0'И' & 9 \\ \hline
К & 0'К' & 10 \\ \hline
\_П & 4'П' & 11 \\ \hline
ОЛ & 2'Л' & 12 \\ \hline
ОВ & 2'В' & 13 \\ \hline
Е & 0'Е' & 14 \\ \hline
Ц & 0'Ц' & 15 \\ \hline
\end{tabular}
\end{table}

Результат: 0'П' 0'О' 0'Л' 0'\_' 1'О' 3'О' 0'В' 0'Н' 0'И' 0'К' 4'П' 2'Л' 2'В' 0'Е' 0'Ц'\\
\pagebreak
\paragraph{Задание 4. Арифметическое кодирование\\}

Исходная строка: РОПВПАРВВВ\
\begin{center}
 \begin{tabular}{ |c|c| } 
  \hline
     Буква & Вероятность \\ \hline
В & 0.40\\\hline
Р & 0.20\\\hline
П & 0.20\\\hline
А & 0.10\\\hline
О & 0.10
\\ \hline \end{tabular}
\end{center}
\begin{center}
 \begin{tabular}{ |c|c|c| } 
  \hline
     Буква & Начало & Конец \\ \hline
В & 0.00 & 0.40\\\hline
Р & 0.40 & 0.60\\\hline
П & 0.60 & 0.80\\\hline
А & 0.80 & 0.90\\\hline
О & 0.90 & 1.00
\\ \hline \end{tabular}
\end{center}
\begin{center}
 \begin{tabular}{ |c|c|c|c| } 
  \hline
     Буква & delta & min & max \\ \hline
Р & 0.2000000000 & 0.4000000000 & 0.6000000000\\\hline
О & 0.0200000000 & 0.5800000000 & 0.6000000000\\\hline
П & 0.0040000000 & 0.5920000000 & 0.5960000000\\\hline
В & 0.0016000000 & 0.5920000000 & 0.5936000000\\\hline
П & 0.0003200000 & 0.5929600000 & 0.5932800000\\\hline
А & 0.0000320000 & 0.5932160000 & 0.5932480000\\\hline
Р & 0.0000064000 & 0.5932288000 & 0.5932352000\\\hline
В & 0.0000025600 & 0.5932288000 & 0.5932313600\\\hline
В & 0.0000010240 & 0.5932288000 & 0.5932298240\\\hline
В & 0.0000004096 & 0.5932288000 & 0.5932292096
\\ \hline \end{tabular}
\end{center}
Результат: 593229
\pagebreak
\paragraph{Задание 5.1}

\\ 

Декодировать сообщение методом адаптивного хаффмана \\
Строка: 
'K'0'N'00'M'101100'H'110111010111111\\
Результат: KNMMKMHKMKNMKKM

\includegraphics[width=0.8\linewidth]{/home/fizlrock/data/files/backup/code_backup/hobby/algoritms/LabExecutor/app/./doc_src/images/71006515.jpg}

\includegraphics[width=0.8\linewidth]{/home/fizlrock/data/files/backup/code_backup/hobby/algoritms/LabExecutor/app/./doc_src/images/350213500.jpg}

\includegraphics[width=0.8\linewidth]{/home/fizlrock/data/files/backup/code_backup/hobby/algoritms/LabExecutor/app/./doc_src/images/2042705399.jpg}

\includegraphics[width=0.8\linewidth]{/home/fizlrock/data/files/backup/code_backup/hobby/algoritms/LabExecutor/app/./doc_src/images/282060707.jpg}

\includegraphics[width=0.8\linewidth]{/home/fizlrock/data/files/backup/code_backup/hobby/algoritms/LabExecutor/app/./doc_src/images/972242116.jpg}

\includegraphics[width=0.8\linewidth]{/home/fizlrock/data/files/backup/code_backup/hobby/algoritms/LabExecutor/app/./doc_src/images/1280703189.jpg}

\includegraphics[width=0.8\linewidth]{/home/fizlrock/data/files/backup/code_backup/hobby/algoritms/LabExecutor/app/./doc_src/images/1556263903.jpg}

\includegraphics[width=0.8\linewidth]{/home/fizlrock/data/files/backup/code_backup/hobby/algoritms/LabExecutor/app/./doc_src/images/1232295005.jpg}

\includegraphics[width=0.8\linewidth]{/home/fizlrock/data/files/backup/code_backup/hobby/algoritms/LabExecutor/app/./doc_src/images/443658452.jpg}

\includegraphics[width=0.8\linewidth]{/home/fizlrock/data/files/backup/code_backup/hobby/algoritms/LabExecutor/app/./doc_src/images/1050396785.jpg}

\includegraphics[width=0.8\linewidth]{/home/fizlrock/data/files/backup/code_backup/hobby/algoritms/LabExecutor/app/./doc_src/images/634333565.jpg}

\includegraphics[width=0.8\linewidth]{/home/fizlrock/data/files/backup/code_backup/hobby/algoritms/LabExecutor/app/./doc_src/images/38689987.jpg}

\includegraphics[width=0.8\linewidth]{/home/fizlrock/data/files/backup/code_backup/hobby/algoritms/LabExecutor/app/./doc_src/images/563686050.jpg}

\includegraphics[width=0.8\linewidth]{/home/fizlrock/data/files/backup/code_backup/hobby/algoritms/LabExecutor/app/./doc_src/images/803178507.jpg}

\includegraphics[width=0.8\linewidth]{/home/fizlrock/data/files/backup/code_backup/hobby/algoritms/LabExecutor/app/./doc_src/images/1880669213.jpg}
\pagebreak
\paragraph{Задание 5.3 Декодировать строку(LZSS)\\}

Исходная строка: [0'р'] [0'о'] [0'с'] [0'т'] [0' '] [1<7,2>] [1<4,1>] [0'л'] [1<5,3>] [0'у'] [1<5,4>] [0'а'] [1<0,1>] [0'ь']\\
\begin{table}[h!]
\centering
\begin{tabular}{|c|c|c|}
\hline
 Cловарь & Буфер & Код  \\ \hline
0'р' & [ ,  ,  ,  ,  ,  ,  ,  ,  , р] & р
\\ \hline
0'о' & [ ,  ,  ,  ,  ,  ,  ,  , р, о] & о
\\ \hline
0'с' & [ ,  ,  ,  ,  ,  ,  , р, о, с] & с
\\ \hline
0'т' & [ ,  ,  ,  ,  ,  , р, о, с, т] & т
\\ \hline
0' ' & [ ,  ,  ,  ,  , р, о, с, т,  ] &  
\\ \hline
1<7,2> & [ ,  ,  , р, о, с, т,  , с, т] & ст
\\ \hline
1<4,1> & [ ,  , р, о, с, т,  , с, т, о] & о
\\ \hline
0'л' & [ , р, о, с, т,  , с, т, о, л] & л
\\ \hline
1<5,3> & [с, т,  , с, т, о, л,  , с, т] &  ст
\\ \hline
0'у' & [т,  , с, т, о, л,  , с, т, у] & у
\\ \hline
1<5,4> & [о, л,  , с, т, у, л,  , с, т] & л ст
\\ \hline
0'а' & [л,  , с, т, у, л,  , с, т, а] & а
\\ \hline
1<0,1> & [ , с, т, у, л,  , с, т, а, л] & л
\\ \hline
0'ь' & [с, т, у, л,  , с, т, а, л, ь] & ь
\\ \hline
\end{tabular}
\end{table}

Результат: рост стол стул сталь
\pagebreak
\paragraph{Задание 5.4 Декодировать строку(LZ78)\\}

Исходная строка: [0'б'] [0'р'] [0'и'] [0'з'] [0' '] [1'р'] [0'а'] [5'б'] [7'р'] [5'р'] [7'б'] [5'а'] [0'р']\\
\begin{table}[h!]
\centering
\begin{tabular}{|c|c|c|}
\hline
 Cловарь & Буфер & Код  \\ \hline
 & [] & 
\\ \hline
0'б' & [, б] & б
\\ \hline
0'р' & [, б, р] & р
\\ \hline
0'и' & [, б, р, и] & и
\\ \hline
0'з' & [, б, р, и, з] & з
\\ \hline
0' ' & [, б, р, и, з,  ] &  
\\ \hline
1'р' & [, б, р, и, з,  , бр] & бр
\\ \hline
0'а' & [, б, р, и, з,  , бр, а] & а
\\ \hline
5'б' & [, б, р, и, з,  , бр, а,  б] &  б
\\ \hline
7'р' & [, б, р, и, з,  , бр, а,  б, ар] & ар
\\ \hline
5'р' & [, б, р, и, з,  , бр, а,  б, ар,  р] &  р
\\ \hline
7'б' & [, б, р, и, з,  , бр, а,  б, ар,  р, аб] & аб
\\ \hline
5'а' & [, б, р, и, з,  , бр, а,  б, ар,  р, аб,  а] &  а
\\ \hline
0'р' & [, б, р, и, з,  , бр, а,  б, ар,  р, аб,  а, р] & р
\\ \hline
\end{tabular}
\end{table}

Результат: бриз бра бар раб ар
\pagebreak
\subsection{Вариант №9}
\paragraph{Задание 1. Блочный хаффман \\}

Строка СОКККККООО, размер блока: 2
\begin{center}
 \begin{tabular}{ |c|c|l| } 
  \hline
     Буква & Вероятность & Код\\ \hline
К & 0.50 & 0\\\hline
О & 0.40 & 11\\\hline
С & 0.10 & 10
\\ \hline \end{tabular}
\end{center}
Энтропия алфавита: 1.3610
\begin{center}
 \begin{tabular}{ |c|c|l| } 
  \hline
     Блок & Вероятность & Код\\ \hline
КК & 0.25 & 10\\\hline
КО & 0.20 & 00\\\hline
ОК & 0.20 & 01\\\hline
ОО & 0.16 & 110\\\hline
КС & 0.05 & 11101\\\hline
СК & 0.05 & 11110\\\hline
ОС & 0.04 & 111111\\\hline
СО & 0.04 & 11100\\\hline
СС & 0.01 & 111110
\\ \hline \end{tabular}
\end{center}
Бит на символ при посимвольном кодировании: 1.5000, при блочном: 1.3900

\includegraphics[width=0.5\linewidth]{/home/fizlrock/data/files/backup/code_backup/hobby/algoritms/LabExecutor/app/./doc_src/images/1365368524.jpg}

\includegraphics[width=0.9\linewidth]{/home/fizlrock/data/files/backup/code_backup/hobby/algoritms/LabExecutor/app/./doc_src/images/1771975830.jpg}
\pagebreak
\paragraph{Задание 2. Сжать адаптивным хаффманом\\}

Строка: 
РОРНРПОООО\\
Результат: 'Р' 0'О' 1 00'Н' 1 000'П' 00 10 11 0

\includegraphics[width=0.8\linewidth]{/home/fizlrock/data/files/backup/code_backup/hobby/algoritms/LabExecutor/app/./doc_src/images/33194124.jpg}

\includegraphics[width=0.8\linewidth]{/home/fizlrock/data/files/backup/code_backup/hobby/algoritms/LabExecutor/app/./doc_src/images/622735046.jpg}

\includegraphics[width=0.8\linewidth]{/home/fizlrock/data/files/backup/code_backup/hobby/algoritms/LabExecutor/app/./doc_src/images/1675188201.jpg}

\includegraphics[width=0.8\linewidth]{/home/fizlrock/data/files/backup/code_backup/hobby/algoritms/LabExecutor/app/./doc_src/images/989205816.jpg}

\includegraphics[width=0.8\linewidth]{/home/fizlrock/data/files/backup/code_backup/hobby/algoritms/LabExecutor/app/./doc_src/images/1023778809.jpg}

\includegraphics[width=0.8\linewidth]{/home/fizlrock/data/files/backup/code_backup/hobby/algoritms/LabExecutor/app/./doc_src/images/581102589.jpg}

\includegraphics[width=0.8\linewidth]{/home/fizlrock/data/files/backup/code_backup/hobby/algoritms/LabExecutor/app/./doc_src/images/689375501.jpg}

\includegraphics[width=0.8\linewidth]{/home/fizlrock/data/files/backup/code_backup/hobby/algoritms/LabExecutor/app/./doc_src/images/522039052.jpg}

\includegraphics[width=0.8\linewidth]{/home/fizlrock/data/files/backup/code_backup/hobby/algoritms/LabExecutor/app/./doc_src/images/1854613279.jpg}

\includegraphics[width=0.8\linewidth]{/home/fizlrock/data/files/backup/code_backup/hobby/algoritms/LabExecutor/app/./doc_src/images/2050215733.jpg}
\pagebreak

\paragraph{Задание 3.2}

Закодировать сообщение методом LZSS\\
Строка:МУМУМУ\_МУКА\_МУРКА\\
Результат: 0'М' 0'У' 1<8,2> 1<6,2> 0'\_' 1<3,2> 0'К' 0'А' 1<5,3> 0'Р' 1<4,2>\\
\begin{table}[h!]
\centering
\begin{tabular}{|c|c|c|c|c|c|c|c|c|c|c|c|c|c|c|c|c|}
\hline
\multicolumn{10}{|c|}{Cловарь} & \multicolumn{6}{c|}{Буфер} & Код  \\ \hline
  &   &   &   &   &   &   &   &   &   & М & У & М & У & М & У & 0'М'\\ \hline
  &   &   &   &   &   &   &   &   & М & У & М & У & М & У & \_ & 0'У'\\ \hline
  &   &   &   &   &   &   &   & \cellcolor[HTML]{FFFF00} М & \cellcolor[HTML]{FFFF00} У & \cellcolor[HTML]{FFFF00} М & \cellcolor[HTML]{FFFF00} У & М & У & \_ & М & 1<8,2>\\ \hline
  &   &   &   &   &   & \cellcolor[HTML]{FFFF00} М & \cellcolor[HTML]{FFFF00} У & М & У & \cellcolor[HTML]{FFFF00} М & \cellcolor[HTML]{FFFF00} У & \_ & М & У & К & 1<6,2>\\ \hline
  &   &   &   & М & У & М & У & М & У & \_ & М & У & К & А & \_ & 0'\_'\\ \hline
  &   &   & \cellcolor[HTML]{FFFF00} М & \cellcolor[HTML]{FFFF00} У & М & У & М & У & \_ & \cellcolor[HTML]{FFFF00} М & \cellcolor[HTML]{FFFF00} У & К & А & \_ & М & 1<3,2>\\ \hline
  & М & У & М & У & М & У & \_ & М & У & К & А & \_ & М & У & Р & 0'К'\\ \hline
М & У & М & У & М & У & \_ & М & У & К & А & \_ & М & У & Р & К & 0'А'\\ \hline
У & М & У & М & У & \cellcolor[HTML]{FFFF00} \_ & \cellcolor[HTML]{FFFF00} М & \cellcolor[HTML]{FFFF00} У & К & А & \cellcolor[HTML]{FFFF00} \_ & \cellcolor[HTML]{FFFF00} М & \cellcolor[HTML]{FFFF00} У & Р & К & А & 1<5,3>\\ \hline
М & У & \_ & М & У & К & А & \_ & М & У & Р & К & А &   &   &   & 0'Р'\\ \hline
У & \_ & М & У & \cellcolor[HTML]{FFFF00} К & \cellcolor[HTML]{FFFF00} А & \_ & М & У & Р & \cellcolor[HTML]{FFFF00} К & \cellcolor[HTML]{FFFF00} А &   &   &   &   & 1<4,2>\\ \hline
\end{tabular}
\end{table}

\paragraph{Задание 3.3}

Закодировать сообщение методом LZ78\\
Строка:МУМУМУ\_МУКА\_МУРКА\\
\begin{table}[h!]
\centering
\begin{tabular}{|c|c|c|} 
\hline
 Входная фраза (в словарь) & Код & Позиция словаря \\ \hline

 &  & 0 \\ \hline
М & 0'М' & 1 \\ \hline
У & 0'У' & 2 \\ \hline
МУ & 1'У' & 3 \\ \hline
МУ\_ & 3'\_' & 4 \\ \hline
МУК & 3'К' & 5 \\ \hline
А & 0'А' & 6 \\ \hline
\_ & 0'\_' & 7 \\ \hline
МУР & 3'Р' & 8 \\ \hline
К & 0'К' & 9 \\ \hline
\end{tabular}
\end{table}

Результат: 0'М' 0'У' 1'У' 3'\_' 3'К' 0'А' 0'\_' 3'Р' 0'К'\\
\pagebreak
\paragraph{Задание 4. Арифметическое кодирование\\}

Исходная строка: РОРНРПОООО\
\begin{center}
 \begin{tabular}{ |c|c| } 
  \hline
     Буква & Вероятность \\ \hline
О & 0.50\\\hline
Р & 0.30\\\hline
Н & 0.10\\\hline
П & 0.10
\\ \hline \end{tabular}
\end{center}
\begin{center}
 \begin{tabular}{ |c|c|c| } 
  \hline
     Буква & Начало & Конец \\ \hline
О & 0.00 & 0.50\\\hline
Р & 0.50 & 0.80\\\hline
Н & 0.80 & 0.90\\\hline
П & 0.90 & 1.00
\\ \hline \end{tabular}
\end{center}
\begin{center}
 \begin{tabular}{ |c|c|c|c| } 
  \hline
     Буква & delta & min & max \\ \hline
Р & 0.3000000000 & 0.5000000000 & 0.8000000000\\\hline
О & 0.1500000000 & 0.5000000000 & 0.6500000000\\\hline
Р & 0.0450000000 & 0.5750000000 & 0.6200000000\\\hline
Н & 0.0045000000 & 0.6110000000 & 0.6155000000\\\hline
Р & 0.0013500000 & 0.6132500000 & 0.6146000000\\\hline
П & 0.0001350000 & 0.6144650000 & 0.6146000000\\\hline
О & 0.0000675000 & 0.6144650000 & 0.6145325000\\\hline
О & 0.0000337500 & 0.6144650000 & 0.6144987500\\\hline
О & 0.0000168750 & 0.6144650000 & 0.6144818750\\\hline
О & 0.0000084375 & 0.6144650000 & 0.6144734375
\\ \hline \end{tabular}
\end{center}
Результат: 61447
\pagebreak
\paragraph{Задание 5.1}

\\ 

Декодировать сообщение методом адаптивного хаффмана \\
Строка: 
'T'0'Y'00'H'100'G'0010111111111111\\
Результат: TYHGGHGTTG

\includegraphics[width=0.8\linewidth]{/home/fizlrock/data/files/backup/code_backup/hobby/algoritms/LabExecutor/app/./doc_src/images/541079278.jpg}

\includegraphics[width=0.8\linewidth]{/home/fizlrock/data/files/backup/code_backup/hobby/algoritms/LabExecutor/app/./doc_src/images/711101763.jpg}

\includegraphics[width=0.8\linewidth]{/home/fizlrock/data/files/backup/code_backup/hobby/algoritms/LabExecutor/app/./doc_src/images/1467553712.jpg}

\includegraphics[width=0.8\linewidth]{/home/fizlrock/data/files/backup/code_backup/hobby/algoritms/LabExecutor/app/./doc_src/images/2133037516.jpg}

\includegraphics[width=0.8\linewidth]{/home/fizlrock/data/files/backup/code_backup/hobby/algoritms/LabExecutor/app/./doc_src/images/1773351283.jpg}

\includegraphics[width=0.8\linewidth]{/home/fizlrock/data/files/backup/code_backup/hobby/algoritms/LabExecutor/app/./doc_src/images/1443594223.jpg}

\includegraphics[width=0.8\linewidth]{/home/fizlrock/data/files/backup/code_backup/hobby/algoritms/LabExecutor/app/./doc_src/images/718752603.jpg}

\includegraphics[width=0.8\linewidth]{/home/fizlrock/data/files/backup/code_backup/hobby/algoritms/LabExecutor/app/./doc_src/images/1811332858.jpg}

\includegraphics[width=0.8\linewidth]{/home/fizlrock/data/files/backup/code_backup/hobby/algoritms/LabExecutor/app/./doc_src/images/899378968.jpg}

\includegraphics[width=0.8\linewidth]{/home/fizlrock/data/files/backup/code_backup/hobby/algoritms/LabExecutor/app/./doc_src/images/1161928719.jpg}
\pagebreak
\paragraph{Задание 5.3 Декодировать строку(LZSS)\\}

Исходная строка: [0'с'] [0'о'] [0'т'] [0'ы'] [0' '] [1<5,2>] [0'к'] [1<6,1>] [1<8,1>] [1<6,2>] [1<4,1>] [1<2,1>] [1<0,4>] [1<2,1>] [0'л']\\
\begin{table}[h!]
\centering
\begin{tabular}{|c|c|c|}
\hline
 Cловарь & Буфер & Код  \\ \hline
0'с' & [ ,  ,  ,  ,  ,  ,  ,  ,  , с] & с
\\ \hline
0'о' & [ ,  ,  ,  ,  ,  ,  ,  , с, о] & о
\\ \hline
0'т' & [ ,  ,  ,  ,  ,  ,  , с, о, т] & т
\\ \hline
0'ы' & [ ,  ,  ,  ,  ,  , с, о, т, ы] & ы
\\ \hline
0' ' & [ ,  ,  ,  ,  , с, о, т, ы,  ] &  
\\ \hline
1<5,2> & [ ,  ,  , с, о, т, ы,  , с, о] & со
\\ \hline
0'к' & [ ,  , с, о, т, ы,  , с, о, к] & к
\\ \hline
1<6,1> & [ , с, о, т, ы,  , с, о, к,  ] &  
\\ \hline
1<8,1> & [с, о, т, ы,  , с, о, к,  , к] & к
\\ \hline
1<6,2> & [т, ы,  , с, о, к,  , к, о, к] & ок
\\ \hline
1<4,1> & [ы,  , с, о, к,  , к, о, к, о] & о
\\ \hline
1<2,1> & [ , с, о, к,  , к, о, к, о, с] & с
\\ \hline
1<0,4> & [ , к, о, к, о, с,  , с, о, к] &  сок
\\ \hline
1<2,1> & [к, о, к, о, с,  , с, о, к, о] & о
\\ \hline
0'л' & [о, к, о, с,  , с, о, к, о, л] & л
\\ \hline
\end{tabular}
\end{table}

Результат: соты сок кокос сокол
\pagebreak
\paragraph{Задание 5.4 Декодировать строку(LZ78)\\}

Исходная строка: [0'л'] [0'е'] [0'т'] [0'о'] [0' '] [3'о'] [0'н'] [5'т'] [4'н'] [0'у'] [0'с'] [5'у'] [11'ы']\\
\begin{table}[h!]
\centering
\begin{tabular}{|c|c|c|}
\hline
 Cловарь & Буфер & Код  \\ \hline
 & [] & 
\\ \hline
0'л' & [, л] & л
\\ \hline
0'е' & [, л, е] & е
\\ \hline
0'т' & [, л, е, т] & т
\\ \hline
0'о' & [, л, е, т, о] & о
\\ \hline
0' ' & [, л, е, т, о,  ] &  
\\ \hline
3'о' & [, л, е, т, о,  , то] & то
\\ \hline
0'н' & [, л, е, т, о,  , то, н] & н
\\ \hline
5'т' & [, л, е, т, о,  , то, н,  т] &  т
\\ \hline
4'н' & [, л, е, т, о,  , то, н,  т, он] & он
\\ \hline
0'у' & [, л, е, т, о,  , то, н,  т, он, у] & у
\\ \hline
0'с' & [, л, е, т, о,  , то, н,  т, он, у, с] & с
\\ \hline
5'у' & [, л, е, т, о,  , то, н,  т, он, у, с,  у] &  у
\\ \hline
11'ы' & [, л, е, т, о,  , то, н,  т, он, у, с,  у, сы] & сы
\\ \hline
\end{tabular}
\end{table}

Результат: лето тон тонус усы
\pagebreak
\subsection{Вариант №10}
\paragraph{Задание 1. Блочный хаффман \\}

Строка СТТТТССТТТ, размер блока: 3
\begin{center}
 \begin{tabular}{ |c|c|l| } 
  \hline
     Буква & Вероятность & Код\\ \hline
Т & 0.70 & 1\\\hline
С & 0.30 & 0
\\ \hline \end{tabular}
\end{center}
Энтропия алфавита: 0.8813
\begin{center}
 \begin{tabular}{ |c|c|l| } 
  \hline
     Блок & Вероятность & Код\\ \hline
ТТТ & 0.34 & 11\\\hline
СТТ & 0.15 & 101\\\hline
ТСТ & 0.15 & 00\\\hline
ТТС & 0.15 & 100\\\hline
СТС & 0.06 & 0101\\\hline
ССТ & 0.06 & 0110\\\hline
ТСС & 0.06 & 0111\\\hline
ССС & 0.03 & 0100
\\ \hline \end{tabular}
\end{center}
Бит на символ при посимвольном кодировании: 1.0000, при блочном: 0.9087

\includegraphics[width=0.5\linewidth]{/home/fizlrock/data/files/backup/code_backup/hobby/algoritms/LabExecutor/app/./doc_src/images/1978333321.jpg}

\includegraphics[width=0.9\linewidth]{/home/fizlrock/data/files/backup/code_backup/hobby/algoritms/LabExecutor/app/./doc_src/images/1811049143.jpg}
\pagebreak
\paragraph{Задание 2. Сжать адаптивным хаффманом\\}

Строка: 
КАВАПВПППА\\
Результат: 'К' 0'А' 00'В' 11 100'П' 111 001 01 11 111

\includegraphics[width=0.8\linewidth]{/home/fizlrock/data/files/backup/code_backup/hobby/algoritms/LabExecutor/app/./doc_src/images/1488224557.jpg}

\includegraphics[width=0.8\linewidth]{/home/fizlrock/data/files/backup/code_backup/hobby/algoritms/LabExecutor/app/./doc_src/images/30895259.jpg}

\includegraphics[width=0.8\linewidth]{/home/fizlrock/data/files/backup/code_backup/hobby/algoritms/LabExecutor/app/./doc_src/images/2031432321.jpg}

\includegraphics[width=0.8\linewidth]{/home/fizlrock/data/files/backup/code_backup/hobby/algoritms/LabExecutor/app/./doc_src/images/2029374627.jpg}

\includegraphics[width=0.8\linewidth]{/home/fizlrock/data/files/backup/code_backup/hobby/algoritms/LabExecutor/app/./doc_src/images/63006666.jpg}

\includegraphics[width=0.8\linewidth]{/home/fizlrock/data/files/backup/code_backup/hobby/algoritms/LabExecutor/app/./doc_src/images/1267018831.jpg}

\includegraphics[width=0.8\linewidth]{/home/fizlrock/data/files/backup/code_backup/hobby/algoritms/LabExecutor/app/./doc_src/images/1468526465.jpg}

\includegraphics[width=0.8\linewidth]{/home/fizlrock/data/files/backup/code_backup/hobby/algoritms/LabExecutor/app/./doc_src/images/652398209.jpg}

\includegraphics[width=0.8\linewidth]{/home/fizlrock/data/files/backup/code_backup/hobby/algoritms/LabExecutor/app/./doc_src/images/651055450.jpg}

\includegraphics[width=0.8\linewidth]{/home/fizlrock/data/files/backup/code_backup/hobby/algoritms/LabExecutor/app/./doc_src/images/713259741.jpg}
\pagebreak
\paragraph{Задание 3.1}

Закодировать сообщение методом LZ77\\
Строка:КОК\_КОКЛЮШ\_КЛУБ\_КЛУБОК\\
Результат: <0,0,К> <0,0,О> <8,1,\_> <6,3,Л> <0,0,Ю> <0,0,Ш> <3,2,Л> <0,0,У> <0,0,Б> <5,5,О> <0,0,К>\\
\begin{table}[h!]
\centering
\begin{tabular}{|c|c|c|c|c|c|c|c|c|c|c|c|c|c|c|c|c|} 
\hline
\multicolumn{10}{|c|}{Cловарь} & \multicolumn{6}{c|}{Буфер} & Код  \\ \hline
  &   &   &   &   &   &   &   &   &   & \cellcolor[HTML]{8CE4F6} К & О & К &   & К & О & <0,0,К>
\\ \hline
  &   &   &   &   &   &   &   &   & К & \cellcolor[HTML]{8CE4F6} О & К &   & К & О & К & <0,0,О>
\\ \hline
  &   &   &   &   &   &   &   & \cellcolor[HTML]{FFFF00} К & О & \cellcolor[HTML]{FFFF00} К & \cellcolor[HTML]{8CE4F6}   & К & О & К & Л & <8,1,\_>
\\ \hline
  &   &   &   &   &   & \cellcolor[HTML]{FFFF00} К & \cellcolor[HTML]{FFFF00} О & \cellcolor[HTML]{FFFF00} К &   & \cellcolor[HTML]{FFFF00} К & \cellcolor[HTML]{FFFF00} О & \cellcolor[HTML]{FFFF00} К & \cellcolor[HTML]{8CE4F6} Л & Ю & Ш & <6,3,Л>
\\ \hline
  &   & К & О & К &   & К & О & К & Л & \cellcolor[HTML]{8CE4F6} Ю & Ш &   & К & Л & У & <0,0,Ю>
\\ \hline
  & К & О & К &   & К & О & К & Л & Ю & \cellcolor[HTML]{8CE4F6} Ш &   & К & Л & У & Б & <0,0,Ш>
\\ \hline
К & О & К & \cellcolor[HTML]{FFFF00}   & \cellcolor[HTML]{FFFF00} К & О & К & Л & Ю & Ш & \cellcolor[HTML]{FFFF00}   & \cellcolor[HTML]{FFFF00} К & \cellcolor[HTML]{8CE4F6} Л & У & Б &   & <3,2,Л>
\\ \hline
  & К & О & К & Л & Ю & Ш &   & К & Л & \cellcolor[HTML]{8CE4F6} У & Б &   & К & Л & У & <0,0,У>
\\ \hline
К & О & К & Л & Ю & Ш &   & К & Л & У & \cellcolor[HTML]{8CE4F6} Б &   & К & Л & У & Б & <0,0,Б>
\\ \hline
О & К & Л & Ю & Ш & \cellcolor[HTML]{FFFF00}   & \cellcolor[HTML]{FFFF00} К & \cellcolor[HTML]{FFFF00} Л & \cellcolor[HTML]{FFFF00} У & \cellcolor[HTML]{FFFF00} Б & \cellcolor[HTML]{FFFF00}   & \cellcolor[HTML]{FFFF00} К & \cellcolor[HTML]{FFFF00} Л & \cellcolor[HTML]{FFFF00} У & \cellcolor[HTML]{FFFF00} Б & \cellcolor[HTML]{8CE4F6} О & <5,5,О>
\\ \hline
К & Л & У & Б &   & К & Л & У & Б & О & \cellcolor[HTML]{8CE4F6} К &   &   &   &   &   & <0,0,К>
\\ \hline
\end{tabular}
\end{table}

\paragraph{Задание 3.2}

Закодировать сообщение методом LZSS\\
Строка:КОК\_КОКЛЮШ\_КЛУБ\_КЛУБОК\\
Результат: 0'К' 0'О' 1<8,1> 0'\_' 1<6,3> 0'Л' 0'Ю' 0'Ш' 1<3,2> 1<5,1> 0'У' 0'Б' 1<5,5> 0'О' 1<0,1>\\
\begin{table}[h!]
\centering
\begin{tabular}{|c|c|c|c|c|c|c|c|c|c|c|c|c|c|c|c|c|}
\hline
\multicolumn{10}{|c|}{Cловарь} & \multicolumn{6}{c|}{Буфер} & Код  \\ \hline
  &   &   &   &   &   &   &   &   &   & К & О & К & \_ & К & О & 0'К'\\ \hline
  &   &   &   &   &   &   &   &   & К & О & К & \_ & К & О & К & 0'О'\\ \hline
  &   &   &   &   &   &   &   & \cellcolor[HTML]{FFFF00} К & О & \cellcolor[HTML]{FFFF00} К & \_ & К & О & К & Л & 1<8,1>\\ \hline
  &   &   &   &   &   &   & К & О & К & \_ & К & О & К & Л & Ю & 0'\_'\\ \hline
  &   &   &   &   &   & \cellcolor[HTML]{FFFF00} К & \cellcolor[HTML]{FFFF00} О & \cellcolor[HTML]{FFFF00} К & \_ & \cellcolor[HTML]{FFFF00} К & \cellcolor[HTML]{FFFF00} О & \cellcolor[HTML]{FFFF00} К & Л & Ю & Ш & 1<6,3>\\ \hline
  &   &   & К & О & К & \_ & К & О & К & Л & Ю & Ш & \_ & К & Л & 0'Л'\\ \hline
  &   & К & О & К & \_ & К & О & К & Л & Ю & Ш & \_ & К & Л & У & 0'Ю'\\ \hline
  & К & О & К & \_ & К & О & К & Л & Ю & Ш & \_ & К & Л & У & Б & 0'Ш'\\ \hline
К & О & К & \cellcolor[HTML]{FFFF00} \_ & \cellcolor[HTML]{FFFF00} К & О & К & Л & Ю & Ш & \cellcolor[HTML]{FFFF00} \_ & \cellcolor[HTML]{FFFF00} К & Л & У & Б & \_ & 1<3,2>\\ \hline
К & \_ & К & О & К & \cellcolor[HTML]{FFFF00} Л & Ю & Ш & \_ & К & \cellcolor[HTML]{FFFF00} Л & У & Б & \_ & К & Л & 1<5,1>\\ \hline
\_ & К & О & К & Л & Ю & Ш & \_ & К & Л & У & Б & \_ & К & Л & У & 0'У'\\ \hline
К & О & К & Л & Ю & Ш & \_ & К & Л & У & Б & \_ & К & Л & У & Б & 0'Б'\\ \hline
О & К & Л & Ю & Ш & \cellcolor[HTML]{FFFF00} \_ & \cellcolor[HTML]{FFFF00} К & \cellcolor[HTML]{FFFF00} Л & \cellcolor[HTML]{FFFF00} У & \cellcolor[HTML]{FFFF00} Б & \cellcolor[HTML]{FFFF00} \_ & \cellcolor[HTML]{FFFF00} К & \cellcolor[HTML]{FFFF00} Л & \cellcolor[HTML]{FFFF00} У & \cellcolor[HTML]{FFFF00} Б & О & 1<5,5>\\ \hline
\_ & К & Л & У & Б & \_ & К & Л & У & Б & О & К &   &   &   &   & 0'О'\\ \hline
\cellcolor[HTML]{FFFF00} К & Л & У & Б & \_ & К & Л & У & Б & О & \cellcolor[HTML]{FFFF00} К &   &   &   &   &   & 1<0,1>\\ \hline
\end{tabular}
\end{table}

\paragraph{Задание 3.3}

Закодировать сообщение методом LZ78\\
Строка:КОК\_КОКЛЮШ\_КЛУБ\_КЛУБОК\\
\begin{table}[h!]
\centering
\begin{tabular}{|c|c|c|} 
\hline
 Входная фраза (в словарь) & Код & Позиция словаря \\ \hline

 &  & 0 \\ \hline
К & 0'К' & 1 \\ \hline
О & 0'О' & 2 \\ \hline
К\_ & 1'\_' & 3 \\ \hline
КО & 1'О' & 4 \\ \hline
КЛ & 1'Л' & 5 \\ \hline
Ю & 0'Ю' & 6 \\ \hline
Ш & 0'Ш' & 7 \\ \hline
\_ & 0'\_' & 8 \\ \hline
КЛУ & 5'У' & 9 \\ \hline
Б & 0'Б' & 10 \\ \hline
\_К & 8'К' & 11 \\ \hline
Л & 0'Л' & 12 \\ \hline
У & 0'У' & 13 \\ \hline
БО & 10'О' & 14 \\ \hline
\end{tabular}
\end{table}

Результат: 0'К' 0'О' 1'\_' 1'О' 1'Л' 0'Ю' 0'Ш' 0'\_' 5'У' 0'Б' 8'К' 0'Л' 0'У' 10'О'\\
\pagebreak
\paragraph{Задание 4. Арифметическое кодирование\\}

Исходная строка: КАВАПВПППА\
\begin{center}
 \begin{tabular}{ |c|c| } 
  \hline
     Буква & Вероятность \\ \hline
П & 0.40\\\hline
А & 0.30\\\hline
В & 0.20\\\hline
К & 0.10
\\ \hline \end{tabular}
\end{center}
\begin{center}
 \begin{tabular}{ |c|c|c| } 
  \hline
     Буква & Начало & Конец \\ \hline
П & 0.00 & 0.40\\\hline
А & 0.40 & 0.70\\\hline
В & 0.70 & 0.90\\\hline
К & 0.90 & 1.00
\\ \hline \end{tabular}
\end{center}
\begin{center}
 \begin{tabular}{ |c|c|c|c| } 
  \hline
     Буква & delta & min & max \\ \hline
К & 0.1000000000 & 0.9000000000 & 1.0000000000\\\hline
А & 0.0300000000 & 0.9400000000 & 0.9700000000\\\hline
В & 0.0060000000 & 0.9610000000 & 0.9670000000\\\hline
А & 0.0018000000 & 0.9634000000 & 0.9652000000\\\hline
П & 0.0007200000 & 0.9634000000 & 0.9641200000\\\hline
В & 0.0001440000 & 0.9639040000 & 0.9640480000\\\hline
П & 0.0000576000 & 0.9639040000 & 0.9639616000\\\hline
П & 0.0000230400 & 0.9639040000 & 0.9639270400\\\hline
П & 0.0000092160 & 0.9639040000 & 0.9639132160\\\hline
А & 0.0000027648 & 0.9639076864 & 0.9639104512
\\ \hline \end{tabular}
\end{center}
Результат: 96391
\pagebreak
\paragraph{Задание 5.1}

\\ 

Декодировать сообщение методом адаптивного хаффмана \\
Строка: 
'K'0'J'00'N'100'M'000'H'0010001\\
Результат: KJNMHJJJJJ

\includegraphics[width=0.8\linewidth]{/home/fizlrock/data/files/backup/code_backup/hobby/algoritms/LabExecutor/app/./doc_src/images/1738773102.jpg}

\includegraphics[width=0.8\linewidth]{/home/fizlrock/data/files/backup/code_backup/hobby/algoritms/LabExecutor/app/./doc_src/images/2052737635.jpg}

\includegraphics[width=0.8\linewidth]{/home/fizlrock/data/files/backup/code_backup/hobby/algoritms/LabExecutor/app/./doc_src/images/747490246.jpg}

\includegraphics[width=0.8\linewidth]{/home/fizlrock/data/files/backup/code_backup/hobby/algoritms/LabExecutor/app/./doc_src/images/547847810.jpg}

\includegraphics[width=0.8\linewidth]{/home/fizlrock/data/files/backup/code_backup/hobby/algoritms/LabExecutor/app/./doc_src/images/1071929525.jpg}

\includegraphics[width=0.8\linewidth]{/home/fizlrock/data/files/backup/code_backup/hobby/algoritms/LabExecutor/app/./doc_src/images/787709299.jpg}

\includegraphics[width=0.8\linewidth]{/home/fizlrock/data/files/backup/code_backup/hobby/algoritms/LabExecutor/app/./doc_src/images/590282765.jpg}

\includegraphics[width=0.8\linewidth]{/home/fizlrock/data/files/backup/code_backup/hobby/algoritms/LabExecutor/app/./doc_src/images/1974298731.jpg}

\includegraphics[width=0.8\linewidth]{/home/fizlrock/data/files/backup/code_backup/hobby/algoritms/LabExecutor/app/./doc_src/images/501281232.jpg}

\includegraphics[width=0.8\linewidth]{/home/fizlrock/data/files/backup/code_backup/hobby/algoritms/LabExecutor/app/./doc_src/images/496711415.jpg}
\pagebreak
\paragraph{Задание 5.3 Декодировать строку(LZSS)\\}

Исходная строка: [0'л'] [0'о'] [0'с'] [0'к'] [0' '] [1<5,2>] [0'т'] [1<6,4>] [1<4,1>] [0'с'][1<0,1>] [1<8,1>] [1<5,2>] [0'л']\\
\begin{table}[h!]
\centering
\begin{tabular}{|c|c|c|}
\hline
 Cловарь & Буфер & Код  \\ \hline
0'л' & [ ,  ,  ,  ,  ,  ,  ,  ,  , л] & л
\\ \hline
0'о' & [ ,  ,  ,  ,  ,  ,  ,  , л, о] & о
\\ \hline
0'с' & [ ,  ,  ,  ,  ,  ,  , л, о, с] & с
\\ \hline
0'к' & [ ,  ,  ,  ,  ,  , л, о, с, к] & к
\\ \hline
0' ' & [ ,  ,  ,  ,  , л, о, с, к,  ] &  
\\ \hline
1<5,2> & [ ,  ,  , л, о, с, к,  , л, о] & ло
\\ \hline
0'т' & [ ,  , л, о, с, к,  , л, о, т] & т
\\ \hline
1<6,4> & [с, к,  , л, о, т,  , л, о, т] &  лот
\\ \hline
1<4,1> & [к,  , л, о, т,  , л, о, т, о] & о
\\ \hline
0'с' & [ , л, о, т,  , л, о, т, о, с] & с
\\ \hline
1<0,1> & [л, о, т,  , л, о, т, о, с,  ] &  
\\ \hline
1<8,1> & [о, т,  , л, о, т, о, с,  , с] & с
\\ \hline
1<5,2> & [ , л, о, т, о, с,  , с, т, о] & то
\\ \hline
0'л' & [л, о, т, о, с,  , с, т, о, л] & л
\\ \hline
\end{tabular}
\end{table}

Результат: лоск лот лотос стол
\pagebreak
\paragraph{Задание 5.4 Декодировать строку(LZ78)\\}

Исходная строка: [0'к'] [0'у'] [0'с'] [0'о'] [1' '] [3'о'] [1'о'] [0'л'] [0' '] [6'к'] [9'к'] [4'л']\\
\begin{table}[h!]
\centering
\begin{tabular}{|c|c|c|}
\hline
 Cловарь & Буфер & Код  \\ \hline
 & [] & 
\\ \hline
0'к' & [, к] & к
\\ \hline
0'у' & [, к, у] & у
\\ \hline
0'с' & [, к, у, с] & с
\\ \hline
0'о' & [, к, у, с, о] & о
\\ \hline
1' ' & [, к, у, с, о, к ] & к 
\\ \hline
3'о' & [, к, у, с, о, к , со] & со
\\ \hline
1'о' & [, к, у, с, о, к , со, ко] & ко
\\ \hline
0'л' & [, к, у, с, о, к , со, ко, л] & л
\\ \hline
0' ' & [, к, у, с, о, к , со, ко, л,  ] &  
\\ \hline
6'к' & [, к, у, с, о, к , со, ко, л,  , сок] & сок
\\ \hline
9'к' & [, к, у, с, о, к , со, ко, л,  , сок,  к] &  к
\\ \hline
4'л' & [, к, у, с, о, к , со, ко, л,  , сок,  к, ол] & ол
\\ \hline
\end{tabular}
\end{table}

Результат: кусок сокол сок кол
\pagebreak
\subsection{Вариант №11}
\paragraph{Задание 1. Блочный хаффман \\}

Строка ВВВАСССССС, размер блока: 2
\begin{center}
 \begin{tabular}{ |c|c|l| } 
  \hline
     Буква & Вероятность & Код\\ \hline
С & 0.60 & 1\\\hline
В & 0.30 & 01\\\hline
А & 0.10 & 00
\\ \hline \end{tabular}
\end{center}
Энтропия алфавита: 1.2955
\begin{center}
 \begin{tabular}{ |c|c|l| } 
  \hline
     Блок & Вероятность & Код\\ \hline
СС & 0.36 & 11\\\hline
СВ & 0.18 & 00\\\hline
ВС & 0.18 & 01\\\hline
ВВ & 0.09 & 1011\\\hline
АС & 0.06 & 1000\\\hline
СА & 0.06 & 1001\\\hline
АВ & 0.03 & 101011\\\hline
ВА & 0.03 & 10100\\\hline
АА & 0.01 & 101010
\\ \hline \end{tabular}
\end{center}
Бит на символ при посимвольном кодировании: 1.4000, при блочном: 1.3350

\includegraphics[width=0.5\linewidth]{/home/fizlrock/data/files/backup/code_backup/hobby/algoritms/LabExecutor/app/./doc_src/images/1559289566.jpg}

\includegraphics[width=0.9\linewidth]{/home/fizlrock/data/files/backup/code_backup/hobby/algoritms/LabExecutor/app/./doc_src/images/1031512252.jpg}
\pagebreak
\paragraph{Задание 2. Сжать адаптивным хаффманом\\}

Строка: 
ЕНКПКЕКИЕЕ\\
Результат: 'Е' 0'Н' 00'К' 100'П' 01 10 11 1100'И' 10 10

\includegraphics[width=0.8\linewidth]{/home/fizlrock/data/files/backup/code_backup/hobby/algoritms/LabExecutor/app/./doc_src/images/865998206.jpg}

\includegraphics[width=0.8\linewidth]{/home/fizlrock/data/files/backup/code_backup/hobby/algoritms/LabExecutor/app/./doc_src/images/810851408.jpg}

\includegraphics[width=0.8\linewidth]{/home/fizlrock/data/files/backup/code_backup/hobby/algoritms/LabExecutor/app/./doc_src/images/1923642030.jpg}

\includegraphics[width=0.8\linewidth]{/home/fizlrock/data/files/backup/code_backup/hobby/algoritms/LabExecutor/app/./doc_src/images/560651021.jpg}

\includegraphics[width=0.8\linewidth]{/home/fizlrock/data/files/backup/code_backup/hobby/algoritms/LabExecutor/app/./doc_src/images/1144085085.jpg}

\includegraphics[width=0.8\linewidth]{/home/fizlrock/data/files/backup/code_backup/hobby/algoritms/LabExecutor/app/./doc_src/images/1186709804.jpg}

\includegraphics[width=0.8\linewidth]{/home/fizlrock/data/files/backup/code_backup/hobby/algoritms/LabExecutor/app/./doc_src/images/1884880708.jpg}

\includegraphics[width=0.8\linewidth]{/home/fizlrock/data/files/backup/code_backup/hobby/algoritms/LabExecutor/app/./doc_src/images/123595478.jpg}

\includegraphics[width=0.8\linewidth]{/home/fizlrock/data/files/backup/code_backup/hobby/algoritms/LabExecutor/app/./doc_src/images/1002393018.jpg}

\includegraphics[width=0.8\linewidth]{/home/fizlrock/data/files/backup/code_backup/hobby/algoritms/LabExecutor/app/./doc_src/images/1136208680.jpg}
\pagebreak
\paragraph{Задание 3.1}

Закодировать сообщение методом LZ77\\
Строка:ВАРВАР\_ВАРИТ\_ВАРЕНЬЕ\\
Результат: <0,0,В> <0,0,А> <0,0,Р> <7,3,\_> <3,3,И> <0,0,Т> <4,4,Е> <0,0,Н> <0,0,Ь> <0,0,Е>\\
\begin{table}[h!]
\centering
\begin{tabular}{|c|c|c|c|c|c|c|c|c|c|c|c|c|c|c|c|c|} 
\hline
\multicolumn{10}{|c|}{Cловарь} & \multicolumn{6}{c|}{Буфер} & Код  \\ \hline
  &   &   &   &   &   &   &   &   &   & \cellcolor[HTML]{8CE4F6} В & А & Р & В & А & Р & <0,0,В>
\\ \hline
  &   &   &   &   &   &   &   &   & В & \cellcolor[HTML]{8CE4F6} А & Р & В & А & Р &   & <0,0,А>
\\ \hline
  &   &   &   &   &   &   &   & В & А & \cellcolor[HTML]{8CE4F6} Р & В & А & Р &   & В & <0,0,Р>
\\ \hline
  &   &   &   &   &   &   & \cellcolor[HTML]{FFFF00} В & \cellcolor[HTML]{FFFF00} А & \cellcolor[HTML]{FFFF00} Р & \cellcolor[HTML]{FFFF00} В & \cellcolor[HTML]{FFFF00} А & \cellcolor[HTML]{FFFF00} Р & \cellcolor[HTML]{8CE4F6}   & В & А & <7,3,\_>
\\ \hline
  &   &   & \cellcolor[HTML]{FFFF00} В & \cellcolor[HTML]{FFFF00} А & \cellcolor[HTML]{FFFF00} Р & В & А & Р &   & \cellcolor[HTML]{FFFF00} В & \cellcolor[HTML]{FFFF00} А & \cellcolor[HTML]{FFFF00} Р & \cellcolor[HTML]{8CE4F6} И & Т &   & <3,3,И>
\\ \hline
А & Р & В & А & Р &   & В & А & Р & И & \cellcolor[HTML]{8CE4F6} Т &   & В & А & Р & Е & <0,0,Т>
\\ \hline
Р & В & А & Р & \cellcolor[HTML]{FFFF00}   & \cellcolor[HTML]{FFFF00} В & \cellcolor[HTML]{FFFF00} А & \cellcolor[HTML]{FFFF00} Р & И & Т & \cellcolor[HTML]{FFFF00}   & \cellcolor[HTML]{FFFF00} В & \cellcolor[HTML]{FFFF00} А & \cellcolor[HTML]{FFFF00} Р & \cellcolor[HTML]{8CE4F6} Е & Н & <4,4,Е>
\\ \hline
В & А & Р & И & Т &   & В & А & Р & Е & \cellcolor[HTML]{8CE4F6} Н & Ь & Е &   &   &   & <0,0,Н>
\\ \hline
А & Р & И & Т &   & В & А & Р & Е & Н & \cellcolor[HTML]{8CE4F6} Ь & Е &   &   &   &   & <0,0,Ь>
\\ \hline
Р & И & Т &   & В & А & Р & Е & Н & Ь & \cellcolor[HTML]{8CE4F6} Е &   &   &   &   &   & <0,0,Е>
\\ \hline
\end{tabular}
\end{table}

\paragraph{Задание 3.2}

Закодировать сообщение методом LZSS\\
Строка:ВАРВАР\_ВАРИТ\_ВАРЕНЬЕ\\
Результат: 0'В' 0'А' 0'Р' 1<7,3> 0'\_' 1<3,3> 0'И' 0'Т' 1<4,4> 0'Е' 0'Н' 0'Ь' 1<7,1>\\
\begin{table}[h!]
\centering
\begin{tabular}{|c|c|c|c|c|c|c|c|c|c|c|c|c|c|c|c|c|}
\hline
\multicolumn{10}{|c|}{Cловарь} & \multicolumn{6}{c|}{Буфер} & Код  \\ \hline
  &   &   &   &   &   &   &   &   &   & В & А & Р & В & А & Р & 0'В'\\ \hline
  &   &   &   &   &   &   &   &   & В & А & Р & В & А & Р & \_ & 0'А'\\ \hline
  &   &   &   &   &   &   &   & В & А & Р & В & А & Р & \_ & В & 0'Р'\\ \hline
  &   &   &   &   &   &   & \cellcolor[HTML]{FFFF00} В & \cellcolor[HTML]{FFFF00} А & \cellcolor[HTML]{FFFF00} Р & \cellcolor[HTML]{FFFF00} В & \cellcolor[HTML]{FFFF00} А & \cellcolor[HTML]{FFFF00} Р & \_ & В & А & 1<7,3>\\ \hline
  &   &   &   & В & А & Р & В & А & Р & \_ & В & А & Р & И & Т & 0'\_'\\ \hline
  &   &   & \cellcolor[HTML]{FFFF00} В & \cellcolor[HTML]{FFFF00} А & \cellcolor[HTML]{FFFF00} Р & В & А & Р & \_ & \cellcolor[HTML]{FFFF00} В & \cellcolor[HTML]{FFFF00} А & \cellcolor[HTML]{FFFF00} Р & И & Т & \_ & 1<3,3>\\ \hline
В & А & Р & В & А & Р & \_ & В & А & Р & И & Т & \_ & В & А & Р & 0'И'\\ \hline
А & Р & В & А & Р & \_ & В & А & Р & И & Т & \_ & В & А & Р & Е & 0'Т'\\ \hline
Р & В & А & Р & \cellcolor[HTML]{FFFF00} \_ & \cellcolor[HTML]{FFFF00} В & \cellcolor[HTML]{FFFF00} А & \cellcolor[HTML]{FFFF00} Р & И & Т & \cellcolor[HTML]{FFFF00} \_ & \cellcolor[HTML]{FFFF00} В & \cellcolor[HTML]{FFFF00} А & \cellcolor[HTML]{FFFF00} Р & Е & Н & 1<4,4>\\ \hline
\_ & В & А & Р & И & Т & \_ & В & А & Р & Е & Н & Ь & Е &   &   & 0'Е'\\ \hline
В & А & Р & И & Т & \_ & В & А & Р & Е & Н & Ь & Е &   &   &   & 0'Н'\\ \hline
А & Р & И & Т & \_ & В & А & Р & Е & Н & Ь & Е &   &   &   &   & 0'Ь'\\ \hline
Р & И & Т & \_ & В & А & Р & \cellcolor[HTML]{FFFF00} Е & Н & Ь & \cellcolor[HTML]{FFFF00} Е &   &   &   &   &   & 1<7,1>\\ \hline
\end{tabular}
\end{table}

\paragraph{Задание 3.3}

Закодировать сообщение методом LZ78\\
Строка:ВАРВАР\_ВАРИТ\_ВАРЕНЬЕ\\
\begin{table}[h!]
\centering
\begin{tabular}{|c|c|c|} 
\hline
 Входная фраза (в словарь) & Код & Позиция словаря \\ \hline

 &  & 0 \\ \hline
В & 0'В' & 1 \\ \hline
А & 0'А' & 2 \\ \hline
Р & 0'Р' & 3 \\ \hline
ВА & 1'А' & 4 \\ \hline
Р\_ & 3'\_' & 5 \\ \hline
ВАР & 4'Р' & 6 \\ \hline
И & 0'И' & 7 \\ \hline
Т & 0'Т' & 8 \\ \hline
\_ & 0'\_' & 9 \\ \hline
ВАРЕ & 6'Е' & 10 \\ \hline
Н & 0'Н' & 11 \\ \hline
Ь & 0'Ь' & 12 \\ \hline
Е & 0'Е' & 13 \\ \hline
\end{tabular}
\end{table}

Результат: 0'В' 0'А' 0'Р' 1'А' 3'\_' 4'Р' 0'И' 0'Т' 0'\_' 6'Е' 0'Н' 0'Ь' 0'Е'\\
\pagebreak
\paragraph{Задание 4. Арифметическое кодирование\\}

Исходная строка: ЕНКПКЕКИЕЕ\
\begin{center}
 \begin{tabular}{ |c|c| } 
  \hline
     Буква & Вероятность \\ \hline
Е & 0.40\\\hline
К & 0.30\\\hline
И & 0.10\\\hline
Н & 0.10\\\hline
П & 0.10
\\ \hline \end{tabular}
\end{center}
\begin{center}
 \begin{tabular}{ |c|c|c| } 
  \hline
     Буква & Начало & Конец \\ \hline
Е & 0.00 & 0.40\\\hline
К & 0.40 & 0.70\\\hline
И & 0.70 & 0.80\\\hline
Н & 0.80 & 0.90\\\hline
П & 0.90 & 1.00
\\ \hline \end{tabular}
\end{center}
\begin{center}
 \begin{tabular}{ |c|c|c|c| } 
  \hline
     Буква & delta & min & max \\ \hline
Е & 0.4000000000 & 0.0000000000 & 0.4000000000\\\hline
Н & 0.0400000000 & 0.3200000000 & 0.3600000000\\\hline
К & 0.0120000000 & 0.3360000000 & 0.3480000000\\\hline
П & 0.0012000000 & 0.3468000000 & 0.3480000000\\\hline
К & 0.0003600000 & 0.3472800000 & 0.3476400000\\\hline
Е & 0.0001440000 & 0.3472800000 & 0.3474240000\\\hline
К & 0.0000432000 & 0.3473376000 & 0.3473808000\\\hline
И & 0.0000043200 & 0.3473678400 & 0.3473721600\\\hline
Е & 0.0000017280 & 0.3473678400 & 0.3473695680\\\hline
Е & 0.0000006912 & 0.3473678400 & 0.3473685312
\\ \hline \end{tabular}
\end{center}
Результат: 347368
\pagebreak
\paragraph{Задание 5.1}

\\ 

Декодировать сообщение методом адаптивного хаффмана \\
Строка: 
'L'0'K'00'M'100'N'01000'B'10010111\\
Результат: LKMNMBBBB

\includegraphics[width=0.8\linewidth]{/home/fizlrock/data/files/backup/code_backup/hobby/algoritms/LabExecutor/app/./doc_src/images/1373630556.jpg}

\includegraphics[width=0.8\linewidth]{/home/fizlrock/data/files/backup/code_backup/hobby/algoritms/LabExecutor/app/./doc_src/images/861313506.jpg}

\includegraphics[width=0.8\linewidth]{/home/fizlrock/data/files/backup/code_backup/hobby/algoritms/LabExecutor/app/./doc_src/images/1619256870.jpg}

\includegraphics[width=0.8\linewidth]{/home/fizlrock/data/files/backup/code_backup/hobby/algoritms/LabExecutor/app/./doc_src/images/1843238309.jpg}

\includegraphics[width=0.8\linewidth]{/home/fizlrock/data/files/backup/code_backup/hobby/algoritms/LabExecutor/app/./doc_src/images/755764925.jpg}

\includegraphics[width=0.8\linewidth]{/home/fizlrock/data/files/backup/code_backup/hobby/algoritms/LabExecutor/app/./doc_src/images/1429227519.jpg}

\includegraphics[width=0.8\linewidth]{/home/fizlrock/data/files/backup/code_backup/hobby/algoritms/LabExecutor/app/./doc_src/images/1498663298.jpg}

\includegraphics[width=0.8\linewidth]{/home/fizlrock/data/files/backup/code_backup/hobby/algoritms/LabExecutor/app/./doc_src/images/591871754.jpg}

\includegraphics[width=0.8\linewidth]{/home/fizlrock/data/files/backup/code_backup/hobby/algoritms/LabExecutor/app/./doc_src/images/1578287858.jpg}
\pagebreak
\paragraph{Задание 5.3 Декодировать строку(LZSS)\\}

Исходная строка: [0'к'] [0'о'] [0'л'] [0'е'] [0'с'] [1<6,1>] [0' '] [1<7,2>] [1<1,3>] [1<4,1>][1<6,4>] [1<0,2>] [0'к']\\
\begin{table}[h!]
\centering
\begin{tabular}{|c|c|c|}
\hline
 Cловарь & Буфер & Код  \\ \hline
0'к' & [ ,  ,  ,  ,  ,  ,  ,  ,  , к] & к
\\ \hline
0'о' & [ ,  ,  ,  ,  ,  ,  ,  , к, о] & о
\\ \hline
0'л' & [ ,  ,  ,  ,  ,  ,  , к, о, л] & л
\\ \hline
0'е' & [ ,  ,  ,  ,  ,  , к, о, л, е] & е
\\ \hline
0'с' & [ ,  ,  ,  ,  , к, о, л, е, с] & с
\\ \hline
1<6,1> & [ ,  ,  ,  , к, о, л, е, с, о] & о
\\ \hline
0' ' & [ ,  ,  , к, о, л, е, с, о,  ] &  
\\ \hline
1<7,2> & [ , к, о, л, е, с, о,  , с, о] & со
\\ \hline
1<1,3> & [л, е, с, о,  , с, о, к, о, л] & кол
\\ \hline
1<4,1> & [е, с, о,  , с, о, к, о, л,  ] &  
\\ \hline
1<6,4> & [с, о, к, о, л,  , к, о, л,  ] & кол 
\\ \hline
1<0,2> & [к, о, л,  , к, о, л,  , с, о] & со
\\ \hline
0'к' & [о, л,  , к, о, л,  , с, о, к] & к
\\ \hline
\end{tabular}
\end{table}

Результат: колесо сокол кол сок
\pagebreak
\paragraph{Задание 5.4 Декодировать строку(LZ78)\\}

Исходная строка: [0'у'] [0'к'] [0'с'] [1'с'] [0' '] [1'к'] [4' '] [2'у'] [3'т'] [0'ы'] [5'к'] [4'т']\\
\begin{table}[h!]
\centering
\begin{tabular}{|c|c|c|}
\hline
 Cловарь & Буфер & Код  \\ \hline
 & [] & 
\\ \hline
0'у' & [, у] & у
\\ \hline
0'к' & [, у, к] & к
\\ \hline
0'с' & [, у, к, с] & с
\\ \hline
1'с' & [, у, к, с, ус] & ус
\\ \hline
0' ' & [, у, к, с, ус,  ] &  
\\ \hline
1'к' & [, у, к, с, ус,  , ук] & ук
\\ \hline
4' ' & [, у, к, с, ус,  , ук, ус ] & ус 
\\ \hline
2'у' & [, у, к, с, ус,  , ук, ус , ку] & ку
\\ \hline
3'т' & [, у, к, с, ус,  , ук, ус , ку, ст] & ст
\\ \hline
0'ы' & [, у, к, с, ус,  , ук, ус , ку, ст, ы] & ы
\\ \hline
5'к' & [, у, к, с, ус,  , ук, ус , ку, ст, ы,  к] &  к
\\ \hline
4'т' & [, у, к, с, ус,  , ук, ус , ку, ст, ы,  к, уст] & уст
\\ \hline
\end{tabular}
\end{table}

Результат: уксус укус кусты куст
\pagebreak
\subsection{Вариант №12}
\paragraph{Задание 1. Блочный хаффман \\}

Строка ТИИИКТКККТ, размер блока: 3
\begin{center}
 \begin{tabular}{ |c|c|l| } 
  \hline
     Буква & Вероятность & Код\\ \hline
К & 0.40 & 0\\\hline
Т & 0.30 & 10\\\hline
И & 0.30 & 11
\\ \hline \end{tabular}
\end{center}
Энтропия алфавита: 1.5710
\begin{center}
 \begin{tabular}{ |c|c|l| } 
  \hline
     Блок & Вероятность & Код\\ \hline
ККК & 0.06 & 1000\\\hline
ККИ & 0.05 & 0010\\\hline
ККТ & 0.05 & 0011\\\hline
КИК & 0.05 & 11110\\\hline
ИКК & 0.05 & 11111\\\hline
КТК & 0.05 & 0000\\\hline
ТКК & 0.05 & 0001\\\hline
ТТК & 0.04 & 10010\\\hline
ИИК & 0.04 & 10011\\\hline
КТТ & 0.04 & 10100\\\hline
КИИ & 0.04 & 10101\\\hline
ИКИ & 0.04 & 10110\\\hline
ИТК & 0.04 & 10111\\\hline
КТИ & 0.04 & 11000\\\hline
КИТ & 0.04 & 11001\\\hline
ИКТ & 0.04 & 11010\\\hline
ТКИ & 0.04 & 11011\\\hline
ТИК & 0.04 & 11100\\\hline
ТКТ & 0.04 & 11101\\\hline
ИТИ & 0.03 & 01000\\\hline
ТТИ & 0.03 & 01001\\\hline
ИТТ & 0.03 & 01010\\\hline
ИИИ & 0.03 & 01011\\\hline
ИИТ & 0.03 & 01100\\\hline
ТТТ & 0.03 & 01101\\\hline
ТИИ & 0.03 & 01110\\\hline
ТИТ & 0.03 & 01111
\\ \hline \end{tabular}
\end{center}
Бит на символ при посимвольном кодировании: 1.6000, при блочном: 1.5813

\includegraphics[width=0.5\linewidth]{/home/fizlrock/data/files/backup/code_backup/hobby/algoritms/LabExecutor/app/./doc_src/images/1985311316.jpg}

\includegraphics[width=0.9\linewidth]{/home/fizlrock/data/files/backup/code_backup/hobby/algoritms/LabExecutor/app/./doc_src/images/1178503711.jpg}
\pagebreak
\paragraph{Задание 2. Сжать адаптивным хаффманом\\}

Строка: 
УКВАУКВСАК\\
Результат: 'У' 0'К' 00'В' 100'А' 10 10 01 000'С' 001 10

\includegraphics[width=0.8\linewidth]{/home/fizlrock/data/files/backup/code_backup/hobby/algoritms/LabExecutor/app/./doc_src/images/1528405753.jpg}

\includegraphics[width=0.8\linewidth]{/home/fizlrock/data/files/backup/code_backup/hobby/algoritms/LabExecutor/app/./doc_src/images/212832177.jpg}

\includegraphics[width=0.8\linewidth]{/home/fizlrock/data/files/backup/code_backup/hobby/algoritms/LabExecutor/app/./doc_src/images/1155631602.jpg}

\includegraphics[width=0.8\linewidth]{/home/fizlrock/data/files/backup/code_backup/hobby/algoritms/LabExecutor/app/./doc_src/images/1985368088.jpg}

\includegraphics[width=0.8\linewidth]{/home/fizlrock/data/files/backup/code_backup/hobby/algoritms/LabExecutor/app/./doc_src/images/1867423014.jpg}

\includegraphics[width=0.8\linewidth]{/home/fizlrock/data/files/backup/code_backup/hobby/algoritms/LabExecutor/app/./doc_src/images/2110107182.jpg}

\includegraphics[width=0.8\linewidth]{/home/fizlrock/data/files/backup/code_backup/hobby/algoritms/LabExecutor/app/./doc_src/images/1505859688.jpg}

\includegraphics[width=0.8\linewidth]{/home/fizlrock/data/files/backup/code_backup/hobby/algoritms/LabExecutor/app/./doc_src/images/233986386.jpg}

\includegraphics[width=0.8\linewidth]{/home/fizlrock/data/files/backup/code_backup/hobby/algoritms/LabExecutor/app/./doc_src/images/167897503.jpg}

\includegraphics[width=0.8\linewidth]{/home/fizlrock/data/files/backup/code_backup/hobby/algoritms/LabExecutor/app/./doc_src/images/779932783.jpg}
\pagebreak
\paragraph{Задание 3.1}

Закодировать сообщение методом LZ77\\
Строка:СОКОЛ\_СОК\_КОЛ\_КОЛОСОК\\
Результат: <0,0,С> <0,0,О> <0,0,К> <8,1,Л> <0,0,\_> <4,3,\_> <2,4,К> <6,2,О> <0,0,С> <2,1,К>\\
\begin{table}[h!]
\centering
\begin{tabular}{|c|c|c|c|c|c|c|c|c|c|c|c|c|c|c|c|c|} 
\hline
\multicolumn{10}{|c|}{Cловарь} & \multicolumn{6}{c|}{Буфер} & Код  \\ \hline
  &   &   &   &   &   &   &   &   &   & \cellcolor[HTML]{8CE4F6} С & О & К & О & Л &   & <0,0,С>
\\ \hline
  &   &   &   &   &   &   &   &   & С & \cellcolor[HTML]{8CE4F6} О & К & О & Л &   & С & <0,0,О>
\\ \hline
  &   &   &   &   &   &   &   & С & О & \cellcolor[HTML]{8CE4F6} К & О & Л &   & С & О & <0,0,К>
\\ \hline
  &   &   &   &   &   &   & С & \cellcolor[HTML]{FFFF00} О & К & \cellcolor[HTML]{FFFF00} О & \cellcolor[HTML]{8CE4F6} Л &   & С & О & К & <8,1,Л>
\\ \hline
  &   &   &   &   & С & О & К & О & Л & \cellcolor[HTML]{8CE4F6}   & С & О & К &   & К & <0,0,\_>
\\ \hline
  &   &   &   & \cellcolor[HTML]{FFFF00} С & \cellcolor[HTML]{FFFF00} О & \cellcolor[HTML]{FFFF00} К & О & Л &   & \cellcolor[HTML]{FFFF00} С & \cellcolor[HTML]{FFFF00} О & \cellcolor[HTML]{FFFF00} К & \cellcolor[HTML]{8CE4F6}   & К & О & <4,3,\_>
\\ \hline
С & О & \cellcolor[HTML]{FFFF00} К & \cellcolor[HTML]{FFFF00} О & \cellcolor[HTML]{FFFF00} Л & \cellcolor[HTML]{FFFF00}   & С & О & К &   & \cellcolor[HTML]{FFFF00} К & \cellcolor[HTML]{FFFF00} О & \cellcolor[HTML]{FFFF00} Л & \cellcolor[HTML]{FFFF00}   & \cellcolor[HTML]{8CE4F6} К & О & <2,4,К>
\\ \hline
  & С & О & К &   & К & \cellcolor[HTML]{FFFF00} О & \cellcolor[HTML]{FFFF00} Л &   & К & \cellcolor[HTML]{FFFF00} О & \cellcolor[HTML]{FFFF00} Л & \cellcolor[HTML]{8CE4F6} О & С & О & К & <6,2,О>
\\ \hline
К &   & К & О & Л &   & К & О & Л & О & \cellcolor[HTML]{8CE4F6} С & О & К &   &   &   & <0,0,С>
\\ \hline
  & К & \cellcolor[HTML]{FFFF00} О & Л &   & К & О & Л & О & С & \cellcolor[HTML]{FFFF00} О & \cellcolor[HTML]{8CE4F6} К &   &   &   &   & <2,1,К>
\\ \hline
\end{tabular}
\end{table}

\paragraph{Задание 3.2}

Закодировать сообщение методом LZSS\\
Строка:СОКОЛ\_СОК\_КОЛ\_КОЛОСОК\\
Результат: 0'С' 0'О' 0'К' 1<8,1> 0'Л' 0'\_' 1<4,3> 1<6,1> 1<2,4> 1<6,3> 1<0,1> 0'С' 1<2,1> 1<0,1>\\
\begin{table}[h!]
\centering
\begin{tabular}{|c|c|c|c|c|c|c|c|c|c|c|c|c|c|c|c|c|}
\hline
\multicolumn{10}{|c|}{Cловарь} & \multicolumn{6}{c|}{Буфер} & Код  \\ \hline
  &   &   &   &   &   &   &   &   &   & С & О & К & О & Л & \_ & 0'С'\\ \hline
  &   &   &   &   &   &   &   &   & С & О & К & О & Л & \_ & С & 0'О'\\ \hline
  &   &   &   &   &   &   &   & С & О & К & О & Л & \_ & С & О & 0'К'\\ \hline
  &   &   &   &   &   &   & С & \cellcolor[HTML]{FFFF00} О & К & \cellcolor[HTML]{FFFF00} О & Л & \_ & С & О & К & 1<8,1>\\ \hline
  &   &   &   &   &   & С & О & К & О & Л & \_ & С & О & К & \_ & 0'Л'\\ \hline
  &   &   &   &   & С & О & К & О & Л & \_ & С & О & К & \_ & К & 0'\_'\\ \hline
  &   &   &   & \cellcolor[HTML]{FFFF00} С & \cellcolor[HTML]{FFFF00} О & \cellcolor[HTML]{FFFF00} К & О & Л & \_ & \cellcolor[HTML]{FFFF00} С & \cellcolor[HTML]{FFFF00} О & \cellcolor[HTML]{FFFF00} К & \_ & К & О & 1<4,3>\\ \hline
  & С & О & К & О & Л & \cellcolor[HTML]{FFFF00} \_ & С & О & К & \cellcolor[HTML]{FFFF00} \_ & К & О & Л & \_ & К & 1<6,1>\\ \hline
С & О & \cellcolor[HTML]{FFFF00} К & \cellcolor[HTML]{FFFF00} О & \cellcolor[HTML]{FFFF00} Л & \cellcolor[HTML]{FFFF00} \_ & С & О & К & \_ & \cellcolor[HTML]{FFFF00} К & \cellcolor[HTML]{FFFF00} О & \cellcolor[HTML]{FFFF00} Л & \cellcolor[HTML]{FFFF00} \_ & К & О & 1<2,4>\\ \hline
Л & \_ & С & О & К & \_ & \cellcolor[HTML]{FFFF00} К & \cellcolor[HTML]{FFFF00} О & \cellcolor[HTML]{FFFF00} Л & \_ & \cellcolor[HTML]{FFFF00} К & \cellcolor[HTML]{FFFF00} О & \cellcolor[HTML]{FFFF00} Л & О & С & О & 1<6,3>\\ \hline
\cellcolor[HTML]{FFFF00} О & К & \_ & К & О & Л & \_ & К & О & Л & \cellcolor[HTML]{FFFF00} О & С & О & К &   &   & 1<0,1>\\ \hline
К & \_ & К & О & Л & \_ & К & О & Л & О & С & О & К &   &   &   & 0'С'\\ \hline
\_ & К & \cellcolor[HTML]{FFFF00} О & Л & \_ & К & О & Л & О & С & \cellcolor[HTML]{FFFF00} О & К &   &   &   &   & 1<2,1>\\ \hline
\cellcolor[HTML]{FFFF00} К & О & Л & \_ & К & О & Л & О & С & О & \cellcolor[HTML]{FFFF00} К &   &   &   &   &   & 1<0,1>\\ \hline
\end{tabular}
\end{table}

\paragraph{Задание 3.3}

Закодировать сообщение методом LZ78\\
Строка:СОКОЛ\_СОК\_КОЛ\_КОЛОСОК\\
\begin{table}[h!]
\centering
\begin{tabular}{|c|c|c|} 
\hline
 Входная фраза (в словарь) & Код & Позиция словаря \\ \hline

 &  & 0 \\ \hline
С & 0'С' & 1 \\ \hline
О & 0'О' & 2 \\ \hline
К & 0'К' & 3 \\ \hline
ОЛ & 2'Л' & 4 \\ \hline
\_ & 0'\_' & 5 \\ \hline
СО & 1'О' & 6 \\ \hline
К\_ & 3'\_' & 7 \\ \hline
КО & 3'О' & 8 \\ \hline
Л & 0'Л' & 9 \\ \hline
\_К & 5'К' & 10 \\ \hline
ОЛО & 4'О' & 11 \\ \hline
СОК & 6'К' & 12 \\ \hline
\end{tabular}
\end{table}

Результат: 0'С' 0'О' 0'К' 2'Л' 0'\_' 1'О' 3'\_' 3'О' 0'Л' 5'К' 4'О' 6'К'\\
\pagebreak
\paragraph{Задание 4. Арифметическое кодирование\\}

Исходная строка: УКВАУКВСАК\
\begin{center}
 \begin{tabular}{ |c|c| } 
  \hline
     Буква & Вероятность \\ \hline
К & 0.30\\\hline
А & 0.20\\\hline
В & 0.20\\\hline
У & 0.20\\\hline
С & 0.10
\\ \hline \end{tabular}
\end{center}
\begin{center}
 \begin{tabular}{ |c|c|c| } 
  \hline
     Буква & Начало & Конец \\ \hline
К & 0.00 & 0.30\\\hline
А & 0.30 & 0.50\\\hline
В & 0.50 & 0.70\\\hline
У & 0.70 & 0.90\\\hline
С & 0.90 & 1.00
\\ \hline \end{tabular}
\end{center}
\begin{center}
 \begin{tabular}{ |c|c|c|c| } 
  \hline
     Буква & delta & min & max \\ \hline
У & 0.2000000000 & 0.7000000000 & 0.9000000000\\\hline
К & 0.0600000000 & 0.7000000000 & 0.7600000000\\\hline
В & 0.0120000000 & 0.7300000000 & 0.7420000000\\\hline
А & 0.0024000000 & 0.7336000000 & 0.7360000000\\\hline
У & 0.0004800000 & 0.7352800000 & 0.7357600000\\\hline
К & 0.0001440000 & 0.7352800000 & 0.7354240000\\\hline
В & 0.0000288000 & 0.7353520000 & 0.7353808000\\\hline
С & 0.0000028800 & 0.7353779200 & 0.7353808000\\\hline
А & 0.0000005760 & 0.7353787840 & 0.7353793600\\\hline
К & 0.0000001728 & 0.7353787840 & 0.7353789568
\\ \hline \end{tabular}
\end{center}
Результат: 7353788
\pagebreak
\paragraph{Задание 5.1}

\\ 

Декодировать сообщение методом адаптивного хаффмана \\
Строка: 
'Y'0'T'00'R'100'F'01001111101111111\\
Результат: YTRFRFRYYR

\includegraphics[width=0.8\linewidth]{/home/fizlrock/data/files/backup/code_backup/hobby/algoritms/LabExecutor/app/./doc_src/images/1572557693.jpg}

\includegraphics[width=0.8\linewidth]{/home/fizlrock/data/files/backup/code_backup/hobby/algoritms/LabExecutor/app/./doc_src/images/154391448.jpg}

\includegraphics[width=0.8\linewidth]{/home/fizlrock/data/files/backup/code_backup/hobby/algoritms/LabExecutor/app/./doc_src/images/204903466.jpg}

\includegraphics[width=0.8\linewidth]{/home/fizlrock/data/files/backup/code_backup/hobby/algoritms/LabExecutor/app/./doc_src/images/505760112.jpg}

\includegraphics[width=0.8\linewidth]{/home/fizlrock/data/files/backup/code_backup/hobby/algoritms/LabExecutor/app/./doc_src/images/1833909618.jpg}

\includegraphics[width=0.8\linewidth]{/home/fizlrock/data/files/backup/code_backup/hobby/algoritms/LabExecutor/app/./doc_src/images/965288597.jpg}

\includegraphics[width=0.8\linewidth]{/home/fizlrock/data/files/backup/code_backup/hobby/algoritms/LabExecutor/app/./doc_src/images/657036906.jpg}

\includegraphics[width=0.8\linewidth]{/home/fizlrock/data/files/backup/code_backup/hobby/algoritms/LabExecutor/app/./doc_src/images/1149342571.jpg}

\includegraphics[width=0.8\linewidth]{/home/fizlrock/data/files/backup/code_backup/hobby/algoritms/LabExecutor/app/./doc_src/images/287883362.jpg}

\includegraphics[width=0.8\linewidth]{/home/fizlrock/data/files/backup/code_backup/hobby/algoritms/LabExecutor/app/./doc_src/images/394366283.jpg}
\pagebreak
\paragraph{Задание 5.3 Декодировать строку(LZSS)\\}

Исходная строка: [0'л'] [0'о'] [0'т'] [1<8,1>] [0' '] [1<5,3>] [1<6,1>] [1<3,2>] [1<4,1>][1<2,1>] [0'с'] [1<5,2>] [0'л']\\
\begin{table}[h!]
\centering
\begin{tabular}{|c|c|c|}
\hline
 Cловарь & Буфер & Код  \\ \hline
0'л' & [ ,  ,  ,  ,  ,  ,  ,  ,  , л] & л
\\ \hline
0'о' & [ ,  ,  ,  ,  ,  ,  ,  , л, о] & о
\\ \hline
0'т' & [ ,  ,  ,  ,  ,  ,  , л, о, т] & т
\\ \hline
1<8,1> & [ ,  ,  ,  ,  ,  , л, о, т, о] & о
\\ \hline
0' ' & [ ,  ,  ,  ,  , л, о, т, о,  ] &  
\\ \hline
1<5,3> & [ ,  , л, о, т, о,  , л, о, т] & лот
\\ \hline
1<6,1> & [ , л, о, т, о,  , л, о, т,  ] &  
\\ \hline
1<3,2> & [о, т, о,  , л, о, т,  , т, о] & то
\\ \hline
1<4,1> & [т, о,  , л, о, т,  , т, о, л] & л
\\ \hline
1<2,1> & [о,  , л, о, т,  , т, о, л,  ] &  
\\ \hline
0'с' & [ , л, о, т,  , т, о, л,  , с] & с
\\ \hline
1<5,2> & [о, т,  , т, о, л,  , с, т, о] & то
\\ \hline
0'л' & [т,  , т, о, л,  , с, т, о, л] & л
\\ \hline
\end{tabular}
\end{table}

Результат: лото лот тол стол
\pagebreak
\paragraph{Задание 5.4 Декодировать строку(LZ78)\\}

Исходная строка: [0'д'] [0'о'] [0'р'] [2'г'] [0'а'] [0' '] [0'г'] [2'р'] [5' '] [7'о'] [3'о'] [0'д']\\
\begin{table}[h!]
\centering
\begin{tabular}{|c|c|c|}
\hline
 Cловарь & Буфер & Код  \\ \hline
 & [] & 
\\ \hline
0'д' & [, д] & д
\\ \hline
0'о' & [, д, о] & о
\\ \hline
0'р' & [, д, о, р] & р
\\ \hline
2'г' & [, д, о, р, ог] & ог
\\ \hline
0'а' & [, д, о, р, ог, а] & а
\\ \hline
0' ' & [, д, о, р, ог, а,  ] &  
\\ \hline
0'г' & [, д, о, р, ог, а,  , г] & г
\\ \hline
2'р' & [, д, о, р, ог, а,  , г, ор] & ор
\\ \hline
5' ' & [, д, о, р, ог, а,  , г, ор, а ] & а 
\\ \hline
7'о' & [, д, о, р, ог, а,  , г, ор, а , го] & го
\\ \hline
3'о' & [, д, о, р, ог, а,  , г, ор, а , го, ро] & ро
\\ \hline
0'д' & [, д, о, р, ог, а,  , г, ор, а , го, ро, д] & д
\\ \hline
\end{tabular}
\end{table}

Результат: дорога гора город
\pagebreak
\subsection{Вариант №13}
\paragraph{Задание 1. Блочный хаффман \\}

Строка БОББББОБОО, размер блока: 3
\begin{center}
 \begin{tabular}{ |c|c|l| } 
  \hline
     Буква & Вероятность & Код\\ \hline
Б & 0.60 & 1\\\hline
О & 0.40 & 0
\\ \hline \end{tabular}
\end{center}
Энтропия алфавита: 0.9710
\begin{center}
 \begin{tabular}{ |c|c|l| } 
  \hline
     Блок & Вероятность & Код\\ \hline
БББ & 0.22 & 01\\\hline
ББО & 0.14 & 100\\\hline
ОББ & 0.14 & 101\\\hline
БОБ & 0.14 & 110\\\hline
ООБ & 0.10 & 001\\\hline
ОБО & 0.10 & 1111\\\hline
БОО & 0.10 & 000\\\hline
ООО & 0.06 & 1110
\\ \hline \end{tabular}
\end{center}
Бит на символ при посимвольном кодировании: 1.0000, при блочном: 0.9813

\includegraphics[width=0.5\linewidth]{/home/fizlrock/data/files/backup/code_backup/hobby/algoritms/LabExecutor/app/./doc_src/images/1868796901.jpg}

\includegraphics[width=0.9\linewidth]{/home/fizlrock/data/files/backup/code_backup/hobby/algoritms/LabExecutor/app/./doc_src/images/145824706.jpg}
\pagebreak
\paragraph{Задание 2. Сжать адаптивным хаффманом\\}

Строка: 
ЛПРИРПТОРТ\\
Результат: 'Л' 0'П' 00'Р' 100'И' 01 01 000'Т' 0100'О' 01 001

\includegraphics[width=0.8\linewidth]{/home/fizlrock/data/files/backup/code_backup/hobby/algoritms/LabExecutor/app/./doc_src/images/1420705904.jpg}

\includegraphics[width=0.8\linewidth]{/home/fizlrock/data/files/backup/code_backup/hobby/algoritms/LabExecutor/app/./doc_src/images/1324401082.jpg}

\includegraphics[width=0.8\linewidth]{/home/fizlrock/data/files/backup/code_backup/hobby/algoritms/LabExecutor/app/./doc_src/images/1201083210.jpg}

\includegraphics[width=0.8\linewidth]{/home/fizlrock/data/files/backup/code_backup/hobby/algoritms/LabExecutor/app/./doc_src/images/460342720.jpg}

\includegraphics[width=0.8\linewidth]{/home/fizlrock/data/files/backup/code_backup/hobby/algoritms/LabExecutor/app/./doc_src/images/2107074943.jpg}

\includegraphics[width=0.8\linewidth]{/home/fizlrock/data/files/backup/code_backup/hobby/algoritms/LabExecutor/app/./doc_src/images/1427356568.jpg}

\includegraphics[width=0.8\linewidth]{/home/fizlrock/data/files/backup/code_backup/hobby/algoritms/LabExecutor/app/./doc_src/images/1898780409.jpg}

\includegraphics[width=0.8\linewidth]{/home/fizlrock/data/files/backup/code_backup/hobby/algoritms/LabExecutor/app/./doc_src/images/1904404997.jpg}

\includegraphics[width=0.8\linewidth]{/home/fizlrock/data/files/backup/code_backup/hobby/algoritms/LabExecutor/app/./doc_src/images/1726762116.jpg}

\includegraphics[width=0.8\linewidth]{/home/fizlrock/data/files/backup/code_backup/hobby/algoritms/LabExecutor/app/./doc_src/images/1855554162.jpg}
\pagebreak

\paragraph{Задание 3.2}

Закодировать сообщение методом LZSS\\
Строка:ПЕС\_ПЕСОК\_СОКОЛ\_СКОЛ\\
Результат: 0'П' 0'Е' 0'С' 0'\_' 1<6,3> 0'О' 0'К' 1<4,1> 1<6,3> 1<4,1> 0'Л' 1<4,2> 1<5,3>\\
\begin{table}[h!]
\centering
\begin{tabular}{|c|c|c|c|c|c|c|c|c|c|c|c|c|c|c|c|c|}
\hline
\multicolumn{10}{|c|}{Cловарь} & \multicolumn{6}{c|}{Буфер} & Код  \\ \hline
  &   &   &   &   &   &   &   &   &   & П & Е & С & \_ & П & Е & 0'П'\\ \hline
  &   &   &   &   &   &   &   &   & П & Е & С & \_ & П & Е & С & 0'Е'\\ \hline
  &   &   &   &   &   &   &   & П & Е & С & \_ & П & Е & С & О & 0'С'\\ \hline
  &   &   &   &   &   &   & П & Е & С & \_ & П & Е & С & О & К & 0'\_'\\ \hline
  &   &   &   &   &   & \cellcolor[HTML]{FFFF00} П & \cellcolor[HTML]{FFFF00} Е & \cellcolor[HTML]{FFFF00} С & \_ & \cellcolor[HTML]{FFFF00} П & \cellcolor[HTML]{FFFF00} Е & \cellcolor[HTML]{FFFF00} С & О & К & \_ & 1<6,3>\\ \hline
  &   &   & П & Е & С & \_ & П & Е & С & О & К & \_ & С & О & К & 0'О'\\ \hline
  &   & П & Е & С & \_ & П & Е & С & О & К & \_ & С & О & К & О & 0'К'\\ \hline
  & П & Е & С & \cellcolor[HTML]{FFFF00} \_ & П & Е & С & О & К & \cellcolor[HTML]{FFFF00} \_ & С & О & К & О & Л & 1<4,1>\\ \hline
П & Е & С & \_ & П & Е & \cellcolor[HTML]{FFFF00} С & \cellcolor[HTML]{FFFF00} О & \cellcolor[HTML]{FFFF00} К & \_ & \cellcolor[HTML]{FFFF00} С & \cellcolor[HTML]{FFFF00} О & \cellcolor[HTML]{FFFF00} К & О & Л & \_ & 1<6,3>\\ \hline
\_ & П & Е & С & \cellcolor[HTML]{FFFF00} О & К & \_ & С & О & К & \cellcolor[HTML]{FFFF00} О & Л & \_ & С & К & О & 1<4,1>\\ \hline
П & Е & С & О & К & \_ & С & О & К & О & Л & \_ & С & К & О & Л & 0'Л'\\ \hline
Е & С & О & К & \cellcolor[HTML]{FFFF00} \_ & \cellcolor[HTML]{FFFF00} С & О & К & О & Л & \cellcolor[HTML]{FFFF00} \_ & \cellcolor[HTML]{FFFF00} С & К & О & Л &   & 1<4,2>\\ \hline
О & К & \_ & С & О & \cellcolor[HTML]{FFFF00} К & \cellcolor[HTML]{FFFF00} О & \cellcolor[HTML]{FFFF00} Л & \_ & С & \cellcolor[HTML]{FFFF00} К & \cellcolor[HTML]{FFFF00} О & \cellcolor[HTML]{FFFF00} Л &   &   &   & 1<5,3>\\ \hline
\end{tabular}
\end{table}

\paragraph{Задание 3.3}

Закодировать сообщение методом LZ78\\
Строка:ПЕС\_ПЕСОК\_СОКОЛ\_СКОЛ\\
\begin{table}[h!]
\centering
\begin{tabular}{|c|c|c|} 
\hline
 Входная фраза (в словарь) & Код & Позиция словаря \\ \hline

 &  & 0 \\ \hline
П & 0'П' & 1 \\ \hline
Е & 0'Е' & 2 \\ \hline
С & 0'С' & 3 \\ \hline
\_ & 0'\_' & 4 \\ \hline
ПЕ & 1'Е' & 5 \\ \hline
СО & 3'О' & 6 \\ \hline
К & 0'К' & 7 \\ \hline
\_С & 4'С' & 8 \\ \hline
О & 0'О' & 9 \\ \hline
КО & 7'О' & 10 \\ \hline
Л & 0'Л' & 11 \\ \hline
\_СК & 8'К' & 12 \\ \hline
ОЛ & 9'Л' & 13 \\ \hline
\end{tabular}
\end{table}

Результат: 0'П' 0'Е' 0'С' 0'\_' 1'Е' 3'О' 0'К' 4'С' 0'О' 7'О' 0'Л' 8'К' 9'Л'\\
\pagebreak
\paragraph{Задание 4. Арифметическое кодирование\\}

Исходная строка: ЛПРИРПТОРТ\
\begin{center}
 \begin{tabular}{ |c|c| } 
  \hline
     Буква & Вероятность \\ \hline
Р & 0.30\\\hline
Т & 0.20\\\hline
П & 0.20\\\hline
И & 0.10\\\hline
Л & 0.10\\\hline
О & 0.10
\\ \hline \end{tabular}
\end{center}
\begin{center}
 \begin{tabular}{ |c|c|c| } 
  \hline
     Буква & Начало & Конец \\ \hline
Р & 0.00 & 0.30\\\hline
Т & 0.30 & 0.50\\\hline
П & 0.50 & 0.70\\\hline
И & 0.70 & 0.80\\\hline
Л & 0.80 & 0.90\\\hline
О & 0.90 & 1.00
\\ \hline \end{tabular}
\end{center}
\begin{center}
 \begin{tabular}{ |c|c|c|c| } 
  \hline
     Буква & delta & min & max \\ \hline
Л & 0.1000000000 & 0.8000000000 & 0.9000000000\\\hline
П & 0.0200000000 & 0.8500000000 & 0.8700000000\\\hline
Р & 0.0060000000 & 0.8500000000 & 0.8560000000\\\hline
И & 0.0006000000 & 0.8542000000 & 0.8548000000\\\hline
Р & 0.0001800000 & 0.8542000000 & 0.8543800000\\\hline
П & 0.0000360000 & 0.8542900000 & 0.8543260000\\\hline
Т & 0.0000072000 & 0.8543008000 & 0.8543080000\\\hline
О & 0.0000007200 & 0.8543072800 & 0.8543080000\\\hline
Р & 0.0000002160 & 0.8543072800 & 0.8543074960\\\hline
Т & 0.0000000432 & 0.8543073448 & 0.8543073880
\\ \hline \end{tabular}
\end{center}
Результат: 85430735
\pagebreak
\paragraph{Задание 5.1}

\\ 

Декодировать сообщение методом адаптивного хаффмана \\
Строка: 
'D'0'C'00'V'1110111100'F'100111110\\
Результат: DCVCVVFFFF

\includegraphics[width=0.8\linewidth]{/home/fizlrock/data/files/backup/code_backup/hobby/algoritms/LabExecutor/app/./doc_src/images/1886599241.jpg}

\includegraphics[width=0.8\linewidth]{/home/fizlrock/data/files/backup/code_backup/hobby/algoritms/LabExecutor/app/./doc_src/images/627019822.jpg}

\includegraphics[width=0.8\linewidth]{/home/fizlrock/data/files/backup/code_backup/hobby/algoritms/LabExecutor/app/./doc_src/images/1864778826.jpg}

\includegraphics[width=0.8\linewidth]{/home/fizlrock/data/files/backup/code_backup/hobby/algoritms/LabExecutor/app/./doc_src/images/619906846.jpg}

\includegraphics[width=0.8\linewidth]{/home/fizlrock/data/files/backup/code_backup/hobby/algoritms/LabExecutor/app/./doc_src/images/112328396.jpg}

\includegraphics[width=0.8\linewidth]{/home/fizlrock/data/files/backup/code_backup/hobby/algoritms/LabExecutor/app/./doc_src/images/1421305105.jpg}

\includegraphics[width=0.8\linewidth]{/home/fizlrock/data/files/backup/code_backup/hobby/algoritms/LabExecutor/app/./doc_src/images/1204131573.jpg}

\includegraphics[width=0.8\linewidth]{/home/fizlrock/data/files/backup/code_backup/hobby/algoritms/LabExecutor/app/./doc_src/images/221110041.jpg}

\includegraphics[width=0.8\linewidth]{/home/fizlrock/data/files/backup/code_backup/hobby/algoritms/LabExecutor/app/./doc_src/images/140224900.jpg}

\includegraphics[width=0.8\linewidth]{/home/fizlrock/data/files/backup/code_backup/hobby/algoritms/LabExecutor/app/./doc_src/images/2107328794.jpg}
\pagebreak
\paragraph{Задание 5.3 Декодировать строку(LZSS)\\}

Исходная строка: [0'б'] [0'е'] [0'р'] [1<8,1>] [0'з'] [0'а'] [0' '] [1<3,4>] [0'г'] [1<4,4>] [0'л'] [0'о'] [1<3,1>] [0'а']\\
\begin{table}[h!]
\centering
\begin{tabular}{|c|c|c|}
\hline
 Cловарь & Буфер & Код  \\ \hline
0'б' & [ ,  ,  ,  ,  ,  ,  ,  ,  , б] & б
\\ \hline
0'е' & [ ,  ,  ,  ,  ,  ,  ,  , б, е] & е
\\ \hline
0'р' & [ ,  ,  ,  ,  ,  ,  , б, е, р] & р
\\ \hline
1<8,1> & [ ,  ,  ,  ,  ,  , б, е, р, е] & е
\\ \hline
0'з' & [ ,  ,  ,  ,  , б, е, р, е, з] & з
\\ \hline
0'а' & [ ,  ,  ,  , б, е, р, е, з, а] & а
\\ \hline
0' ' & [ ,  ,  , б, е, р, е, з, а,  ] &  
\\ \hline
1<3,4> & [е, р, е, з, а,  , б, е, р, е] & бере
\\ \hline
0'г' & [р, е, з, а,  , б, е, р, е, г] & г
\\ \hline
1<4,4> & [ , б, е, р, е, г,  , б, е, р] &  бер
\\ \hline
0'л' & [б, е, р, е, г,  , б, е, р, л] & л
\\ \hline
0'о' & [е, р, е, г,  , б, е, р, л, о] & о
\\ \hline
1<3,1> & [р, е, г,  , б, е, р, л, о, г] & г
\\ \hline
0'а' & [е, г,  , б, е, р, л, о, г, а] & а
\\ \hline
\end{tabular}
\end{table}

Результат: береза берег берлога
\pagebreak
\paragraph{Задание 5.4 Декодировать строку(LZ78)\\}

Исходная строка: [0'п'] [0'о'] [0'р'] [0'т'] [0' '] [1'о'] [3'а'] [5'р'] [0'а'] [6'р'] [0'т']\\
\begin{table}[h!]
\centering
\begin{tabular}{|c|c|c|}
\hline
 Cловарь & Буфер & Код  \\ \hline
 & [] & 
\\ \hline
0'п' & [, п] & п
\\ \hline
0'о' & [, п, о] & о
\\ \hline
0'р' & [, п, о, р] & р
\\ \hline
0'т' & [, п, о, р, т] & т
\\ \hline
0' ' & [, п, о, р, т,  ] &  
\\ \hline
1'о' & [, п, о, р, т,  , по] & по
\\ \hline
3'а' & [, п, о, р, т,  , по, ра] & ра
\\ \hline
5'р' & [, п, о, р, т,  , по, ра,  р] &  р
\\ \hline
0'а' & [, п, о, р, т,  , по, ра,  р, а] & а
\\ \hline
6'р' & [, п, о, р, т,  , по, ра,  р, а, пор] & пор
\\ \hline
0'т' & [, п, о, р, т,  , по, ра,  р, а, пор, т] & т
\\ \hline
\end{tabular}
\end{table}

Результат: порт пора рапорт
\pagebreak
\subsection{Вариант №14}
\paragraph{Задание 1. Блочный хаффман \\}

Строка КРООРТТТТТ, размер блока: 2
\begin{center}
 \begin{tabular}{ |c|c|l| } 
  \hline
     Буква & Вероятность & Код\\ \hline
Т & 0.50 & 0\\\hline
Р & 0.20 & 111\\\hline
О & 0.20 & 10\\\hline
К & 0.10 & 110
\\ \hline \end{tabular}
\end{center}
Энтропия алфавита: 1.7610
\begin{center}
 \begin{tabular}{ |c|c|l| } 
  \hline
     Блок & Вероятность & Код\\ \hline
ТТ & 0.25 & 10\\\hline
РТ & 0.10 & 1111\\\hline
ОТ & 0.10 & 000\\\hline
ТО & 0.10 & 001\\\hline
ТР & 0.10 & 010\\\hline
КТ & 0.05 & 11101\\\hline
ТК & 0.05 & 0110\\\hline
РР & 0.04 & 11000\\\hline
ОО & 0.04 & 11001\\\hline
ОР & 0.04 & 11010\\\hline
РО & 0.04 & 11011\\\hline
КО & 0.02 & 011111\\\hline
КР & 0.02 & 111000\\\hline
РК & 0.02 & 111001\\\hline
ОК & 0.02 & 01110\\\hline
КК & 0.01 & 011110
\\ \hline \end{tabular}
\end{center}
Бит на символ при посимвольном кодировании: 1.8000, при блочном: 1.7850

\includegraphics[width=0.5\linewidth]{/home/fizlrock/data/files/backup/code_backup/hobby/algoritms/LabExecutor/app/./doc_src/images/477869281.jpg}

\includegraphics[width=0.9\linewidth]{/home/fizlrock/data/files/backup/code_backup/hobby/algoritms/LabExecutor/app/./doc_src/images/2047682566.jpg}
\pagebreak
\paragraph{Задание 2. Сжать адаптивным хаффманом\\}

Строка: 
СААВИПВАИИ\\
Результат: 'С' 0'А' 01 00'В' 000'И' 1100'П' 01 11 101 00

\includegraphics[width=0.8\linewidth]{/home/fizlrock/data/files/backup/code_backup/hobby/algoritms/LabExecutor/app/./doc_src/images/1752764404.jpg}

\includegraphics[width=0.8\linewidth]{/home/fizlrock/data/files/backup/code_backup/hobby/algoritms/LabExecutor/app/./doc_src/images/1549729993.jpg}

\includegraphics[width=0.8\linewidth]{/home/fizlrock/data/files/backup/code_backup/hobby/algoritms/LabExecutor/app/./doc_src/images/1876416923.jpg}

\includegraphics[width=0.8\linewidth]{/home/fizlrock/data/files/backup/code_backup/hobby/algoritms/LabExecutor/app/./doc_src/images/1627580138.jpg}

\includegraphics[width=0.8\linewidth]{/home/fizlrock/data/files/backup/code_backup/hobby/algoritms/LabExecutor/app/./doc_src/images/1733995598.jpg}

\includegraphics[width=0.8\linewidth]{/home/fizlrock/data/files/backup/code_backup/hobby/algoritms/LabExecutor/app/./doc_src/images/393805205.jpg}

\includegraphics[width=0.8\linewidth]{/home/fizlrock/data/files/backup/code_backup/hobby/algoritms/LabExecutor/app/./doc_src/images/1386961094.jpg}

\includegraphics[width=0.8\linewidth]{/home/fizlrock/data/files/backup/code_backup/hobby/algoritms/LabExecutor/app/./doc_src/images/224072030.jpg}

\includegraphics[width=0.8\linewidth]{/home/fizlrock/data/files/backup/code_backup/hobby/algoritms/LabExecutor/app/./doc_src/images/1733206907.jpg}

\includegraphics[width=0.8\linewidth]{/home/fizlrock/data/files/backup/code_backup/hobby/algoritms/LabExecutor/app/./doc_src/images/860903282.jpg}
\pagebreak

\paragraph{Задание 3.2}

Закодировать сообщение методом LZSS\\
Строка:РАБ\_РАБА\_БАК\_БАКЕН\_БАК\\
Результат: 0'Р' 0'А' 0'Б' 0'\_' 1<6,3> 1<4,1> 1<5,1> 1<7,2> 0'К' 1<6,4> 0'Е' 0'Н' 1<0,4>\\
\begin{table}[h!]
\centering
\begin{tabular}{|c|c|c|c|c|c|c|c|c|c|c|c|c|c|c|c|c|}
\hline
\multicolumn{10}{|c|}{Cловарь} & \multicolumn{6}{c|}{Буфер} & Код  \\ \hline
  &   &   &   &   &   &   &   &   &   & Р & А & Б & \_ & Р & А & 0'Р'\\ \hline
  &   &   &   &   &   &   &   &   & Р & А & Б & \_ & Р & А & Б & 0'А'\\ \hline
  &   &   &   &   &   &   &   & Р & А & Б & \_ & Р & А & Б & А & 0'Б'\\ \hline
  &   &   &   &   &   &   & Р & А & Б & \_ & Р & А & Б & А & \_ & 0'\_'\\ \hline
  &   &   &   &   &   & \cellcolor[HTML]{FFFF00} Р & \cellcolor[HTML]{FFFF00} А & \cellcolor[HTML]{FFFF00} Б & \_ & \cellcolor[HTML]{FFFF00} Р & \cellcolor[HTML]{FFFF00} А & \cellcolor[HTML]{FFFF00} Б & А & \_ & Б & 1<6,3>\\ \hline
  &   &   & Р & \cellcolor[HTML]{FFFF00} А & Б & \_ & Р & А & Б & \cellcolor[HTML]{FFFF00} А & \_ & Б & А & К & \_ & 1<4,1>\\ \hline
  &   & Р & А & Б & \cellcolor[HTML]{FFFF00} \_ & Р & А & Б & А & \cellcolor[HTML]{FFFF00} \_ & Б & А & К & \_ & Б & 1<5,1>\\ \hline
  & Р & А & Б & \_ & Р & А & \cellcolor[HTML]{FFFF00} Б & \cellcolor[HTML]{FFFF00} А & \_ & \cellcolor[HTML]{FFFF00} Б & \cellcolor[HTML]{FFFF00} А & К & \_ & Б & А & 1<7,2>\\ \hline
А & Б & \_ & Р & А & Б & А & \_ & Б & А & К & \_ & Б & А & К & Е & 0'К'\\ \hline
Б & \_ & Р & А & Б & А & \cellcolor[HTML]{FFFF00} \_ & \cellcolor[HTML]{FFFF00} Б & \cellcolor[HTML]{FFFF00} А & \cellcolor[HTML]{FFFF00} К & \cellcolor[HTML]{FFFF00} \_ & \cellcolor[HTML]{FFFF00} Б & \cellcolor[HTML]{FFFF00} А & \cellcolor[HTML]{FFFF00} К & Е & Н & 1<6,4>\\ \hline
Б & А & \_ & Б & А & К & \_ & Б & А & К & Е & Н & \_ & Б & А & К & 0'Е'\\ \hline
А & \_ & Б & А & К & \_ & Б & А & К & Е & Н & \_ & Б & А & К &   & 0'Н'\\ \hline
\cellcolor[HTML]{FFFF00} \_ & \cellcolor[HTML]{FFFF00} Б & \cellcolor[HTML]{FFFF00} А & \cellcolor[HTML]{FFFF00} К & \_ & Б & А & К & Е & Н & \cellcolor[HTML]{FFFF00} \_ & \cellcolor[HTML]{FFFF00} Б & \cellcolor[HTML]{FFFF00} А & \cellcolor[HTML]{FFFF00} К &   &   & 1<0,4>\\ \hline
\end{tabular}
\end{table}

\paragraph{Задание 3.3}

Закодировать сообщение методом LZ78\\
Строка:РАБ\_РАБА\_БАК\_БАКЕН\_БАК\\
\begin{table}[h!]
\centering
\begin{tabular}{|c|c|c|} 
\hline
 Входная фраза (в словарь) & Код & Позиция словаря \\ \hline

 &  & 0 \\ \hline
Р & 0'Р' & 1 \\ \hline
А & 0'А' & 2 \\ \hline
Б & 0'Б' & 3 \\ \hline
\_ & 0'\_' & 4 \\ \hline
РА & 1'А' & 5 \\ \hline
БА & 3'А' & 6 \\ \hline
\_Б & 4'Б' & 7 \\ \hline
АК & 2'К' & 8 \\ \hline
\_БА & 7'А' & 9 \\ \hline
К & 0'К' & 10 \\ \hline
Е & 0'Е' & 11 \\ \hline
Н & 0'Н' & 12 \\ \hline
\_БАК & 9'К' & 13 \\ \hline
\end{tabular}
\end{table}

Результат: 0'Р' 0'А' 0'Б' 0'\_' 1'А' 3'А' 4'Б' 2'К' 7'А' 0'К' 0'Е' 0'Н' 9'К'\\
\pagebreak
\paragraph{Задание 4. Арифметическое кодирование\\}

Исходная строка: СААВИПВАИИ\
\begin{center}
 \begin{tabular}{ |c|c| } 
  \hline
     Буква & Вероятность \\ \hline
А & 0.30\\\hline
И & 0.30\\\hline
В & 0.20\\\hline
С & 0.10\\\hline
П & 0.10
\\ \hline \end{tabular}
\end{center}
\begin{center}
 \begin{tabular}{ |c|c|c| } 
  \hline
     Буква & Начало & Конец \\ \hline
А & 0.00 & 0.30\\\hline
И & 0.30 & 0.60\\\hline
В & 0.60 & 0.80\\\hline
С & 0.80 & 0.90\\\hline
П & 0.90 & 1.00
\\ \hline \end{tabular}
\end{center}
\begin{center}
 \begin{tabular}{ |c|c|c|c| } 
  \hline
     Буква & delta & min & max \\ \hline
С & 0.1000000000 & 0.8000000000 & 0.9000000000\\\hline
А & 0.0300000000 & 0.8000000000 & 0.8300000000\\\hline
А & 0.0090000000 & 0.8000000000 & 0.8090000000\\\hline
В & 0.0018000000 & 0.8054000000 & 0.8072000000\\\hline
И & 0.0005400000 & 0.8059400000 & 0.8064800000\\\hline
П & 0.0000540000 & 0.8064260000 & 0.8064800000\\\hline
В & 0.0000108000 & 0.8064584000 & 0.8064692000\\\hline
А & 0.0000032400 & 0.8064584000 & 0.8064616400\\\hline
И & 0.0000009720 & 0.8064593720 & 0.8064603440\\\hline
И & 0.0000002916 & 0.8064596636 & 0.8064599552
\\ \hline \end{tabular}
\end{center}
Результат: 8064597
\pagebreak
\paragraph{Задание 5.1}

\\ 

Декодировать сообщение методом адаптивного хаффмана \\
Строка: 
'S'0'X'00'C'100'D'010011001001111\\
Результат: SXCDCDDDSS

\includegraphics[width=0.8\linewidth]{/home/fizlrock/data/files/backup/code_backup/hobby/algoritms/LabExecutor/app/./doc_src/images/1578344910.jpg}

\includegraphics[width=0.8\linewidth]{/home/fizlrock/data/files/backup/code_backup/hobby/algoritms/LabExecutor/app/./doc_src/images/710119199.jpg}

\includegraphics[width=0.8\linewidth]{/home/fizlrock/data/files/backup/code_backup/hobby/algoritms/LabExecutor/app/./doc_src/images/1493320807.jpg}

\includegraphics[width=0.8\linewidth]{/home/fizlrock/data/files/backup/code_backup/hobby/algoritms/LabExecutor/app/./doc_src/images/35893790.jpg}

\includegraphics[width=0.8\linewidth]{/home/fizlrock/data/files/backup/code_backup/hobby/algoritms/LabExecutor/app/./doc_src/images/840439786.jpg}

\includegraphics[width=0.8\linewidth]{/home/fizlrock/data/files/backup/code_backup/hobby/algoritms/LabExecutor/app/./doc_src/images/1057826196.jpg}

\includegraphics[width=0.8\linewidth]{/home/fizlrock/data/files/backup/code_backup/hobby/algoritms/LabExecutor/app/./doc_src/images/1257597960.jpg}

\includegraphics[width=0.8\linewidth]{/home/fizlrock/data/files/backup/code_backup/hobby/algoritms/LabExecutor/app/./doc_src/images/936164042.jpg}

\includegraphics[width=0.8\linewidth]{/home/fizlrock/data/files/backup/code_backup/hobby/algoritms/LabExecutor/app/./doc_src/images/194681240.jpg}

\includegraphics[width=0.8\linewidth]{/home/fizlrock/data/files/backup/code_backup/hobby/algoritms/LabExecutor/app/./doc_src/images/1493609341.jpg}
\pagebreak
\paragraph{Задание 5.3 Декодировать строку(LZSS)\\}

Исходная строка: [0'л'] [0'у'] [0'ж'] [0'а'] [0' '] [1<7,2>] [0'б'] [1<5,2>] [1<6,3>] [1<2,1>] [0'у'] [0'р']\\
\begin{table}[h!]
\centering
\begin{tabular}{|c|c|c|}
\hline
 Cловарь & Буфер & Код  \\ \hline
0'л' & [ ,  ,  ,  ,  ,  ,  ,  ,  , л] & л
\\ \hline
0'у' & [ ,  ,  ,  ,  ,  ,  ,  , л, у] & у
\\ \hline
0'ж' & [ ,  ,  ,  ,  ,  ,  , л, у, ж] & ж
\\ \hline
0'а' & [ ,  ,  ,  ,  ,  , л, у, ж, а] & а
\\ \hline
0' ' & [ ,  ,  ,  ,  , л, у, ж, а,  ] &  
\\ \hline
1<7,2> & [ ,  ,  , л, у, ж, а,  , ж, а] & жа
\\ \hline
0'б' & [ ,  , л, у, ж, а,  , ж, а, б] & б
\\ \hline
1<5,2> & [л, у, ж, а,  , ж, а, б, а,  ] & а 
\\ \hline
1<6,3> & [а,  , ж, а, б, а,  , а, б, а] & аба
\\ \hline
1<2,1> & [ , ж, а, б, а,  , а, б, а, ж] & ж
\\ \hline
0'у' & [ж, а, б, а,  , а, б, а, ж, у] & у
\\ \hline
0'р' & [а, б, а,  , а, б, а, ж, у, р] & р
\\ \hline
\end{tabular}
\end{table}

Результат: лужа жаба абажур
\pagebreak
\paragraph{Задание 5.4 Декодировать строку(LZ78)\\}

Исходная строка: [0'т'] [0'о'] [0'р'] [1' '] [0'с'] [2'р'] [4'с'] [6' '] [5'п'] [6'т']\\
\begin{table}[h!]
\centering
\begin{tabular}{|c|c|c|}
\hline
 Cловарь & Буфер & Код  \\ \hline
 & [] & 
\\ \hline
0'т' & [, т] & т
\\ \hline
0'о' & [, т, о] & о
\\ \hline
0'р' & [, т, о, р] & р
\\ \hline
1' ' & [, т, о, р, т ] & т 
\\ \hline
0'с' & [, т, о, р, т , с] & с
\\ \hline
2'р' & [, т, о, р, т , с, ор] & ор
\\ \hline
4'с' & [, т, о, р, т , с, ор, т с] & т с
\\ \hline
6' ' & [, т, о, р, т , с, ор, т с, ор ] & ор 
\\ \hline
5'п' & [, т, о, р, т , с, ор, т с, ор , сп] & сп
\\ \hline
6'т' & [, т, о, р, т , с, ор, т с, ор , сп, орт] & орт
\\ \hline
\end{tabular}
\end{table}

Результат: торт сорт сор спорт
\pagebreak
\subsection{Вариант №15}
\paragraph{Задание 1. Блочный хаффман \\}

Строка БИББББИИИБ, размер блока: 3
\begin{center}
 \begin{tabular}{ |c|c|l| } 
  \hline
     Буква & Вероятность & Код\\ \hline
Б & 0.60 & 1\\\hline
И & 0.40 & 0
\\ \hline \end{tabular}
\end{center}
Энтропия алфавита: 0.9710
\begin{center}
 \begin{tabular}{ |c|c|l| } 
  \hline
     Блок & Вероятность & Код\\ \hline
БББ & 0.22 & 01\\\hline
БИБ & 0.14 & 100\\\hline
ББИ & 0.14 & 101\\\hline
ИББ & 0.14 & 110\\\hline
ИИБ & 0.10 & 001\\\hline
ИБИ & 0.10 & 1111\\\hline
БИИ & 0.10 & 000\\\hline
ИИИ & 0.06 & 1110
\\ \hline \end{tabular}
\end{center}
Бит на символ при посимвольном кодировании: 1.0000, при блочном: 0.9813

\includegraphics[width=0.5\linewidth]{/home/fizlrock/data/files/backup/code_backup/hobby/algoritms/LabExecutor/app/./doc_src/images/604218564.jpg}

\includegraphics[width=0.9\linewidth]{/home/fizlrock/data/files/backup/code_backup/hobby/algoritms/LabExecutor/app/./doc_src/images/497578830.jpg}
\pagebreak
\paragraph{Задание 2. Сжать адаптивным хаффманом\\}

Строка: 
УВАААУУКПУ\\
Результат: 'У' 0'В' 00'А' 101 0 00 01 100'К' 1000'П' 11

\includegraphics[width=0.8\linewidth]{/home/fizlrock/data/files/backup/code_backup/hobby/algoritms/LabExecutor/app/./doc_src/images/455120305.jpg}

\includegraphics[width=0.8\linewidth]{/home/fizlrock/data/files/backup/code_backup/hobby/algoritms/LabExecutor/app/./doc_src/images/1634378228.jpg}

\includegraphics[width=0.8\linewidth]{/home/fizlrock/data/files/backup/code_backup/hobby/algoritms/LabExecutor/app/./doc_src/images/417785392.jpg}

\includegraphics[width=0.8\linewidth]{/home/fizlrock/data/files/backup/code_backup/hobby/algoritms/LabExecutor/app/./doc_src/images/1743124095.jpg}

\includegraphics[width=0.8\linewidth]{/home/fizlrock/data/files/backup/code_backup/hobby/algoritms/LabExecutor/app/./doc_src/images/1891625311.jpg}

\includegraphics[width=0.8\linewidth]{/home/fizlrock/data/files/backup/code_backup/hobby/algoritms/LabExecutor/app/./doc_src/images/1457441861.jpg}

\includegraphics[width=0.8\linewidth]{/home/fizlrock/data/files/backup/code_backup/hobby/algoritms/LabExecutor/app/./doc_src/images/865246149.jpg}

\includegraphics[width=0.8\linewidth]{/home/fizlrock/data/files/backup/code_backup/hobby/algoritms/LabExecutor/app/./doc_src/images/2004520412.jpg}

\includegraphics[width=0.8\linewidth]{/home/fizlrock/data/files/backup/code_backup/hobby/algoritms/LabExecutor/app/./doc_src/images/1890899351.jpg}

\includegraphics[width=0.8\linewidth]{/home/fizlrock/data/files/backup/code_backup/hobby/algoritms/LabExecutor/app/./doc_src/images/1289786324.jpg}
\pagebreak

\paragraph{Задание 3.2}

Закодировать сообщение методом LZSS\\
Строка:ТАРА\_ТАРТАР\_ТАРЕЛКА\_ЕЛКА\\
Результат: 0'Т' 0'А' 0'Р' 1<8,1> 0'\_' 1<5,3> 1<2,3> 1<3,4> 0'Е' 0'Л' 0'К' 1<1,1> 1<2,1> 1<5,4>\\
\begin{table}[h!]
\centering
\begin{tabular}{|c|c|c|c|c|c|c|c|c|c|c|c|c|c|c|c|c|}
\hline
\multicolumn{10}{|c|}{Cловарь} & \multicolumn{6}{c|}{Буфер} & Код  \\ \hline
  &   &   &   &   &   &   &   &   &   & Т & А & Р & А & \_ & Т & 0'Т'\\ \hline
  &   &   &   &   &   &   &   &   & Т & А & Р & А & \_ & Т & А & 0'А'\\ \hline
  &   &   &   &   &   &   &   & Т & А & Р & А & \_ & Т & А & Р & 0'Р'\\ \hline
  &   &   &   &   &   &   & Т & \cellcolor[HTML]{FFFF00} А & Р & \cellcolor[HTML]{FFFF00} А & \_ & Т & А & Р & Т & 1<8,1>\\ \hline
  &   &   &   &   &   & Т & А & Р & А & \_ & Т & А & Р & Т & А & 0'\_'\\ \hline
  &   &   &   &   & \cellcolor[HTML]{FFFF00} Т & \cellcolor[HTML]{FFFF00} А & \cellcolor[HTML]{FFFF00} Р & А & \_ & \cellcolor[HTML]{FFFF00} Т & \cellcolor[HTML]{FFFF00} А & \cellcolor[HTML]{FFFF00} Р & Т & А & Р & 1<5,3>\\ \hline
  &   & \cellcolor[HTML]{FFFF00} Т & \cellcolor[HTML]{FFFF00} А & \cellcolor[HTML]{FFFF00} Р & А & \_ & Т & А & Р & \cellcolor[HTML]{FFFF00} Т & \cellcolor[HTML]{FFFF00} А & \cellcolor[HTML]{FFFF00} Р & \_ & Т & А & 1<2,3>\\ \hline
А & Р & А & \cellcolor[HTML]{FFFF00} \_ & \cellcolor[HTML]{FFFF00} Т & \cellcolor[HTML]{FFFF00} А & \cellcolor[HTML]{FFFF00} Р & Т & А & Р & \cellcolor[HTML]{FFFF00} \_ & \cellcolor[HTML]{FFFF00} Т & \cellcolor[HTML]{FFFF00} А & \cellcolor[HTML]{FFFF00} Р & Е & Л & 1<3,4>\\ \hline
Т & А & Р & Т & А & Р & \_ & Т & А & Р & Е & Л & К & А & \_ & Е & 0'Е'\\ \hline
А & Р & Т & А & Р & \_ & Т & А & Р & Е & Л & К & А & \_ & Е & Л & 0'Л'\\ \hline
Р & Т & А & Р & \_ & Т & А & Р & Е & Л & К & А & \_ & Е & Л & К & 0'К'\\ \hline
Т & \cellcolor[HTML]{FFFF00} А & Р & \_ & Т & А & Р & Е & Л & К & \cellcolor[HTML]{FFFF00} А & \_ & Е & Л & К & А & 1<1,1>\\ \hline
А & Р & \cellcolor[HTML]{FFFF00} \_ & Т & А & Р & Е & Л & К & А & \cellcolor[HTML]{FFFF00} \_ & Е & Л & К & А &   & 1<2,1>\\ \hline
Р & \_ & Т & А & Р & \cellcolor[HTML]{FFFF00} Е & \cellcolor[HTML]{FFFF00} Л & \cellcolor[HTML]{FFFF00} К & \cellcolor[HTML]{FFFF00} А & \_ & \cellcolor[HTML]{FFFF00} Е & \cellcolor[HTML]{FFFF00} Л & \cellcolor[HTML]{FFFF00} К & \cellcolor[HTML]{FFFF00} А &   &   & 1<5,4>\\ \hline
\end{tabular}
\end{table}

\paragraph{Задание 3.3}

Закодировать сообщение методом LZ78\\
Строка:ТАРА\_ТАРТАР\_ТАРЕЛКА\_ЕЛКА\\
\begin{table}[h!]
\centering
\begin{tabular}{|c|c|c|} 
\hline
 Входная фраза (в словарь) & Код & Позиция словаря \\ \hline

 &  & 0 \\ \hline
Т & 0'Т' & 1 \\ \hline
А & 0'А' & 2 \\ \hline
Р & 0'Р' & 3 \\ \hline
А\_ & 2'\_' & 4 \\ \hline
ТА & 1'А' & 5 \\ \hline
РТ & 3'Т' & 6 \\ \hline
АР & 2'Р' & 7 \\ \hline
\_ & 0'\_' & 8 \\ \hline
ТАР & 5'Р' & 9 \\ \hline
Е & 0'Е' & 10 \\ \hline
Л & 0'Л' & 11 \\ \hline
К & 0'К' & 12 \\ \hline
А\_Е & 4'Е' & 13 \\ \hline
ЛК & 11'К' & 14 \\ \hline
\end{tabular}
\end{table}

Результат: 0'Т' 0'А' 0'Р' 2'\_' 1'А' 3'Т' 2'Р' 0'\_' 5'Р' 0'Е' 0'Л' 0'К' 4'Е' 11'К'\\
\pagebreak
\paragraph{Задание 4. Арифметическое кодирование\\}

Исходная строка: УВАААУУКПУ\
\begin{center}
 \begin{tabular}{ |c|c| } 
  \hline
     Буква & Вероятность \\ \hline
У & 0.40\\\hline
А & 0.30\\\hline
В & 0.10\\\hline
К & 0.10\\\hline
П & 0.10
\\ \hline \end{tabular}
\end{center}
\begin{center}
 \begin{tabular}{ |c|c|c| } 
  \hline
     Буква & Начало & Конец \\ \hline
У & 0.00 & 0.40\\\hline
А & 0.40 & 0.70\\\hline
В & 0.70 & 0.80\\\hline
К & 0.80 & 0.90\\\hline
П & 0.90 & 1.00
\\ \hline \end{tabular}
\end{center}
\begin{center}
 \begin{tabular}{ |c|c|c|c| } 
  \hline
     Буква & delta & min & max \\ \hline
У & 0.4000000000 & 0.0000000000 & 0.4000000000\\\hline
В & 0.0400000000 & 0.2800000000 & 0.3200000000\\\hline
А & 0.0120000000 & 0.2960000000 & 0.3080000000\\\hline
А & 0.0036000000 & 0.3008000000 & 0.3044000000\\\hline
А & 0.0010800000 & 0.3022400000 & 0.3033200000\\\hline
У & 0.0004320000 & 0.3022400000 & 0.3026720000\\\hline
У & 0.0001728000 & 0.3022400000 & 0.3024128000\\\hline
К & 0.0000172800 & 0.3023782400 & 0.3023955200\\\hline
П & 0.0000017280 & 0.3023937920 & 0.3023955200\\\hline
У & 0.0000006912 & 0.3023937920 & 0.3023944832
\\ \hline \end{tabular}
\end{center}
Результат: 302394
\pagebreak
\paragraph{Задание 5.1}

\\ 

Декодировать сообщение методом адаптивного хаффмана \\
Строка: 
'K'0'C'00'B'100'V'100110100111110\\
Результат: KCBVKBBVVV

\includegraphics[width=0.8\linewidth]{/home/fizlrock/data/files/backup/code_backup/hobby/algoritms/LabExecutor/app/./doc_src/images/1325398875.jpg}

\includegraphics[width=0.8\linewidth]{/home/fizlrock/data/files/backup/code_backup/hobby/algoritms/LabExecutor/app/./doc_src/images/2066601491.jpg}

\includegraphics[width=0.8\linewidth]{/home/fizlrock/data/files/backup/code_backup/hobby/algoritms/LabExecutor/app/./doc_src/images/77184098.jpg}

\includegraphics[width=0.8\linewidth]{/home/fizlrock/data/files/backup/code_backup/hobby/algoritms/LabExecutor/app/./doc_src/images/1183073195.jpg}

\includegraphics[width=0.8\linewidth]{/home/fizlrock/data/files/backup/code_backup/hobby/algoritms/LabExecutor/app/./doc_src/images/1790560827.jpg}

\includegraphics[width=0.8\linewidth]{/home/fizlrock/data/files/backup/code_backup/hobby/algoritms/LabExecutor/app/./doc_src/images/1417658834.jpg}

\includegraphics[width=0.8\linewidth]{/home/fizlrock/data/files/backup/code_backup/hobby/algoritms/LabExecutor/app/./doc_src/images/2040156477.jpg}

\includegraphics[width=0.8\linewidth]{/home/fizlrock/data/files/backup/code_backup/hobby/algoritms/LabExecutor/app/./doc_src/images/1697172145.jpg}

\includegraphics[width=0.8\linewidth]{/home/fizlrock/data/files/backup/code_backup/hobby/algoritms/LabExecutor/app/./doc_src/images/2030943595.jpg}

\includegraphics[width=0.8\linewidth]{/home/fizlrock/data/files/backup/code_backup/hobby/algoritms/LabExecutor/app/./doc_src/images/1323662704.jpg}
\pagebreak
\paragraph{Задание 5.3 Декодировать строку(LZSS)\\}

Исходная строка: [0'л'] [0'у'] [0'к'] [1<8,1>] [0'м'] [0' '] [1<4,3>] [1<6,1>] [1<2,4>] [1<3,1>] [0'м']\\
\begin{table}[h!]
\centering
\begin{tabular}{|c|c|c|}
\hline
 Cловарь & Буфер & Код  \\ \hline
0'л' & [ ,  ,  ,  ,  ,  ,  ,  ,  , л] & л
\\ \hline
0'у' & [ ,  ,  ,  ,  ,  ,  ,  , л, у] & у
\\ \hline
0'к' & [ ,  ,  ,  ,  ,  ,  , л, у, к] & к
\\ \hline
1<8,1> & [ ,  ,  ,  ,  ,  , л, у, к, у] & у
\\ \hline
0'м' & [ ,  ,  ,  ,  , л, у, к, у, м] & м
\\ \hline
0' ' & [ ,  ,  ,  , л, у, к, у, м,  ] &  
\\ \hline
1<4,3> & [ , л, у, к, у, м,  , л, у, к] & лук
\\ \hline
1<6,1> & [л, у, к, у, м,  , л, у, к,  ] &  
\\ \hline
1<2,4> & [м,  , л, у, к,  , к, у, м,  ] & кум 
\\ \hline
1<3,1> & [ , л, у, к,  , к, у, м,  , у] & у
\\ \hline
0'м' & [л, у, к,  , к, у, м,  , у, м] & м
\\ \hline
\end{tabular}
\end{table}

Результат: лукум лук кум ум
\pagebreak
\paragraph{Задание 5.4 Декодировать строку(LZ78)\\}

Исходная строка: [0'т'] [0'о'] [0'н'] [0'и'] [0'к'] [0' '] [1'о'] [3' '] [7'н'] [3'а']\\
\begin{table}[h!]
\centering
\begin{tabular}{|c|c|c|}
\hline
 Cловарь & Буфер & Код  \\ \hline
 & [] & 
\\ \hline
0'т' & [, т] & т
\\ \hline
0'о' & [, т, о] & о
\\ \hline
0'н' & [, т, о, н] & н
\\ \hline
0'и' & [, т, о, н, и] & и
\\ \hline
0'к' & [, т, о, н, и, к] & к
\\ \hline
0' ' & [, т, о, н, и, к,  ] &  
\\ \hline
1'о' & [, т, о, н, и, к,  , то] & то
\\ \hline
3' ' & [, т, о, н, и, к,  , то, н ] & н 
\\ \hline
7'н' & [, т, о, н, и, к,  , то, н , тон] & тон
\\ \hline
3'а' & [, т, о, н, и, к,  , то, н , тон, на] & на
\\ \hline
\end{tabular}
\end{table}

Результат: тоник тон тонна
\pagebreak
\subsection{Вариант №16}
\paragraph{Задание 1. Блочный хаффман \\}

Строка ДЕЕДКУДДКК, размер блока: 2
\begin{center}
 \begin{tabular}{ |c|c|l| } 
  \hline
     Буква & Вероятность & Код\\ \hline
Д & 0.40 & 0\\\hline
К & 0.30 & 10\\\hline
Е & 0.20 & 111\\\hline
У & 0.10 & 110
\\ \hline \end{tabular}
\end{center}
Энтропия алфавита: 1.8464
\begin{center}
 \begin{tabular}{ |c|c|l| } 
  \hline
     Блок & Вероятность & Код\\ \hline
ДД & 0.16 & 110\\\hline
ДК & 0.12 & 010\\\hline
КД & 0.12 & 011\\\hline
КК & 0.09 & 000\\\hline
ДЕ & 0.08 & 1011\\\hline
ЕД & 0.08 & 1110\\\hline
ЕК & 0.06 & 1000\\\hline
КЕ & 0.06 & 1001\\\hline
ЕЕ & 0.04 & 11110\\\hline
ДУ & 0.04 & 11111\\\hline
УД & 0.04 & 0010\\\hline
КУ & 0.03 & 00111\\\hline
УК & 0.03 & 10100\\\hline
ЕУ & 0.02 & 101011\\\hline
УЕ & 0.02 & 00110\\\hline
УУ & 0.01 & 101010
\\ \hline \end{tabular}
\end{center}
Бит на символ при посимвольном кодировании: 1.9000, при блочном: 1.8650

\includegraphics[width=0.5\linewidth]{/home/fizlrock/data/files/backup/code_backup/hobby/algoritms/LabExecutor/app/./doc_src/images/994087776.jpg}

\includegraphics[width=0.9\linewidth]{/home/fizlrock/data/files/backup/code_backup/hobby/algoritms/LabExecutor/app/./doc_src/images/1459113519.jpg}
\pagebreak
\paragraph{Задание 2. Сжать адаптивным хаффманом\\}

Строка: 
РПЕАКАРРПП\\
Результат: 'Р' 0'П' 00'Е' 100'А' 000'К' 111 01 01 00 00

\includegraphics[width=0.8\linewidth]{/home/fizlrock/data/files/backup/code_backup/hobby/algoritms/LabExecutor/app/./doc_src/images/1356516852.jpg}

\includegraphics[width=0.8\linewidth]{/home/fizlrock/data/files/backup/code_backup/hobby/algoritms/LabExecutor/app/./doc_src/images/1954306580.jpg}

\includegraphics[width=0.8\linewidth]{/home/fizlrock/data/files/backup/code_backup/hobby/algoritms/LabExecutor/app/./doc_src/images/564867782.jpg}

\includegraphics[width=0.8\linewidth]{/home/fizlrock/data/files/backup/code_backup/hobby/algoritms/LabExecutor/app/./doc_src/images/491756157.jpg}

\includegraphics[width=0.8\linewidth]{/home/fizlrock/data/files/backup/code_backup/hobby/algoritms/LabExecutor/app/./doc_src/images/1373184583.jpg}

\includegraphics[width=0.8\linewidth]{/home/fizlrock/data/files/backup/code_backup/hobby/algoritms/LabExecutor/app/./doc_src/images/434486272.jpg}

\includegraphics[width=0.8\linewidth]{/home/fizlrock/data/files/backup/code_backup/hobby/algoritms/LabExecutor/app/./doc_src/images/340770285.jpg}

\includegraphics[width=0.8\linewidth]{/home/fizlrock/data/files/backup/code_backup/hobby/algoritms/LabExecutor/app/./doc_src/images/1088774288.jpg}

\includegraphics[width=0.8\linewidth]{/home/fizlrock/data/files/backup/code_backup/hobby/algoritms/LabExecutor/app/./doc_src/images/1090164725.jpg}

\includegraphics[width=0.8\linewidth]{/home/fizlrock/data/files/backup/code_backup/hobby/algoritms/LabExecutor/app/./doc_src/images/1942647641.jpg}
\pagebreak
\paragraph{Задание 3.1}

Закодировать сообщение методом LZ77\\
Строка:УКУС\_КУСКУС\_УКСУС\_КСИ\\
Результат: <0,0,У> <0,0,К> <8,1,С> <0,0,\_> <6,3,К> <3,3,У> <2,1,С> <4,3,К> <1,1,И>\\
\begin{table}[h!]
\centering
\begin{tabular}{|c|c|c|c|c|c|c|c|c|c|c|c|c|c|c|c|c|} 
\hline
\multicolumn{10}{|c|}{Cловарь} & \multicolumn{6}{c|}{Буфер} & Код  \\ \hline
  &   &   &   &   &   &   &   &   &   & \cellcolor[HTML]{8CE4F6} У & К & У & С &   & К & <0,0,У>
\\ \hline
  &   &   &   &   &   &   &   &   & У & \cellcolor[HTML]{8CE4F6} К & У & С &   & К & У & <0,0,К>
\\ \hline
  &   &   &   &   &   &   &   & \cellcolor[HTML]{FFFF00} У & К & \cellcolor[HTML]{FFFF00} У & \cellcolor[HTML]{8CE4F6} С &   & К & У & С & <8,1,С>
\\ \hline
  &   &   &   &   &   & У & К & У & С & \cellcolor[HTML]{8CE4F6}   & К & У & С & К & У & <0,0,\_>
\\ \hline
  &   &   &   &   & У & \cellcolor[HTML]{FFFF00} К & \cellcolor[HTML]{FFFF00} У & \cellcolor[HTML]{FFFF00} С &   & \cellcolor[HTML]{FFFF00} К & \cellcolor[HTML]{FFFF00} У & \cellcolor[HTML]{FFFF00} С & \cellcolor[HTML]{8CE4F6} К & У & С & <6,3,К>
\\ \hline
  & У & К & \cellcolor[HTML]{FFFF00} У & \cellcolor[HTML]{FFFF00} С & \cellcolor[HTML]{FFFF00}   & К & У & С & К & \cellcolor[HTML]{FFFF00} У & \cellcolor[HTML]{FFFF00} С & \cellcolor[HTML]{FFFF00}   & \cellcolor[HTML]{8CE4F6} У & К & С & <3,3,У>
\\ \hline
С &   & \cellcolor[HTML]{FFFF00} К & У & С & К & У & С &   & У & \cellcolor[HTML]{FFFF00} К & \cellcolor[HTML]{8CE4F6} С & У & С &   & К & <2,1,С>
\\ \hline
К & У & С & К & \cellcolor[HTML]{FFFF00} У & \cellcolor[HTML]{FFFF00} С & \cellcolor[HTML]{FFFF00}   & У & К & С & \cellcolor[HTML]{FFFF00} У & \cellcolor[HTML]{FFFF00} С & \cellcolor[HTML]{FFFF00}   & \cellcolor[HTML]{8CE4F6} К & С & И & <4,3,К>
\\ \hline
У & \cellcolor[HTML]{FFFF00} С &   & У & К & С & У & С &   & К & \cellcolor[HTML]{FFFF00} С & \cellcolor[HTML]{8CE4F6} И &   &   &   &   & <1,1,И>
\\ \hline
\end{tabular}
\end{table}

\paragraph{Задание 3.2}

Закодировать сообщение методом LZSS\\
Строка:УКУС\_КУСКУС\_УКСУС\_КСИ\\
Результат: 0'У' 0'К' 1<8,1> 0'С' 0'\_' 1<6,3> 1<3,4> 1<0,1> 1<2,1> 1<3,1> 1<4,3> 1<5,2> 0'И'\\
\begin{table}[h!]
\centering
\begin{tabular}{|c|c|c|c|c|c|c|c|c|c|c|c|c|c|c|c|c|}
\hline
\multicolumn{10}{|c|}{Cловарь} & \multicolumn{6}{c|}{Буфер} & Код  \\ \hline
  &   &   &   &   &   &   &   &   &   & У & К & У & С & \_ & К & 0'У'\\ \hline
  &   &   &   &   &   &   &   &   & У & К & У & С & \_ & К & У & 0'К'\\ \hline
  &   &   &   &   &   &   &   & \cellcolor[HTML]{FFFF00} У & К & \cellcolor[HTML]{FFFF00} У & С & \_ & К & У & С & 1<8,1>\\ \hline
  &   &   &   &   &   &   & У & К & У & С & \_ & К & У & С & К & 0'С'\\ \hline
  &   &   &   &   &   & У & К & У & С & \_ & К & У & С & К & У & 0'\_'\\ \hline
  &   &   &   &   & У & \cellcolor[HTML]{FFFF00} К & \cellcolor[HTML]{FFFF00} У & \cellcolor[HTML]{FFFF00} С & \_ & \cellcolor[HTML]{FFFF00} К & \cellcolor[HTML]{FFFF00} У & \cellcolor[HTML]{FFFF00} С & К & У & С & 1<6,3>\\ \hline
  &   & У & \cellcolor[HTML]{FFFF00} К & \cellcolor[HTML]{FFFF00} У & \cellcolor[HTML]{FFFF00} С & \cellcolor[HTML]{FFFF00} \_ & К & У & С & \cellcolor[HTML]{FFFF00} К & \cellcolor[HTML]{FFFF00} У & \cellcolor[HTML]{FFFF00} С & \cellcolor[HTML]{FFFF00} \_ & У & К & 1<3,4>\\ \hline
\cellcolor[HTML]{FFFF00} У & С & \_ & К & У & С & К & У & С & \_ & \cellcolor[HTML]{FFFF00} У & К & С & У & С & \_ & 1<0,1>\\ \hline
С & \_ & \cellcolor[HTML]{FFFF00} К & У & С & К & У & С & \_ & У & \cellcolor[HTML]{FFFF00} К & С & У & С & \_ & К & 1<2,1>\\ \hline
\_ & К & У & \cellcolor[HTML]{FFFF00} С & К & У & С & \_ & У & К & \cellcolor[HTML]{FFFF00} С & У & С & \_ & К & С & 1<3,1>\\ \hline
К & У & С & К & \cellcolor[HTML]{FFFF00} У & \cellcolor[HTML]{FFFF00} С & \cellcolor[HTML]{FFFF00} \_ & У & К & С & \cellcolor[HTML]{FFFF00} У & \cellcolor[HTML]{FFFF00} С & \cellcolor[HTML]{FFFF00} \_ & К & С & И & 1<4,3>\\ \hline
К & У & С & \_ & У & \cellcolor[HTML]{FFFF00} К & \cellcolor[HTML]{FFFF00} С & У & С & \_ & \cellcolor[HTML]{FFFF00} К & \cellcolor[HTML]{FFFF00} С & И &   &   &   & 1<5,2>\\ \hline
С & \_ & У & К & С & У & С & \_ & К & С & И &   &   &   &   &   & 0'И'\\ \hline
\end{tabular}
\end{table}

\paragraph{Задание 3.3}

Закодировать сообщение методом LZ78\\
Строка:УКУС\_КУСКУС\_УКСУС\_КСИ\\
\begin{table}[h!]
\centering
\begin{tabular}{|c|c|c|} 
\hline
 Входная фраза (в словарь) & Код & Позиция словаря \\ \hline

 &  & 0 \\ \hline
У & 0'У' & 1 \\ \hline
К & 0'К' & 2 \\ \hline
УС & 1'С' & 3 \\ \hline
\_ & 0'\_' & 4 \\ \hline
КУ & 2'У' & 5 \\ \hline
С & 0'С' & 6 \\ \hline
КУС & 5'С' & 7 \\ \hline
\_У & 4'У' & 8 \\ \hline
КС & 2'С' & 9 \\ \hline
УС\_ & 3'\_' & 10 \\ \hline
КСИ & 9'И' & 11 \\ \hline
\end{tabular}
\end{table}

Результат: 0'У' 0'К' 1'С' 0'\_' 2'У' 0'С' 5'С' 4'У' 2'С' 3'\_' 9'И'\\
\pagebreak
\paragraph{Задание 4. Арифметическое кодирование\\}

Исходная строка: РПЕАКАРРПП\
\begin{center}
 \begin{tabular}{ |c|c| } 
  \hline
     Буква & Вероятность \\ \hline
Р & 0.30\\\hline
П & 0.30\\\hline
А & 0.20\\\hline
Е & 0.10\\\hline
К & 0.10
\\ \hline \end{tabular}
\end{center}
\begin{center}
 \begin{tabular}{ |c|c|c| } 
  \hline
     Буква & Начало & Конец \\ \hline
Р & 0.00 & 0.30\\\hline
П & 0.30 & 0.60\\\hline
А & 0.60 & 0.80\\\hline
Е & 0.80 & 0.90\\\hline
К & 0.90 & 1.00
\\ \hline \end{tabular}
\end{center}
\begin{center}
 \begin{tabular}{ |c|c|c|c| } 
  \hline
     Буква & delta & min & max \\ \hline
Р & 0.3000000000 & 0.0000000000 & 0.3000000000\\\hline
П & 0.0900000000 & 0.0900000000 & 0.1800000000\\\hline
Е & 0.0090000000 & 0.1620000000 & 0.1710000000\\\hline
А & 0.0018000000 & 0.1674000000 & 0.1692000000\\\hline
К & 0.0001800000 & 0.1690200000 & 0.1692000000\\\hline
А & 0.0000360000 & 0.1691280000 & 0.1691640000\\\hline
Р & 0.0000108000 & 0.1691280000 & 0.1691388000\\\hline
Р & 0.0000032400 & 0.1691280000 & 0.1691312400\\\hline
П & 0.0000009720 & 0.1691289720 & 0.1691299440\\\hline
П & 0.0000002916 & 0.1691292636 & 0.1691295552
\\ \hline \end{tabular}
\end{center}
Результат: 1691293
\pagebreak
\paragraph{Задание 5.1}

\\ 

Декодировать сообщение методом адаптивного хаффмана \\
Строка: 
'Z'0'X'00'Y'10110110111100'D'11110\\
Результат: ZXYYZZZZZZZDZZZZ

\includegraphics[width=0.8\linewidth]{/home/fizlrock/data/files/backup/code_backup/hobby/algoritms/LabExecutor/app/./doc_src/images/144749354.jpg}

\includegraphics[width=0.8\linewidth]{/home/fizlrock/data/files/backup/code_backup/hobby/algoritms/LabExecutor/app/./doc_src/images/1811683117.jpg}

\includegraphics[width=0.8\linewidth]{/home/fizlrock/data/files/backup/code_backup/hobby/algoritms/LabExecutor/app/./doc_src/images/1916737012.jpg}

\includegraphics[width=0.8\linewidth]{/home/fizlrock/data/files/backup/code_backup/hobby/algoritms/LabExecutor/app/./doc_src/images/494344391.jpg}

\includegraphics[width=0.8\linewidth]{/home/fizlrock/data/files/backup/code_backup/hobby/algoritms/LabExecutor/app/./doc_src/images/1013284377.jpg}

\includegraphics[width=0.8\linewidth]{/home/fizlrock/data/files/backup/code_backup/hobby/algoritms/LabExecutor/app/./doc_src/images/339658605.jpg}

\includegraphics[width=0.8\linewidth]{/home/fizlrock/data/files/backup/code_backup/hobby/algoritms/LabExecutor/app/./doc_src/images/1742745742.jpg}

\includegraphics[width=0.8\linewidth]{/home/fizlrock/data/files/backup/code_backup/hobby/algoritms/LabExecutor/app/./doc_src/images/552245418.jpg}

\includegraphics[width=0.8\linewidth]{/home/fizlrock/data/files/backup/code_backup/hobby/algoritms/LabExecutor/app/./doc_src/images/1253388979.jpg}

\includegraphics[width=0.8\linewidth]{/home/fizlrock/data/files/backup/code_backup/hobby/algoritms/LabExecutor/app/./doc_src/images/180759490.jpg}

\includegraphics[width=0.8\linewidth]{/home/fizlrock/data/files/backup/code_backup/hobby/algoritms/LabExecutor/app/./doc_src/images/303529035.jpg}

\includegraphics[width=0.8\linewidth]{/home/fizlrock/data/files/backup/code_backup/hobby/algoritms/LabExecutor/app/./doc_src/images/1669127553.jpg}

\includegraphics[width=0.8\linewidth]{/home/fizlrock/data/files/backup/code_backup/hobby/algoritms/LabExecutor/app/./doc_src/images/324866774.jpg}

\includegraphics[width=0.8\linewidth]{/home/fizlrock/data/files/backup/code_backup/hobby/algoritms/LabExecutor/app/./doc_src/images/1172814301.jpg}

\includegraphics[width=0.8\linewidth]{/home/fizlrock/data/files/backup/code_backup/hobby/algoritms/LabExecutor/app/./doc_src/images/1685730144.jpg}

\includegraphics[width=0.8\linewidth]{/home/fizlrock/data/files/backup/code_backup/hobby/algoritms/LabExecutor/app/./doc_src/images/757836328.jpg}
\pagebreak
\paragraph{Задание 5.3 Декодировать строку(LZSS)\\}

Исходная строка: [0'к'] [0'у'] [1<8,2>] [0'р'] [1<6,1>] [0'з'] [0'а'] [0' '] [1<3,3>] [0'о'] [1<6,1>] [1<4,1>] [1<6,2>] [0'к']\\
\begin{table}[h!]
\centering
\begin{tabular}{|c|c|c|}
\hline
 Cловарь & Буфер & Код  \\ \hline
0'к' & [ ,  ,  ,  ,  ,  ,  ,  ,  , к] & к
\\ \hline
0'у' & [ ,  ,  ,  ,  ,  ,  ,  , к, у] & у
\\ \hline
1<8,2> & [ ,  ,  ,  ,  ,  , к, у, к, у] & ку
\\ \hline
0'р' & [ ,  ,  ,  ,  , к, у, к, у, р] & р
\\ \hline
1<6,1> & [ ,  ,  ,  , к, у, к, у, р, у] & у
\\ \hline
0'з' & [ ,  ,  , к, у, к, у, р, у, з] & з
\\ \hline
0'а' & [ ,  , к, у, к, у, р, у, з, а] & а
\\ \hline
0' ' & [ , к, у, к, у, р, у, з, а,  ] &  
\\ \hline
1<3,3> & [к, у, р, у, з, а,  , к, у, р] & кур
\\ \hline
0'о' & [у, р, у, з, а,  , к, у, р, о] & о
\\ \hline
1<6,1> & [р, у, з, а,  , к, у, р, о, к] & к
\\ \hline
1<4,1> & [у, з, а,  , к, у, р, о, к,  ] &  
\\ \hline
1<6,2> & [а,  , к, у, р, о, к,  , р, о] & ро
\\ \hline
0'к' & [ , к, у, р, о, к,  , р, о, к] & к
\\ \hline
\end{tabular}
\end{table}

Результат: кукуруза курок рок
\pagebreak
\paragraph{Задание 5.4 Декодировать строку(LZ78)\\}

Исходная строка: [0'с'] [0'и'] [0'л'] [0'а'] [0' '] [3'а'] [1'к'] [4' '] [6'с'] [0'т'] [8'с'] [10'а'] [0'н']\\
\begin{table}[h!]
\centering
\begin{tabular}{|c|c|c|}
\hline
 Cловарь & Буфер & Код  \\ \hline
 & [] & 
\\ \hline
0'с' & [, с] & с
\\ \hline
0'и' & [, с, и] & и
\\ \hline
0'л' & [, с, и, л] & л
\\ \hline
0'а' & [, с, и, л, а] & а
\\ \hline
0' ' & [, с, и, л, а,  ] &  
\\ \hline
3'а' & [, с, и, л, а,  , ла] & ла
\\ \hline
1'к' & [, с, и, л, а,  , ла, ск] & ск
\\ \hline
4' ' & [, с, и, л, а,  , ла, ск, а ] & а 
\\ \hline
6'с' & [, с, и, л, а,  , ла, ск, а , лас] & лас
\\ \hline
0'т' & [, с, и, л, а,  , ла, ск, а , лас, т] & т
\\ \hline
8'с' & [, с, и, л, а,  , ла, ск, а , лас, т, а с] & а с
\\ \hline
10'а' & [, с, и, л, а,  , ла, ск, а , лас, т, а с, та] & та
\\ \hline
0'н' & [, с, и, л, а,  , ла, ск, а , лас, т, а с, та, н] & н
\\ \hline
\end{tabular}
\end{table}

Результат: сила ласка ласта стан
\pagebreak
\subsection{Вариант №17}
\paragraph{Задание 1. Блочный хаффман \\}

Строка ГНННОООМНГ, размер блока: 2
\begin{center}
 \begin{tabular}{ |c|c|l| } 
  \hline
     Буква & Вероятность & Код\\ \hline
Н & 0.40 & 0\\\hline
О & 0.30 & 10\\\hline
Г & 0.20 & 111\\\hline
М & 0.10 & 110
\\ \hline \end{tabular}
\end{center}
Энтропия алфавита: 1.8464
\begin{center}
 \begin{tabular}{ |c|c|l| } 
  \hline
     Блок & Вероятность & Код\\ \hline
НН & 0.16 & 110\\\hline
НО & 0.12 & 010\\\hline
ОН & 0.12 & 011\\\hline
ОО & 0.09 & 000\\\hline
ГН & 0.08 & 1011\\\hline
НГ & 0.08 & 1110\\\hline
ГО & 0.06 & 1000\\\hline
ОГ & 0.06 & 1001\\\hline
ГГ & 0.04 & 11110\\\hline
МН & 0.04 & 11111\\\hline
НМ & 0.04 & 0010\\\hline
МО & 0.03 & 00111\\\hline
ОМ & 0.03 & 10100\\\hline
ГМ & 0.02 & 101011\\\hline
МГ & 0.02 & 00110\\\hline
ММ & 0.01 & 101010
\\ \hline \end{tabular}
\end{center}
Бит на символ при посимвольном кодировании: 1.9000, при блочном: 1.8650

\includegraphics[width=0.5\linewidth]{/home/fizlrock/data/files/backup/code_backup/hobby/algoritms/LabExecutor/app/./doc_src/images/940603877.jpg}

\includegraphics[width=0.9\linewidth]{/home/fizlrock/data/files/backup/code_backup/hobby/algoritms/LabExecutor/app/./doc_src/images/727720812.jpg}
\pagebreak
\paragraph{Задание 2. Сжать адаптивным хаффманом\\}

Строка: 
ГНРНГРНПРР\\
Результат: 'Г' 0'Н' 00'Р' 11 11 101 0 100'П' 111 10

\includegraphics[width=0.8\linewidth]{/home/fizlrock/data/files/backup/code_backup/hobby/algoritms/LabExecutor/app/./doc_src/images/381761767.jpg}

\includegraphics[width=0.8\linewidth]{/home/fizlrock/data/files/backup/code_backup/hobby/algoritms/LabExecutor/app/./doc_src/images/1260802331.jpg}

\includegraphics[width=0.8\linewidth]{/home/fizlrock/data/files/backup/code_backup/hobby/algoritms/LabExecutor/app/./doc_src/images/278410118.jpg}

\includegraphics[width=0.8\linewidth]{/home/fizlrock/data/files/backup/code_backup/hobby/algoritms/LabExecutor/app/./doc_src/images/1186594691.jpg}

\includegraphics[width=0.8\linewidth]{/home/fizlrock/data/files/backup/code_backup/hobby/algoritms/LabExecutor/app/./doc_src/images/97328215.jpg}

\includegraphics[width=0.8\linewidth]{/home/fizlrock/data/files/backup/code_backup/hobby/algoritms/LabExecutor/app/./doc_src/images/1456976480.jpg}

\includegraphics[width=0.8\linewidth]{/home/fizlrock/data/files/backup/code_backup/hobby/algoritms/LabExecutor/app/./doc_src/images/812557914.jpg}

\includegraphics[width=0.8\linewidth]{/home/fizlrock/data/files/backup/code_backup/hobby/algoritms/LabExecutor/app/./doc_src/images/1185005520.jpg}

\includegraphics[width=0.8\linewidth]{/home/fizlrock/data/files/backup/code_backup/hobby/algoritms/LabExecutor/app/./doc_src/images/836213292.jpg}

\includegraphics[width=0.8\linewidth]{/home/fizlrock/data/files/backup/code_backup/hobby/algoritms/LabExecutor/app/./doc_src/images/1693091578.jpg}
\pagebreak
\paragraph{Задание 3.1}

Закодировать сообщение методом LZ77\\
Строка:ДОМ\_ДОМИК\_ОМИК\_МИР\\
Результат: <0,0,Д> <0,0,О> <0,0,М> <0,0,\_> <6,3,И> <0,0,К> <4,1,О> <5,4,М> <1,1,Р>\\
\begin{table}[h!]
\centering
\begin{tabular}{|c|c|c|c|c|c|c|c|c|c|c|c|c|c|c|c|c|} 
\hline
\multicolumn{10}{|c|}{Cловарь} & \multicolumn{6}{c|}{Буфер} & Код  \\ \hline
  &   &   &   &   &   &   &   &   &   & \cellcolor[HTML]{8CE4F6} Д & О & М &   & Д & О & <0,0,Д>
\\ \hline
  &   &   &   &   &   &   &   &   & Д & \cellcolor[HTML]{8CE4F6} О & М &   & Д & О & М & <0,0,О>
\\ \hline
  &   &   &   &   &   &   &   & Д & О & \cellcolor[HTML]{8CE4F6} М &   & Д & О & М & И & <0,0,М>
\\ \hline
  &   &   &   &   &   &   & Д & О & М & \cellcolor[HTML]{8CE4F6}   & Д & О & М & И & К & <0,0,\_>
\\ \hline
  &   &   &   &   &   & \cellcolor[HTML]{FFFF00} Д & \cellcolor[HTML]{FFFF00} О & \cellcolor[HTML]{FFFF00} М &   & \cellcolor[HTML]{FFFF00} Д & \cellcolor[HTML]{FFFF00} О & \cellcolor[HTML]{FFFF00} М & \cellcolor[HTML]{8CE4F6} И & К &   & <6,3,И>
\\ \hline
  &   & Д & О & М &   & Д & О & М & И & \cellcolor[HTML]{8CE4F6} К &   & О & М & И & К & <0,0,К>
\\ \hline
  & Д & О & М & \cellcolor[HTML]{FFFF00}   & Д & О & М & И & К & \cellcolor[HTML]{FFFF00}   & \cellcolor[HTML]{8CE4F6} О & М & И & К &   & <4,1,О>
\\ \hline
О & М &   & Д & О & \cellcolor[HTML]{FFFF00} М & \cellcolor[HTML]{FFFF00} И & \cellcolor[HTML]{FFFF00} К & \cellcolor[HTML]{FFFF00}   & О & \cellcolor[HTML]{FFFF00} М & \cellcolor[HTML]{FFFF00} И & \cellcolor[HTML]{FFFF00} К & \cellcolor[HTML]{FFFF00}   & \cellcolor[HTML]{8CE4F6} М & И & <5,4,М>
\\ \hline
М & \cellcolor[HTML]{FFFF00} И & К &   & О & М & И & К &   & М & \cellcolor[HTML]{FFFF00} И & \cellcolor[HTML]{8CE4F6} Р &   &   &   &   & <1,1,Р>
\\ \hline
\end{tabular}
\end{table}

\paragraph{Задание 3.2}

Закодировать сообщение методом LZSS\\
Строка:ДОМ\_ДОМИК\_ОМИК\_МИР\\
Результат: 0'Д' 0'О' 0'М' 0'\_' 1<6,3> 0'И' 0'К' 1<4,1> 1<5,5> 1<1,2> 0'Р'\\
\begin{table}[h!]
\centering
\begin{tabular}{|c|c|c|c|c|c|c|c|c|c|c|c|c|c|c|c|c|}
\hline
\multicolumn{10}{|c|}{Cловарь} & \multicolumn{6}{c|}{Буфер} & Код  \\ \hline
  &   &   &   &   &   &   &   &   &   & Д & О & М & \_ & Д & О & 0'Д'\\ \hline
  &   &   &   &   &   &   &   &   & Д & О & М & \_ & Д & О & М & 0'О'\\ \hline
  &   &   &   &   &   &   &   & Д & О & М & \_ & Д & О & М & И & 0'М'\\ \hline
  &   &   &   &   &   &   & Д & О & М & \_ & Д & О & М & И & К & 0'\_'\\ \hline
  &   &   &   &   &   & \cellcolor[HTML]{FFFF00} Д & \cellcolor[HTML]{FFFF00} О & \cellcolor[HTML]{FFFF00} М & \_ & \cellcolor[HTML]{FFFF00} Д & \cellcolor[HTML]{FFFF00} О & \cellcolor[HTML]{FFFF00} М & И & К & \_ & 1<6,3>\\ \hline
  &   &   & Д & О & М & \_ & Д & О & М & И & К & \_ & О & М & И & 0'И'\\ \hline
  &   & Д & О & М & \_ & Д & О & М & И & К & \_ & О & М & И & К & 0'К'\\ \hline
  & Д & О & М & \cellcolor[HTML]{FFFF00} \_ & Д & О & М & И & К & \cellcolor[HTML]{FFFF00} \_ & О & М & И & К & \_ & 1<4,1>\\ \hline
Д & О & М & \_ & Д & \cellcolor[HTML]{FFFF00} О & \cellcolor[HTML]{FFFF00} М & \cellcolor[HTML]{FFFF00} И & \cellcolor[HTML]{FFFF00} К & \cellcolor[HTML]{FFFF00} \_ & \cellcolor[HTML]{FFFF00} О & \cellcolor[HTML]{FFFF00} М & \cellcolor[HTML]{FFFF00} И & \cellcolor[HTML]{FFFF00} К & \cellcolor[HTML]{FFFF00} \_ & М & 1<5,5>\\ \hline
О & \cellcolor[HTML]{FFFF00} М & \cellcolor[HTML]{FFFF00} И & К & \_ & О & М & И & К & \_ & \cellcolor[HTML]{FFFF00} М & \cellcolor[HTML]{FFFF00} И & Р &   &   &   & 1<1,2>\\ \hline
И & К & \_ & О & М & И & К & \_ & М & И & Р &   &   &   &   &   & 0'Р'\\ \hline
\end{tabular}
\end{table}

\paragraph{Задание 3.3}

Закодировать сообщение методом LZ78\\
Строка:ДОМ\_ДОМИК\_ОМИК\_МИР\\
\begin{table}[h!]
\centering
\begin{tabular}{|c|c|c|} 
\hline
 Входная фраза (в словарь) & Код & Позиция словаря \\ \hline

 &  & 0 \\ \hline
Д & 0'Д' & 1 \\ \hline
О & 0'О' & 2 \\ \hline
М & 0'М' & 3 \\ \hline
\_ & 0'\_' & 4 \\ \hline
ДО & 1'О' & 5 \\ \hline
МИ & 3'И' & 6 \\ \hline
К & 0'К' & 7 \\ \hline
\_О & 4'О' & 8 \\ \hline
МИК & 6'К' & 9 \\ \hline
\_М & 4'М' & 10 \\ \hline
И & 0'И' & 11 \\ \hline
Р & 0'Р' & 12 \\ \hline
\end{tabular}
\end{table}

Результат: 0'Д' 0'О' 0'М' 0'\_' 1'О' 3'И' 0'К' 4'О' 6'К' 4'М' 0'И' 0'Р'\\
\pagebreak
\paragraph{Задание 4. Арифметическое кодирование\\}

Исходная строка: ГНРНГРНПРР\
\begin{center}
 \begin{tabular}{ |c|c| } 
  \hline
     Буква & Вероятность \\ \hline
Р & 0.40\\\hline
Н & 0.30\\\hline
Г & 0.20\\\hline
П & 0.10
\\ \hline \end{tabular}
\end{center}
\begin{center}
 \begin{tabular}{ |c|c|c| } 
  \hline
     Буква & Начало & Конец \\ \hline
Р & 0.00 & 0.40\\\hline
Н & 0.40 & 0.70\\\hline
Г & 0.70 & 0.90\\\hline
П & 0.90 & 1.00
\\ \hline \end{tabular}
\end{center}
\begin{center}
 \begin{tabular}{ |c|c|c|c| } 
  \hline
     Буква & delta & min & max \\ \hline
Г & 0.2000000000 & 0.7000000000 & 0.9000000000\\\hline
Н & 0.0600000000 & 0.7800000000 & 0.8400000000\\\hline
Р & 0.0240000000 & 0.7800000000 & 0.8040000000\\\hline
Н & 0.0072000000 & 0.7896000000 & 0.7968000000\\\hline
Г & 0.0014400000 & 0.7946400000 & 0.7960800000\\\hline
Р & 0.0005760000 & 0.7946400000 & 0.7952160000\\\hline
Н & 0.0001728000 & 0.7948704000 & 0.7950432000\\\hline
П & 0.0000172800 & 0.7950259200 & 0.7950432000\\\hline
Р & 0.0000069120 & 0.7950259200 & 0.7950328320\\\hline
Р & 0.0000027648 & 0.7950259200 & 0.7950286848
\\ \hline \end{tabular}
\end{center}
Результат: 795026
\pagebreak
\paragraph{Задание 5.1}

\\ 

Декодировать сообщение методом адаптивного хаффмана \\
Строка: 
'R'0'F'00'T'100'D'101111011111101001\\
Результат: RFTDRRDTDDD

\includegraphics[width=0.8\linewidth]{/home/fizlrock/data/files/backup/code_backup/hobby/algoritms/LabExecutor/app/./doc_src/images/1084579278.jpg}

\includegraphics[width=0.8\linewidth]{/home/fizlrock/data/files/backup/code_backup/hobby/algoritms/LabExecutor/app/./doc_src/images/837885257.jpg}

\includegraphics[width=0.8\linewidth]{/home/fizlrock/data/files/backup/code_backup/hobby/algoritms/LabExecutor/app/./doc_src/images/300598673.jpg}

\includegraphics[width=0.8\linewidth]{/home/fizlrock/data/files/backup/code_backup/hobby/algoritms/LabExecutor/app/./doc_src/images/1234260458.jpg}

\includegraphics[width=0.8\linewidth]{/home/fizlrock/data/files/backup/code_backup/hobby/algoritms/LabExecutor/app/./doc_src/images/2037276915.jpg}

\includegraphics[width=0.8\linewidth]{/home/fizlrock/data/files/backup/code_backup/hobby/algoritms/LabExecutor/app/./doc_src/images/1103516792.jpg}

\includegraphics[width=0.8\linewidth]{/home/fizlrock/data/files/backup/code_backup/hobby/algoritms/LabExecutor/app/./doc_src/images/572805908.jpg}

\includegraphics[width=0.8\linewidth]{/home/fizlrock/data/files/backup/code_backup/hobby/algoritms/LabExecutor/app/./doc_src/images/330772530.jpg}

\includegraphics[width=0.8\linewidth]{/home/fizlrock/data/files/backup/code_backup/hobby/algoritms/LabExecutor/app/./doc_src/images/575939969.jpg}

\includegraphics[width=0.8\linewidth]{/home/fizlrock/data/files/backup/code_backup/hobby/algoritms/LabExecutor/app/./doc_src/images/1317628172.jpg}

\includegraphics[width=0.8\linewidth]{/home/fizlrock/data/files/backup/code_backup/hobby/algoritms/LabExecutor/app/./doc_src/images/647704151.jpg}
\pagebreak
\paragraph{Задание 5.3 Декодировать строку(LZSS)\\}

Исходная строка: [0'к'] [0'о'] [0'р'] [0'м'] [0' '] [1<7,1>] [1<5,1>] [1<6,4>] [1<2,1>] [0'а'] [0'н'] [1<0,1>] [1<2,1>] [1<0,1>] [0'р']\\
\begin{table}[h!]
\centering
\begin{tabular}{|c|c|c|}
\hline
 Cловарь & Буфер & Код  \\ \hline
0'к' & [ ,  ,  ,  ,  ,  ,  ,  ,  , к] & к
\\ \hline
0'о' & [ ,  ,  ,  ,  ,  ,  ,  , к, о] & о
\\ \hline
0'р' & [ ,  ,  ,  ,  ,  ,  , к, о, р] & р
\\ \hline
0'м' & [ ,  ,  ,  ,  ,  , к, о, р, м] & м
\\ \hline
0' ' & [ ,  ,  ,  ,  , к, о, р, м,  ] &  
\\ \hline
1<7,1> & [ ,  ,  ,  , к, о, р, м,  , р] & р
\\ \hline
1<5,1> & [ ,  ,  , к, о, р, м,  , р, о] & о
\\ \hline
1<6,4> & [о, р, м,  , р, о, м,  , р, о] & м ро
\\ \hline
1<2,1> & [р, м,  , р, о, м,  , р, о, м] & м
\\ \hline
0'а' & [м,  , р, о, м,  , р, о, м, а] & а
\\ \hline
0'н' & [ , р, о, м,  , р, о, м, а, н] & н
\\ \hline
1<0,1> & [р, о, м,  , р, о, м, а, н,  ] &  
\\ \hline
1<2,1> & [о, м,  , р, о, м, а, н,  , м] & м
\\ \hline
1<0,1> & [м,  , р, о, м, а, н,  , м, о] & о
\\ \hline
0'р' & [ , р, о, м, а, н,  , м, о, р] & р
\\ \hline
\end{tabular}
\end{table}

Результат: корм ром роман мор
\pagebreak
\paragraph{Задание 5.4 Декодировать строку(LZ78)\\}

Исходная строка: [0'т'] [0'о'] [0'с'] [0'к'] [0'а'] [0' '] [3'к'] [5'л'] [5' '] [0'л'] [5'с'] [1'и'] [0'к']\\
\begin{table}[h!]
\centering
\begin{tabular}{|c|c|c|}
\hline
 Cловарь & Буфер & Код  \\ \hline
 & [] & 
\\ \hline
0'т' & [, т] & т
\\ \hline
0'о' & [, т, о] & о
\\ \hline
0'с' & [, т, о, с] & с
\\ \hline
0'к' & [, т, о, с, к] & к
\\ \hline
0'а' & [, т, о, с, к, а] & а
\\ \hline
0' ' & [, т, о, с, к, а,  ] &  
\\ \hline
3'к' & [, т, о, с, к, а,  , ск] & ск
\\ \hline
5'л' & [, т, о, с, к, а,  , ск, ал] & ал
\\ \hline
5' ' & [, т, о, с, к, а,  , ск, ал, а ] & а 
\\ \hline
0'л' & [, т, о, с, к, а,  , ск, ал, а , л] & л
\\ \hline
5'с' & [, т, о, с, к, а,  , ск, ал, а , л, ас] & ас
\\ \hline
1'и' & [, т, о, с, к, а,  , ск, ал, а , л, ас, ти] & ти
\\ \hline
0'к' & [, т, о, с, к, а,  , ск, ал, а , л, ас, ти, к] & к
\\ \hline
\end{tabular}
\end{table}

Результат: тоска скала ластик
\pagebreak
\subsection{Вариант №18}
\paragraph{Задание 1. Блочный хаффман \\}

Строка КРРРАККККК, размер блока: 2
\begin{center}
 \begin{tabular}{ |c|c|l| } 
  \hline
     Буква & Вероятность & Код\\ \hline
К & 0.60 & 1\\\hline
Р & 0.30 & 01\\\hline
А & 0.10 & 00
\\ \hline \end{tabular}
\end{center}
Энтропия алфавита: 1.2955
\begin{center}
 \begin{tabular}{ |c|c|l| } 
  \hline
     Блок & Вероятность & Код\\ \hline
КК & 0.36 & 11\\\hline
КР & 0.18 & 00\\\hline
РК & 0.18 & 01\\\hline
РР & 0.09 & 1011\\\hline
КА & 0.06 & 1000\\\hline
АК & 0.06 & 1001\\\hline
РА & 0.03 & 101011\\\hline
АР & 0.03 & 10100\\\hline
АА & 0.01 & 101010
\\ \hline \end{tabular}
\end{center}
Бит на символ при посимвольном кодировании: 1.4000, при блочном: 1.3350

\includegraphics[width=0.5\linewidth]{/home/fizlrock/data/files/backup/code_backup/hobby/algoritms/LabExecutor/app/./doc_src/images/177467773.jpg}

\includegraphics[width=0.9\linewidth]{/home/fizlrock/data/files/backup/code_backup/hobby/algoritms/LabExecutor/app/./doc_src/images/1835809748.jpg}
\pagebreak
\paragraph{Задание 2. Сжать адаптивным хаффманом\\}

Строка: 
ГНРПАНПППП\\
Результат: 'Г' 0'Н' 00'Р' 100'П' 000'А' 00 111 01 11 0

\includegraphics[width=0.8\linewidth]{/home/fizlrock/data/files/backup/code_backup/hobby/algoritms/LabExecutor/app/./doc_src/images/989284324.jpg}

\includegraphics[width=0.8\linewidth]{/home/fizlrock/data/files/backup/code_backup/hobby/algoritms/LabExecutor/app/./doc_src/images/1377886036.jpg}

\includegraphics[width=0.8\linewidth]{/home/fizlrock/data/files/backup/code_backup/hobby/algoritms/LabExecutor/app/./doc_src/images/265067942.jpg}

\includegraphics[width=0.8\linewidth]{/home/fizlrock/data/files/backup/code_backup/hobby/algoritms/LabExecutor/app/./doc_src/images/1778524012.jpg}

\includegraphics[width=0.8\linewidth]{/home/fizlrock/data/files/backup/code_backup/hobby/algoritms/LabExecutor/app/./doc_src/images/560262156.jpg}

\includegraphics[width=0.8\linewidth]{/home/fizlrock/data/files/backup/code_backup/hobby/algoritms/LabExecutor/app/./doc_src/images/929056.jpg}

\includegraphics[width=0.8\linewidth]{/home/fizlrock/data/files/backup/code_backup/hobby/algoritms/LabExecutor/app/./doc_src/images/1410101685.jpg}

\includegraphics[width=0.8\linewidth]{/home/fizlrock/data/files/backup/code_backup/hobby/algoritms/LabExecutor/app/./doc_src/images/312105761.jpg}

\includegraphics[width=0.8\linewidth]{/home/fizlrock/data/files/backup/code_backup/hobby/algoritms/LabExecutor/app/./doc_src/images/91000749.jpg}

\includegraphics[width=0.8\linewidth]{/home/fizlrock/data/files/backup/code_backup/hobby/algoritms/LabExecutor/app/./doc_src/images/138099534.jpg}
\pagebreak
\paragraph{Задание 3.1}

Закодировать сообщение методом LZ77\\
Строка:РИМ\_РОМ\_МУРОМ\_МУРКА\\
Результат: <0,0,Р> <0,0,И> <0,0,М> <0,0,\_> <6,1,О> <6,2,М> <0,0,У> <4,5,У> <4,1,К> <0,0,А>\\
\begin{table}[h!]
\centering
\begin{tabular}{|c|c|c|c|c|c|c|c|c|c|c|c|c|c|c|c|c|} 
\hline
\multicolumn{10}{|c|}{Cловарь} & \multicolumn{6}{c|}{Буфер} & Код  \\ \hline
  &   &   &   &   &   &   &   &   &   & \cellcolor[HTML]{8CE4F6} Р & И & М &   & Р & О & <0,0,Р>
\\ \hline
  &   &   &   &   &   &   &   &   & Р & \cellcolor[HTML]{8CE4F6} И & М &   & Р & О & М & <0,0,И>
\\ \hline
  &   &   &   &   &   &   &   & Р & И & \cellcolor[HTML]{8CE4F6} М &   & Р & О & М &   & <0,0,М>
\\ \hline
  &   &   &   &   &   &   & Р & И & М & \cellcolor[HTML]{8CE4F6}   & Р & О & М &   & М & <0,0,\_>
\\ \hline
  &   &   &   &   &   & \cellcolor[HTML]{FFFF00} Р & И & М &   & \cellcolor[HTML]{FFFF00} Р & \cellcolor[HTML]{8CE4F6} О & М &   & М & У & <6,1,О>
\\ \hline
  &   &   &   & Р & И & \cellcolor[HTML]{FFFF00} М & \cellcolor[HTML]{FFFF00}   & Р & О & \cellcolor[HTML]{FFFF00} М & \cellcolor[HTML]{FFFF00}   & \cellcolor[HTML]{8CE4F6} М & У & Р & О & <6,2,М>
\\ \hline
  & Р & И & М &   & Р & О & М &   & М & \cellcolor[HTML]{8CE4F6} У & Р & О & М &   & М & <0,0,У>
\\ \hline
Р & И & М &   & \cellcolor[HTML]{FFFF00} Р & \cellcolor[HTML]{FFFF00} О & \cellcolor[HTML]{FFFF00} М & \cellcolor[HTML]{FFFF00}   & \cellcolor[HTML]{FFFF00} М & У & \cellcolor[HTML]{FFFF00} Р & \cellcolor[HTML]{FFFF00} О & \cellcolor[HTML]{FFFF00} М & \cellcolor[HTML]{FFFF00}   & \cellcolor[HTML]{FFFF00} М & \cellcolor[HTML]{8CE4F6} У & <4,5,У>
\\ \hline
М &   & М & У & \cellcolor[HTML]{FFFF00} Р & О & М &   & М & У & \cellcolor[HTML]{FFFF00} Р & \cellcolor[HTML]{8CE4F6} К & А &   &   &   & <4,1,К>
\\ \hline
М & У & Р & О & М &   & М & У & Р & К & \cellcolor[HTML]{8CE4F6} А &   &   &   &   &   & <0,0,А>
\\ \hline
\end{tabular}
\end{table}

\paragraph{Задание 3.2}

Закодировать сообщение методом LZSS\\
Строка:РИМ\_РОМ\_МУРОМ\_МУРКА\\
Результат: 0'Р' 0'И' 0'М' 0'\_' 1<6,1> 0'О' 1<6,2> 1<4,1> 0'У' 1<4,6> 1<4,1> 0'К' 0'А'\\
\begin{table}[h!]
\centering
\begin{tabular}{|c|c|c|c|c|c|c|c|c|c|c|c|c|c|c|c|c|}
\hline
\multicolumn{10}{|c|}{Cловарь} & \multicolumn{6}{c|}{Буфер} & Код  \\ \hline
  &   &   &   &   &   &   &   &   &   & Р & И & М & \_ & Р & О & 0'Р'\\ \hline
  &   &   &   &   &   &   &   &   & Р & И & М & \_ & Р & О & М & 0'И'\\ \hline
  &   &   &   &   &   &   &   & Р & И & М & \_ & Р & О & М & \_ & 0'М'\\ \hline
  &   &   &   &   &   &   & Р & И & М & \_ & Р & О & М & \_ & М & 0'\_'\\ \hline
  &   &   &   &   &   & \cellcolor[HTML]{FFFF00} Р & И & М & \_ & \cellcolor[HTML]{FFFF00} Р & О & М & \_ & М & У & 1<6,1>\\ \hline
  &   &   &   &   & Р & И & М & \_ & Р & О & М & \_ & М & У & Р & 0'О'\\ \hline
  &   &   &   & Р & И & \cellcolor[HTML]{FFFF00} М & \cellcolor[HTML]{FFFF00} \_ & Р & О & \cellcolor[HTML]{FFFF00} М & \cellcolor[HTML]{FFFF00} \_ & М & У & Р & О & 1<6,2>\\ \hline
  &   & Р & И & \cellcolor[HTML]{FFFF00} М & \_ & Р & О & М & \_ & \cellcolor[HTML]{FFFF00} М & У & Р & О & М & \_ & 1<4,1>\\ \hline
  & Р & И & М & \_ & Р & О & М & \_ & М & У & Р & О & М & \_ & М & 0'У'\\ \hline
Р & И & М & \_ & \cellcolor[HTML]{FFFF00} Р & \cellcolor[HTML]{FFFF00} О & \cellcolor[HTML]{FFFF00} М & \cellcolor[HTML]{FFFF00} \_ & \cellcolor[HTML]{FFFF00} М & \cellcolor[HTML]{FFFF00} У & \cellcolor[HTML]{FFFF00} Р & \cellcolor[HTML]{FFFF00} О & \cellcolor[HTML]{FFFF00} М & \cellcolor[HTML]{FFFF00} \_ & \cellcolor[HTML]{FFFF00} М & \cellcolor[HTML]{FFFF00} У & 1<4,6>\\ \hline
М & \_ & М & У & \cellcolor[HTML]{FFFF00} Р & О & М & \_ & М & У & \cellcolor[HTML]{FFFF00} Р & К & А &   &   &   & 1<4,1>\\ \hline
\_ & М & У & Р & О & М & \_ & М & У & Р & К & А &   &   &   &   & 0'К'\\ \hline
М & У & Р & О & М & \_ & М & У & Р & К & А &   &   &   &   &   & 0'А'\\ \hline
\end{tabular}
\end{table}

\paragraph{Задание 3.3}

Закодировать сообщение методом LZ78\\
Строка:РИМ\_РОМ\_МУРОМ\_МУРКА\\
\begin{table}[h!]
\centering
\begin{tabular}{|c|c|c|} 
\hline
 Входная фраза (в словарь) & Код & Позиция словаря \\ \hline

 &  & 0 \\ \hline
Р & 0'Р' & 1 \\ \hline
И & 0'И' & 2 \\ \hline
М & 0'М' & 3 \\ \hline
\_ & 0'\_' & 4 \\ \hline
РО & 1'О' & 5 \\ \hline
М\_ & 3'\_' & 6 \\ \hline
МУ & 3'У' & 7 \\ \hline
РОМ & 5'М' & 8 \\ \hline
\_М & 4'М' & 9 \\ \hline
У & 0'У' & 10 \\ \hline
РК & 1'К' & 11 \\ \hline
А & 0'А' & 12 \\ \hline
\end{tabular}
\end{table}

Результат: 0'Р' 0'И' 0'М' 0'\_' 1'О' 3'\_' 3'У' 5'М' 4'М' 0'У' 1'К' 0'А'\\
\pagebreak
\paragraph{Задание 4. Арифметическое кодирование\\}

Исходная строка: ГНРПАНПППП\
\begin{center}
 \begin{tabular}{ |c|c| } 
  \hline
     Буква & Вероятность \\ \hline
П & 0.50\\\hline
Н & 0.20\\\hline
Р & 0.10\\\hline
А & 0.10\\\hline
Г & 0.10
\\ \hline \end{tabular}
\end{center}
\begin{center}
 \begin{tabular}{ |c|c|c| } 
  \hline
     Буква & Начало & Конец \\ \hline
П & 0.00 & 0.50\\\hline
Н & 0.50 & 0.70\\\hline
Р & 0.70 & 0.80\\\hline
А & 0.80 & 0.90\\\hline
Г & 0.90 & 1.00
\\ \hline \end{tabular}
\end{center}
\begin{center}
 \begin{tabular}{ |c|c|c|c| } 
  \hline
     Буква & delta & min & max \\ \hline
Г & 0.1000000000 & 0.9000000000 & 1.0000000000\\\hline
Н & 0.0200000000 & 0.9500000000 & 0.9700000000\\\hline
Р & 0.0020000000 & 0.9640000000 & 0.9660000000\\\hline
П & 0.0010000000 & 0.9640000000 & 0.9650000000\\\hline
А & 0.0001000000 & 0.9648000000 & 0.9649000000\\\hline
Н & 0.0000200000 & 0.9648500000 & 0.9648700000\\\hline
П & 0.0000100000 & 0.9648500000 & 0.9648600000\\\hline
П & 0.0000050000 & 0.9648500000 & 0.9648550000\\\hline
П & 0.0000025000 & 0.9648500000 & 0.9648525000\\\hline
П & 0.0000012500 & 0.9648500000 & 0.9648512500
\\ \hline \end{tabular}
\end{center}
Результат: 96485
\pagebreak
\paragraph{Задание 5.1}

\\ 

Декодировать сообщение методом адаптивного хаффмана \\
Строка: 
'K'0'L'00'N'100'B'0011111110111001\\
Результат: KLNBBBNNNL

\includegraphics[width=0.8\linewidth]{/home/fizlrock/data/files/backup/code_backup/hobby/algoritms/LabExecutor/app/./doc_src/images/1811544025.jpg}

\includegraphics[width=0.8\linewidth]{/home/fizlrock/data/files/backup/code_backup/hobby/algoritms/LabExecutor/app/./doc_src/images/927880794.jpg}

\includegraphics[width=0.8\linewidth]{/home/fizlrock/data/files/backup/code_backup/hobby/algoritms/LabExecutor/app/./doc_src/images/258606590.jpg}

\includegraphics[width=0.8\linewidth]{/home/fizlrock/data/files/backup/code_backup/hobby/algoritms/LabExecutor/app/./doc_src/images/1222880248.jpg}

\includegraphics[width=0.8\linewidth]{/home/fizlrock/data/files/backup/code_backup/hobby/algoritms/LabExecutor/app/./doc_src/images/1277116474.jpg}

\includegraphics[width=0.8\linewidth]{/home/fizlrock/data/files/backup/code_backup/hobby/algoritms/LabExecutor/app/./doc_src/images/1781664586.jpg}

\includegraphics[width=0.8\linewidth]{/home/fizlrock/data/files/backup/code_backup/hobby/algoritms/LabExecutor/app/./doc_src/images/626789874.jpg}

\includegraphics[width=0.8\linewidth]{/home/fizlrock/data/files/backup/code_backup/hobby/algoritms/LabExecutor/app/./doc_src/images/908058739.jpg}

\includegraphics[width=0.8\linewidth]{/home/fizlrock/data/files/backup/code_backup/hobby/algoritms/LabExecutor/app/./doc_src/images/1988684323.jpg}

\includegraphics[width=0.8\linewidth]{/home/fizlrock/data/files/backup/code_backup/hobby/algoritms/LabExecutor/app/./doc_src/images/426897016.jpg}
\pagebreak
\paragraph{Задание 5.3 Декодировать строку(LZSS)\\}

Исходная строка: [0'к'] [0'и'] [0'л'] [0'ь'] [0' '] [1<5,4>] [1<1,1>] [0'а'] [1<3,4>] [0'о']\\
\begin{table}[h!]
\centering
\begin{tabular}{|c|c|c|}
\hline
 Cловарь & Буфер & Код  \\ \hline
0'к' & [ ,  ,  ,  ,  ,  ,  ,  ,  , к] & к
\\ \hline
0'и' & [ ,  ,  ,  ,  ,  ,  ,  , к, и] & и
\\ \hline
0'л' & [ ,  ,  ,  ,  ,  ,  , к, и, л] & л
\\ \hline
0'ь' & [ ,  ,  ,  ,  ,  , к, и, л, ь] & ь
\\ \hline
0' ' & [ ,  ,  ,  ,  , к, и, л, ь,  ] &  
\\ \hline
1<5,4> & [ , к, и, л, ь,  , к, и, л, ь] & киль
\\ \hline
1<1,1> & [к, и, л, ь,  , к, и, л, ь, к] & к
\\ \hline
0'а' & [и, л, ь,  , к, и, л, ь, к, а] & а
\\ \hline
1<3,4> & [к, и, л, ь, к, а,  , к, и, л] &  кил
\\ \hline
0'о' & [и, л, ь, к, а,  , к, и, л, о] & о
\\ \hline
\end{tabular}
\end{table}

Результат: киль килька кило
\pagebreak
\paragraph{Задание 5.4 Декодировать строку(LZ78)\\}

Исходная строка: [0'к'] [0'о'] [0'с'] [0'т'] [0'ь'] [0' '] [1'о'] [3'а'] [6'о'] [8' '] [2'к'] [0'о']\\
\begin{table}[h!]
\centering
\begin{tabular}{|c|c|c|}
\hline
 Cловарь & Буфер & Код  \\ \hline
 & [] & 
\\ \hline
0'к' & [, к] & к
\\ \hline
0'о' & [, к, о] & о
\\ \hline
0'с' & [, к, о, с] & с
\\ \hline
0'т' & [, к, о, с, т] & т
\\ \hline
0'ь' & [, к, о, с, т, ь] & ь
\\ \hline
0' ' & [, к, о, с, т, ь,  ] &  
\\ \hline
1'о' & [, к, о, с, т, ь,  , ко] & ко
\\ \hline
3'а' & [, к, о, с, т, ь,  , ко, са] & са
\\ \hline
6'о' & [, к, о, с, т, ь,  , ко, са,  о] &  о
\\ \hline
8' ' & [, к, о, с, т, ь,  , ко, са,  о, са ] & са 
\\ \hline
2'к' & [, к, о, с, т, ь,  , ко, са,  о, са , ок] & ок
\\ \hline
0'о' & [, к, о, с, т, ь,  , ко, са,  о, са , ок, о] & о
\\ \hline
\end{tabular}
\end{table}

Результат: кость коса оса око
\pagebreak
\subsection{Вариант №19}
\paragraph{Задание 1. Блочный хаффман \\}

Строка КУКУУУУУУУ, размер блока: 3
\begin{center}
 \begin{tabular}{ |c|c|l| } 
  \hline
     Буква & Вероятность & Код\\ \hline
У & 0.80 & 1\\\hline
К & 0.20 & 0
\\ \hline \end{tabular}
\end{center}
Энтропия алфавита: 0.7219
\begin{center}
 \begin{tabular}{ |c|c|l| } 
  \hline
     Блок & Вероятность & Код\\ \hline
УУУ & 0.51 & 1\\\hline
КУУ & 0.13 & 001\\\hline
УУК & 0.13 & 010\\\hline
УКУ & 0.13 & 011\\\hline
УКК & 0.03 & 00001\\\hline
ККУ & 0.03 & 00010\\\hline
КУК & 0.03 & 00011\\\hline
ККК & 0.01 & 00000
\\ \hline \end{tabular}
\end{center}
Бит на символ при посимвольном кодировании: 1.0000, при блочном: 0.7280

\includegraphics[width=0.5\linewidth]{/home/fizlrock/data/files/backup/code_backup/hobby/algoritms/LabExecutor/app/./doc_src/images/1051864986.jpg}

\includegraphics[width=0.9\linewidth]{/home/fizlrock/data/files/backup/code_backup/hobby/algoritms/LabExecutor/app/./doc_src/images/1198210759.jpg}
\pagebreak
\paragraph{Задание 2. Сжать адаптивным хаффманом\\}

Строка: 
ВУКАУВУААА\\
Результат: 'В' 0'У' 00'К' 100'А' 11 10 11 1101 111 10

\includegraphics[width=0.8\linewidth]{/home/fizlrock/data/files/backup/code_backup/hobby/algoritms/LabExecutor/app/./doc_src/images/35995115.jpg}

\includegraphics[width=0.8\linewidth]{/home/fizlrock/data/files/backup/code_backup/hobby/algoritms/LabExecutor/app/./doc_src/images/1635404628.jpg}

\includegraphics[width=0.8\linewidth]{/home/fizlrock/data/files/backup/code_backup/hobby/algoritms/LabExecutor/app/./doc_src/images/1213614403.jpg}

\includegraphics[width=0.8\linewidth]{/home/fizlrock/data/files/backup/code_backup/hobby/algoritms/LabExecutor/app/./doc_src/images/1882927534.jpg}

\includegraphics[width=0.8\linewidth]{/home/fizlrock/data/files/backup/code_backup/hobby/algoritms/LabExecutor/app/./doc_src/images/539007365.jpg}

\includegraphics[width=0.8\linewidth]{/home/fizlrock/data/files/backup/code_backup/hobby/algoritms/LabExecutor/app/./doc_src/images/115659824.jpg}

\includegraphics[width=0.8\linewidth]{/home/fizlrock/data/files/backup/code_backup/hobby/algoritms/LabExecutor/app/./doc_src/images/1301471680.jpg}

\includegraphics[width=0.8\linewidth]{/home/fizlrock/data/files/backup/code_backup/hobby/algoritms/LabExecutor/app/./doc_src/images/890199350.jpg}

\includegraphics[width=0.8\linewidth]{/home/fizlrock/data/files/backup/code_backup/hobby/algoritms/LabExecutor/app/./doc_src/images/2025564098.jpg}

\includegraphics[width=0.8\linewidth]{/home/fizlrock/data/files/backup/code_backup/hobby/algoritms/LabExecutor/app/./doc_src/images/1395938492.jpg}
\pagebreak
\paragraph{Задание 3.1}

Закодировать сообщение методом LZ77\\
Строка:ОЛОВО\_ЛОВЕЦ\_ОВЦА\_ЦАП\\
Результат: <0,0,О> <0,0,Л> <8,1,В> <6,1,\_> <5,3,Е> <0,0,Ц> <4,1,О> <0,1,Ц> <0,0,А> <5,1,Ц> <7,1,П>\\
\begin{table}[h!]
\centering
\begin{tabular}{|c|c|c|c|c|c|c|c|c|c|c|c|c|c|c|c|c|} 
\hline
\multicolumn{10}{|c|}{Cловарь} & \multicolumn{6}{c|}{Буфер} & Код  \\ \hline
  &   &   &   &   &   &   &   &   &   & \cellcolor[HTML]{8CE4F6} О & Л & О & В & О &   & <0,0,О>
\\ \hline
  &   &   &   &   &   &   &   &   & О & \cellcolor[HTML]{8CE4F6} Л & О & В & О &   & Л & <0,0,Л>
\\ \hline
  &   &   &   &   &   &   &   & \cellcolor[HTML]{FFFF00} О & Л & \cellcolor[HTML]{FFFF00} О & \cellcolor[HTML]{8CE4F6} В & О &   & Л & О & <8,1,В>
\\ \hline
  &   &   &   &   &   & \cellcolor[HTML]{FFFF00} О & Л & О & В & \cellcolor[HTML]{FFFF00} О & \cellcolor[HTML]{8CE4F6}   & Л & О & В & Е & <6,1,\_>
\\ \hline
  &   &   &   & О & \cellcolor[HTML]{FFFF00} Л & \cellcolor[HTML]{FFFF00} О & \cellcolor[HTML]{FFFF00} В & О &   & \cellcolor[HTML]{FFFF00} Л & \cellcolor[HTML]{FFFF00} О & \cellcolor[HTML]{FFFF00} В & \cellcolor[HTML]{8CE4F6} Е & Ц &   & <5,3,Е>
\\ \hline
О & Л & О & В & О &   & Л & О & В & Е & \cellcolor[HTML]{8CE4F6} Ц &   & О & В & Ц & А & <0,0,Ц>
\\ \hline
Л & О & В & О & \cellcolor[HTML]{FFFF00}   & Л & О & В & Е & Ц & \cellcolor[HTML]{FFFF00}   & \cellcolor[HTML]{8CE4F6} О & В & Ц & А &   & <4,1,О>
\\ \hline
\cellcolor[HTML]{FFFF00} В & О &   & Л & О & В & Е & Ц &   & О & \cellcolor[HTML]{FFFF00} В & \cellcolor[HTML]{8CE4F6} Ц & А &   & Ц & А & <0,1,Ц>
\\ \hline
  & Л & О & В & Е & Ц &   & О & В & Ц & \cellcolor[HTML]{8CE4F6} А &   & Ц & А & П &   & <0,0,А>
\\ \hline
Л & О & В & Е & Ц & \cellcolor[HTML]{FFFF00}   & О & В & Ц & А & \cellcolor[HTML]{FFFF00}   & \cellcolor[HTML]{8CE4F6} Ц & А & П &   &   & <5,1,Ц>
\\ \hline
В & Е & Ц &   & О & В & Ц & \cellcolor[HTML]{FFFF00} А &   & Ц & \cellcolor[HTML]{FFFF00} А & \cellcolor[HTML]{8CE4F6} П &   &   &   &   & <7,1,П>
\\ \hline
\end{tabular}
\end{table}

\paragraph{Задание 3.2}

Закодировать сообщение методом LZSS\\
Строка:ОЛОВО\_ЛОВЕЦ\_ОВЦА\_ЦАП\\
Результат: 0'О' 0'Л' 1<8,1> 0'В' 1<6,1> 0'\_' 1<5,3> 0'Е' 0'Ц' 1<4,1> 1<0,2> 1<6,1> 0'А' 1<5,1> 1<7,2> 0'П'\\
\begin{table}[h!]
\centering
\begin{tabular}{|c|c|c|c|c|c|c|c|c|c|c|c|c|c|c|c|c|}
\hline
\multicolumn{10}{|c|}{Cловарь} & \multicolumn{6}{c|}{Буфер} & Код  \\ \hline
  &   &   &   &   &   &   &   &   &   & О & Л & О & В & О & \_ & 0'О'\\ \hline
  &   &   &   &   &   &   &   &   & О & Л & О & В & О & \_ & Л & 0'Л'\\ \hline
  &   &   &   &   &   &   &   & \cellcolor[HTML]{FFFF00} О & Л & \cellcolor[HTML]{FFFF00} О & В & О & \_ & Л & О & 1<8,1>\\ \hline
  &   &   &   &   &   &   & О & Л & О & В & О & \_ & Л & О & В & 0'В'\\ \hline
  &   &   &   &   &   & \cellcolor[HTML]{FFFF00} О & Л & О & В & \cellcolor[HTML]{FFFF00} О & \_ & Л & О & В & Е & 1<6,1>\\ \hline
  &   &   &   &   & О & Л & О & В & О & \_ & Л & О & В & Е & Ц & 0'\_'\\ \hline
  &   &   &   & О & \cellcolor[HTML]{FFFF00} Л & \cellcolor[HTML]{FFFF00} О & \cellcolor[HTML]{FFFF00} В & О & \_ & \cellcolor[HTML]{FFFF00} Л & \cellcolor[HTML]{FFFF00} О & \cellcolor[HTML]{FFFF00} В & Е & Ц & \_ & 1<5,3>\\ \hline
  & О & Л & О & В & О & \_ & Л & О & В & Е & Ц & \_ & О & В & Ц & 0'Е'\\ \hline
О & Л & О & В & О & \_ & Л & О & В & Е & Ц & \_ & О & В & Ц & А & 0'Ц'\\ \hline
Л & О & В & О & \cellcolor[HTML]{FFFF00} \_ & Л & О & В & Е & Ц & \cellcolor[HTML]{FFFF00} \_ & О & В & Ц & А & \_ & 1<4,1>\\ \hline
\cellcolor[HTML]{FFFF00} О & \cellcolor[HTML]{FFFF00} В & О & \_ & Л & О & В & Е & Ц & \_ & \cellcolor[HTML]{FFFF00} О & \cellcolor[HTML]{FFFF00} В & Ц & А & \_ & Ц & 1<0,2>\\ \hline
О & \_ & Л & О & В & Е & \cellcolor[HTML]{FFFF00} Ц & \_ & О & В & \cellcolor[HTML]{FFFF00} Ц & А & \_ & Ц & А & П & 1<6,1>\\ \hline
\_ & Л & О & В & Е & Ц & \_ & О & В & Ц & А & \_ & Ц & А & П &   & 0'А'\\ \hline
Л & О & В & Е & Ц & \cellcolor[HTML]{FFFF00} \_ & О & В & Ц & А & \cellcolor[HTML]{FFFF00} \_ & Ц & А & П &   &   & 1<5,1>\\ \hline
О & В & Е & Ц & \_ & О & В & \cellcolor[HTML]{FFFF00} Ц & \cellcolor[HTML]{FFFF00} А & \_ & \cellcolor[HTML]{FFFF00} Ц & \cellcolor[HTML]{FFFF00} А & П &   &   &   & 1<7,2>\\ \hline
Е & Ц & \_ & О & В & Ц & А & \_ & Ц & А & П &   &   &   &   &   & 0'П'\\ \hline
\end{tabular}
\end{table}

\paragraph{Задание 3.3}

Закодировать сообщение методом LZ78\\
Строка:ОЛОВО\_ЛОВЕЦ\_ОВЦА\_ЦАП\\
\begin{table}[h!]
\centering
\begin{tabular}{|c|c|c|} 
\hline
 Входная фраза (в словарь) & Код & Позиция словаря \\ \hline

 &  & 0 \\ \hline
О & 0'О' & 1 \\ \hline
Л & 0'Л' & 2 \\ \hline
ОВ & 1'В' & 3 \\ \hline
О\_ & 1'\_' & 4 \\ \hline
ЛО & 2'О' & 5 \\ \hline
В & 0'В' & 6 \\ \hline
Е & 0'Е' & 7 \\ \hline
Ц & 0'Ц' & 8 \\ \hline
\_ & 0'\_' & 9 \\ \hline
ОВЦ & 3'Ц' & 10 \\ \hline
А & 0'А' & 11 \\ \hline
\_Ц & 9'Ц' & 12 \\ \hline
АП & 11'П' & 13 \\ \hline
\end{tabular}
\end{table}

Результат: 0'О' 0'Л' 1'В' 1'\_' 2'О' 0'В' 0'Е' 0'Ц' 0'\_' 3'Ц' 0'А' 9'Ц' 11'П'\\
\pagebreak
\paragraph{Задание 4. Арифметическое кодирование\\}

Исходная строка: ВУКАУВУААА\
\begin{center}
 \begin{tabular}{ |c|c| } 
  \hline
     Буква & Вероятность \\ \hline
А & 0.40\\\hline
У & 0.30\\\hline
В & 0.20\\\hline
К & 0.10
\\ \hline \end{tabular}
\end{center}
\begin{center}
 \begin{tabular}{ |c|c|c| } 
  \hline
     Буква & Начало & Конец \\ \hline
А & 0.00 & 0.40\\\hline
У & 0.40 & 0.70\\\hline
В & 0.70 & 0.90\\\hline
К & 0.90 & 1.00
\\ \hline \end{tabular}
\end{center}
\begin{center}
 \begin{tabular}{ |c|c|c|c| } 
  \hline
     Буква & delta & min & max \\ \hline
В & 0.2000000000 & 0.7000000000 & 0.9000000000\\\hline
У & 0.0600000000 & 0.7800000000 & 0.8400000000\\\hline
К & 0.0060000000 & 0.8340000000 & 0.8400000000\\\hline
А & 0.0024000000 & 0.8340000000 & 0.8364000000\\\hline
У & 0.0007200000 & 0.8349600000 & 0.8356800000\\\hline
В & 0.0001440000 & 0.8354640000 & 0.8356080000\\\hline
У & 0.0000432000 & 0.8355216000 & 0.8355648000\\\hline
А & 0.0000172800 & 0.8355216000 & 0.8355388800\\\hline
А & 0.0000069120 & 0.8355216000 & 0.8355285120\\\hline
А & 0.0000027648 & 0.8355216000 & 0.8355243648
\\ \hline \end{tabular}
\end{center}
Результат: 835522
\pagebreak
\paragraph{Задание 5.1}

\\ 

Декодировать сообщение методом адаптивного хаффмана \\
Строка: 
Ошибка декодирования\\
Результат: Ошибка декодирования
\pagebreak
\paragraph{Задание 5.3 Декодировать строку(LZSS)\\}

Исходная строка: [0'р'] [0'у'] [0'к'] [0'и'] [0' '] [1<7,2>] [0'л'] [0'о'] [1<5,1>] [1<7,2>] [0'в'] [1<2,2>] [0'й']\\
\begin{table}[h!]
\centering
\begin{tabular}{|c|c|c|}
\hline
 Cловарь & Буфер & Код  \\ \hline
0'р' & [ ,  ,  ,  ,  ,  ,  ,  ,  , р] & р
\\ \hline
0'у' & [ ,  ,  ,  ,  ,  ,  ,  , р, у] & у
\\ \hline
0'к' & [ ,  ,  ,  ,  ,  ,  , р, у, к] & к
\\ \hline
0'и' & [ ,  ,  ,  ,  ,  , р, у, к, и] & и
\\ \hline
0' ' & [ ,  ,  ,  ,  , р, у, к, и,  ] &  
\\ \hline
1<7,2> & [ ,  ,  , р, у, к, и,  , к, и] & ки
\\ \hline
0'л' & [ ,  , р, у, к, и,  , к, и, л] & л
\\ \hline
0'о' & [ , р, у, к, и,  , к, и, л, о] & о
\\ \hline
1<5,1> & [р, у, к, и,  , к, и, л, о,  ] &  
\\ \hline
1<7,2> & [к, и,  , к, и, л, о,  , л, о] & ло
\\ \hline
0'в' & [и,  , к, и, л, о,  , л, о, в] & в
\\ \hline
1<2,2> & [к, и, л, о,  , л, о, в, к, и] & ки
\\ \hline
0'й' & [и, л, о,  , л, о, в, к, и, й] & й
\\ \hline
\end{tabular}
\end{table}

Результат: руки кило ловкий
\pagebreak
\paragraph{Задание 5.4 Декодировать строку(LZ78)\\}

Исходная строка: [0'б'] [0'е'] [0'р'] [2'т'] [0' '] [1'е'] [3'е'] [0'г'] [5'б'] [7'г']\\
\begin{table}[h!]
\centering
\begin{tabular}{|c|c|c|}
\hline
 Cловарь & Буфер & Код  \\ \hline
 & [] & 
\\ \hline
0'б' & [, б] & б
\\ \hline
0'е' & [, б, е] & е
\\ \hline
0'р' & [, б, е, р] & р
\\ \hline
2'т' & [, б, е, р, ет] & ет
\\ \hline
0' ' & [, б, е, р, ет,  ] &  
\\ \hline
1'е' & [, б, е, р, ет,  , бе] & бе
\\ \hline
3'е' & [, б, е, р, ет,  , бе, ре] & ре
\\ \hline
0'г' & [, б, е, р, ет,  , бе, ре, г] & г
\\ \hline
5'б' & [, б, е, р, ет,  , бе, ре, г,  б] &  б
\\ \hline
7'г' & [, б, е, р, ет,  , бе, ре, г,  б, рег] & рег
\\ \hline
\end{tabular}
\end{table}

Результат: берет берег брег
\pagebreak
\subsection{Вариант №20}
\paragraph{Задание 1. Блочный хаффман \\}

Строка РРРРАААААА, размер блока: 3
\begin{center}
 \begin{tabular}{ |c|c|l| } 
  \hline
     Буква & Вероятность & Код\\ \hline
А & 0.60 & 1\\\hline
Р & 0.40 & 0
\\ \hline \end{tabular}
\end{center}
Энтропия алфавита: 0.9710
\begin{center}
 \begin{tabular}{ |c|c|l| } 
  \hline
     Блок & Вероятность & Код\\ \hline
ААА & 0.22 & 01\\\hline
РАА & 0.14 & 100\\\hline
АРА & 0.14 & 101\\\hline
ААР & 0.14 & 110\\\hline
РРА & 0.10 & 001\\\hline
РАР & 0.10 & 1111\\\hline
АРР & 0.10 & 000\\\hline
РРР & 0.06 & 1110
\\ \hline \end{tabular}
\end{center}
Бит на символ при посимвольном кодировании: 1.0000, при блочном: 0.9813

\includegraphics[width=0.5\linewidth]{/home/fizlrock/data/files/backup/code_backup/hobby/algoritms/LabExecutor/app/./doc_src/images/1433955746.jpg}

\includegraphics[width=0.9\linewidth]{/home/fizlrock/data/files/backup/code_backup/hobby/algoritms/LabExecutor/app/./doc_src/images/970593408.jpg}
\pagebreak
\paragraph{Задание 2. Сжать адаптивным хаффманом\\}

Строка: 
КУИРЕИИККК\\
Результат: 'К' 0'У' 00'И' 100'Р' 000'Е' 01 10 101 111 10

\includegraphics[width=0.8\linewidth]{/home/fizlrock/data/files/backup/code_backup/hobby/algoritms/LabExecutor/app/./doc_src/images/990681865.jpg}

\includegraphics[width=0.8\linewidth]{/home/fizlrock/data/files/backup/code_backup/hobby/algoritms/LabExecutor/app/./doc_src/images/1160122568.jpg}

\includegraphics[width=0.8\linewidth]{/home/fizlrock/data/files/backup/code_backup/hobby/algoritms/LabExecutor/app/./doc_src/images/1239624238.jpg}

\includegraphics[width=0.8\linewidth]{/home/fizlrock/data/files/backup/code_backup/hobby/algoritms/LabExecutor/app/./doc_src/images/16633886.jpg}

\includegraphics[width=0.8\linewidth]{/home/fizlrock/data/files/backup/code_backup/hobby/algoritms/LabExecutor/app/./doc_src/images/537579053.jpg}

\includegraphics[width=0.8\linewidth]{/home/fizlrock/data/files/backup/code_backup/hobby/algoritms/LabExecutor/app/./doc_src/images/260165247.jpg}

\includegraphics[width=0.8\linewidth]{/home/fizlrock/data/files/backup/code_backup/hobby/algoritms/LabExecutor/app/./doc_src/images/332671708.jpg}

\includegraphics[width=0.8\linewidth]{/home/fizlrock/data/files/backup/code_backup/hobby/algoritms/LabExecutor/app/./doc_src/images/1708861077.jpg}

\includegraphics[width=0.8\linewidth]{/home/fizlrock/data/files/backup/code_backup/hobby/algoritms/LabExecutor/app/./doc_src/images/89274012.jpg}

\includegraphics[width=0.8\linewidth]{/home/fizlrock/data/files/backup/code_backup/hobby/algoritms/LabExecutor/app/./doc_src/images/588080301.jpg}
\pagebreak

\paragraph{Задание 3.2}

Закодировать сообщение методом LZSS\\
Строка:КАКТУС\_ТУСА\_ТУЗ\_УСА\\
Результат: 0'К' 0'А' 1<8,1> 0'Т' 0'У' 0'С' 0'\_' 1<6,3> 1<1,1> 1<5,3> 0'З' 1<1,1> 1<2,3>\\
\begin{table}[h!]
\centering
\begin{tabular}{|c|c|c|c|c|c|c|c|c|c|c|c|c|c|c|c|c|}
\hline
\multicolumn{10}{|c|}{Cловарь} & \multicolumn{6}{c|}{Буфер} & Код  \\ \hline
  &   &   &   &   &   &   &   &   &   & К & А & К & Т & У & С & 0'К'\\ \hline
  &   &   &   &   &   &   &   &   & К & А & К & Т & У & С & \_ & 0'А'\\ \hline
  &   &   &   &   &   &   &   & \cellcolor[HTML]{FFFF00} К & А & \cellcolor[HTML]{FFFF00} К & Т & У & С & \_ & Т & 1<8,1>\\ \hline
  &   &   &   &   &   &   & К & А & К & Т & У & С & \_ & Т & У & 0'Т'\\ \hline
  &   &   &   &   &   & К & А & К & Т & У & С & \_ & Т & У & С & 0'У'\\ \hline
  &   &   &   &   & К & А & К & Т & У & С & \_ & Т & У & С & А & 0'С'\\ \hline
  &   &   &   & К & А & К & Т & У & С & \_ & Т & У & С & А & \_ & 0'\_'\\ \hline
  &   &   & К & А & К & \cellcolor[HTML]{FFFF00} Т & \cellcolor[HTML]{FFFF00} У & \cellcolor[HTML]{FFFF00} С & \_ & \cellcolor[HTML]{FFFF00} Т & \cellcolor[HTML]{FFFF00} У & \cellcolor[HTML]{FFFF00} С & А & \_ & Т & 1<6,3>\\ \hline
К & \cellcolor[HTML]{FFFF00} А & К & Т & У & С & \_ & Т & У & С & \cellcolor[HTML]{FFFF00} А & \_ & Т & У & З & \_ & 1<1,1>\\ \hline
А & К & Т & У & С & \cellcolor[HTML]{FFFF00} \_ & \cellcolor[HTML]{FFFF00} Т & \cellcolor[HTML]{FFFF00} У & С & А & \cellcolor[HTML]{FFFF00} \_ & \cellcolor[HTML]{FFFF00} Т & \cellcolor[HTML]{FFFF00} У & З & \_ & У & 1<5,3>\\ \hline
У & С & \_ & Т & У & С & А & \_ & Т & У & З & \_ & У & С & А &   & 0'З'\\ \hline
С & \cellcolor[HTML]{FFFF00} \_ & Т & У & С & А & \_ & Т & У & З & \cellcolor[HTML]{FFFF00} \_ & У & С & А &   &   & 1<1,1>\\ \hline
\_ & Т & \cellcolor[HTML]{FFFF00} У & \cellcolor[HTML]{FFFF00} С & \cellcolor[HTML]{FFFF00} А & \_ & Т & У & З & \_ & \cellcolor[HTML]{FFFF00} У & \cellcolor[HTML]{FFFF00} С & \cellcolor[HTML]{FFFF00} А &   &   &   & 1<2,3>\\ \hline
\end{tabular}
\end{table}

\paragraph{Задание 3.3}

Закодировать сообщение методом LZ78\\
Строка:КАКТУС\_ТУСА\_ТУЗ\_УСА\\
\begin{table}[h!]
\centering
\begin{tabular}{|c|c|c|} 
\hline
 Входная фраза (в словарь) & Код & Позиция словаря \\ \hline

 &  & 0 \\ \hline
К & 0'К' & 1 \\ \hline
А & 0'А' & 2 \\ \hline
КТ & 1'Т' & 3 \\ \hline
У & 0'У' & 4 \\ \hline
С & 0'С' & 5 \\ \hline
\_ & 0'\_' & 6 \\ \hline
Т & 0'Т' & 7 \\ \hline
УС & 4'С' & 8 \\ \hline
А\_ & 2'\_' & 9 \\ \hline
ТУ & 7'У' & 10 \\ \hline
З & 0'З' & 11 \\ \hline
\_У & 6'У' & 12 \\ \hline
СА & 5'А' & 13 \\ \hline
\end{tabular}
\end{table}

Результат: 0'К' 0'А' 1'Т' 0'У' 0'С' 0'\_' 0'Т' 4'С' 2'\_' 7'У' 0'З' 6'У' 5'А'\\
\pagebreak
\paragraph{Задание 4. Арифметическое кодирование\\}

Исходная строка: КУИРЕИИККК\
\begin{center}
 \begin{tabular}{ |c|c| } 
  \hline
     Буква & Вероятность \\ \hline
К & 0.40\\\hline
И & 0.30\\\hline
Р & 0.10\\\hline
У & 0.10\\\hline
Е & 0.10
\\ \hline \end{tabular}
\end{center}
\begin{center}
 \begin{tabular}{ |c|c|c| } 
  \hline
     Буква & Начало & Конец \\ \hline
К & 0.00 & 0.40\\\hline
И & 0.40 & 0.70\\\hline
Р & 0.70 & 0.80\\\hline
У & 0.80 & 0.90\\\hline
Е & 0.90 & 1.00
\\ \hline \end{tabular}
\end{center}
\begin{center}
 \begin{tabular}{ |c|c|c|c| } 
  \hline
     Буква & delta & min & max \\ \hline
К & 0.4000000000 & 0.0000000000 & 0.4000000000\\\hline
У & 0.0400000000 & 0.3200000000 & 0.3600000000\\\hline
И & 0.0120000000 & 0.3360000000 & 0.3480000000\\\hline
Р & 0.0012000000 & 0.3444000000 & 0.3456000000\\\hline
Е & 0.0001200000 & 0.3454800000 & 0.3456000000\\\hline
И & 0.0000360000 & 0.3455280000 & 0.3455640000\\\hline
И & 0.0000108000 & 0.3455424000 & 0.3455532000\\\hline
К & 0.0000043200 & 0.3455424000 & 0.3455467200\\\hline
К & 0.0000017280 & 0.3455424000 & 0.3455441280\\\hline
К & 0.0000006912 & 0.3455424000 & 0.3455430912
\\ \hline \end{tabular}
\end{center}
Результат: 345543
\pagebreak
\paragraph{Задание 5.1}

\\ 

Декодировать сообщение методом адаптивного хаффмана \\
Строка: 
'P'0'O'0100'K'000'M'110110110111110\\
Результат: POOKMMMOMOO

\includegraphics[width=0.8\linewidth]{/home/fizlrock/data/files/backup/code_backup/hobby/algoritms/LabExecutor/app/./doc_src/images/1966694816.jpg}

\includegraphics[width=0.8\linewidth]{/home/fizlrock/data/files/backup/code_backup/hobby/algoritms/LabExecutor/app/./doc_src/images/600489816.jpg}

\includegraphics[width=0.8\linewidth]{/home/fizlrock/data/files/backup/code_backup/hobby/algoritms/LabExecutor/app/./doc_src/images/1817609436.jpg}

\includegraphics[width=0.8\linewidth]{/home/fizlrock/data/files/backup/code_backup/hobby/algoritms/LabExecutor/app/./doc_src/images/611924704.jpg}

\includegraphics[width=0.8\linewidth]{/home/fizlrock/data/files/backup/code_backup/hobby/algoritms/LabExecutor/app/./doc_src/images/2059786328.jpg}

\includegraphics[width=0.8\linewidth]{/home/fizlrock/data/files/backup/code_backup/hobby/algoritms/LabExecutor/app/./doc_src/images/1001652420.jpg}

\includegraphics[width=0.8\linewidth]{/home/fizlrock/data/files/backup/code_backup/hobby/algoritms/LabExecutor/app/./doc_src/images/208582634.jpg}

\includegraphics[width=0.8\linewidth]{/home/fizlrock/data/files/backup/code_backup/hobby/algoritms/LabExecutor/app/./doc_src/images/103060062.jpg}

\includegraphics[width=0.8\linewidth]{/home/fizlrock/data/files/backup/code_backup/hobby/algoritms/LabExecutor/app/./doc_src/images/1839018826.jpg}

\includegraphics[width=0.8\linewidth]{/home/fizlrock/data/files/backup/code_backup/hobby/algoritms/LabExecutor/app/./doc_src/images/1309053126.jpg}

\includegraphics[width=0.8\linewidth]{/home/fizlrock/data/files/backup/code_backup/hobby/algoritms/LabExecutor/app/./doc_src/images/937559916.jpg}
\pagebreak
\paragraph{Задание 5.3 Декодировать строку(LZSS)\\}

Исходная строка: [0'к'] [0'р'] [0'а'] [0'б'] [0' '] [1<6,4>] [1<4,1>] [1<1,2>] [1<6,4>] [0'к']\\
\begin{table}[h!]
\centering
\begin{tabular}{|c|c|c|}
\hline
 Cловарь & Буфер & Код  \\ \hline
0'к' & [ ,  ,  ,  ,  ,  ,  ,  ,  , к] & к
\\ \hline
0'р' & [ ,  ,  ,  ,  ,  ,  ,  , к, р] & р
\\ \hline
0'а' & [ ,  ,  ,  ,  ,  ,  , к, р, а] & а
\\ \hline
0'б' & [ ,  ,  ,  ,  ,  , к, р, а, б] & б
\\ \hline
0' ' & [ ,  ,  ,  ,  , к, р, а, б,  ] &  
\\ \hline
1<6,4> & [ , к, р, а, б,  , р, а, б,  ] & раб 
\\ \hline
1<4,1> & [к, р, а, б,  , р, а, б,  , б] & б
\\ \hline
1<1,2> & [а, б,  , р, а, б,  , б, р, а] & ра
\\ \hline
1<6,4> & [а, б,  , б, р, а,  , б, р, а] &  бра
\\ \hline
0'к' & [б,  , б, р, а,  , б, р, а, к] & к
\\ \hline
\end{tabular}
\end{table}

Результат: краб раб бра брак
\pagebreak
\paragraph{Задание 5.4 Декодировать строку(LZ78)\\}

Исходная строка: [0'в'] [0'а'] [0'р'] [1'а'] [3' '] [3'в'] [2'н'] [0'ь'] [0' '] [4'н'] [0'н'] [0'а']\\
\begin{table}[h!]
\centering
\begin{tabular}{|c|c|c|}
\hline
 Cловарь & Буфер & Код  \\ \hline
 & [] & 
\\ \hline
0'в' & [, в] & в
\\ \hline
0'а' & [, в, а] & а
\\ \hline
0'р' & [, в, а, р] & р
\\ \hline
1'а' & [, в, а, р, ва] & ва
\\ \hline
3' ' & [, в, а, р, ва, р ] & р 
\\ \hline
3'в' & [, в, а, р, ва, р , рв] & рв
\\ \hline
2'н' & [, в, а, р, ва, р , рв, ан] & ан
\\ \hline
0'ь' & [, в, а, р, ва, р , рв, ан, ь] & ь
\\ \hline
0' ' & [, в, а, р, ва, р , рв, ан, ь,  ] &  
\\ \hline
4'н' & [, в, а, р, ва, р , рв, ан, ь,  , ван] & ван
\\ \hline
0'н' & [, в, а, р, ва, р , рв, ан, ь,  , ван, н] & н
\\ \hline
0'а' & [, в, а, р, ва, р , рв, ан, ь,  , ван, н, а] & а
\\ \hline
\end{tabular}
\end{table}

Результат: варвар рвань ванна
\pagebreak
\subsection{Вариант №21}
\paragraph{Задание 1. Блочный хаффман \\}

Строка ЛЕЛЕЛЕЕЕЕЕ, размер блока: 3
\begin{center}
 \begin{tabular}{ |c|c|l| } 
  \hline
     Буква & Вероятность & Код\\ \hline
Е & 0.70 & 1\\\hline
Л & 0.30 & 0
\\ \hline \end{tabular}
\end{center}
Энтропия алфавита: 0.8813
\begin{center}
 \begin{tabular}{ |c|c|l| } 
  \hline
     Блок & Вероятность & Код\\ \hline
ЕЕЕ & 0.34 & 11\\\hline
ЕЛЕ & 0.15 & 101\\\hline
ЛЕЕ & 0.15 & 00\\\hline
ЕЕЛ & 0.15 & 100\\\hline
ЕЛЛ & 0.06 & 0101\\\hline
ЛЛЕ & 0.06 & 0110\\\hline
ЛЕЛ & 0.06 & 0111\\\hline
ЛЛЛ & 0.03 & 0100
\\ \hline \end{tabular}
\end{center}
Бит на символ при посимвольном кодировании: 1.0000, при блочном: 0.9087

\includegraphics[width=0.5\linewidth]{/home/fizlrock/data/files/backup/code_backup/hobby/algoritms/LabExecutor/app/./doc_src/images/1619697866.jpg}

\includegraphics[width=0.9\linewidth]{/home/fizlrock/data/files/backup/code_backup/hobby/algoritms/LabExecutor/app/./doc_src/images/39063839.jpg}
\pagebreak
\paragraph{Задание 2. Сжать адаптивным хаффманом\\}

Строка: 
СВИВТРИИИИ\\
Результат: 'С' 0'В' 00'И' 11 100'Т' 1100'Р' 01 01 11 0

\includegraphics[width=0.8\linewidth]{/home/fizlrock/data/files/backup/code_backup/hobby/algoritms/LabExecutor/app/./doc_src/images/2134550515.jpg}

\includegraphics[width=0.8\linewidth]{/home/fizlrock/data/files/backup/code_backup/hobby/algoritms/LabExecutor/app/./doc_src/images/505906211.jpg}

\includegraphics[width=0.8\linewidth]{/home/fizlrock/data/files/backup/code_backup/hobby/algoritms/LabExecutor/app/./doc_src/images/764698413.jpg}

\includegraphics[width=0.8\linewidth]{/home/fizlrock/data/files/backup/code_backup/hobby/algoritms/LabExecutor/app/./doc_src/images/1879963185.jpg}

\includegraphics[width=0.8\linewidth]{/home/fizlrock/data/files/backup/code_backup/hobby/algoritms/LabExecutor/app/./doc_src/images/1065615823.jpg}

\includegraphics[width=0.8\linewidth]{/home/fizlrock/data/files/backup/code_backup/hobby/algoritms/LabExecutor/app/./doc_src/images/796620886.jpg}

\includegraphics[width=0.8\linewidth]{/home/fizlrock/data/files/backup/code_backup/hobby/algoritms/LabExecutor/app/./doc_src/images/266388313.jpg}

\includegraphics[width=0.8\linewidth]{/home/fizlrock/data/files/backup/code_backup/hobby/algoritms/LabExecutor/app/./doc_src/images/819948851.jpg}

\includegraphics[width=0.8\linewidth]{/home/fizlrock/data/files/backup/code_backup/hobby/algoritms/LabExecutor/app/./doc_src/images/555953297.jpg}

\includegraphics[width=0.8\linewidth]{/home/fizlrock/data/files/backup/code_backup/hobby/algoritms/LabExecutor/app/./doc_src/images/1934027479.jpg}
\pagebreak
\paragraph{Задание 3.1}

Закодировать сообщение методом LZ77\\
Строка:ЛОДКА\_ЛОДОЧКА\_ОЧКИ\\
Результат: <0,0,Л> <0,0,О> <0,0,Д> <0,0,К> <0,0,А> <0,0,\_> <4,3,О> <0,0,Ч> <2,3,О> <5,2,И>\\
\begin{table}[h!]
\centering
\begin{tabular}{|c|c|c|c|c|c|c|c|c|c|c|c|c|c|c|c|c|} 
\hline
\multicolumn{10}{|c|}{Cловарь} & \multicolumn{6}{c|}{Буфер} & Код  \\ \hline
  &   &   &   &   &   &   &   &   &   & \cellcolor[HTML]{8CE4F6} Л & О & Д & К & А &   & <0,0,Л>
\\ \hline
  &   &   &   &   &   &   &   &   & Л & \cellcolor[HTML]{8CE4F6} О & Д & К & А &   & Л & <0,0,О>
\\ \hline
  &   &   &   &   &   &   &   & Л & О & \cellcolor[HTML]{8CE4F6} Д & К & А &   & Л & О & <0,0,Д>
\\ \hline
  &   &   &   &   &   &   & Л & О & Д & \cellcolor[HTML]{8CE4F6} К & А &   & Л & О & Д & <0,0,К>
\\ \hline
  &   &   &   &   &   & Л & О & Д & К & \cellcolor[HTML]{8CE4F6} А &   & Л & О & Д & О & <0,0,А>
\\ \hline
  &   &   &   &   & Л & О & Д & К & А & \cellcolor[HTML]{8CE4F6}   & Л & О & Д & О & Ч & <0,0,\_>
\\ \hline
  &   &   &   & \cellcolor[HTML]{FFFF00} Л & \cellcolor[HTML]{FFFF00} О & \cellcolor[HTML]{FFFF00} Д & К & А &   & \cellcolor[HTML]{FFFF00} Л & \cellcolor[HTML]{FFFF00} О & \cellcolor[HTML]{FFFF00} Д & \cellcolor[HTML]{8CE4F6} О & Ч & К & <4,3,О>
\\ \hline
Л & О & Д & К & А &   & Л & О & Д & О & \cellcolor[HTML]{8CE4F6} Ч & К & А &   & О & Ч & <0,0,Ч>
\\ \hline
О & Д & \cellcolor[HTML]{FFFF00} К & \cellcolor[HTML]{FFFF00} А & \cellcolor[HTML]{FFFF00}   & Л & О & Д & О & Ч & \cellcolor[HTML]{FFFF00} К & \cellcolor[HTML]{FFFF00} А & \cellcolor[HTML]{FFFF00}   & \cellcolor[HTML]{8CE4F6} О & Ч & К & <2,3,О>
\\ \hline
  & Л & О & Д & О & \cellcolor[HTML]{FFFF00} Ч & \cellcolor[HTML]{FFFF00} К & А &   & О & \cellcolor[HTML]{FFFF00} Ч & \cellcolor[HTML]{FFFF00} К & \cellcolor[HTML]{8CE4F6} И &   &   &   & <5,2,И>
\\ \hline
\end{tabular}
\end{table}

\paragraph{Задание 3.2}

Закодировать сообщение методом LZSS\\
Строка:ЛОДКА\_ЛОДОЧКА\_ОЧКИ\\
Результат: 0'Л' 0'О' 0'Д' 0'К' 0'А' 0'\_' 1<4,3> 1<2,1> 0'Ч' 1<2,3> 1<5,3> 0'И'\\
\begin{table}[h!]
\centering
\begin{tabular}{|c|c|c|c|c|c|c|c|c|c|c|c|c|c|c|c|c|}
\hline
\multicolumn{10}{|c|}{Cловарь} & \multicolumn{6}{c|}{Буфер} & Код  \\ \hline
  &   &   &   &   &   &   &   &   &   & Л & О & Д & К & А & \_ & 0'Л'\\ \hline
  &   &   &   &   &   &   &   &   & Л & О & Д & К & А & \_ & Л & 0'О'\\ \hline
  &   &   &   &   &   &   &   & Л & О & Д & К & А & \_ & Л & О & 0'Д'\\ \hline
  &   &   &   &   &   &   & Л & О & Д & К & А & \_ & Л & О & Д & 0'К'\\ \hline
  &   &   &   &   &   & Л & О & Д & К & А & \_ & Л & О & Д & О & 0'А'\\ \hline
  &   &   &   &   & Л & О & Д & К & А & \_ & Л & О & Д & О & Ч & 0'\_'\\ \hline
  &   &   &   & \cellcolor[HTML]{FFFF00} Л & \cellcolor[HTML]{FFFF00} О & \cellcolor[HTML]{FFFF00} Д & К & А & \_ & \cellcolor[HTML]{FFFF00} Л & \cellcolor[HTML]{FFFF00} О & \cellcolor[HTML]{FFFF00} Д & О & Ч & К & 1<4,3>\\ \hline
  & Л & \cellcolor[HTML]{FFFF00} О & Д & К & А & \_ & Л & О & Д & \cellcolor[HTML]{FFFF00} О & Ч & К & А & \_ & О & 1<2,1>\\ \hline
Л & О & Д & К & А & \_ & Л & О & Д & О & Ч & К & А & \_ & О & Ч & 0'Ч'\\ \hline
О & Д & \cellcolor[HTML]{FFFF00} К & \cellcolor[HTML]{FFFF00} А & \cellcolor[HTML]{FFFF00} \_ & Л & О & Д & О & Ч & \cellcolor[HTML]{FFFF00} К & \cellcolor[HTML]{FFFF00} А & \cellcolor[HTML]{FFFF00} \_ & О & Ч & К & 1<2,3>\\ \hline
А & \_ & Л & О & Д & \cellcolor[HTML]{FFFF00} О & \cellcolor[HTML]{FFFF00} Ч & \cellcolor[HTML]{FFFF00} К & А & \_ & \cellcolor[HTML]{FFFF00} О & \cellcolor[HTML]{FFFF00} Ч & \cellcolor[HTML]{FFFF00} К & И &   &   & 1<5,3>\\ \hline
О & Д & О & Ч & К & А & \_ & О & Ч & К & И &   &   &   &   &   & 0'И'\\ \hline
\end{tabular}
\end{table}

\paragraph{Задание 3.3}

Закодировать сообщение методом LZ78\\
Строка:ЛОДКА\_ЛОДОЧКА\_ОЧКИ\\
\begin{table}[h!]
\centering
\begin{tabular}{|c|c|c|} 
\hline
 Входная фраза (в словарь) & Код & Позиция словаря \\ \hline

 &  & 0 \\ \hline
Л & 0'Л' & 1 \\ \hline
О & 0'О' & 2 \\ \hline
Д & 0'Д' & 3 \\ \hline
К & 0'К' & 4 \\ \hline
А & 0'А' & 5 \\ \hline
\_ & 0'\_' & 6 \\ \hline
ЛО & 1'О' & 7 \\ \hline
ДО & 3'О' & 8 \\ \hline
Ч & 0'Ч' & 9 \\ \hline
КА & 4'А' & 10 \\ \hline
\_О & 6'О' & 11 \\ \hline
ЧК & 9'К' & 12 \\ \hline
И & 0'И' & 13 \\ \hline
\end{tabular}
\end{table}

Результат: 0'Л' 0'О' 0'Д' 0'К' 0'А' 0'\_' 1'О' 3'О' 0'Ч' 4'А' 6'О' 9'К' 0'И'\\
\pagebreak
\paragraph{Задание 4. Арифметическое кодирование\\}

Исходная строка: СВИВТРИИИИ\
\begin{center}
 \begin{tabular}{ |c|c| } 
  \hline
     Буква & Вероятность \\ \hline
И & 0.50\\\hline
В & 0.20\\\hline
Р & 0.10\\\hline
С & 0.10\\\hline
Т & 0.10
\\ \hline \end{tabular}
\end{center}
\begin{center}
 \begin{tabular}{ |c|c|c| } 
  \hline
     Буква & Начало & Конец \\ \hline
И & 0.00 & 0.50\\\hline
В & 0.50 & 0.70\\\hline
Р & 0.70 & 0.80\\\hline
С & 0.80 & 0.90\\\hline
Т & 0.90 & 1.00
\\ \hline \end{tabular}
\end{center}
\begin{center}
 \begin{tabular}{ |c|c|c|c| } 
  \hline
     Буква & delta & min & max \\ \hline
С & 0.1000000000 & 0.8000000000 & 0.9000000000\\\hline
В & 0.0200000000 & 0.8500000000 & 0.8700000000\\\hline
И & 0.0100000000 & 0.8500000000 & 0.8600000000\\\hline
В & 0.0020000000 & 0.8550000000 & 0.8570000000\\\hline
Т & 0.0002000000 & 0.8568000000 & 0.8570000000\\\hline
Р & 0.0000200000 & 0.8569400000 & 0.8569600000\\\hline
И & 0.0000100000 & 0.8569400000 & 0.8569500000\\\hline
И & 0.0000050000 & 0.8569400000 & 0.8569450000\\\hline
И & 0.0000025000 & 0.8569400000 & 0.8569425000\\\hline
И & 0.0000012500 & 0.8569400000 & 0.8569412500
\\ \hline \end{tabular}
\end{center}
Результат: 85694
\pagebreak
\paragraph{Задание 5.1}

\\ 

Декодировать сообщение методом адаптивного хаффмана \\
Строка: 
'C'0'X'0100'V'001100'R'10010111111\\
Результат: CXXVVRRRRV

\includegraphics[width=0.8\linewidth]{/home/fizlrock/data/files/backup/code_backup/hobby/algoritms/LabExecutor/app/./doc_src/images/600245983.jpg}

\includegraphics[width=0.8\linewidth]{/home/fizlrock/data/files/backup/code_backup/hobby/algoritms/LabExecutor/app/./doc_src/images/149206445.jpg}

\includegraphics[width=0.8\linewidth]{/home/fizlrock/data/files/backup/code_backup/hobby/algoritms/LabExecutor/app/./doc_src/images/1634044966.jpg}

\includegraphics[width=0.8\linewidth]{/home/fizlrock/data/files/backup/code_backup/hobby/algoritms/LabExecutor/app/./doc_src/images/400901445.jpg}

\includegraphics[width=0.8\linewidth]{/home/fizlrock/data/files/backup/code_backup/hobby/algoritms/LabExecutor/app/./doc_src/images/1059520222.jpg}

\includegraphics[width=0.8\linewidth]{/home/fizlrock/data/files/backup/code_backup/hobby/algoritms/LabExecutor/app/./doc_src/images/1411622993.jpg}

\includegraphics[width=0.8\linewidth]{/home/fizlrock/data/files/backup/code_backup/hobby/algoritms/LabExecutor/app/./doc_src/images/515679640.jpg}

\includegraphics[width=0.8\linewidth]{/home/fizlrock/data/files/backup/code_backup/hobby/algoritms/LabExecutor/app/./doc_src/images/167012640.jpg}

\includegraphics[width=0.8\linewidth]{/home/fizlrock/data/files/backup/code_backup/hobby/algoritms/LabExecutor/app/./doc_src/images/770881944.jpg}

\includegraphics[width=0.8\linewidth]{/home/fizlrock/data/files/backup/code_backup/hobby/algoritms/LabExecutor/app/./doc_src/images/982575679.jpg}
\pagebreak
\paragraph{Задание 5.3 Декодировать строку(LZSS)\\}

Исходная строка: [0'к'] [0'о'] [0'н'] [0' '] [1<6,3>] [0'и'] [1<5,3>] [1<3,3>] [1<4,3>] [0'н']\\
\begin{table}[h!]
\centering
\begin{tabular}{|c|c|c|}
\hline
 Cловарь & Буфер & Код  \\ \hline
0'к' & [ ,  ,  ,  ,  ,  ,  ,  ,  , к] & к
\\ \hline
0'о' & [ ,  ,  ,  ,  ,  ,  ,  , к, о] & о
\\ \hline
0'н' & [ ,  ,  ,  ,  ,  ,  , к, о, н] & н
\\ \hline
0' ' & [ ,  ,  ,  ,  ,  , к, о, н,  ] &  
\\ \hline
1<6,3> & [ ,  ,  , к, о, н,  , к, о, н] & кон
\\ \hline
0'и' & [ ,  , к, о, н,  , к, о, н, и] & и
\\ \hline
1<5,3> & [о, н,  , к, о, н, и,  , к, о] &  ко
\\ \hline
1<3,3> & [к, о, н, и,  , к, о, к, о, н] & кон
\\ \hline
1<4,3> & [и,  , к, о, к, о, н,  , к, о] &  ко
\\ \hline
0'н' & [ , к, о, к, о, н,  , к, о, н] & н
\\ \hline
\end{tabular}
\end{table}

Результат: кон кони кокон кон
\pagebreak
\paragraph{Задание 5.4 Декодировать строку(LZ78)\\}

Исходная строка: [0'н'] [0'о'] [0'с'] [2'к'] [0' '] [2'с'] [4'а'] [5'с'] [4'о'] [0'л']\\
\begin{table}[h!]
\centering
\begin{tabular}{|c|c|c|}
\hline
 Cловарь & Буфер & Код  \\ \hline
 & [] & 
\\ \hline
0'н' & [, н] & н
\\ \hline
0'о' & [, н, о] & о
\\ \hline
0'с' & [, н, о, с] & с
\\ \hline
2'к' & [, н, о, с, ок] & ок
\\ \hline
0' ' & [, н, о, с, ок,  ] &  
\\ \hline
2'с' & [, н, о, с, ок,  , ос] & ос
\\ \hline
4'а' & [, н, о, с, ок,  , ос, ока] & ока
\\ \hline
5'с' & [, н, о, с, ок,  , ос, ока,  с] &  с
\\ \hline
4'о' & [, н, о, с, ок,  , ос, ока,  с, око] & око
\\ \hline
0'л' & [, н, о, с, ок,  , ос, ока,  с, око, л] & л
\\ \hline
\end{tabular}
\end{table}

Результат: носок осока сокол
\pagebreak
\subsection{Вариант №22}
\paragraph{Задание 1. Блочный хаффман \\}

Строка КЛЛЛККККОО, размер блока: 2
\begin{center}
 \begin{tabular}{ |c|c|l| } 
  \hline
     Буква & Вероятность & Код\\ \hline
К & 0.50 & 0\\\hline
Л & 0.30 & 11\\\hline
О & 0.20 & 10
\\ \hline \end{tabular}
\end{center}
Энтропия алфавита: 1.4855
\begin{center}
 \begin{tabular}{ |c|c|l| } 
  \hline
     Блок & Вероятность & Код\\ \hline
КК & 0.25 & 01\\\hline
КЛ & 0.15 & 101\\\hline
ЛК & 0.15 & 110\\\hline
КО & 0.10 & 000\\\hline
ОК & 0.10 & 001\\\hline
ЛЛ & 0.09 & 1111\\\hline
ЛО & 0.06 & 1001\\\hline
ОЛ & 0.06 & 1110\\\hline
ОО & 0.04 & 1000
\\ \hline \end{tabular}
\end{center}
Бит на символ при посимвольном кодировании: 1.5000, при блочном: 1.5000

\includegraphics[width=0.5\linewidth]{/home/fizlrock/data/files/backup/code_backup/hobby/algoritms/LabExecutor/app/./doc_src/images/1197953146.jpg}

\includegraphics[width=0.9\linewidth]{/home/fizlrock/data/files/backup/code_backup/hobby/algoritms/LabExecutor/app/./doc_src/images/802720392.jpg}
\pagebreak
\paragraph{Задание 2. Сжать адаптивным хаффманом\\}

Строка: 
ДЕИМЕИДДДД\\
Результат: 'Д' 0'Е' 00'И' 100'М' 11 01 01 01 11 0

\includegraphics[width=0.8\linewidth]{/home/fizlrock/data/files/backup/code_backup/hobby/algoritms/LabExecutor/app/./doc_src/images/826587600.jpg}

\includegraphics[width=0.8\linewidth]{/home/fizlrock/data/files/backup/code_backup/hobby/algoritms/LabExecutor/app/./doc_src/images/1631169563.jpg}

\includegraphics[width=0.8\linewidth]{/home/fizlrock/data/files/backup/code_backup/hobby/algoritms/LabExecutor/app/./doc_src/images/1348476866.jpg}

\includegraphics[width=0.8\linewidth]{/home/fizlrock/data/files/backup/code_backup/hobby/algoritms/LabExecutor/app/./doc_src/images/1721303206.jpg}

\includegraphics[width=0.8\linewidth]{/home/fizlrock/data/files/backup/code_backup/hobby/algoritms/LabExecutor/app/./doc_src/images/1058857839.jpg}

\includegraphics[width=0.8\linewidth]{/home/fizlrock/data/files/backup/code_backup/hobby/algoritms/LabExecutor/app/./doc_src/images/1992258935.jpg}

\includegraphics[width=0.8\linewidth]{/home/fizlrock/data/files/backup/code_backup/hobby/algoritms/LabExecutor/app/./doc_src/images/917370691.jpg}

\includegraphics[width=0.8\linewidth]{/home/fizlrock/data/files/backup/code_backup/hobby/algoritms/LabExecutor/app/./doc_src/images/2065182313.jpg}

\includegraphics[width=0.8\linewidth]{/home/fizlrock/data/files/backup/code_backup/hobby/algoritms/LabExecutor/app/./doc_src/images/1667268545.jpg}

\includegraphics[width=0.8\linewidth]{/home/fizlrock/data/files/backup/code_backup/hobby/algoritms/LabExecutor/app/./doc_src/images/1359624776.jpg}
\pagebreak
\paragraph{Задание 3.1}

Закодировать сообщение методом LZ77\\
Строка:КЛУБ\_КЛУБОК\_КЛУБНИ\\
Результат: <0,0,К> <0,0,Л> <0,0,У> <0,0,Б> <0,0,\_> <5,4,О> <0,1,\_> <3,4,Н> <0,0,И>\\
\begin{table}[h!]
\centering
\begin{tabular}{|c|c|c|c|c|c|c|c|c|c|c|c|c|c|c|c|c|} 
\hline
\multicolumn{10}{|c|}{Cловарь} & \multicolumn{6}{c|}{Буфер} & Код  \\ \hline
  &   &   &   &   &   &   &   &   &   & \cellcolor[HTML]{8CE4F6} К & Л & У & Б &   & К & <0,0,К>
\\ \hline
  &   &   &   &   &   &   &   &   & К & \cellcolor[HTML]{8CE4F6} Л & У & Б &   & К & Л & <0,0,Л>
\\ \hline
  &   &   &   &   &   &   &   & К & Л & \cellcolor[HTML]{8CE4F6} У & Б &   & К & Л & У & <0,0,У>
\\ \hline
  &   &   &   &   &   &   & К & Л & У & \cellcolor[HTML]{8CE4F6} Б &   & К & Л & У & Б & <0,0,Б>
\\ \hline
  &   &   &   &   &   & К & Л & У & Б & \cellcolor[HTML]{8CE4F6}   & К & Л & У & Б & О & <0,0,\_>
\\ \hline
  &   &   &   &   & \cellcolor[HTML]{FFFF00} К & \cellcolor[HTML]{FFFF00} Л & \cellcolor[HTML]{FFFF00} У & \cellcolor[HTML]{FFFF00} Б &   & \cellcolor[HTML]{FFFF00} К & \cellcolor[HTML]{FFFF00} Л & \cellcolor[HTML]{FFFF00} У & \cellcolor[HTML]{FFFF00} Б & \cellcolor[HTML]{8CE4F6} О & К & <5,4,О>
\\ \hline
\cellcolor[HTML]{FFFF00} К & Л & У & Б &   & К & Л & У & Б & О & \cellcolor[HTML]{FFFF00} К & \cellcolor[HTML]{8CE4F6}   & К & Л & У & Б & <0,1,\_>
\\ \hline
У & Б &   & \cellcolor[HTML]{FFFF00} К & \cellcolor[HTML]{FFFF00} Л & \cellcolor[HTML]{FFFF00} У & \cellcolor[HTML]{FFFF00} Б & О & К &   & \cellcolor[HTML]{FFFF00} К & \cellcolor[HTML]{FFFF00} Л & \cellcolor[HTML]{FFFF00} У & \cellcolor[HTML]{FFFF00} Б & \cellcolor[HTML]{8CE4F6} Н & И & <3,4,Н>
\\ \hline
У & Б & О & К &   & К & Л & У & Б & Н & \cellcolor[HTML]{8CE4F6} И &   &   &   &   &   & <0,0,И>
\\ \hline
\end{tabular}
\end{table}

\paragraph{Задание 3.2}

Закодировать сообщение методом LZSS\\
Строка:КЛУБ\_КЛУБОК\_КЛУБНИ\\
Результат: 0'К' 0'Л' 0'У' 0'Б' 0'\_' 1<5,4> 0'О' 1<0,1> 1<3,5> 0'Н' 0'И'\\
\begin{table}[h!]
\centering
\begin{tabular}{|c|c|c|c|c|c|c|c|c|c|c|c|c|c|c|c|c|}
\hline
\multicolumn{10}{|c|}{Cловарь} & \multicolumn{6}{c|}{Буфер} & Код  \\ \hline
  &   &   &   &   &   &   &   &   &   & К & Л & У & Б & \_ & К & 0'К'\\ \hline
  &   &   &   &   &   &   &   &   & К & Л & У & Б & \_ & К & Л & 0'Л'\\ \hline
  &   &   &   &   &   &   &   & К & Л & У & Б & \_ & К & Л & У & 0'У'\\ \hline
  &   &   &   &   &   &   & К & Л & У & Б & \_ & К & Л & У & Б & 0'Б'\\ \hline
  &   &   &   &   &   & К & Л & У & Б & \_ & К & Л & У & Б & О & 0'\_'\\ \hline
  &   &   &   &   & \cellcolor[HTML]{FFFF00} К & \cellcolor[HTML]{FFFF00} Л & \cellcolor[HTML]{FFFF00} У & \cellcolor[HTML]{FFFF00} Б & \_ & \cellcolor[HTML]{FFFF00} К & \cellcolor[HTML]{FFFF00} Л & \cellcolor[HTML]{FFFF00} У & \cellcolor[HTML]{FFFF00} Б & О & К & 1<5,4>\\ \hline
  & К & Л & У & Б & \_ & К & Л & У & Б & О & К & \_ & К & Л & У & 0'О'\\ \hline
\cellcolor[HTML]{FFFF00} К & Л & У & Б & \_ & К & Л & У & Б & О & \cellcolor[HTML]{FFFF00} К & \_ & К & Л & У & Б & 1<0,1>\\ \hline
Л & У & Б & \cellcolor[HTML]{FFFF00} \_ & \cellcolor[HTML]{FFFF00} К & \cellcolor[HTML]{FFFF00} Л & \cellcolor[HTML]{FFFF00} У & \cellcolor[HTML]{FFFF00} Б & О & К & \cellcolor[HTML]{FFFF00} \_ & \cellcolor[HTML]{FFFF00} К & \cellcolor[HTML]{FFFF00} Л & \cellcolor[HTML]{FFFF00} У & \cellcolor[HTML]{FFFF00} Б & Н & 1<3,5>\\ \hline
Л & У & Б & О & К & \_ & К & Л & У & Б & Н & И &   &   &   &   & 0'Н'\\ \hline
У & Б & О & К & \_ & К & Л & У & Б & Н & И &   &   &   &   &   & 0'И'\\ \hline
\end{tabular}
\end{table}

\paragraph{Задание 3.3}

Закодировать сообщение методом LZ78\\
Строка:КЛУБ\_КЛУБОК\_КЛУБНИ\\
\begin{table}[h!]
\centering
\begin{tabular}{|c|c|c|} 
\hline
 Входная фраза (в словарь) & Код & Позиция словаря \\ \hline

 &  & 0 \\ \hline
К & 0'К' & 1 \\ \hline
Л & 0'Л' & 2 \\ \hline
У & 0'У' & 3 \\ \hline
Б & 0'Б' & 4 \\ \hline
\_ & 0'\_' & 5 \\ \hline
КЛ & 1'Л' & 6 \\ \hline
УБ & 3'Б' & 7 \\ \hline
О & 0'О' & 8 \\ \hline
К\_ & 1'\_' & 9 \\ \hline
КЛУ & 6'У' & 10 \\ \hline
БН & 4'Н' & 11 \\ \hline
И & 0'И' & 12 \\ \hline
\end{tabular}
\end{table}

Результат: 0'К' 0'Л' 0'У' 0'Б' 0'\_' 1'Л' 3'Б' 0'О' 1'\_' 6'У' 4'Н' 0'И'\\
\pagebreak
\paragraph{Задание 4. Арифметическое кодирование\\}

Исходная строка: ДЕИМЕИДДДД\
\begin{center}
 \begin{tabular}{ |c|c| } 
  \hline
     Буква & Вероятность \\ \hline
Д & 0.50\\\hline
Е & 0.20\\\hline
И & 0.20\\\hline
М & 0.10
\\ \hline \end{tabular}
\end{center}
\begin{center}
 \begin{tabular}{ |c|c|c| } 
  \hline
     Буква & Начало & Конец \\ \hline
Д & 0.00 & 0.50\\\hline
Е & 0.50 & 0.70\\\hline
И & 0.70 & 0.90\\\hline
М & 0.90 & 1.00
\\ \hline \end{tabular}
\end{center}
\begin{center}
 \begin{tabular}{ |c|c|c|c| } 
  \hline
     Буква & delta & min & max \\ \hline
Д & 0.5000000000 & 0.0000000000 & 0.5000000000\\\hline
Е & 0.1000000000 & 0.2500000000 & 0.3500000000\\\hline
И & 0.0200000000 & 0.3200000000 & 0.3400000000\\\hline
М & 0.0020000000 & 0.3380000000 & 0.3400000000\\\hline
Е & 0.0004000000 & 0.3390000000 & 0.3394000000\\\hline
И & 0.0000800000 & 0.3392800000 & 0.3393600000\\\hline
Д & 0.0000400000 & 0.3392800000 & 0.3393200000\\\hline
Д & 0.0000200000 & 0.3392800000 & 0.3393000000\\\hline
Д & 0.0000100000 & 0.3392800000 & 0.3392900000\\\hline
Д & 0.0000050000 & 0.3392800000 & 0.3392850000
\\ \hline \end{tabular}
\end{center}
Результат: 33928
\pagebreak
\paragraph{Задание 5.1}

\\ 

Декодировать сообщение методом адаптивного хаффмана \\
Строка: 
Ошибка декодирования\\
Результат: Ошибка декодирования
\pagebreak
\paragraph{Задание 5.3 Декодировать строку(LZSS)\\}

Исходная строка: [0'п'] [0'а'] [0'р'] [1<8,2>] [0'у'] [1<6,2>] [0' '] [1<1,2>] [1<8,2>] [0'м'] [1<4,3>] [0'р'] [1<2,1>] [0'м']\\
\begin{table}[h!]
\centering
\begin{tabular}{|c|c|c|}
\hline
 Cловарь & Буфер & Код  \\ \hline
0'п' & [ ,  ,  ,  ,  ,  ,  ,  ,  , п] & п
\\ \hline
0'а' & [ ,  ,  ,  ,  ,  ,  ,  , п, а] & а
\\ \hline
0'р' & [ ,  ,  ,  ,  ,  ,  , п, а, р] & р
\\ \hline
1<8,2> & [ ,  ,  ,  ,  , п, а, р, а, р] & ар
\\ \hline
0'у' & [ ,  ,  ,  , п, а, р, а, р, у] & у
\\ \hline
1<6,2> & [ ,  , п, а, р, а, р, у, р, а] & ра
\\ \hline
0' ' & [ , п, а, р, а, р, у, р, а,  ] &  
\\ \hline
1<1,2> & [а, р, а, р, у, р, а,  , п, а] & па
\\ \hline
1<8,2> & [а, р, у, р, а,  , п, а, п, а] & па
\\ \hline
0'м' & [р, у, р, а,  , п, а, п, а, м] & м
\\ \hline
1<4,3> & [а,  , п, а, п, а, м,  , п, а] &  па
\\ \hline
0'р' & [ , п, а, п, а, м,  , п, а, р] & р
\\ \hline
1<2,1> & [п, а, п, а, м,  , п, а, р, а] & а
\\ \hline
0'м' & [а, п, а, м,  , п, а, р, а, м] & м
\\ \hline
\end{tabular}
\end{table}

Результат: парарура папам парам
\pagebreak
\paragraph{Задание 5.4 Декодировать строку(LZ78)\\}

Исходная строка: [0'к'] [0'о'] [0'л'] [2'б'] [2'к'] [0' '] [0'б'] [5' '] [7'о'] [1'а'] [0'л']\\
\begin{table}[h!]
\centering
\begin{tabular}{|c|c|c|}
\hline
 Cловарь & Буфер & Код  \\ \hline
 & [] & 
\\ \hline
0'к' & [, к] & к
\\ \hline
0'о' & [, к, о] & о
\\ \hline
0'л' & [, к, о, л] & л
\\ \hline
2'б' & [, к, о, л, об] & об
\\ \hline
2'к' & [, к, о, л, об, ок] & ок
\\ \hline
0' ' & [, к, о, л, об, ок,  ] &  
\\ \hline
0'б' & [, к, о, л, об, ок,  , б] & б
\\ \hline
5' ' & [, к, о, л, об, ок,  , б, ок ] & ок 
\\ \hline
7'о' & [, к, о, л, об, ок,  , б, ок , бо] & бо
\\ \hline
1'а' & [, к, о, л, об, ок,  , б, ок , бо, ка] & ка
\\ \hline
0'л' & [, к, о, л, об, ок,  , б, ок , бо, ка, л] & л
\\ \hline
\end{tabular}
\end{table}

Результат: колобок бок бокал
\pagebreak
\subsection{Вариант №23}
\paragraph{Задание 1. Блочный хаффман \\}

Строка РРРООРТТТР, размер блока: 2
\begin{center}
 \begin{tabular}{ |c|c|l| } 
  \hline
     Буква & Вероятность & Код\\ \hline
Р & 0.50 & 0\\\hline
Т & 0.30 & 11\\\hline
О & 0.20 & 10
\\ \hline \end{tabular}
\end{center}
Энтропия алфавита: 1.4855
\begin{center}
 \begin{tabular}{ |c|c|l| } 
  \hline
     Блок & Вероятность & Код\\ \hline
РР & 0.25 & 01\\\hline
РТ & 0.15 & 101\\\hline
ТР & 0.15 & 110\\\hline
ОР & 0.10 & 000\\\hline
РО & 0.10 & 001\\\hline
ТТ & 0.09 & 1111\\\hline
ОТ & 0.06 & 1001\\\hline
ТО & 0.06 & 1110\\\hline
ОО & 0.04 & 1000
\\ \hline \end{tabular}
\end{center}
Бит на символ при посимвольном кодировании: 1.5000, при блочном: 1.5000

\includegraphics[width=0.5\linewidth]{/home/fizlrock/data/files/backup/code_backup/hobby/algoritms/LabExecutor/app/./doc_src/images/1157417555.jpg}

\includegraphics[width=0.9\linewidth]{/home/fizlrock/data/files/backup/code_backup/hobby/algoritms/LabExecutor/app/./doc_src/images/2106516234.jpg}
\pagebreak
\paragraph{Задание 2. Сжать адаптивным хаффманом\\}

Строка: 
НЕЕИИННЕАА\\
Результат: 'Н' 0'Е' 01 00'И' 001 101 101 10 100'А' 1001

\includegraphics[width=0.8\linewidth]{/home/fizlrock/data/files/backup/code_backup/hobby/algoritms/LabExecutor/app/./doc_src/images/2086787359.jpg}

\includegraphics[width=0.8\linewidth]{/home/fizlrock/data/files/backup/code_backup/hobby/algoritms/LabExecutor/app/./doc_src/images/817777835.jpg}

\includegraphics[width=0.8\linewidth]{/home/fizlrock/data/files/backup/code_backup/hobby/algoritms/LabExecutor/app/./doc_src/images/1045152529.jpg}

\includegraphics[width=0.8\linewidth]{/home/fizlrock/data/files/backup/code_backup/hobby/algoritms/LabExecutor/app/./doc_src/images/330360779.jpg}

\includegraphics[width=0.8\linewidth]{/home/fizlrock/data/files/backup/code_backup/hobby/algoritms/LabExecutor/app/./doc_src/images/1394320248.jpg}

\includegraphics[width=0.8\linewidth]{/home/fizlrock/data/files/backup/code_backup/hobby/algoritms/LabExecutor/app/./doc_src/images/526458181.jpg}

\includegraphics[width=0.8\linewidth]{/home/fizlrock/data/files/backup/code_backup/hobby/algoritms/LabExecutor/app/./doc_src/images/1206685499.jpg}

\includegraphics[width=0.8\linewidth]{/home/fizlrock/data/files/backup/code_backup/hobby/algoritms/LabExecutor/app/./doc_src/images/1943572151.jpg}

\includegraphics[width=0.8\linewidth]{/home/fizlrock/data/files/backup/code_backup/hobby/algoritms/LabExecutor/app/./doc_src/images/1746781506.jpg}

\includegraphics[width=0.8\linewidth]{/home/fizlrock/data/files/backup/code_backup/hobby/algoritms/LabExecutor/app/./doc_src/images/987897176.jpg}
\pagebreak
\paragraph{Задание 3.1}

Закодировать сообщение методом LZ77\\
Строка:БОЛОТО\_БОЛТ\_БОЛЬ\_ОЛЯ\\
Результат: <0,0,Б> <0,0,О> <0,0,Л> <8,1,Т> <6,1,\_> <3,3,Т> <5,4,Ь> <0,1,О> <1,1,Я>\\
\begin{table}[h!]
\centering
\begin{tabular}{|c|c|c|c|c|c|c|c|c|c|c|c|c|c|c|c|c|} 
\hline
\multicolumn{10}{|c|}{Cловарь} & \multicolumn{6}{c|}{Буфер} & Код  \\ \hline
  &   &   &   &   &   &   &   &   &   & \cellcolor[HTML]{8CE4F6} Б & О & Л & О & Т & О & <0,0,Б>
\\ \hline
  &   &   &   &   &   &   &   &   & Б & \cellcolor[HTML]{8CE4F6} О & Л & О & Т & О &   & <0,0,О>
\\ \hline
  &   &   &   &   &   &   &   & Б & О & \cellcolor[HTML]{8CE4F6} Л & О & Т & О &   & Б & <0,0,Л>
\\ \hline
  &   &   &   &   &   &   & Б & \cellcolor[HTML]{FFFF00} О & Л & \cellcolor[HTML]{FFFF00} О & \cellcolor[HTML]{8CE4F6} Т & О &   & Б & О & <8,1,Т>
\\ \hline
  &   &   &   &   & Б & \cellcolor[HTML]{FFFF00} О & Л & О & Т & \cellcolor[HTML]{FFFF00} О & \cellcolor[HTML]{8CE4F6}   & Б & О & Л & Т & <6,1,\_>
\\ \hline
  &   &   & \cellcolor[HTML]{FFFF00} Б & \cellcolor[HTML]{FFFF00} О & \cellcolor[HTML]{FFFF00} Л & О & Т & О &   & \cellcolor[HTML]{FFFF00} Б & \cellcolor[HTML]{FFFF00} О & \cellcolor[HTML]{FFFF00} Л & \cellcolor[HTML]{8CE4F6} Т &   & Б & <3,3,Т>
\\ \hline
О & Л & О & Т & О & \cellcolor[HTML]{FFFF00}   & \cellcolor[HTML]{FFFF00} Б & \cellcolor[HTML]{FFFF00} О & \cellcolor[HTML]{FFFF00} Л & Т & \cellcolor[HTML]{FFFF00}   & \cellcolor[HTML]{FFFF00} Б & \cellcolor[HTML]{FFFF00} О & \cellcolor[HTML]{FFFF00} Л & \cellcolor[HTML]{8CE4F6} Ь &   & <5,4,Ь>
\\ \hline
\cellcolor[HTML]{FFFF00}   & Б & О & Л & Т &   & Б & О & Л & Ь & \cellcolor[HTML]{FFFF00}   & \cellcolor[HTML]{8CE4F6} О & Л & Я &   &   & <0,1,О>
\\ \hline
О & \cellcolor[HTML]{FFFF00} Л & Т &   & Б & О & Л & Ь &   & О & \cellcolor[HTML]{FFFF00} Л & \cellcolor[HTML]{8CE4F6} Я &   &   &   &   & <1,1,Я>
\\ \hline
\end{tabular}
\end{table}

\paragraph{Задание 3.2}

Закодировать сообщение методом LZSS\\
Строка:БОЛОТО\_БОЛТ\_БОЛЬ\_ОЛЯ\\
Результат: 0'Б' 0'О' 0'Л' 1<8,1> 0'Т' 1<6,1> 0'\_' 1<3,3> 1<4,1> 1<5,4> 0'Ь' 1<0,1> 1<1,2> 0'Я'\\
\begin{table}[h!]
\centering
\begin{tabular}{|c|c|c|c|c|c|c|c|c|c|c|c|c|c|c|c|c|}
\hline
\multicolumn{10}{|c|}{Cловарь} & \multicolumn{6}{c|}{Буфер} & Код  \\ \hline
  &   &   &   &   &   &   &   &   &   & Б & О & Л & О & Т & О & 0'Б'\\ \hline
  &   &   &   &   &   &   &   &   & Б & О & Л & О & Т & О & \_ & 0'О'\\ \hline
  &   &   &   &   &   &   &   & Б & О & Л & О & Т & О & \_ & Б & 0'Л'\\ \hline
  &   &   &   &   &   &   & Б & \cellcolor[HTML]{FFFF00} О & Л & \cellcolor[HTML]{FFFF00} О & Т & О & \_ & Б & О & 1<8,1>\\ \hline
  &   &   &   &   &   & Б & О & Л & О & Т & О & \_ & Б & О & Л & 0'Т'\\ \hline
  &   &   &   &   & Б & \cellcolor[HTML]{FFFF00} О & Л & О & Т & \cellcolor[HTML]{FFFF00} О & \_ & Б & О & Л & Т & 1<6,1>\\ \hline
  &   &   &   & Б & О & Л & О & Т & О & \_ & Б & О & Л & Т & \_ & 0'\_'\\ \hline
  &   &   & \cellcolor[HTML]{FFFF00} Б & \cellcolor[HTML]{FFFF00} О & \cellcolor[HTML]{FFFF00} Л & О & Т & О & \_ & \cellcolor[HTML]{FFFF00} Б & \cellcolor[HTML]{FFFF00} О & \cellcolor[HTML]{FFFF00} Л & Т & \_ & Б & 1<3,3>\\ \hline
Б & О & Л & О & \cellcolor[HTML]{FFFF00} Т & О & \_ & Б & О & Л & \cellcolor[HTML]{FFFF00} Т & \_ & Б & О & Л & Ь & 1<4,1>\\ \hline
О & Л & О & Т & О & \cellcolor[HTML]{FFFF00} \_ & \cellcolor[HTML]{FFFF00} Б & \cellcolor[HTML]{FFFF00} О & \cellcolor[HTML]{FFFF00} Л & Т & \cellcolor[HTML]{FFFF00} \_ & \cellcolor[HTML]{FFFF00} Б & \cellcolor[HTML]{FFFF00} О & \cellcolor[HTML]{FFFF00} Л & Ь & \_ & 1<5,4>\\ \hline
О & \_ & Б & О & Л & Т & \_ & Б & О & Л & Ь & \_ & О & Л & Я &   & 0'Ь'\\ \hline
\cellcolor[HTML]{FFFF00} \_ & Б & О & Л & Т & \_ & Б & О & Л & Ь & \cellcolor[HTML]{FFFF00} \_ & О & Л & Я &   &   & 1<0,1>\\ \hline
Б & \cellcolor[HTML]{FFFF00} О & \cellcolor[HTML]{FFFF00} Л & Т & \_ & Б & О & Л & Ь & \_ & \cellcolor[HTML]{FFFF00} О & \cellcolor[HTML]{FFFF00} Л & Я &   &   &   & 1<1,2>\\ \hline
Л & Т & \_ & Б & О & Л & Ь & \_ & О & Л & Я &   &   &   &   &   & 0'Я'\\ \hline
\end{tabular}
\end{table}

\paragraph{Задание 3.3}

Закодировать сообщение методом LZ78\\
Строка:БОЛОТО\_БОЛТ\_БОЛЬ\_ОЛЯ\\
\begin{table}[h!]
\centering
\begin{tabular}{|c|c|c|} 
\hline
 Входная фраза (в словарь) & Код & Позиция словаря \\ \hline

 &  & 0 \\ \hline
Б & 0'Б' & 1 \\ \hline
О & 0'О' & 2 \\ \hline
Л & 0'Л' & 3 \\ \hline
ОТ & 2'Т' & 4 \\ \hline
О\_ & 2'\_' & 5 \\ \hline
БО & 1'О' & 6 \\ \hline
ЛТ & 3'Т' & 7 \\ \hline
\_ & 0'\_' & 8 \\ \hline
БОЛ & 6'Л' & 9 \\ \hline
Ь & 0'Ь' & 10 \\ \hline
\_О & 8'О' & 11 \\ \hline
ЛЯ & 3'Я' & 12 \\ \hline
\end{tabular}
\end{table}

Результат: 0'Б' 0'О' 0'Л' 2'Т' 2'\_' 1'О' 3'Т' 0'\_' 6'Л' 0'Ь' 8'О' 3'Я'\\
\pagebreak
\paragraph{Задание 4. Арифметическое кодирование\\}

Исходная строка: НЕЕИИННЕАА\
\begin{center}
 \begin{tabular}{ |c|c| } 
  \hline
     Буква & Вероятность \\ \hline
Е & 0.30\\\hline
Н & 0.30\\\hline
А & 0.20\\\hline
И & 0.20
\\ \hline \end{tabular}
\end{center}
\begin{center}
 \begin{tabular}{ |c|c|c| } 
  \hline
     Буква & Начало & Конец \\ \hline
Е & 0.00 & 0.30\\\hline
Н & 0.30 & 0.60\\\hline
А & 0.60 & 0.80\\\hline
И & 0.80 & 1.00
\\ \hline \end{tabular}
\end{center}
\begin{center}
 \begin{tabular}{ |c|c|c|c| } 
  \hline
     Буква & delta & min & max \\ \hline
Н & 0.3000000000 & 0.3000000000 & 0.6000000000\\\hline
Е & 0.0900000000 & 0.3000000000 & 0.3900000000\\\hline
Е & 0.0270000000 & 0.3000000000 & 0.3270000000\\\hline
И & 0.0054000000 & 0.3216000000 & 0.3270000000\\\hline
И & 0.0010800000 & 0.3259200000 & 0.3270000000\\\hline
Н & 0.0003240000 & 0.3262440000 & 0.3265680000\\\hline
Н & 0.0000972000 & 0.3263412000 & 0.3264384000\\\hline
Е & 0.0000291600 & 0.3263412000 & 0.3263703600\\\hline
А & 0.0000058320 & 0.3263586960 & 0.3263645280\\\hline
А & 0.0000011664 & 0.3263621952 & 0.3263633616
\\ \hline \end{tabular}
\end{center}
Результат: 326363
\pagebreak
\paragraph{Задание 5.1}

\\ 

Декодировать сообщение методом адаптивного хаффмана \\
Строка: 
'P'0'O'0100'U'0011110110111100'K'\\
Результат: POOUUUPPPK

\includegraphics[width=0.8\linewidth]{/home/fizlrock/data/files/backup/code_backup/hobby/algoritms/LabExecutor/app/./doc_src/images/384160905.jpg}

\includegraphics[width=0.8\linewidth]{/home/fizlrock/data/files/backup/code_backup/hobby/algoritms/LabExecutor/app/./doc_src/images/110281426.jpg}

\includegraphics[width=0.8\linewidth]{/home/fizlrock/data/files/backup/code_backup/hobby/algoritms/LabExecutor/app/./doc_src/images/1367299495.jpg}

\includegraphics[width=0.8\linewidth]{/home/fizlrock/data/files/backup/code_backup/hobby/algoritms/LabExecutor/app/./doc_src/images/53005536.jpg}

\includegraphics[width=0.8\linewidth]{/home/fizlrock/data/files/backup/code_backup/hobby/algoritms/LabExecutor/app/./doc_src/images/384377604.jpg}

\includegraphics[width=0.8\linewidth]{/home/fizlrock/data/files/backup/code_backup/hobby/algoritms/LabExecutor/app/./doc_src/images/1906906412.jpg}

\includegraphics[width=0.8\linewidth]{/home/fizlrock/data/files/backup/code_backup/hobby/algoritms/LabExecutor/app/./doc_src/images/834629037.jpg}

\includegraphics[width=0.8\linewidth]{/home/fizlrock/data/files/backup/code_backup/hobby/algoritms/LabExecutor/app/./doc_src/images/213411383.jpg}

\includegraphics[width=0.8\linewidth]{/home/fizlrock/data/files/backup/code_backup/hobby/algoritms/LabExecutor/app/./doc_src/images/254319543.jpg}

\includegraphics[width=0.8\linewidth]{/home/fizlrock/data/files/backup/code_backup/hobby/algoritms/LabExecutor/app/./doc_src/images/798267919.jpg}
\pagebreak
\paragraph{Задание 5.3 Декодировать строку(LZSS)\\}

Исходная строка: [0'р'] [0'е'] [0'з'] [1<8,1>] [0'д'] [0'а'] [0' '] [1<3,3>] [1<5,1>] [0'к'] [1<4,2>] [0'ю'] [1<6,1>] [1<3,2>] [0'к']\\
\begin{table}[h!]
\centering
\begin{tabular}{|c|c|c|}
\hline
 Cловарь & Буфер & Код  \\ \hline
0'р' & [ ,  ,  ,  ,  ,  ,  ,  ,  , р] & р
\\ \hline
0'е' & [ ,  ,  ,  ,  ,  ,  ,  , р, е] & е
\\ \hline
0'з' & [ ,  ,  ,  ,  ,  ,  , р, е, з] & з
\\ \hline
1<8,1> & [ ,  ,  ,  ,  ,  , р, е, з, е] & е
\\ \hline
0'д' & [ ,  ,  ,  ,  , р, е, з, е, д] & д
\\ \hline
0'а' & [ ,  ,  ,  , р, е, з, е, д, а] & а
\\ \hline
0' ' & [ ,  ,  , р, е, з, е, д, а,  ] &  
\\ \hline
1<3,3> & [р, е, з, е, д, а,  , р, е, з] & рез
\\ \hline
1<5,1> & [е, з, е, д, а,  , р, е, з, а] & а
\\ \hline
0'к' & [з, е, д, а,  , р, е, з, а, к] & к
\\ \hline
1<4,2> & [д, а,  , р, е, з, а, к,  , р] &  р
\\ \hline
0'ю' & [а,  , р, е, з, а, к,  , р, ю] & ю
\\ \hline
1<6,1> & [ , р, е, з, а, к,  , р, ю, к] & к
\\ \hline
1<3,2> & [е, з, а, к,  , р, ю, к, з, а] & за
\\ \hline
0'к' & [з, а, к,  , р, ю, к, з, а, к] & к
\\ \hline
\end{tabular}
\end{table}

Результат: резеда резак рюкзак
\pagebreak
\paragraph{Задание 5.4 Декодировать строку(LZ78)\\}

Исходная строка: [0'к'] [0'л'] [0'у'] [0'б'] [0' '] [1'л'] [3'б'] [0'о'] [1' '] [4'о'] [0'к']\\
\begin{table}[h!]
\centering
\begin{tabular}{|c|c|c|}
\hline
 Cловарь & Буфер & Код  \\ \hline
 & [] & 
\\ \hline
0'к' & [, к] & к
\\ \hline
0'л' & [, к, л] & л
\\ \hline
0'у' & [, к, л, у] & у
\\ \hline
0'б' & [, к, л, у, б] & б
\\ \hline
0' ' & [, к, л, у, б,  ] &  
\\ \hline
1'л' & [, к, л, у, б,  , кл] & кл
\\ \hline
3'б' & [, к, л, у, б,  , кл, уб] & уб
\\ \hline
0'о' & [, к, л, у, б,  , кл, уб, о] & о
\\ \hline
1' ' & [, к, л, у, б,  , кл, уб, о, к ] & к 
\\ \hline
4'о' & [, к, л, у, б,  , кл, уб, о, к , бо] & бо
\\ \hline
0'к' & [, к, л, у, б,  , кл, уб, о, к , бо, к] & к
\\ \hline
\end{tabular}
\end{table}

Результат: клуб клубок бок
\pagebreak
\subsection{Вариант №24}
\paragraph{Задание 1. Блочный хаффман \\}

Строка ККЛКЮВВВВ, размер блока: 2
\begin{center}
 \begin{tabular}{ |c|c|l| } 
  \hline
     Буква & Вероятность & Код\\ \hline
В & 0.44 & 0\\\hline
К & 0.33 & 11\\\hline
Л & 0.11 & 100\\\hline
Ю & 0.11 & 101
\\ \hline \end{tabular}
\end{center}
Энтропия алфавита: 1.7527
\begin{center}
 \begin{tabular}{ |c|c|l| } 
  \hline
     Блок & Вероятность & Код\\ \hline
ВВ & 0.20 & 00\\\hline
ВК & 0.15 & 101\\\hline
КВ & 0.15 & 110\\\hline
КК & 0.11 & 011\\\hline
ЮВ & 0.05 & 11110\\\hline
ВЛ & 0.05 & 11111\\\hline
ЛВ & 0.05 & 0100\\\hline
ВЮ & 0.05 & 0101\\\hline
КЛ & 0.04 & 10010\\\hline
ЮК & 0.04 & 10011\\\hline
КЮ & 0.04 & 11100\\\hline
ЛК & 0.04 & 11101\\\hline
ЛЛ & 0.01 & 100000\\\hline
ЮЮ & 0.01 & 100001\\\hline
ЮЛ & 0.01 & 100010\\\hline
ЛЮ & 0.01 & 100011
\\ \hline \end{tabular}
\end{center}
Бит на символ при посимвольном кодировании: 1.7778, при блочном: 1.7716

\includegraphics[width=0.5\linewidth]{/home/fizlrock/data/files/backup/code_backup/hobby/algoritms/LabExecutor/app/./doc_src/images/1581783120.jpg}

\includegraphics[width=0.9\linewidth]{/home/fizlrock/data/files/backup/code_backup/hobby/algoritms/LabExecutor/app/./doc_src/images/1302610458.jpg}
\pagebreak
\paragraph{Задание 2. Сжать адаптивным хаффманом\\}

Строка: 
ЕАКАКККРАА\\
Результат: 'Е' 0'А' 00'К' 11 101 11 0 000'Р' 01 11

\includegraphics[width=0.8\linewidth]{/home/fizlrock/data/files/backup/code_backup/hobby/algoritms/LabExecutor/app/./doc_src/images/143580766.jpg}

\includegraphics[width=0.8\linewidth]{/home/fizlrock/data/files/backup/code_backup/hobby/algoritms/LabExecutor/app/./doc_src/images/703245781.jpg}

\includegraphics[width=0.8\linewidth]{/home/fizlrock/data/files/backup/code_backup/hobby/algoritms/LabExecutor/app/./doc_src/images/909322885.jpg}

\includegraphics[width=0.8\linewidth]{/home/fizlrock/data/files/backup/code_backup/hobby/algoritms/LabExecutor/app/./doc_src/images/1464642178.jpg}

\includegraphics[width=0.8\linewidth]{/home/fizlrock/data/files/backup/code_backup/hobby/algoritms/LabExecutor/app/./doc_src/images/1975757281.jpg}

\includegraphics[width=0.8\linewidth]{/home/fizlrock/data/files/backup/code_backup/hobby/algoritms/LabExecutor/app/./doc_src/images/800621833.jpg}

\includegraphics[width=0.8\linewidth]{/home/fizlrock/data/files/backup/code_backup/hobby/algoritms/LabExecutor/app/./doc_src/images/1613202431.jpg}

\includegraphics[width=0.8\linewidth]{/home/fizlrock/data/files/backup/code_backup/hobby/algoritms/LabExecutor/app/./doc_src/images/1023947863.jpg}

\includegraphics[width=0.8\linewidth]{/home/fizlrock/data/files/backup/code_backup/hobby/algoritms/LabExecutor/app/./doc_src/images/1327331408.jpg}

\includegraphics[width=0.8\linewidth]{/home/fizlrock/data/files/backup/code_backup/hobby/algoritms/LabExecutor/app/./doc_src/images/1111829697.jpg}
\pagebreak
\paragraph{Задание 3.1}

Закодировать сообщение методом LZ77\\
Строка:ЛАПКИ\_ЛАПЫ\_ЛАПИТАЛЬ\\
Результат: <0,0,Л> <0,0,А> <0,0,П> <0,0,К> <0,0,И> <0,0,\_> <4,3,Ы> <5,4,И> <0,0,Т> <1,1,Л> <0,0,Ь>\\
\begin{table}[h!]
\centering
\begin{tabular}{|c|c|c|c|c|c|c|c|c|c|c|c|c|c|c|c|c|} 
\hline
\multicolumn{10}{|c|}{Cловарь} & \multicolumn{6}{c|}{Буфер} & Код  \\ \hline
  &   &   &   &   &   &   &   &   &   & \cellcolor[HTML]{8CE4F6} Л & А & П & К & И &   & <0,0,Л>
\\ \hline
  &   &   &   &   &   &   &   &   & Л & \cellcolor[HTML]{8CE4F6} А & П & К & И &   & Л & <0,0,А>
\\ \hline
  &   &   &   &   &   &   &   & Л & А & \cellcolor[HTML]{8CE4F6} П & К & И &   & Л & А & <0,0,П>
\\ \hline
  &   &   &   &   &   &   & Л & А & П & \cellcolor[HTML]{8CE4F6} К & И &   & Л & А & П & <0,0,К>
\\ \hline
  &   &   &   &   &   & Л & А & П & К & \cellcolor[HTML]{8CE4F6} И &   & Л & А & П & Ы & <0,0,И>
\\ \hline
  &   &   &   &   & Л & А & П & К & И & \cellcolor[HTML]{8CE4F6}   & Л & А & П & Ы &   & <0,0,\_>
\\ \hline
  &   &   &   & \cellcolor[HTML]{FFFF00} Л & \cellcolor[HTML]{FFFF00} А & \cellcolor[HTML]{FFFF00} П & К & И &   & \cellcolor[HTML]{FFFF00} Л & \cellcolor[HTML]{FFFF00} А & \cellcolor[HTML]{FFFF00} П & \cellcolor[HTML]{8CE4F6} Ы &   & Л & <4,3,Ы>
\\ \hline
Л & А & П & К & И & \cellcolor[HTML]{FFFF00}   & \cellcolor[HTML]{FFFF00} Л & \cellcolor[HTML]{FFFF00} А & \cellcolor[HTML]{FFFF00} П & Ы & \cellcolor[HTML]{FFFF00}   & \cellcolor[HTML]{FFFF00} Л & \cellcolor[HTML]{FFFF00} А & \cellcolor[HTML]{FFFF00} П & \cellcolor[HTML]{8CE4F6} И & Т & <5,4,И>
\\ \hline
  & Л & А & П & Ы &   & Л & А & П & И & \cellcolor[HTML]{8CE4F6} Т & А & Л & Ь &   &   & <0,0,Т>
\\ \hline
Л & \cellcolor[HTML]{FFFF00} А & П & Ы &   & Л & А & П & И & Т & \cellcolor[HTML]{FFFF00} А & \cellcolor[HTML]{8CE4F6} Л & Ь &   &   &   & <1,1,Л>
\\ \hline
П & Ы &   & Л & А & П & И & Т & А & Л & \cellcolor[HTML]{8CE4F6} Ь &   &   &   &   &   & <0,0,Ь>
\\ \hline
\end{tabular}
\end{table}

\paragraph{Задание 3.2}

Закодировать сообщение методом LZSS\\
Строка:ЛАПКИ\_ЛАПЫ\_ЛАПИТАЛЬ\\
Результат: 0'Л' 0'А' 0'П' 0'К' 0'И' 0'\_' 1<4,3> 0'Ы' 1<5,4> 1<0,1> 0'Т' 1<1,1> 1<4,1> 0'Ь'\\
\begin{table}[h!]
\centering
\begin{tabular}{|c|c|c|c|c|c|c|c|c|c|c|c|c|c|c|c|c|}
\hline
\multicolumn{10}{|c|}{Cловарь} & \multicolumn{6}{c|}{Буфер} & Код  \\ \hline
  &   &   &   &   &   &   &   &   &   & Л & А & П & К & И & \_ & 0'Л'\\ \hline
  &   &   &   &   &   &   &   &   & Л & А & П & К & И & \_ & Л & 0'А'\\ \hline
  &   &   &   &   &   &   &   & Л & А & П & К & И & \_ & Л & А & 0'П'\\ \hline
  &   &   &   &   &   &   & Л & А & П & К & И & \_ & Л & А & П & 0'К'\\ \hline
  &   &   &   &   &   & Л & А & П & К & И & \_ & Л & А & П & Ы & 0'И'\\ \hline
  &   &   &   &   & Л & А & П & К & И & \_ & Л & А & П & Ы & \_ & 0'\_'\\ \hline
  &   &   &   & \cellcolor[HTML]{FFFF00} Л & \cellcolor[HTML]{FFFF00} А & \cellcolor[HTML]{FFFF00} П & К & И & \_ & \cellcolor[HTML]{FFFF00} Л & \cellcolor[HTML]{FFFF00} А & \cellcolor[HTML]{FFFF00} П & Ы & \_ & Л & 1<4,3>\\ \hline
  & Л & А & П & К & И & \_ & Л & А & П & Ы & \_ & Л & А & П & И & 0'Ы'\\ \hline
Л & А & П & К & И & \cellcolor[HTML]{FFFF00} \_ & \cellcolor[HTML]{FFFF00} Л & \cellcolor[HTML]{FFFF00} А & \cellcolor[HTML]{FFFF00} П & Ы & \cellcolor[HTML]{FFFF00} \_ & \cellcolor[HTML]{FFFF00} Л & \cellcolor[HTML]{FFFF00} А & \cellcolor[HTML]{FFFF00} П & И & Т & 1<5,4>\\ \hline
\cellcolor[HTML]{FFFF00} И & \_ & Л & А & П & Ы & \_ & Л & А & П & \cellcolor[HTML]{FFFF00} И & Т & А & Л & Ь &   & 1<0,1>\\ \hline
\_ & Л & А & П & Ы & \_ & Л & А & П & И & Т & А & Л & Ь &   &   & 0'Т'\\ \hline
Л & \cellcolor[HTML]{FFFF00} А & П & Ы & \_ & Л & А & П & И & Т & \cellcolor[HTML]{FFFF00} А & Л & Ь &   &   &   & 1<1,1>\\ \hline
А & П & Ы & \_ & \cellcolor[HTML]{FFFF00} Л & А & П & И & Т & А & \cellcolor[HTML]{FFFF00} Л & Ь &   &   &   &   & 1<4,1>\\ \hline
П & Ы & \_ & Л & А & П & И & Т & А & Л & Ь &   &   &   &   &   & 0'Ь'\\ \hline
\end{tabular}
\end{table}

\paragraph{Задание 3.3}

Закодировать сообщение методом LZ78\\
Строка:ЛАПКИ\_ЛАПЫ\_ЛАПИТАЛЬ\\
\begin{table}[h!]
\centering
\begin{tabular}{|c|c|c|} 
\hline
 Входная фраза (в словарь) & Код & Позиция словаря \\ \hline

 &  & 0 \\ \hline
Л & 0'Л' & 1 \\ \hline
А & 0'А' & 2 \\ \hline
П & 0'П' & 3 \\ \hline
К & 0'К' & 4 \\ \hline
И & 0'И' & 5 \\ \hline
\_ & 0'\_' & 6 \\ \hline
ЛА & 1'А' & 7 \\ \hline
ПЫ & 3'Ы' & 8 \\ \hline
\_Л & 6'Л' & 9 \\ \hline
АП & 2'П' & 10 \\ \hline
ИТ & 5'Т' & 11 \\ \hline
АЛ & 2'Л' & 12 \\ \hline
Ь & 0'Ь' & 13 \\ \hline
\end{tabular}
\end{table}

Результат: 0'Л' 0'А' 0'П' 0'К' 0'И' 0'\_' 1'А' 3'Ы' 6'Л' 2'П' 5'Т' 2'Л' 0'Ь'\\
\pagebreak
\paragraph{Задание 4. Арифметическое кодирование\\}

Исходная строка: ЕАКАКККРАА\
\begin{center}
 \begin{tabular}{ |c|c| } 
  \hline
     Буква & Вероятность \\ \hline
А & 0.40\\\hline
К & 0.40\\\hline
Р & 0.10\\\hline
Е & 0.10
\\ \hline \end{tabular}
\end{center}
\begin{center}
 \begin{tabular}{ |c|c|c| } 
  \hline
     Буква & Начало & Конец \\ \hline
А & 0.00 & 0.40\\\hline
К & 0.40 & 0.80\\\hline
Р & 0.80 & 0.90\\\hline
Е & 0.90 & 1.00
\\ \hline \end{tabular}
\end{center}
\begin{center}
 \begin{tabular}{ |c|c|c|c| } 
  \hline
     Буква & delta & min & max \\ \hline
Е & 0.1000000000 & 0.9000000000 & 1.0000000000\\\hline
А & 0.0400000000 & 0.9000000000 & 0.9400000000\\\hline
К & 0.0160000000 & 0.9160000000 & 0.9320000000\\\hline
А & 0.0064000000 & 0.9160000000 & 0.9224000000\\\hline
К & 0.0025600000 & 0.9185600000 & 0.9211200000\\\hline
К & 0.0010240000 & 0.9195840000 & 0.9206080000\\\hline
К & 0.0004096000 & 0.9199936000 & 0.9204032000\\\hline
Р & 0.0000409600 & 0.9203212800 & 0.9203622400\\\hline
А & 0.0000163840 & 0.9203212800 & 0.9203376640\\\hline
А & 0.0000065536 & 0.9203212800 & 0.9203278336
\\ \hline \end{tabular}
\end{center}
Результат: 920322
\pagebreak
\paragraph{Задание 5.1}

\\ 

Декодировать сообщение методом адаптивного хаффмана \\
Строка: 
'A'0'B'00'C'10110111100'S'100111110\\
Результат: ABCCAACAASABAAAA

\includegraphics[width=0.8\linewidth]{/home/fizlrock/data/files/backup/code_backup/hobby/algoritms/LabExecutor/app/./doc_src/images/1055210688.jpg}

\includegraphics[width=0.8\linewidth]{/home/fizlrock/data/files/backup/code_backup/hobby/algoritms/LabExecutor/app/./doc_src/images/995581085.jpg}

\includegraphics[width=0.8\linewidth]{/home/fizlrock/data/files/backup/code_backup/hobby/algoritms/LabExecutor/app/./doc_src/images/1290626829.jpg}

\includegraphics[width=0.8\linewidth]{/home/fizlrock/data/files/backup/code_backup/hobby/algoritms/LabExecutor/app/./doc_src/images/1283976687.jpg}

\includegraphics[width=0.8\linewidth]{/home/fizlrock/data/files/backup/code_backup/hobby/algoritms/LabExecutor/app/./doc_src/images/746906642.jpg}

\includegraphics[width=0.8\linewidth]{/home/fizlrock/data/files/backup/code_backup/hobby/algoritms/LabExecutor/app/./doc_src/images/958322237.jpg}

\includegraphics[width=0.8\linewidth]{/home/fizlrock/data/files/backup/code_backup/hobby/algoritms/LabExecutor/app/./doc_src/images/1290063648.jpg}

\includegraphics[width=0.8\linewidth]{/home/fizlrock/data/files/backup/code_backup/hobby/algoritms/LabExecutor/app/./doc_src/images/129251199.jpg}

\includegraphics[width=0.8\linewidth]{/home/fizlrock/data/files/backup/code_backup/hobby/algoritms/LabExecutor/app/./doc_src/images/830197880.jpg}

\includegraphics[width=0.8\linewidth]{/home/fizlrock/data/files/backup/code_backup/hobby/algoritms/LabExecutor/app/./doc_src/images/2087320348.jpg}

\includegraphics[width=0.8\linewidth]{/home/fizlrock/data/files/backup/code_backup/hobby/algoritms/LabExecutor/app/./doc_src/images/790742337.jpg}

\includegraphics[width=0.8\linewidth]{/home/fizlrock/data/files/backup/code_backup/hobby/algoritms/LabExecutor/app/./doc_src/images/1090728732.jpg}

\includegraphics[width=0.8\linewidth]{/home/fizlrock/data/files/backup/code_backup/hobby/algoritms/LabExecutor/app/./doc_src/images/1326098297.jpg}

\includegraphics[width=0.8\linewidth]{/home/fizlrock/data/files/backup/code_backup/hobby/algoritms/LabExecutor/app/./doc_src/images/2062175539.jpg}

\includegraphics[width=0.8\linewidth]{/home/fizlrock/data/files/backup/code_backup/hobby/algoritms/LabExecutor/app/./doc_src/images/883816539.jpg}

\includegraphics[width=0.8\linewidth]{/home/fizlrock/data/files/backup/code_backup/hobby/algoritms/LabExecutor/app/./doc_src/images/772986852.jpg}
\pagebreak
\paragraph{Задание 5.3 Декодировать строку(LZSS)\\}

Исходная строка: [0'з'] [0'и'] [0'г'] [1<7,1>] [0'а'] [1<7,1>] [0' '] [1<6,3>] [1<4,1>] [0'р'] [1<4,1>] [1<6,3>] [0'е'] [0'м'] [1<4,4>] [0'ь']\\
\begin{table}[h!]
\centering
\begin{tabular}{|c|c|c|}
\hline
 Cловарь & Буфер & Код  \\ \hline
0'з' & [ ,  ,  ,  ,  ,  ,  ,  ,  , з] & з
\\ \hline
0'и' & [ ,  ,  ,  ,  ,  ,  ,  , з, и] & и
\\ \hline
0'г' & [ ,  ,  ,  ,  ,  ,  , з, и, г] & г
\\ \hline
1<7,1> & [ ,  ,  ,  ,  ,  , з, и, г, з] & з
\\ \hline
0'а' & [ ,  ,  ,  ,  , з, и, г, з, а] & а
\\ \hline
1<7,1> & [ ,  ,  ,  , з, и, г, з, а, г] & г
\\ \hline
0' ' & [ ,  ,  , з, и, г, з, а, г,  ] &  
\\ \hline
1<6,3> & [з, и, г, з, а, г,  , з, а, г] & заг
\\ \hline
1<4,1> & [и, г, з, а, г,  , з, а, г, а] & а
\\ \hline
0'р' & [г, з, а, г,  , з, а, г, а, р] & р
\\ \hline
1<4,1> & [з, а, г,  , з, а, г, а, р,  ] &  
\\ \hline
1<6,3> & [ , з, а, г, а, р,  , г, а, р] & гар
\\ \hline
0'е' & [з, а, г, а, р,  , г, а, р, е] & е
\\ \hline
0'м' & [а, г, а, р,  , г, а, р, е, м] & м
\\ \hline
1<4,4> & [ , г, а, р, е, м,  , г, а, р] &  гар
\\ \hline
0'ь' & [г, а, р, е, м,  , г, а, р, ь] & ь
\\ \hline
\end{tabular}
\end{table}

Результат: зигзаг загар гарем гарь
\pagebreak
\paragraph{Задание 5.4 Декодировать строку(LZ78)\\}

Исходная строка: [0'к'] [0'и'] [1'и'] [0'м'] [0'о'] [0'р'] [0'а'] [0' '] [4'о'] [6' '] [9'р'] [0'ж']\\
\begin{table}[h!]
\centering
\begin{tabular}{|c|c|c|}
\hline
 Cловарь & Буфер & Код  \\ \hline
 & [] & 
\\ \hline
0'к' & [, к] & к
\\ \hline
0'и' & [, к, и] & и
\\ \hline
1'и' & [, к, и, ки] & ки
\\ \hline
0'м' & [, к, и, ки, м] & м
\\ \hline
0'о' & [, к, и, ки, м, о] & о
\\ \hline
0'р' & [, к, и, ки, м, о, р] & р
\\ \hline
0'а' & [, к, и, ки, м, о, р, а] & а
\\ \hline
0' ' & [, к, и, ки, м, о, р, а,  ] &  
\\ \hline
4'о' & [, к, и, ки, м, о, р, а,  , мо] & мо
\\ \hline
6' ' & [, к, и, ки, м, о, р, а,  , мо, р ] & р 
\\ \hline
9'р' & [, к, и, ки, м, о, р, а,  , мо, р , мор] & мор
\\ \hline
0'ж' & [, к, и, ки, м, о, р, а,  , мо, р , мор, ж] & ж
\\ \hline
\end{tabular}
\end{table}

Результат: кикимора мор морж
\pagebreak
\subsection{Вариант №25}
\paragraph{Задание 1. Блочный хаффман \\}

Строка ЛЛИМЛЛЛМИИ, размер блока: 2
\begin{center}
 \begin{tabular}{ |c|c|l| } 
  \hline
     Буква & Вероятность & Код\\ \hline
Л & 0.50 & 0\\\hline
И & 0.30 & 11\\\hline
М & 0.20 & 10
\\ \hline \end{tabular}
\end{center}
Энтропия алфавита: 1.4855
\begin{center}
 \begin{tabular}{ |c|c|l| } 
  \hline
     Блок & Вероятность & Код\\ \hline
ЛЛ & 0.25 & 01\\\hline
ИЛ & 0.15 & 101\\\hline
ЛИ & 0.15 & 110\\\hline
ЛМ & 0.10 & 000\\\hline
МЛ & 0.10 & 001\\\hline
ИИ & 0.09 & 1111\\\hline
ИМ & 0.06 & 1001\\\hline
МИ & 0.06 & 1110\\\hline
ММ & 0.04 & 1000
\\ \hline \end{tabular}
\end{center}
Бит на символ при посимвольном кодировании: 1.5000, при блочном: 1.5000

\includegraphics[width=0.5\linewidth]{/home/fizlrock/data/files/backup/code_backup/hobby/algoritms/LabExecutor/app/./doc_src/images/2083124447.jpg}

\includegraphics[width=0.9\linewidth]{/home/fizlrock/data/files/backup/code_backup/hobby/algoritms/LabExecutor/app/./doc_src/images/1961546401.jpg}
\pagebreak
\paragraph{Задание 2. Сжать адаптивным хаффманом\\}

Строка: 
ГОРОНПОРРР\\
Результат: 'Г' 0'О' 00'Р' 11 100'Н' 1100'П' 11 111 111 10

\includegraphics[width=0.8\linewidth]{/home/fizlrock/data/files/backup/code_backup/hobby/algoritms/LabExecutor/app/./doc_src/images/1298699304.jpg}

\includegraphics[width=0.8\linewidth]{/home/fizlrock/data/files/backup/code_backup/hobby/algoritms/LabExecutor/app/./doc_src/images/595125866.jpg}

\includegraphics[width=0.8\linewidth]{/home/fizlrock/data/files/backup/code_backup/hobby/algoritms/LabExecutor/app/./doc_src/images/1057913288.jpg}

\includegraphics[width=0.8\linewidth]{/home/fizlrock/data/files/backup/code_backup/hobby/algoritms/LabExecutor/app/./doc_src/images/141131472.jpg}

\includegraphics[width=0.8\linewidth]{/home/fizlrock/data/files/backup/code_backup/hobby/algoritms/LabExecutor/app/./doc_src/images/1565867815.jpg}

\includegraphics[width=0.8\linewidth]{/home/fizlrock/data/files/backup/code_backup/hobby/algoritms/LabExecutor/app/./doc_src/images/657792267.jpg}

\includegraphics[width=0.8\linewidth]{/home/fizlrock/data/files/backup/code_backup/hobby/algoritms/LabExecutor/app/./doc_src/images/233835529.jpg}

\includegraphics[width=0.8\linewidth]{/home/fizlrock/data/files/backup/code_backup/hobby/algoritms/LabExecutor/app/./doc_src/images/29515849.jpg}

\includegraphics[width=0.8\linewidth]{/home/fizlrock/data/files/backup/code_backup/hobby/algoritms/LabExecutor/app/./doc_src/images/869553761.jpg}

\includegraphics[width=0.8\linewidth]{/home/fizlrock/data/files/backup/code_backup/hobby/algoritms/LabExecutor/app/./doc_src/images/284570614.jpg}
\pagebreak

\paragraph{Задание 3.2}

Закодировать сообщение методом LZSS\\
Строка:КУКУРУЗА\_УРЮК\_КРЮК\\
Результат: 0'К' 0'У' 1<8,2> 0'Р' 1<6,1> 0'З' 0'А' 0'\_' 1<4,2> 0'Ю' 1<0,1> 1<5,1> 1<8,1> 1<5,3>\\
\begin{table}[h!]
\centering
\begin{tabular}{|c|c|c|c|c|c|c|c|c|c|c|c|c|c|c|c|c|}
\hline
\multicolumn{10}{|c|}{Cловарь} & \multicolumn{6}{c|}{Буфер} & Код  \\ \hline
  &   &   &   &   &   &   &   &   &   & К & У & К & У & Р & У & 0'К'\\ \hline
  &   &   &   &   &   &   &   &   & К & У & К & У & Р & У & З & 0'У'\\ \hline
  &   &   &   &   &   &   &   & \cellcolor[HTML]{FFFF00} К & \cellcolor[HTML]{FFFF00} У & \cellcolor[HTML]{FFFF00} К & \cellcolor[HTML]{FFFF00} У & Р & У & З & А & 1<8,2>\\ \hline
  &   &   &   &   &   & К & У & К & У & Р & У & З & А & \_ & У & 0'Р'\\ \hline
  &   &   &   &   & К & \cellcolor[HTML]{FFFF00} У & К & У & Р & \cellcolor[HTML]{FFFF00} У & З & А & \_ & У & Р & 1<6,1>\\ \hline
  &   &   &   & К & У & К & У & Р & У & З & А & \_ & У & Р & Ю & 0'З'\\ \hline
  &   &   & К & У & К & У & Р & У & З & А & \_ & У & Р & Ю & К & 0'А'\\ \hline
  &   & К & У & К & У & Р & У & З & А & \_ & У & Р & Ю & К & \_ & 0'\_'\\ \hline
  & К & У & К & \cellcolor[HTML]{FFFF00} У & \cellcolor[HTML]{FFFF00} Р & У & З & А & \_ & \cellcolor[HTML]{FFFF00} У & \cellcolor[HTML]{FFFF00} Р & Ю & К & \_ & К & 1<4,2>\\ \hline
У & К & У & Р & У & З & А & \_ & У & Р & Ю & К & \_ & К & Р & Ю & 0'Ю'\\ \hline
\cellcolor[HTML]{FFFF00} К & У & Р & У & З & А & \_ & У & Р & Ю & \cellcolor[HTML]{FFFF00} К & \_ & К & Р & Ю & К & 1<0,1>\\ \hline
У & Р & У & З & А & \cellcolor[HTML]{FFFF00} \_ & У & Р & Ю & К & \cellcolor[HTML]{FFFF00} \_ & К & Р & Ю & К &   & 1<5,1>\\ \hline
Р & У & З & А & \_ & У & Р & Ю & \cellcolor[HTML]{FFFF00} К & \_ & \cellcolor[HTML]{FFFF00} К & Р & Ю & К &   &   & 1<8,1>\\ \hline
У & З & А & \_ & У & \cellcolor[HTML]{FFFF00} Р & \cellcolor[HTML]{FFFF00} Ю & \cellcolor[HTML]{FFFF00} К & \_ & К & \cellcolor[HTML]{FFFF00} Р & \cellcolor[HTML]{FFFF00} Ю & \cellcolor[HTML]{FFFF00} К &   &   &   & 1<5,3>\\ \hline
\end{tabular}
\end{table}

\paragraph{Задание 3.3}

Закодировать сообщение методом LZ78\\
Строка:КУКУРУЗА\_УРЮК\_КРЮК\\
\begin{table}[h!]
\centering
\begin{tabular}{|c|c|c|} 
\hline
 Входная фраза (в словарь) & Код & Позиция словаря \\ \hline

 &  & 0 \\ \hline
К & 0'К' & 1 \\ \hline
У & 0'У' & 2 \\ \hline
КУ & 1'У' & 3 \\ \hline
Р & 0'Р' & 4 \\ \hline
УЗ & 2'З' & 5 \\ \hline
А & 0'А' & 6 \\ \hline
\_ & 0'\_' & 7 \\ \hline
УР & 2'Р' & 8 \\ \hline
Ю & 0'Ю' & 9 \\ \hline
К\_ & 1'\_' & 10 \\ \hline
КР & 1'Р' & 11 \\ \hline
ЮК & 9'К' & 12 \\ \hline
\end{tabular}
\end{table}

Результат: 0'К' 0'У' 1'У' 0'Р' 2'З' 0'А' 0'\_' 2'Р' 0'Ю' 1'\_' 1'Р' 9'К'\\
\pagebreak
\paragraph{Задание 4. Арифметическое кодирование\\}

Исходная строка: ГОРОНПОРРР\
\begin{center}
 \begin{tabular}{ |c|c| } 
  \hline
     Буква & Вероятность \\ \hline
Р & 0.40\\\hline
О & 0.30\\\hline
Г & 0.10\\\hline
Н & 0.10\\\hline
П & 0.10
\\ \hline \end{tabular}
\end{center}
\begin{center}
 \begin{tabular}{ |c|c|c| } 
  \hline
     Буква & Начало & Конец \\ \hline
Р & 0.00 & 0.40\\\hline
О & 0.40 & 0.70\\\hline
Г & 0.70 & 0.80\\\hline
Н & 0.80 & 0.90\\\hline
П & 0.90 & 1.00
\\ \hline \end{tabular}
\end{center}
\begin{center}
 \begin{tabular}{ |c|c|c|c| } 
  \hline
     Буква & delta & min & max \\ \hline
Г & 0.1000000000 & 0.7000000000 & 0.8000000000\\\hline
О & 0.0300000000 & 0.7400000000 & 0.7700000000\\\hline
Р & 0.0120000000 & 0.7400000000 & 0.7520000000\\\hline
О & 0.0036000000 & 0.7448000000 & 0.7484000000\\\hline
Н & 0.0003600000 & 0.7476800000 & 0.7480400000\\\hline
П & 0.0000360000 & 0.7480040000 & 0.7480400000\\\hline
О & 0.0000108000 & 0.7480184000 & 0.7480292000\\\hline
Р & 0.0000043200 & 0.7480184000 & 0.7480227200\\\hline
Р & 0.0000017280 & 0.7480184000 & 0.7480201280\\\hline
Р & 0.0000006912 & 0.7480184000 & 0.7480190912
\\ \hline \end{tabular}
\end{center}
Результат: 748019
\pagebreak
\paragraph{Задание 5.1}

\\ 

Декодировать сообщение методом адаптивного хаффмана \\
Строка: 
'K'0'L'0100'M'000'N'110110110111110\\
Результат: KLLMNNNLNLL

\includegraphics[width=0.8\linewidth]{/home/fizlrock/data/files/backup/code_backup/hobby/algoritms/LabExecutor/app/./doc_src/images/546729767.jpg}

\includegraphics[width=0.8\linewidth]{/home/fizlrock/data/files/backup/code_backup/hobby/algoritms/LabExecutor/app/./doc_src/images/133686993.jpg}

\includegraphics[width=0.8\linewidth]{/home/fizlrock/data/files/backup/code_backup/hobby/algoritms/LabExecutor/app/./doc_src/images/1065290246.jpg}

\includegraphics[width=0.8\linewidth]{/home/fizlrock/data/files/backup/code_backup/hobby/algoritms/LabExecutor/app/./doc_src/images/1340664788.jpg}

\includegraphics[width=0.8\linewidth]{/home/fizlrock/data/files/backup/code_backup/hobby/algoritms/LabExecutor/app/./doc_src/images/678080060.jpg}

\includegraphics[width=0.8\linewidth]{/home/fizlrock/data/files/backup/code_backup/hobby/algoritms/LabExecutor/app/./doc_src/images/2123089758.jpg}

\includegraphics[width=0.8\linewidth]{/home/fizlrock/data/files/backup/code_backup/hobby/algoritms/LabExecutor/app/./doc_src/images/996576884.jpg}

\includegraphics[width=0.8\linewidth]{/home/fizlrock/data/files/backup/code_backup/hobby/algoritms/LabExecutor/app/./doc_src/images/2009800537.jpg}

\includegraphics[width=0.8\linewidth]{/home/fizlrock/data/files/backup/code_backup/hobby/algoritms/LabExecutor/app/./doc_src/images/1878422077.jpg}

\includegraphics[width=0.8\linewidth]{/home/fizlrock/data/files/backup/code_backup/hobby/algoritms/LabExecutor/app/./doc_src/images/131786304.jpg}

\includegraphics[width=0.8\linewidth]{/home/fizlrock/data/files/backup/code_backup/hobby/algoritms/LabExecutor/app/./doc_src/images/1571972943.jpg}
\pagebreak
\paragraph{Задание 5.3 Декодировать строку(LZSS)\\}

Исходная строка: [0'п'] [0'а'] [0'р'] [0' '] [1<8,1>] [1<6,1>] [0'м'] [1<4,1>] [1<5,1>] [0'о'] [1<6,2>] [1<0,2>] [0'к'] [1<4,3>] [0'р']\\
\begin{table}[h!]
\centering
\begin{tabular}{|c|c|c|}
\hline
 Cловарь & Буфер & Код  \\ \hline
0'п' & [ ,  ,  ,  ,  ,  ,  ,  ,  , п] & п
\\ \hline
0'а' & [ ,  ,  ,  ,  ,  ,  ,  , п, а] & а
\\ \hline
0'р' & [ ,  ,  ,  ,  ,  ,  , п, а, р] & р
\\ \hline
0' ' & [ ,  ,  ,  ,  ,  , п, а, р,  ] &  
\\ \hline
1<8,1> & [ ,  ,  ,  ,  , п, а, р,  , р] & р
\\ \hline
1<6,1> & [ ,  ,  ,  , п, а, р,  , р, а] & а
\\ \hline
0'м' & [ ,  ,  , п, а, р,  , р, а, м] & м
\\ \hline
1<4,1> & [ ,  , п, а, р,  , р, а, м, а] & а
\\ \hline
1<5,1> & [ , п, а, р,  , р, а, м, а,  ] &  
\\ \hline
0'о' & [п, а, р,  , р, а, м, а,  , о] & о
\\ \hline
1<6,2> & [р,  , р, а, м, а,  , о, м, а] & ма
\\ \hline
1<0,2> & [р, а, м, а,  , о, м, а, р,  ] & р 
\\ \hline
0'к' & [а, м, а,  , о, м, а, р,  , к] & к
\\ \hline
1<4,3> & [ , о, м, а, р,  , к, о, м, а] & ома
\\ \hline
0'р' & [о, м, а, р,  , к, о, м, а, р] & р
\\ \hline
\end{tabular}
\end{table}

Результат: пар рама омар комар
\pagebreak
\paragraph{Задание 5.4 Декодировать строку(LZ78)\\}

Исходная строка: [0'м'] [0'а'] [1'а'] [0' '] [0'р'] [2'м'] [2' '] [5'а'] [0'к'] [0'и']\\
\begin{table}[h!]
\centering
\begin{tabular}{|c|c|c|}
\hline
 Cловарь & Буфер & Код  \\ \hline
 & [] & 
\\ \hline
0'м' & [, м] & м
\\ \hline
0'а' & [, м, а] & а
\\ \hline
1'а' & [, м, а, ма] & ма
\\ \hline
0' ' & [, м, а, ма,  ] &  
\\ \hline
0'р' & [, м, а, ма,  , р] & р
\\ \hline
2'м' & [, м, а, ма,  , р, ам] & ам
\\ \hline
2' ' & [, м, а, ма,  , р, ам, а ] & а 
\\ \hline
5'а' & [, м, а, ма,  , р, ам, а , ра] & ра
\\ \hline
0'к' & [, м, а, ма,  , р, ам, а , ра, к] & к
\\ \hline
0'и' & [, м, а, ма,  , р, ам, а , ра, к, и] & и
\\ \hline
\end{tabular}
\end{table}

Результат: мама рама раки
\pagebreak
\subsection{Вариант №26}
\paragraph{Задание 1. Блочный хаффман \\}

Строка БРББРРРБББ, размер блока: 3
\begin{center}
 \begin{tabular}{ |c|c|l| } 
  \hline
     Буква & Вероятность & Код\\ \hline
Б & 0.60 & 1\\\hline
Р & 0.40 & 0
\\ \hline \end{tabular}
\end{center}
Энтропия алфавита: 0.9710
\begin{center}
 \begin{tabular}{ |c|c|l| } 
  \hline
     Блок & Вероятность & Код\\ \hline
БББ & 0.22 & 01\\\hline
БРБ & 0.14 & 100\\\hline
РББ & 0.14 & 101\\\hline
ББР & 0.14 & 110\\\hline
РРБ & 0.10 & 001\\\hline
РБР & 0.10 & 1111\\\hline
БРР & 0.10 & 000\\\hline
РРР & 0.06 & 1110
\\ \hline \end{tabular}
\end{center}
Бит на символ при посимвольном кодировании: 1.0000, при блочном: 0.9813

\includegraphics[width=0.5\linewidth]{/home/fizlrock/data/files/backup/code_backup/hobby/algoritms/LabExecutor/app/./doc_src/images/1946632554.jpg}

\includegraphics[width=0.9\linewidth]{/home/fizlrock/data/files/backup/code_backup/hobby/algoritms/LabExecutor/app/./doc_src/images/691386610.jpg}
\pagebreak
\paragraph{Задание 2. Сжать адаптивным хаффманом\\}

Строка: 
ВУАКУВАМММ\\
Результат: 'В' 0'У' 00'А' 100'К' 11 10 01 000'М' 0001 111

\includegraphics[width=0.8\linewidth]{/home/fizlrock/data/files/backup/code_backup/hobby/algoritms/LabExecutor/app/./doc_src/images/454187861.jpg}

\includegraphics[width=0.8\linewidth]{/home/fizlrock/data/files/backup/code_backup/hobby/algoritms/LabExecutor/app/./doc_src/images/1060480247.jpg}

\includegraphics[width=0.8\linewidth]{/home/fizlrock/data/files/backup/code_backup/hobby/algoritms/LabExecutor/app/./doc_src/images/529044629.jpg}

\includegraphics[width=0.8\linewidth]{/home/fizlrock/data/files/backup/code_backup/hobby/algoritms/LabExecutor/app/./doc_src/images/1095110123.jpg}

\includegraphics[width=0.8\linewidth]{/home/fizlrock/data/files/backup/code_backup/hobby/algoritms/LabExecutor/app/./doc_src/images/2043689020.jpg}

\includegraphics[width=0.8\linewidth]{/home/fizlrock/data/files/backup/code_backup/hobby/algoritms/LabExecutor/app/./doc_src/images/1658648830.jpg}

\includegraphics[width=0.8\linewidth]{/home/fizlrock/data/files/backup/code_backup/hobby/algoritms/LabExecutor/app/./doc_src/images/180973889.jpg}

\includegraphics[width=0.8\linewidth]{/home/fizlrock/data/files/backup/code_backup/hobby/algoritms/LabExecutor/app/./doc_src/images/2017623495.jpg}

\includegraphics[width=0.8\linewidth]{/home/fizlrock/data/files/backup/code_backup/hobby/algoritms/LabExecutor/app/./doc_src/images/1414075406.jpg}

\includegraphics[width=0.8\linewidth]{/home/fizlrock/data/files/backup/code_backup/hobby/algoritms/LabExecutor/app/./doc_src/images/548637007.jpg}
\pagebreak
\paragraph{Задание 3.1}

Закодировать сообщение методом LZ77\\
Строка:ДОДО\_ДОМ\_ДОМИК\_МИГ\\
Результат: <0,0,Д> <0,0,О> <8,2,\_> <5,2,М> <6,4,И> <0,0,К> <0,1,М> <6,1,Г>\\
\begin{table}[h!]
\centering
\begin{tabular}{|c|c|c|c|c|c|c|c|c|c|c|c|c|c|c|c|c|} 
\hline
\multicolumn{10}{|c|}{Cловарь} & \multicolumn{6}{c|}{Буфер} & Код  \\ \hline
  &   &   &   &   &   &   &   &   &   & \cellcolor[HTML]{8CE4F6} Д & О & Д & О &   & Д & <0,0,Д>
\\ \hline
  &   &   &   &   &   &   &   &   & Д & \cellcolor[HTML]{8CE4F6} О & Д & О &   & Д & О & <0,0,О>
\\ \hline
  &   &   &   &   &   &   &   & \cellcolor[HTML]{FFFF00} Д & \cellcolor[HTML]{FFFF00} О & \cellcolor[HTML]{FFFF00} Д & \cellcolor[HTML]{FFFF00} О & \cellcolor[HTML]{8CE4F6}   & Д & О & М & <8,2,\_>
\\ \hline
  &   &   &   &   & \cellcolor[HTML]{FFFF00} Д & \cellcolor[HTML]{FFFF00} О & Д & О &   & \cellcolor[HTML]{FFFF00} Д & \cellcolor[HTML]{FFFF00} О & \cellcolor[HTML]{8CE4F6} М &   & Д & О & <5,2,М>
\\ \hline
  &   & Д & О & Д & О & \cellcolor[HTML]{FFFF00}   & \cellcolor[HTML]{FFFF00} Д & \cellcolor[HTML]{FFFF00} О & \cellcolor[HTML]{FFFF00} М & \cellcolor[HTML]{FFFF00}   & \cellcolor[HTML]{FFFF00} Д & \cellcolor[HTML]{FFFF00} О & \cellcolor[HTML]{FFFF00} М & \cellcolor[HTML]{8CE4F6} И & К & <6,4,И>
\\ \hline
О &   & Д & О & М &   & Д & О & М & И & \cellcolor[HTML]{8CE4F6} К &   & М & И & Г &   & <0,0,К>
\\ \hline
\cellcolor[HTML]{FFFF00}   & Д & О & М &   & Д & О & М & И & К & \cellcolor[HTML]{FFFF00}   & \cellcolor[HTML]{8CE4F6} М & И & Г &   &   & <0,1,М>
\\ \hline
О & М &   & Д & О & М & \cellcolor[HTML]{FFFF00} И & К &   & М & \cellcolor[HTML]{FFFF00} И & \cellcolor[HTML]{8CE4F6} Г &   &   &   &   & <6,1,Г>
\\ \hline
\end{tabular}
\end{table}

\paragraph{Задание 3.2}

Закодировать сообщение методом LZSS\\
Строка:ДОДО\_ДОМ\_ДОМИК\_МИГ\\
Результат: 0'Д' 0'О' 1<8,2> 0'\_' 1<5,2> 0'М' 1<6,4> 0'И' 0'К' 1<0,1> 1<6,2> 0'Г'\\
\begin{table}[h!]
\centering
\begin{tabular}{|c|c|c|c|c|c|c|c|c|c|c|c|c|c|c|c|c|}
\hline
\multicolumn{10}{|c|}{Cловарь} & \multicolumn{6}{c|}{Буфер} & Код  \\ \hline
  &   &   &   &   &   &   &   &   &   & Д & О & Д & О & \_ & Д & 0'Д'\\ \hline
  &   &   &   &   &   &   &   &   & Д & О & Д & О & \_ & Д & О & 0'О'\\ \hline
  &   &   &   &   &   &   &   & \cellcolor[HTML]{FFFF00} Д & \cellcolor[HTML]{FFFF00} О & \cellcolor[HTML]{FFFF00} Д & \cellcolor[HTML]{FFFF00} О & \_ & Д & О & М & 1<8,2>\\ \hline
  &   &   &   &   &   & Д & О & Д & О & \_ & Д & О & М & \_ & Д & 0'\_'\\ \hline
  &   &   &   &   & \cellcolor[HTML]{FFFF00} Д & \cellcolor[HTML]{FFFF00} О & Д & О & \_ & \cellcolor[HTML]{FFFF00} Д & \cellcolor[HTML]{FFFF00} О & М & \_ & Д & О & 1<5,2>\\ \hline
  &   &   & Д & О & Д & О & \_ & Д & О & М & \_ & Д & О & М & И & 0'М'\\ \hline
  &   & Д & О & Д & О & \cellcolor[HTML]{FFFF00} \_ & \cellcolor[HTML]{FFFF00} Д & \cellcolor[HTML]{FFFF00} О & \cellcolor[HTML]{FFFF00} М & \cellcolor[HTML]{FFFF00} \_ & \cellcolor[HTML]{FFFF00} Д & \cellcolor[HTML]{FFFF00} О & \cellcolor[HTML]{FFFF00} М & И & К & 1<6,4>\\ \hline
Д & О & \_ & Д & О & М & \_ & Д & О & М & И & К & \_ & М & И & Г & 0'И'\\ \hline
О & \_ & Д & О & М & \_ & Д & О & М & И & К & \_ & М & И & Г &   & 0'К'\\ \hline
\cellcolor[HTML]{FFFF00} \_ & Д & О & М & \_ & Д & О & М & И & К & \cellcolor[HTML]{FFFF00} \_ & М & И & Г &   &   & 1<0,1>\\ \hline
Д & О & М & \_ & Д & О & \cellcolor[HTML]{FFFF00} М & \cellcolor[HTML]{FFFF00} И & К & \_ & \cellcolor[HTML]{FFFF00} М & \cellcolor[HTML]{FFFF00} И & Г &   &   &   & 1<6,2>\\ \hline
М & \_ & Д & О & М & И & К & \_ & М & И & Г &   &   &   &   &   & 0'Г'\\ \hline
\end{tabular}
\end{table}

\paragraph{Задание 3.3}

Закодировать сообщение методом LZ78\\
Строка:ДОДО\_ДОМ\_ДОМИК\_МИГ\\
\begin{table}[h!]
\centering
\begin{tabular}{|c|c|c|} 
\hline
 Входная фраза (в словарь) & Код & Позиция словаря \\ \hline

 &  & 0 \\ \hline
Д & 0'Д' & 1 \\ \hline
О & 0'О' & 2 \\ \hline
ДО & 1'О' & 3 \\ \hline
\_ & 0'\_' & 4 \\ \hline
ДОМ & 3'М' & 5 \\ \hline
\_Д & 4'Д' & 6 \\ \hline
ОМ & 2'М' & 7 \\ \hline
И & 0'И' & 8 \\ \hline
К & 0'К' & 9 \\ \hline
\_М & 4'М' & 10 \\ \hline
ИГ & 8'Г' & 11 \\ \hline
\end{tabular}
\end{table}

Результат: 0'Д' 0'О' 1'О' 0'\_' 3'М' 4'Д' 2'М' 0'И' 0'К' 4'М' 8'Г'\\
\pagebreak
\paragraph{Задание 4. Арифметическое кодирование\\}

Исходная строка: ВУАКУВАМММ\
\begin{center}
 \begin{tabular}{ |c|c| } 
  \hline
     Буква & Вероятность \\ \hline
М & 0.30\\\hline
А & 0.20\\\hline
В & 0.20\\\hline
У & 0.20\\\hline
К & 0.10
\\ \hline \end{tabular}
\end{center}
\begin{center}
 \begin{tabular}{ |c|c|c| } 
  \hline
     Буква & Начало & Конец \\ \hline
М & 0.00 & 0.30\\\hline
А & 0.30 & 0.50\\\hline
В & 0.50 & 0.70\\\hline
У & 0.70 & 0.90\\\hline
К & 0.90 & 1.00
\\ \hline \end{tabular}
\end{center}
\begin{center}
 \begin{tabular}{ |c|c|c|c| } 
  \hline
     Буква & delta & min & max \\ \hline
В & 0.2000000000 & 0.5000000000 & 0.7000000000\\\hline
У & 0.0400000000 & 0.6400000000 & 0.6800000000\\\hline
А & 0.0080000000 & 0.6520000000 & 0.6600000000\\\hline
К & 0.0008000000 & 0.6592000000 & 0.6600000000\\\hline
У & 0.0001600000 & 0.6597600000 & 0.6599200000\\\hline
В & 0.0000320000 & 0.6598400000 & 0.6598720000\\\hline
А & 0.0000064000 & 0.6598496000 & 0.6598560000\\\hline
М & 0.0000019200 & 0.6598496000 & 0.6598515200\\\hline
М & 0.0000005760 & 0.6598496000 & 0.6598501760\\\hline
М & 0.0000001728 & 0.6598496000 & 0.6598497728
\\ \hline \end{tabular}
\end{center}
Результат: 6598496
\pagebreak
\paragraph{Задание 5.1}

\\ 

Декодировать сообщение методом адаптивного хаффмана \\
Строка: 
'H'0'B'00'V'100'N'0011111011111101001\\
Результат: HBVNNNBVVBH

\includegraphics[width=0.8\linewidth]{/home/fizlrock/data/files/backup/code_backup/hobby/algoritms/LabExecutor/app/./doc_src/images/1259472577.jpg}

\includegraphics[width=0.8\linewidth]{/home/fizlrock/data/files/backup/code_backup/hobby/algoritms/LabExecutor/app/./doc_src/images/1058398147.jpg}

\includegraphics[width=0.8\linewidth]{/home/fizlrock/data/files/backup/code_backup/hobby/algoritms/LabExecutor/app/./doc_src/images/1477522673.jpg}

\includegraphics[width=0.8\linewidth]{/home/fizlrock/data/files/backup/code_backup/hobby/algoritms/LabExecutor/app/./doc_src/images/283172012.jpg}

\includegraphics[width=0.8\linewidth]{/home/fizlrock/data/files/backup/code_backup/hobby/algoritms/LabExecutor/app/./doc_src/images/379396656.jpg}

\includegraphics[width=0.8\linewidth]{/home/fizlrock/data/files/backup/code_backup/hobby/algoritms/LabExecutor/app/./doc_src/images/244114781.jpg}

\includegraphics[width=0.8\linewidth]{/home/fizlrock/data/files/backup/code_backup/hobby/algoritms/LabExecutor/app/./doc_src/images/1905526555.jpg}

\includegraphics[width=0.8\linewidth]{/home/fizlrock/data/files/backup/code_backup/hobby/algoritms/LabExecutor/app/./doc_src/images/1928786915.jpg}

\includegraphics[width=0.8\linewidth]{/home/fizlrock/data/files/backup/code_backup/hobby/algoritms/LabExecutor/app/./doc_src/images/64899184.jpg}

\includegraphics[width=0.8\linewidth]{/home/fizlrock/data/files/backup/code_backup/hobby/algoritms/LabExecutor/app/./doc_src/images/1529426415.jpg}

\includegraphics[width=0.8\linewidth]{/home/fizlrock/data/files/backup/code_backup/hobby/algoritms/LabExecutor/app/./doc_src/images/264544969.jpg}
\pagebreak
\paragraph{Задание 5.3 Декодировать строку(LZSS)\\}

Исходная строка: [0'р'] [0'а'] [0'м'] [0'п'] [1<7,1>] [0' '] [1<5,3>] [0'и'] [1<0,1>] [1<4,1>] [1<6,4>] [1<1,1>] [1<2,1>] [0'р']\\
\begin{table}[h!]
\centering
\begin{tabular}{|c|c|c|}
\hline
 Cловарь & Буфер & Код  \\ \hline
0'р' & [ ,  ,  ,  ,  ,  ,  ,  ,  , р] & р
\\ \hline
0'а' & [ ,  ,  ,  ,  ,  ,  ,  , р, а] & а
\\ \hline
0'м' & [ ,  ,  ,  ,  ,  ,  , р, а, м] & м
\\ \hline
0'п' & [ ,  ,  ,  ,  ,  , р, а, м, п] & п
\\ \hline
1<7,1> & [ ,  ,  ,  ,  , р, а, м, п, а] & а
\\ \hline
0' ' & [ ,  ,  ,  , р, а, м, п, а,  ] &  
\\ \hline
1<5,3> & [ , р, а, м, п, а,  , а, м, п] & амп
\\ \hline
0'и' & [р, а, м, п, а,  , а, м, п, и] & и
\\ \hline
1<0,1> & [а, м, п, а,  , а, м, п, и, р] & р
\\ \hline
1<4,1> & [м, п, а,  , а, м, п, и, р,  ] &  
\\ \hline
1<6,4> & [а, м, п, и, р,  , п, и, р,  ] & пир 
\\ \hline
1<1,1> & [м, п, и, р,  , п, и, р,  , м] & м
\\ \hline
1<2,1> & [п, и, р,  , п, и, р,  , м, и] & и
\\ \hline
0'р' & [и, р,  , п, и, р,  , м, и, р] & р
\\ \hline
\end{tabular}
\end{table}

Результат: рампа ампир пир мир
\pagebreak
\paragraph{Задание 5.4 Декодировать строку(LZ78)\\}

Исходная строка: [0'л'] [0'о'] [0'г'] [2'в'] [2' '] [0'в'] [2'л'] [0' '] [6'о'] [1'к']\\
\begin{table}[h!]
\centering
\begin{tabular}{|c|c|c|}
\hline
 Cловарь & Буфер & Код  \\ \hline
 & [] & 
\\ \hline
0'л' & [, л] & л
\\ \hline
0'о' & [, л, о] & о
\\ \hline
0'г' & [, л, о, г] & г
\\ \hline
2'в' & [, л, о, г, ов] & ов
\\ \hline
2' ' & [, л, о, г, ов, о ] & о 
\\ \hline
0'в' & [, л, о, г, ов, о , в] & в
\\ \hline
2'л' & [, л, о, г, ов, о , в, ол] & ол
\\ \hline
0' ' & [, л, о, г, ов, о , в, ол,  ] &  
\\ \hline
6'о' & [, л, о, г, ов, о , в, ол,  , во] & во
\\ \hline
1'к' & [, л, о, г, ов, о , в, ол,  , во, лк] & лк
\\ \hline
\end{tabular}
\end{table}

Результат: логово вол волк
\pagebreak
\subsection{Вариант №27}
\paragraph{Задание 1. Блочный хаффман \\}

Строка КВКККВВВВВ, размер блока: 3
\begin{center}
 \begin{tabular}{ |c|c|l| } 
  \hline
     Буква & Вероятность & Код\\ \hline
В & 0.60 & 1\\\hline
К & 0.40 & 0
\\ \hline \end{tabular}
\end{center}
Энтропия алфавита: 0.9710
\begin{center}
 \begin{tabular}{ |c|c|l| } 
  \hline
     Блок & Вероятность & Код\\ \hline
ВВВ & 0.22 & 01\\\hline
ВВК & 0.14 & 100\\\hline
ВКВ & 0.14 & 101\\\hline
КВВ & 0.14 & 110\\\hline
ККВ & 0.10 & 001\\\hline
ВКК & 0.10 & 1111\\\hline
КВК & 0.10 & 000\\\hline
ККК & 0.06 & 1110
\\ \hline \end{tabular}
\end{center}
Бит на символ при посимвольном кодировании: 1.0000, при блочном: 0.9813

\includegraphics[width=0.5\linewidth]{/home/fizlrock/data/files/backup/code_backup/hobby/algoritms/LabExecutor/app/./doc_src/images/641826973.jpg}

\includegraphics[width=0.9\linewidth]{/home/fizlrock/data/files/backup/code_backup/hobby/algoritms/LabExecutor/app/./doc_src/images/523945564.jpg}
\pagebreak
\paragraph{Задание 2. Сжать адаптивным хаффманом\\}

Строка: 
УЧЧРККЧУУУ\\
Результат: 'У' 0'Ч' 01 00'Р' 000'К' 1101 11 110 111 10

\includegraphics[width=0.8\linewidth]{/home/fizlrock/data/files/backup/code_backup/hobby/algoritms/LabExecutor/app/./doc_src/images/1408856815.jpg}

\includegraphics[width=0.8\linewidth]{/home/fizlrock/data/files/backup/code_backup/hobby/algoritms/LabExecutor/app/./doc_src/images/591124980.jpg}

\includegraphics[width=0.8\linewidth]{/home/fizlrock/data/files/backup/code_backup/hobby/algoritms/LabExecutor/app/./doc_src/images/869653288.jpg}

\includegraphics[width=0.8\linewidth]{/home/fizlrock/data/files/backup/code_backup/hobby/algoritms/LabExecutor/app/./doc_src/images/1775290298.jpg}

\includegraphics[width=0.8\linewidth]{/home/fizlrock/data/files/backup/code_backup/hobby/algoritms/LabExecutor/app/./doc_src/images/1549206389.jpg}

\includegraphics[width=0.8\linewidth]{/home/fizlrock/data/files/backup/code_backup/hobby/algoritms/LabExecutor/app/./doc_src/images/782373433.jpg}

\includegraphics[width=0.8\linewidth]{/home/fizlrock/data/files/backup/code_backup/hobby/algoritms/LabExecutor/app/./doc_src/images/46985396.jpg}

\includegraphics[width=0.8\linewidth]{/home/fizlrock/data/files/backup/code_backup/hobby/algoritms/LabExecutor/app/./doc_src/images/244738979.jpg}

\includegraphics[width=0.8\linewidth]{/home/fizlrock/data/files/backup/code_backup/hobby/algoritms/LabExecutor/app/./doc_src/images/1455379266.jpg}

\includegraphics[width=0.8\linewidth]{/home/fizlrock/data/files/backup/code_backup/hobby/algoritms/LabExecutor/app/./doc_src/images/1908337609.jpg}
\pagebreak
\paragraph{Задание 3.1}

Закодировать сообщение методом LZ77\\
Строка:ЗИГЗАГ\_ЗАЗОР\_ЗОРКИЙ\\
Результат: <0,0,З> <0,0,И> <0,0,Г> <7,1,А> <7,1,\_> <6,2,З> <0,0,О> <0,0,Р> <4,2,О> <6,1,К> <0,0,И> <0,0,Й>\\
\begin{table}[h!]
\centering
\begin{tabular}{|c|c|c|c|c|c|c|c|c|c|c|c|c|c|c|c|c|} 
\hline
\multicolumn{10}{|c|}{Cловарь} & \multicolumn{6}{c|}{Буфер} & Код  \\ \hline
  &   &   &   &   &   &   &   &   &   & \cellcolor[HTML]{8CE4F6} З & И & Г & З & А & Г & <0,0,З>
\\ \hline
  &   &   &   &   &   &   &   &   & З & \cellcolor[HTML]{8CE4F6} И & Г & З & А & Г &   & <0,0,И>
\\ \hline
  &   &   &   &   &   &   &   & З & И & \cellcolor[HTML]{8CE4F6} Г & З & А & Г &   & З & <0,0,Г>
\\ \hline
  &   &   &   &   &   &   & \cellcolor[HTML]{FFFF00} З & И & Г & \cellcolor[HTML]{FFFF00} З & \cellcolor[HTML]{8CE4F6} А & Г &   & З & А & <7,1,А>
\\ \hline
  &   &   &   &   & З & И & \cellcolor[HTML]{FFFF00} Г & З & А & \cellcolor[HTML]{FFFF00} Г & \cellcolor[HTML]{8CE4F6}   & З & А & З & О & <7,1,\_>
\\ \hline
  &   &   & З & И & Г & \cellcolor[HTML]{FFFF00} З & \cellcolor[HTML]{FFFF00} А & Г &   & \cellcolor[HTML]{FFFF00} З & \cellcolor[HTML]{FFFF00} А & \cellcolor[HTML]{8CE4F6} З & О & Р &   & <6,2,З>
\\ \hline
З & И & Г & З & А & Г &   & З & А & З & \cellcolor[HTML]{8CE4F6} О & Р &   & З & О & Р & <0,0,О>
\\ \hline
И & Г & З & А & Г &   & З & А & З & О & \cellcolor[HTML]{8CE4F6} Р &   & З & О & Р & К & <0,0,Р>
\\ \hline
Г & З & А & Г & \cellcolor[HTML]{FFFF00}   & \cellcolor[HTML]{FFFF00} З & А & З & О & Р & \cellcolor[HTML]{FFFF00}   & \cellcolor[HTML]{FFFF00} З & \cellcolor[HTML]{8CE4F6} О & Р & К & И & <4,2,О>
\\ \hline
Г &   & З & А & З & О & \cellcolor[HTML]{FFFF00} Р &   & З & О & \cellcolor[HTML]{FFFF00} Р & \cellcolor[HTML]{8CE4F6} К & И & Й &   &   & <6,1,К>
\\ \hline
З & А & З & О & Р &   & З & О & Р & К & \cellcolor[HTML]{8CE4F6} И & Й &   &   &   &   & <0,0,И>
\\ \hline
А & З & О & Р &   & З & О & Р & К & И & \cellcolor[HTML]{8CE4F6} Й &   &   &   &   &   & <0,0,Й>
\\ \hline
\end{tabular}
\end{table}

\paragraph{Задание 3.2}

Закодировать сообщение методом LZSS\\
Строка:ЗИГЗАГ\_ЗАЗОР\_ЗОРКИЙ\\
Результат: 0'З' 0'И' 0'Г' 1<7,1> 0'А' 1<7,1> 0'\_' 1<6,2> 1<1,1> 0'О' 0'Р' 1<4,2> 1<6,2> 0'К' 0'И' 0'Й'\\
\begin{table}[h!]
\centering
\begin{tabular}{|c|c|c|c|c|c|c|c|c|c|c|c|c|c|c|c|c|}
\hline
\multicolumn{10}{|c|}{Cловарь} & \multicolumn{6}{c|}{Буфер} & Код  \\ \hline
  &   &   &   &   &   &   &   &   &   & З & И & Г & З & А & Г & 0'З'\\ \hline
  &   &   &   &   &   &   &   &   & З & И & Г & З & А & Г & \_ & 0'И'\\ \hline
  &   &   &   &   &   &   &   & З & И & Г & З & А & Г & \_ & З & 0'Г'\\ \hline
  &   &   &   &   &   &   & \cellcolor[HTML]{FFFF00} З & И & Г & \cellcolor[HTML]{FFFF00} З & А & Г & \_ & З & А & 1<7,1>\\ \hline
  &   &   &   &   &   & З & И & Г & З & А & Г & \_ & З & А & З & 0'А'\\ \hline
  &   &   &   &   & З & И & \cellcolor[HTML]{FFFF00} Г & З & А & \cellcolor[HTML]{FFFF00} Г & \_ & З & А & З & О & 1<7,1>\\ \hline
  &   &   &   & З & И & Г & З & А & Г & \_ & З & А & З & О & Р & 0'\_'\\ \hline
  &   &   & З & И & Г & \cellcolor[HTML]{FFFF00} З & \cellcolor[HTML]{FFFF00} А & Г & \_ & \cellcolor[HTML]{FFFF00} З & \cellcolor[HTML]{FFFF00} А & З & О & Р & \_ & 1<6,2>\\ \hline
  & \cellcolor[HTML]{FFFF00} З & И & Г & З & А & Г & \_ & З & А & \cellcolor[HTML]{FFFF00} З & О & Р & \_ & З & О & 1<1,1>\\ \hline
З & И & Г & З & А & Г & \_ & З & А & З & О & Р & \_ & З & О & Р & 0'О'\\ \hline
И & Г & З & А & Г & \_ & З & А & З & О & Р & \_ & З & О & Р & К & 0'Р'\\ \hline
Г & З & А & Г & \cellcolor[HTML]{FFFF00} \_ & \cellcolor[HTML]{FFFF00} З & А & З & О & Р & \cellcolor[HTML]{FFFF00} \_ & \cellcolor[HTML]{FFFF00} З & О & Р & К & И & 1<4,2>\\ \hline
А & Г & \_ & З & А & З & \cellcolor[HTML]{FFFF00} О & \cellcolor[HTML]{FFFF00} Р & \_ & З & \cellcolor[HTML]{FFFF00} О & \cellcolor[HTML]{FFFF00} Р & К & И & Й &   & 1<6,2>\\ \hline
\_ & З & А & З & О & Р & \_ & З & О & Р & К & И & Й &   &   &   & 0'К'\\ \hline
З & А & З & О & Р & \_ & З & О & Р & К & И & Й &   &   &   &   & 0'И'\\ \hline
А & З & О & Р & \_ & З & О & Р & К & И & Й &   &   &   &   &   & 0'Й'\\ \hline
\end{tabular}
\end{table}

\paragraph{Задание 3.3}

Закодировать сообщение методом LZ78\\
Строка:ЗИГЗАГ\_ЗАЗОР\_ЗОРКИЙ\\
\begin{table}[h!]
\centering
\begin{tabular}{|c|c|c|} 
\hline
 Входная фраза (в словарь) & Код & Позиция словаря \\ \hline

 &  & 0 \\ \hline
З & 0'З' & 1 \\ \hline
И & 0'И' & 2 \\ \hline
Г & 0'Г' & 3 \\ \hline
ЗА & 1'А' & 4 \\ \hline
Г\_ & 3'\_' & 5 \\ \hline
ЗАЗ & 4'З' & 6 \\ \hline
О & 0'О' & 7 \\ \hline
Р & 0'Р' & 8 \\ \hline
\_ & 0'\_' & 9 \\ \hline
ЗО & 1'О' & 10 \\ \hline
РК & 8'К' & 11 \\ \hline
ИЙ & 2'Й' & 12 \\ \hline
\end{tabular}
\end{table}

Результат: 0'З' 0'И' 0'Г' 1'А' 3'\_' 4'З' 0'О' 0'Р' 0'\_' 1'О' 8'К' 2'Й'\\
\pagebreak
\paragraph{Задание 4. Арифметическое кодирование\\}

Исходная строка: УЧЧРККЧУУУ\
\begin{center}
 \begin{tabular}{ |c|c| } 
  \hline
     Буква & Вероятность \\ \hline
У & 0.40\\\hline
Ч & 0.30\\\hline
К & 0.20\\\hline
Р & 0.10
\\ \hline \end{tabular}
\end{center}
\begin{center}
 \begin{tabular}{ |c|c|c| } 
  \hline
     Буква & Начало & Конец \\ \hline
У & 0.00 & 0.40\\\hline
Ч & 0.40 & 0.70\\\hline
К & 0.70 & 0.90\\\hline
Р & 0.90 & 1.00
\\ \hline \end{tabular}
\end{center}
\begin{center}
 \begin{tabular}{ |c|c|c|c| } 
  \hline
     Буква & delta & min & max \\ \hline
У & 0.4000000000 & 0.0000000000 & 0.4000000000\\\hline
Ч & 0.1200000000 & 0.1600000000 & 0.2800000000\\\hline
Ч & 0.0360000000 & 0.2080000000 & 0.2440000000\\\hline
Р & 0.0036000000 & 0.2404000000 & 0.2440000000\\\hline
К & 0.0007200000 & 0.2429200000 & 0.2436400000\\\hline
К & 0.0001440000 & 0.2434240000 & 0.2435680000\\\hline
Ч & 0.0000432000 & 0.2434816000 & 0.2435248000\\\hline
У & 0.0000172800 & 0.2434816000 & 0.2434988800\\\hline
У & 0.0000069120 & 0.2434816000 & 0.2434885120\\\hline
У & 0.0000027648 & 0.2434816000 & 0.2434843648
\\ \hline \end{tabular}
\end{center}
Результат: 243482
\pagebreak
\paragraph{Задание 5.1}

\\ 

Декодировать сообщение методом адаптивного хаффмана \\
Строка: 
'D'0'F'00'C'100'S'010011010011100'H'01\\
Результат: DFCSCSSDDHC

\includegraphics[width=0.8\linewidth]{/home/fizlrock/data/files/backup/code_backup/hobby/algoritms/LabExecutor/app/./doc_src/images/1230033231.jpg}

\includegraphics[width=0.8\linewidth]{/home/fizlrock/data/files/backup/code_backup/hobby/algoritms/LabExecutor/app/./doc_src/images/648344847.jpg}

\includegraphics[width=0.8\linewidth]{/home/fizlrock/data/files/backup/code_backup/hobby/algoritms/LabExecutor/app/./doc_src/images/867747754.jpg}

\includegraphics[width=0.8\linewidth]{/home/fizlrock/data/files/backup/code_backup/hobby/algoritms/LabExecutor/app/./doc_src/images/360019855.jpg}

\includegraphics[width=0.8\linewidth]{/home/fizlrock/data/files/backup/code_backup/hobby/algoritms/LabExecutor/app/./doc_src/images/217762062.jpg}

\includegraphics[width=0.8\linewidth]{/home/fizlrock/data/files/backup/code_backup/hobby/algoritms/LabExecutor/app/./doc_src/images/1896904423.jpg}

\includegraphics[width=0.8\linewidth]{/home/fizlrock/data/files/backup/code_backup/hobby/algoritms/LabExecutor/app/./doc_src/images/1504874327.jpg}

\includegraphics[width=0.8\linewidth]{/home/fizlrock/data/files/backup/code_backup/hobby/algoritms/LabExecutor/app/./doc_src/images/1776823933.jpg}

\includegraphics[width=0.8\linewidth]{/home/fizlrock/data/files/backup/code_backup/hobby/algoritms/LabExecutor/app/./doc_src/images/1103606886.jpg}

\includegraphics[width=0.8\linewidth]{/home/fizlrock/data/files/backup/code_backup/hobby/algoritms/LabExecutor/app/./doc_src/images/1302672141.jpg}

\includegraphics[width=0.8\linewidth]{/home/fizlrock/data/files/backup/code_backup/hobby/algoritms/LabExecutor/app/./doc_src/images/125355831.jpg}
\pagebreak
\paragraph{Задание 5.3 Декодировать строку(LZSS)\\}

Исходная строка: [0'м'] [0'у'] [0'р'] [1<7,3>] [0' '] [1<3,3>] [0'к'] [0'а'] [1<4,4>] [0'з'] [0'и'] [0'к']\\
\begin{table}[h!]
\centering
\begin{tabular}{|c|c|c|}
\hline
 Cловарь & Буфер & Код  \\ \hline
0'м' & [ ,  ,  ,  ,  ,  ,  ,  ,  , м] & м
\\ \hline
0'у' & [ ,  ,  ,  ,  ,  ,  ,  , м, у] & у
\\ \hline
0'р' & [ ,  ,  ,  ,  ,  ,  , м, у, р] & р
\\ \hline
1<7,3> & [ ,  ,  ,  , м, у, р, м, у, р] & мур
\\ \hline
0' ' & [ ,  ,  , м, у, р, м, у, р,  ] &  
\\ \hline
1<3,3> & [м, у, р, м, у, р,  , м, у, р] & мур
\\ \hline
0'к' & [у, р, м, у, р,  , м, у, р, к] & к
\\ \hline
0'а' & [р, м, у, р,  , м, у, р, к, а] & а
\\ \hline
1<4,4> & [ , м, у, р, к, а,  , м, у, р] &  мур
\\ \hline
0'з' & [м, у, р, к, а,  , м, у, р, з] & з
\\ \hline
0'и' & [у, р, к, а,  , м, у, р, з, и] & и
\\ \hline
0'к' & [р, к, а,  , м, у, р, з, и, к] & к
\\ \hline
\end{tabular}
\end{table}

Результат: мурмур мурка мурзик
\pagebreak
\paragraph{Задание 5.4 Декодировать строку(LZ78)\\}

Исходная строка: [0'т'] [0'и'] [0'н'] [0'а'] [0' '] [1'и'] [0'к'] [5'н'] [2'т'] [2' '] [3'и'] [1'к'] [0'и']\\
\begin{table}[h!]
\centering
\begin{tabular}{|c|c|c|}
\hline
 Cловарь & Буфер & Код  \\ \hline
 & [] & 
\\ \hline
0'т' & [, т] & т
\\ \hline
0'и' & [, т, и] & и
\\ \hline
0'н' & [, т, и, н] & н
\\ \hline
0'а' & [, т, и, н, а] & а
\\ \hline
0' ' & [, т, и, н, а,  ] &  
\\ \hline
1'и' & [, т, и, н, а,  , ти] & ти
\\ \hline
0'к' & [, т, и, н, а,  , ти, к] & к
\\ \hline
5'н' & [, т, и, н, а,  , ти, к,  н] &  н
\\ \hline
2'т' & [, т, и, н, а,  , ти, к,  н, ит] & ит
\\ \hline
2' ' & [, т, и, н, а,  , ти, к,  н, ит, и ] & и 
\\ \hline
3'и' & [, т, и, н, а,  , ти, к,  н, ит, и , ни] & ни
\\ \hline
1'к' & [, т, и, н, а,  , ти, к,  н, ит, и , ни, тк] & тк
\\ \hline
0'и' & [, т, и, н, а,  , ти, к,  н, ит, и , ни, тк, и] & и
\\ \hline
\end{tabular}
\end{table}

Результат: тина тик нити нитки
\pagebreak
\subsection{Вариант №28}
\paragraph{Задание 1. Блочный хаффман \\}

Строка УККУУККККК, размер блока: 3
\begin{center}
 \begin{tabular}{ |c|c|l| } 
  \hline
     Буква & Вероятность & Код\\ \hline
К & 0.70 & 1\\\hline
У & 0.30 & 0
\\ \hline \end{tabular}
\end{center}
Энтропия алфавита: 0.8813
\begin{center}
 \begin{tabular}{ |c|c|l| } 
  \hline
     Блок & Вероятность & Код\\ \hline
ККК & 0.34 & 11\\\hline
УКК & 0.15 & 101\\\hline
КУК & 0.15 & 00\\\hline
ККУ & 0.15 & 100\\\hline
КУУ & 0.06 & 0101\\\hline
УУК & 0.06 & 0110\\\hline
УКУ & 0.06 & 0111\\\hline
УУУ & 0.03 & 0100
\\ \hline \end{tabular}
\end{center}
Бит на символ при посимвольном кодировании: 1.0000, при блочном: 0.9087

\includegraphics[width=0.5\linewidth]{/home/fizlrock/data/files/backup/code_backup/hobby/algoritms/LabExecutor/app/./doc_src/images/679104190.jpg}

\includegraphics[width=0.9\linewidth]{/home/fizlrock/data/files/backup/code_backup/hobby/algoritms/LabExecutor/app/./doc_src/images/1856737049.jpg}
\pagebreak
\paragraph{Задание 2. Сжать адаптивным хаффманом\\}

Строка: 
КЛЮЧЧИИИИК\\
Результат: 'К' 0'Л' 00'Ю' 100'Ч' 001 100'И' 1001 01 11 101

\includegraphics[width=0.8\linewidth]{/home/fizlrock/data/files/backup/code_backup/hobby/algoritms/LabExecutor/app/./doc_src/images/1050219518.jpg}

\includegraphics[width=0.8\linewidth]{/home/fizlrock/data/files/backup/code_backup/hobby/algoritms/LabExecutor/app/./doc_src/images/1266364493.jpg}

\includegraphics[width=0.8\linewidth]{/home/fizlrock/data/files/backup/code_backup/hobby/algoritms/LabExecutor/app/./doc_src/images/30985435.jpg}

\includegraphics[width=0.8\linewidth]{/home/fizlrock/data/files/backup/code_backup/hobby/algoritms/LabExecutor/app/./doc_src/images/385308957.jpg}

\includegraphics[width=0.8\linewidth]{/home/fizlrock/data/files/backup/code_backup/hobby/algoritms/LabExecutor/app/./doc_src/images/352994938.jpg}

\includegraphics[width=0.8\linewidth]{/home/fizlrock/data/files/backup/code_backup/hobby/algoritms/LabExecutor/app/./doc_src/images/1611358455.jpg}

\includegraphics[width=0.8\linewidth]{/home/fizlrock/data/files/backup/code_backup/hobby/algoritms/LabExecutor/app/./doc_src/images/1030430231.jpg}

\includegraphics[width=0.8\linewidth]{/home/fizlrock/data/files/backup/code_backup/hobby/algoritms/LabExecutor/app/./doc_src/images/1252551298.jpg}

\includegraphics[width=0.8\linewidth]{/home/fizlrock/data/files/backup/code_backup/hobby/algoritms/LabExecutor/app/./doc_src/images/1793740723.jpg}

\includegraphics[width=0.8\linewidth]{/home/fizlrock/data/files/backup/code_backup/hobby/algoritms/LabExecutor/app/./doc_src/images/1251137528.jpg}
\pagebreak
\paragraph{Задание 3.1}

Закодировать сообщение методом LZ77\\
Строка:ТИКТАК\_ТИК\_ТАК\_ТАКСА\\
Результат: <0,0,Т> <0,0,И> <0,0,К> <7,1,А> <7,1,\_> <3,3,\_> <2,5,А> <2,1,С> <0,0,А>\\
\begin{table}[h!]
\centering
\begin{tabular}{|c|c|c|c|c|c|c|c|c|c|c|c|c|c|c|c|c|} 
\hline
\multicolumn{10}{|c|}{Cловарь} & \multicolumn{6}{c|}{Буфер} & Код  \\ \hline
  &   &   &   &   &   &   &   &   &   & \cellcolor[HTML]{8CE4F6} Т & И & К & Т & А & К & <0,0,Т>
\\ \hline
  &   &   &   &   &   &   &   &   & Т & \cellcolor[HTML]{8CE4F6} И & К & Т & А & К &   & <0,0,И>
\\ \hline
  &   &   &   &   &   &   &   & Т & И & \cellcolor[HTML]{8CE4F6} К & Т & А & К &   & Т & <0,0,К>
\\ \hline
  &   &   &   &   &   &   & \cellcolor[HTML]{FFFF00} Т & И & К & \cellcolor[HTML]{FFFF00} Т & \cellcolor[HTML]{8CE4F6} А & К &   & Т & И & <7,1,А>
\\ \hline
  &   &   &   &   & Т & И & \cellcolor[HTML]{FFFF00} К & Т & А & \cellcolor[HTML]{FFFF00} К & \cellcolor[HTML]{8CE4F6}   & Т & И & К &   & <7,1,\_>
\\ \hline
  &   &   & \cellcolor[HTML]{FFFF00} Т & \cellcolor[HTML]{FFFF00} И & \cellcolor[HTML]{FFFF00} К & Т & А & К &   & \cellcolor[HTML]{FFFF00} Т & \cellcolor[HTML]{FFFF00} И & \cellcolor[HTML]{FFFF00} К & \cellcolor[HTML]{8CE4F6}   & Т & А & <3,3,\_>
\\ \hline
И & К & \cellcolor[HTML]{FFFF00} Т & \cellcolor[HTML]{FFFF00} А & \cellcolor[HTML]{FFFF00} К & \cellcolor[HTML]{FFFF00}   & \cellcolor[HTML]{FFFF00} Т & И & К &   & \cellcolor[HTML]{FFFF00} Т & \cellcolor[HTML]{FFFF00} А & \cellcolor[HTML]{FFFF00} К & \cellcolor[HTML]{FFFF00}   & \cellcolor[HTML]{FFFF00} Т & \cellcolor[HTML]{8CE4F6} А & <2,5,А>
\\ \hline
Т & И & \cellcolor[HTML]{FFFF00} К &   & Т & А & К &   & Т & А & \cellcolor[HTML]{FFFF00} К & \cellcolor[HTML]{8CE4F6} С & А &   &   &   & <2,1,С>
\\ \hline
К &   & Т & А & К &   & Т & А & К & С & \cellcolor[HTML]{8CE4F6} А &   &   &   &   &   & <0,0,А>
\\ \hline
\end{tabular}
\end{table}

\paragraph{Задание 3.2}

Закодировать сообщение методом LZSS\\
Строка:ТИКТАК\_ТИК\_ТАК\_ТАКСА\\
Результат: 0'Т' 0'И' 0'К' 1<7,1> 0'А' 1<7,1> 0'\_' 1<3,3> 1<6,2> 1<2,4> 1<6,2> 0'С' 1<3,1>\\
\begin{table}[h!]
\centering
\begin{tabular}{|c|c|c|c|c|c|c|c|c|c|c|c|c|c|c|c|c|}
\hline
\multicolumn{10}{|c|}{Cловарь} & \multicolumn{6}{c|}{Буфер} & Код  \\ \hline
  &   &   &   &   &   &   &   &   &   & Т & И & К & Т & А & К & 0'Т'\\ \hline
  &   &   &   &   &   &   &   &   & Т & И & К & Т & А & К & \_ & 0'И'\\ \hline
  &   &   &   &   &   &   &   & Т & И & К & Т & А & К & \_ & Т & 0'К'\\ \hline
  &   &   &   &   &   &   & \cellcolor[HTML]{FFFF00} Т & И & К & \cellcolor[HTML]{FFFF00} Т & А & К & \_ & Т & И & 1<7,1>\\ \hline
  &   &   &   &   &   & Т & И & К & Т & А & К & \_ & Т & И & К & 0'А'\\ \hline
  &   &   &   &   & Т & И & \cellcolor[HTML]{FFFF00} К & Т & А & \cellcolor[HTML]{FFFF00} К & \_ & Т & И & К & \_ & 1<7,1>\\ \hline
  &   &   &   & Т & И & К & Т & А & К & \_ & Т & И & К & \_ & Т & 0'\_'\\ \hline
  &   &   & \cellcolor[HTML]{FFFF00} Т & \cellcolor[HTML]{FFFF00} И & \cellcolor[HTML]{FFFF00} К & Т & А & К & \_ & \cellcolor[HTML]{FFFF00} Т & \cellcolor[HTML]{FFFF00} И & \cellcolor[HTML]{FFFF00} К & \_ & Т & А & 1<3,3>\\ \hline
Т & И & К & Т & А & К & \cellcolor[HTML]{FFFF00} \_ & \cellcolor[HTML]{FFFF00} Т & И & К & \cellcolor[HTML]{FFFF00} \_ & \cellcolor[HTML]{FFFF00} Т & А & К & \_ & Т & 1<6,2>\\ \hline
К & Т & \cellcolor[HTML]{FFFF00} А & \cellcolor[HTML]{FFFF00} К & \cellcolor[HTML]{FFFF00} \_ & \cellcolor[HTML]{FFFF00} Т & И & К & \_ & Т & \cellcolor[HTML]{FFFF00} А & \cellcolor[HTML]{FFFF00} К & \cellcolor[HTML]{FFFF00} \_ & \cellcolor[HTML]{FFFF00} Т & А & К & 1<2,4>\\ \hline
\_ & Т & И & К & \_ & Т & \cellcolor[HTML]{FFFF00} А & \cellcolor[HTML]{FFFF00} К & \_ & Т & \cellcolor[HTML]{FFFF00} А & \cellcolor[HTML]{FFFF00} К & С & А &   &   & 1<6,2>\\ \hline
И & К & \_ & Т & А & К & \_ & Т & А & К & С & А &   &   &   &   & 0'С'\\ \hline
К & \_ & Т & \cellcolor[HTML]{FFFF00} А & К & \_ & Т & А & К & С & \cellcolor[HTML]{FFFF00} А &   &   &   &   &   & 1<3,1>\\ \hline
\end{tabular}
\end{table}

\paragraph{Задание 3.3}

Закодировать сообщение методом LZ78\\
Строка:ТИКТАК\_ТИК\_ТАК\_ТАКСА\\
\begin{table}[h!]
\centering
\begin{tabular}{|c|c|c|} 
\hline
 Входная фраза (в словарь) & Код & Позиция словаря \\ \hline

 &  & 0 \\ \hline
Т & 0'Т' & 1 \\ \hline
И & 0'И' & 2 \\ \hline
К & 0'К' & 3 \\ \hline
ТА & 1'А' & 4 \\ \hline
К\_ & 3'\_' & 5 \\ \hline
ТИ & 1'И' & 6 \\ \hline
К\_Т & 5'Т' & 7 \\ \hline
А & 0'А' & 8 \\ \hline
К\_ТА & 7'А' & 9 \\ \hline
КС & 3'С' & 10 \\ \hline
\end{tabular}
\end{table}

Результат: 0'Т' 0'И' 0'К' 1'А' 3'\_' 1'И' 5'Т' 0'А' 7'А' 3'С'\\
\pagebreak
\paragraph{Задание 4. Арифметическое кодирование\\}

Исходная строка: КЛЮЧЧИИИИК\
\begin{center}
 \begin{tabular}{ |c|c| } 
  \hline
     Буква & Вероятность \\ \hline
И & 0.40\\\hline
Ч & 0.20\\\hline
К & 0.20\\\hline
Л & 0.10\\\hline
Ю & 0.10
\\ \hline \end{tabular}
\end{center}
\begin{center}
 \begin{tabular}{ |c|c|c| } 
  \hline
     Буква & Начало & Конец \\ \hline
И & 0.00 & 0.40\\\hline
Ч & 0.40 & 0.60\\\hline
К & 0.60 & 0.80\\\hline
Л & 0.80 & 0.90\\\hline
Ю & 0.90 & 1.00
\\ \hline \end{tabular}
\end{center}
\begin{center}
 \begin{tabular}{ |c|c|c|c| } 
  \hline
     Буква & delta & min & max \\ \hline
К & 0.2000000000 & 0.6000000000 & 0.8000000000\\\hline
Л & 0.0200000000 & 0.7600000000 & 0.7800000000\\\hline
Ю & 0.0020000000 & 0.7780000000 & 0.7800000000\\\hline
Ч & 0.0004000000 & 0.7788000000 & 0.7792000000\\\hline
Ч & 0.0000800000 & 0.7789600000 & 0.7790400000\\\hline
И & 0.0000320000 & 0.7789600000 & 0.7789920000\\\hline
И & 0.0000128000 & 0.7789600000 & 0.7789728000\\\hline
И & 0.0000051200 & 0.7789600000 & 0.7789651200\\\hline
И & 0.0000020480 & 0.7789600000 & 0.7789620480\\\hline
К & 0.0000004096 & 0.7789612288 & 0.7789616384
\\ \hline \end{tabular}
\end{center}
Результат: 7789613
\pagebreak
\paragraph{Задание 5.1}

\\ 

Декодировать сообщение методом адаптивного хаффмана \\
Строка: 
'K'0'L'0100'F'0111100'V'10011101001\\
Результат: KLLFKKVVFF

\includegraphics[width=0.8\linewidth]{/home/fizlrock/data/files/backup/code_backup/hobby/algoritms/LabExecutor/app/./doc_src/images/947830280.jpg}

\includegraphics[width=0.8\linewidth]{/home/fizlrock/data/files/backup/code_backup/hobby/algoritms/LabExecutor/app/./doc_src/images/56362332.jpg}

\includegraphics[width=0.8\linewidth]{/home/fizlrock/data/files/backup/code_backup/hobby/algoritms/LabExecutor/app/./doc_src/images/251123921.jpg}

\includegraphics[width=0.8\linewidth]{/home/fizlrock/data/files/backup/code_backup/hobby/algoritms/LabExecutor/app/./doc_src/images/456307343.jpg}

\includegraphics[width=0.8\linewidth]{/home/fizlrock/data/files/backup/code_backup/hobby/algoritms/LabExecutor/app/./doc_src/images/1053429992.jpg}

\includegraphics[width=0.8\linewidth]{/home/fizlrock/data/files/backup/code_backup/hobby/algoritms/LabExecutor/app/./doc_src/images/567998831.jpg}

\includegraphics[width=0.8\linewidth]{/home/fizlrock/data/files/backup/code_backup/hobby/algoritms/LabExecutor/app/./doc_src/images/931119330.jpg}

\includegraphics[width=0.8\linewidth]{/home/fizlrock/data/files/backup/code_backup/hobby/algoritms/LabExecutor/app/./doc_src/images/1011702092.jpg}

\includegraphics[width=0.8\linewidth]{/home/fizlrock/data/files/backup/code_backup/hobby/algoritms/LabExecutor/app/./doc_src/images/137938075.jpg}

\includegraphics[width=0.8\linewidth]{/home/fizlrock/data/files/backup/code_backup/hobby/algoritms/LabExecutor/app/./doc_src/images/1943180133.jpg}
\pagebreak
\paragraph{Задание 5.3 Декодировать строку(LZSS)\\}

Исходная строка: [0'н'] [0'о'] [0'с'] [1<8,1>] [0'к'] [0' '] [1<5,4>] [0'а'] [1<4,1>] [1<0,4>] [0'н'] [1<1,1>] [0'с']\\
\begin{table}[h!]
\centering
\begin{tabular}{|c|c|c|}
\hline
 Cловарь & Буфер & Код  \\ \hline
0'н' & [ ,  ,  ,  ,  ,  ,  ,  ,  , н] & н
\\ \hline
0'о' & [ ,  ,  ,  ,  ,  ,  ,  , н, о] & о
\\ \hline
0'с' & [ ,  ,  ,  ,  ,  ,  , н, о, с] & с
\\ \hline
1<8,1> & [ ,  ,  ,  ,  ,  , н, о, с, о] & о
\\ \hline
0'к' & [ ,  ,  ,  ,  , н, о, с, о, к] & к
\\ \hline
0' ' & [ ,  ,  ,  , н, о, с, о, к,  ] &  
\\ \hline
1<5,4> & [н, о, с, о, к,  , о, с, о, к] & осок
\\ \hline
0'а' & [о, с, о, к,  , о, с, о, к, а] & а
\\ \hline
1<4,1> & [с, о, к,  , о, с, о, к, а,  ] &  
\\ \hline
1<0,4> & [о, с, о, к, а,  , с, о, к,  ] & сок 
\\ \hline
0'н' & [с, о, к, а,  , с, о, к,  , н] & н
\\ \hline
1<1,1> & [о, к, а,  , с, о, к,  , н, о] & о
\\ \hline
0'с' & [к, а,  , с, о, к,  , н, о, с] & с
\\ \hline
\end{tabular}
\end{table}

Результат: носок осока сок нос
\pagebreak
\paragraph{Задание 5.4 Декодировать строку(LZ78)\\}

Исходная строка: [0'к'] [0'а'] [0'б'] [2'н'] [0' '] [3'а'] [0'н'] [1'а'] [5'б'] [2'к'] [0'е'] [0'н']\\
\begin{table}[h!]
\centering
\begin{tabular}{|c|c|c|}
\hline
 Cловарь & Буфер & Код  \\ \hline
 & [] & 
\\ \hline
0'к' & [, к] & к
\\ \hline
0'а' & [, к, а] & а
\\ \hline
0'б' & [, к, а, б] & б
\\ \hline
2'н' & [, к, а, б, ан] & ан
\\ \hline
0' ' & [, к, а, б, ан,  ] &  
\\ \hline
3'а' & [, к, а, б, ан,  , ба] & ба
\\ \hline
0'н' & [, к, а, б, ан,  , ба, н] & н
\\ \hline
1'а' & [, к, а, б, ан,  , ба, н, ка] & ка
\\ \hline
5'б' & [, к, а, б, ан,  , ба, н, ка,  б] &  б
\\ \hline
2'к' & [, к, а, б, ан,  , ба, н, ка,  б, ак] & ак
\\ \hline
0'е' & [, к, а, б, ан,  , ба, н, ка,  б, ак, е] & е
\\ \hline
0'н' & [, к, а, б, ан,  , ба, н, ка,  б, ак, е, н] & н
\\ \hline
\end{tabular}
\end{table}

Результат: кабан банка бакен
\pagebreak
\subsection{Вариант №29}
\paragraph{Задание 1. Блочный хаффман \\}

Строка ИИММИИИРРР, размер блока: 2
\begin{center}
 \begin{tabular}{ |c|c|l| } 
  \hline
     Буква & Вероятность & Код\\ \hline
И & 0.50 & 0\\\hline
Р & 0.30 & 11\\\hline
М & 0.20 & 10
\\ \hline \end{tabular}
\end{center}
Энтропия алфавита: 1.4855
\begin{center}
 \begin{tabular}{ |c|c|l| } 
  \hline
     Блок & Вероятность & Код\\ \hline
ИИ & 0.25 & 01\\\hline
РИ & 0.15 & 101\\\hline
ИР & 0.15 & 110\\\hline
ИМ & 0.10 & 000\\\hline
МИ & 0.10 & 001\\\hline
РР & 0.09 & 1111\\\hline
МР & 0.06 & 1001\\\hline
РМ & 0.06 & 1110\\\hline
ММ & 0.04 & 1000
\\ \hline \end{tabular}
\end{center}
Бит на символ при посимвольном кодировании: 1.5000, при блочном: 1.5000

\includegraphics[width=0.5\linewidth]{/home/fizlrock/data/files/backup/code_backup/hobby/algoritms/LabExecutor/app/./doc_src/images/835873157.jpg}

\includegraphics[width=0.9\linewidth]{/home/fizlrock/data/files/backup/code_backup/hobby/algoritms/LabExecutor/app/./doc_src/images/838935252.jpg}
\pagebreak
\paragraph{Задание 2. Сжать адаптивным хаффманом\\}

Строка: 
БАЗАААРРРР\\
Результат: 'Б' 0'А' 00'З' 11 0 1 000'Р' 0101 00 11

\includegraphics[width=0.8\linewidth]{/home/fizlrock/data/files/backup/code_backup/hobby/algoritms/LabExecutor/app/./doc_src/images/1003058476.jpg}

\includegraphics[width=0.8\linewidth]{/home/fizlrock/data/files/backup/code_backup/hobby/algoritms/LabExecutor/app/./doc_src/images/450589355.jpg}

\includegraphics[width=0.8\linewidth]{/home/fizlrock/data/files/backup/code_backup/hobby/algoritms/LabExecutor/app/./doc_src/images/1068957793.jpg}

\includegraphics[width=0.8\linewidth]{/home/fizlrock/data/files/backup/code_backup/hobby/algoritms/LabExecutor/app/./doc_src/images/711349378.jpg}

\includegraphics[width=0.8\linewidth]{/home/fizlrock/data/files/backup/code_backup/hobby/algoritms/LabExecutor/app/./doc_src/images/2101779954.jpg}

\includegraphics[width=0.8\linewidth]{/home/fizlrock/data/files/backup/code_backup/hobby/algoritms/LabExecutor/app/./doc_src/images/1976819242.jpg}

\includegraphics[width=0.8\linewidth]{/home/fizlrock/data/files/backup/code_backup/hobby/algoritms/LabExecutor/app/./doc_src/images/1151025169.jpg}

\includegraphics[width=0.8\linewidth]{/home/fizlrock/data/files/backup/code_backup/hobby/algoritms/LabExecutor/app/./doc_src/images/543747759.jpg}

\includegraphics[width=0.8\linewidth]{/home/fizlrock/data/files/backup/code_backup/hobby/algoritms/LabExecutor/app/./doc_src/images/950479311.jpg}

\includegraphics[width=0.8\linewidth]{/home/fizlrock/data/files/backup/code_backup/hobby/algoritms/LabExecutor/app/./doc_src/images/1287717416.jpg}
\pagebreak
\paragraph{Задание 3.1}

Закодировать сообщение методом LZ77\\
Строка:КУРКУЛЬ\_КУЛЕК\_ЛЕКАЛО\\
Результат: <0,0,К> <0,0,У> <0,0,Р> <7,2,Л> <0,0,Ь> <0,0,\_> <5,3,Е> <1,1,\_> <6,3,А> <2,1,О>\\
\begin{table}[h!]
\centering
\begin{tabular}{|c|c|c|c|c|c|c|c|c|c|c|c|c|c|c|c|c|} 
\hline
\multicolumn{10}{|c|}{Cловарь} & \multicolumn{6}{c|}{Буфер} & Код  \\ \hline
  &   &   &   &   &   &   &   &   &   & \cellcolor[HTML]{8CE4F6} К & У & Р & К & У & Л & <0,0,К>
\\ \hline
  &   &   &   &   &   &   &   &   & К & \cellcolor[HTML]{8CE4F6} У & Р & К & У & Л & Ь & <0,0,У>
\\ \hline
  &   &   &   &   &   &   &   & К & У & \cellcolor[HTML]{8CE4F6} Р & К & У & Л & Ь &   & <0,0,Р>
\\ \hline
  &   &   &   &   &   &   & \cellcolor[HTML]{FFFF00} К & \cellcolor[HTML]{FFFF00} У & Р & \cellcolor[HTML]{FFFF00} К & \cellcolor[HTML]{FFFF00} У & \cellcolor[HTML]{8CE4F6} Л & Ь &   & К & <7,2,Л>
\\ \hline
  &   &   &   & К & У & Р & К & У & Л & \cellcolor[HTML]{8CE4F6} Ь &   & К & У & Л & Е & <0,0,Ь>
\\ \hline
  &   &   & К & У & Р & К & У & Л & Ь & \cellcolor[HTML]{8CE4F6}   & К & У & Л & Е & К & <0,0,\_>
\\ \hline
  &   & К & У & Р & \cellcolor[HTML]{FFFF00} К & \cellcolor[HTML]{FFFF00} У & \cellcolor[HTML]{FFFF00} Л & Ь &   & \cellcolor[HTML]{FFFF00} К & \cellcolor[HTML]{FFFF00} У & \cellcolor[HTML]{FFFF00} Л & \cellcolor[HTML]{8CE4F6} Е & К &   & <5,3,Е>
\\ \hline
Р & \cellcolor[HTML]{FFFF00} К & У & Л & Ь &   & К & У & Л & Е & \cellcolor[HTML]{FFFF00} К & \cellcolor[HTML]{8CE4F6}   & Л & Е & К & А & <1,1,\_>
\\ \hline
У & Л & Ь &   & К & У & \cellcolor[HTML]{FFFF00} Л & \cellcolor[HTML]{FFFF00} Е & \cellcolor[HTML]{FFFF00} К &   & \cellcolor[HTML]{FFFF00} Л & \cellcolor[HTML]{FFFF00} Е & \cellcolor[HTML]{FFFF00} К & \cellcolor[HTML]{8CE4F6} А & Л & О & <6,3,А>
\\ \hline
К & У & \cellcolor[HTML]{FFFF00} Л & Е & К &   & Л & Е & К & А & \cellcolor[HTML]{FFFF00} Л & \cellcolor[HTML]{8CE4F6} О &   &   &   &   & <2,1,О>
\\ \hline
\end{tabular}
\end{table}

\paragraph{Задание 3.2}

Закодировать сообщение методом LZSS\\
Строка:КУРКУЛЬ\_КУЛЕК\_ЛЕКАЛО\\
Результат: 0'К' 0'У' 0'Р' 1<7,2> 0'Л' 0'Ь' 0'\_' 1<5,3> 0'Е' 1<1,1> 1<4,1> 1<6,3> 0'А' 1<2,1> 0'О'\\
\begin{table}[h!]
\centering
\begin{tabular}{|c|c|c|c|c|c|c|c|c|c|c|c|c|c|c|c|c|}
\hline
\multicolumn{10}{|c|}{Cловарь} & \multicolumn{6}{c|}{Буфер} & Код  \\ \hline
  &   &   &   &   &   &   &   &   &   & К & У & Р & К & У & Л & 0'К'\\ \hline
  &   &   &   &   &   &   &   &   & К & У & Р & К & У & Л & Ь & 0'У'\\ \hline
  &   &   &   &   &   &   &   & К & У & Р & К & У & Л & Ь & \_ & 0'Р'\\ \hline
  &   &   &   &   &   &   & \cellcolor[HTML]{FFFF00} К & \cellcolor[HTML]{FFFF00} У & Р & \cellcolor[HTML]{FFFF00} К & \cellcolor[HTML]{FFFF00} У & Л & Ь & \_ & К & 1<7,2>\\ \hline
  &   &   &   &   & К & У & Р & К & У & Л & Ь & \_ & К & У & Л & 0'Л'\\ \hline
  &   &   &   & К & У & Р & К & У & Л & Ь & \_ & К & У & Л & Е & 0'Ь'\\ \hline
  &   &   & К & У & Р & К & У & Л & Ь & \_ & К & У & Л & Е & К & 0'\_'\\ \hline
  &   & К & У & Р & \cellcolor[HTML]{FFFF00} К & \cellcolor[HTML]{FFFF00} У & \cellcolor[HTML]{FFFF00} Л & Ь & \_ & \cellcolor[HTML]{FFFF00} К & \cellcolor[HTML]{FFFF00} У & \cellcolor[HTML]{FFFF00} Л & Е & К & \_ & 1<5,3>\\ \hline
У & Р & К & У & Л & Ь & \_ & К & У & Л & Е & К & \_ & Л & Е & К & 0'Е'\\ \hline
Р & \cellcolor[HTML]{FFFF00} К & У & Л & Ь & \_ & К & У & Л & Е & \cellcolor[HTML]{FFFF00} К & \_ & Л & Е & К & А & 1<1,1>\\ \hline
К & У & Л & Ь & \cellcolor[HTML]{FFFF00} \_ & К & У & Л & Е & К & \cellcolor[HTML]{FFFF00} \_ & Л & Е & К & А & Л & 1<4,1>\\ \hline
У & Л & Ь & \_ & К & У & \cellcolor[HTML]{FFFF00} Л & \cellcolor[HTML]{FFFF00} Е & \cellcolor[HTML]{FFFF00} К & \_ & \cellcolor[HTML]{FFFF00} Л & \cellcolor[HTML]{FFFF00} Е & \cellcolor[HTML]{FFFF00} К & А & Л & О & 1<6,3>\\ \hline
\_ & К & У & Л & Е & К & \_ & Л & Е & К & А & Л & О &   &   &   & 0'А'\\ \hline
К & У & \cellcolor[HTML]{FFFF00} Л & Е & К & \_ & Л & Е & К & А & \cellcolor[HTML]{FFFF00} Л & О &   &   &   &   & 1<2,1>\\ \hline
У & Л & Е & К & \_ & Л & Е & К & А & Л & О &   &   &   &   &   & 0'О'\\ \hline
\end{tabular}
\end{table}

\paragraph{Задание 3.3}

Закодировать сообщение методом LZ78\\
Строка:КУРКУЛЬ\_КУЛЕК\_ЛЕКАЛО\\
\begin{table}[h!]
\centering
\begin{tabular}{|c|c|c|} 
\hline
 Входная фраза (в словарь) & Код & Позиция словаря \\ \hline

 &  & 0 \\ \hline
К & 0'К' & 1 \\ \hline
У & 0'У' & 2 \\ \hline
Р & 0'Р' & 3 \\ \hline
КУ & 1'У' & 4 \\ \hline
Л & 0'Л' & 5 \\ \hline
Ь & 0'Ь' & 6 \\ \hline
\_ & 0'\_' & 7 \\ \hline
КУЛ & 4'Л' & 8 \\ \hline
Е & 0'Е' & 9 \\ \hline
К\_ & 1'\_' & 10 \\ \hline
ЛЕ & 5'Е' & 11 \\ \hline
КА & 1'А' & 12 \\ \hline
ЛО & 5'О' & 13 \\ \hline
\end{tabular}
\end{table}

Результат: 0'К' 0'У' 0'Р' 1'У' 0'Л' 0'Ь' 0'\_' 4'Л' 0'Е' 1'\_' 5'Е' 1'А' 5'О'\\
\pagebreak
\paragraph{Задание 4. Арифметическое кодирование\\}

Исходная строка: БАЗАААРРРР\
\begin{center}
 \begin{tabular}{ |c|c| } 
  \hline
     Буква & Вероятность \\ \hline
А & 0.40\\\hline
Р & 0.40\\\hline
Б & 0.10\\\hline
З & 0.10
\\ \hline \end{tabular}
\end{center}
\begin{center}
 \begin{tabular}{ |c|c|c| } 
  \hline
     Буква & Начало & Конец \\ \hline
А & 0.00 & 0.40\\\hline
Р & 0.40 & 0.80\\\hline
Б & 0.80 & 0.90\\\hline
З & 0.90 & 1.00
\\ \hline \end{tabular}
\end{center}
\begin{center}
 \begin{tabular}{ |c|c|c|c| } 
  \hline
     Буква & delta & min & max \\ \hline
Б & 0.1000000000 & 0.8000000000 & 0.9000000000\\\hline
А & 0.0400000000 & 0.8000000000 & 0.8400000000\\\hline
З & 0.0040000000 & 0.8360000000 & 0.8400000000\\\hline
А & 0.0016000000 & 0.8360000000 & 0.8376000000\\\hline
А & 0.0006400000 & 0.8360000000 & 0.8366400000\\\hline
А & 0.0002560000 & 0.8360000000 & 0.8362560000\\\hline
Р & 0.0001024000 & 0.8361024000 & 0.8362048000\\\hline
Р & 0.0000409600 & 0.8361433600 & 0.8361843200\\\hline
Р & 0.0000163840 & 0.8361597440 & 0.8361761280\\\hline
Р & 0.0000065536 & 0.8361662976 & 0.8361728512
\\ \hline \end{tabular}
\end{center}
Результат: 83617
\pagebreak
\paragraph{Задание 5.1}

\\ 

Декодировать сообщение методом адаптивного хаффмана \\
Строка: 
Ошибка декодирования\\
Результат: Ошибка декодирования
\pagebreak
\paragraph{Задание 5.3 Декодировать строку(LZSS)\\}

Исходная строка: [0'к'] [0'о'] [0'л'] [1<8,1>] [0'с'] [0' '] [1<6,3>] [1<1,1>] [1<5,1>] [1<2,2>] [0'а'] [1<1,1>] [1<3,2>] [1<0,1>] [0'л']\\
\begin{table}[h!]
\centering
\begin{tabular}{|c|c|c|}
\hline
 Cловарь & Буфер & Код  \\ \hline
0'к' & [ ,  ,  ,  ,  ,  ,  ,  ,  , к] & к
\\ \hline
0'о' & [ ,  ,  ,  ,  ,  ,  ,  , к, о] & о
\\ \hline
0'л' & [ ,  ,  ,  ,  ,  ,  , к, о, л] & л
\\ \hline
1<8,1> & [ ,  ,  ,  ,  ,  , к, о, л, о] & о
\\ \hline
0'с' & [ ,  ,  ,  ,  , к, о, л, о, с] & с
\\ \hline
0' ' & [ ,  ,  ,  , к, о, л, о, с,  ] &  
\\ \hline
1<6,3> & [ , к, о, л, о, с,  , л, о, с] & лос
\\ \hline
1<1,1> & [к, о, л, о, с,  , л, о, с, к] & к
\\ \hline
1<5,1> & [о, л, о, с,  , л, о, с, к,  ] &  
\\ \hline
1<2,2> & [о, с,  , л, о, с, к,  , о, с] & ос
\\ \hline
0'а' & [с,  , л, о, с, к,  , о, с, а] & а
\\ \hline
1<1,1> & [ , л, о, с, к,  , о, с, а,  ] &  
\\ \hline
1<3,2> & [о, с, к,  , о, с, а,  , с, к] & ск
\\ \hline
1<0,1> & [с, к,  , о, с, а,  , с, к, о] & о
\\ \hline
0'л' & [к,  , о, с, а,  , с, к, о, л] & л
\\ \hline
\end{tabular}
\end{table}

Результат: колос лоск оса скол
\pagebreak
\paragraph{Задание 5.4 Декодировать строку(LZ78)\\}

Исходная строка: [0'з'] [0'а'] [0'р'] [0'я'] [0' '] [1'а'] [3'я'] [0'д'] [0'к'] [2' '] [7'д']\\
\begin{table}[h!]
\centering
\begin{tabular}{|c|c|c|}
\hline
 Cловарь & Буфер & Код  \\ \hline
 & [] & 
\\ \hline
0'з' & [, з] & з
\\ \hline
0'а' & [, з, а] & а
\\ \hline
0'р' & [, з, а, р] & р
\\ \hline
0'я' & [, з, а, р, я] & я
\\ \hline
0' ' & [, з, а, р, я,  ] &  
\\ \hline
1'а' & [, з, а, р, я,  , за] & за
\\ \hline
3'я' & [, з, а, р, я,  , за, ря] & ря
\\ \hline
0'д' & [, з, а, р, я,  , за, ря, д] & д
\\ \hline
0'к' & [, з, а, р, я,  , за, ря, д, к] & к
\\ \hline
2' ' & [, з, а, р, я,  , за, ря, д, к, а ] & а 
\\ \hline
7'д' & [, з, а, р, я,  , за, ря, д, к, а , ряд] & ряд
\\ \hline
\end{tabular}
\end{table}

Результат: заря зарядка ряд
\pagebreak
\subsection{Вариант №30}
\paragraph{Задание 1. Блочный хаффман \\}

Строка ОККОЛТКККК, размер блока: 2
\begin{center}
 \begin{tabular}{ |c|c|l| } 
  \hline
     Буква & Вероятность & Код\\ \hline
К & 0.60 & 1\\\hline
О & 0.20 & 00\\\hline
Т & 0.10 & 010\\\hline
Л & 0.10 & 011
\\ \hline \end{tabular}
\end{center}
Энтропия алфавита: 1.5710
\begin{center}
 \begin{tabular}{ |c|c|l| } 
  \hline
     Блок & Вероятность & Код\\ \hline
КК & 0.36 & 11\\\hline
КО & 0.12 & 010\\\hline
ОК & 0.12 & 011\\\hline
КЛ & 0.06 & 1000\\\hline
КТ & 0.06 & 1001\\\hline
ТК & 0.06 & 1010\\\hline
ЛК & 0.06 & 1011\\\hline
ОО & 0.04 & 0000\\\hline
ЛО & 0.02 & 00010\\\hline
ОТ & 0.02 & 00011\\\hline
ТО & 0.02 & 00100\\\hline
ОЛ & 0.02 & 00101\\\hline
ТТ & 0.01 & 001100\\\hline
ЛЛ & 0.01 & 001101\\\hline
ЛТ & 0.01 & 001110\\\hline
ТЛ & 0.01 & 001111
\\ \hline \end{tabular}
\end{center}
Бит на символ при посимвольном кодировании: 1.6000, при блочном: 1.6000

\includegraphics[width=0.5\linewidth]{/home/fizlrock/data/files/backup/code_backup/hobby/algoritms/LabExecutor/app/./doc_src/images/847245130.jpg}

\includegraphics[width=0.9\linewidth]{/home/fizlrock/data/files/backup/code_backup/hobby/algoritms/LabExecutor/app/./doc_src/images/396438290.jpg}
\pagebreak
\paragraph{Задание 2. Сжать адаптивным хаффманом\\}

Строка: 
КРЫЛЛЛЛЫРР\\
Результат: 'К' 0'Р' 00'Ы' 100'Л' 001 11 0 011 010 111

\includegraphics[width=0.8\linewidth]{/home/fizlrock/data/files/backup/code_backup/hobby/algoritms/LabExecutor/app/./doc_src/images/166014502.jpg}

\includegraphics[width=0.8\linewidth]{/home/fizlrock/data/files/backup/code_backup/hobby/algoritms/LabExecutor/app/./doc_src/images/1452309866.jpg}

\includegraphics[width=0.8\linewidth]{/home/fizlrock/data/files/backup/code_backup/hobby/algoritms/LabExecutor/app/./doc_src/images/650308395.jpg}

\includegraphics[width=0.8\linewidth]{/home/fizlrock/data/files/backup/code_backup/hobby/algoritms/LabExecutor/app/./doc_src/images/1890499230.jpg}

\includegraphics[width=0.8\linewidth]{/home/fizlrock/data/files/backup/code_backup/hobby/algoritms/LabExecutor/app/./doc_src/images/619153216.jpg}

\includegraphics[width=0.8\linewidth]{/home/fizlrock/data/files/backup/code_backup/hobby/algoritms/LabExecutor/app/./doc_src/images/1725264100.jpg}

\includegraphics[width=0.8\linewidth]{/home/fizlrock/data/files/backup/code_backup/hobby/algoritms/LabExecutor/app/./doc_src/images/178751839.jpg}

\includegraphics[width=0.8\linewidth]{/home/fizlrock/data/files/backup/code_backup/hobby/algoritms/LabExecutor/app/./doc_src/images/2130919291.jpg}

\includegraphics[width=0.8\linewidth]{/home/fizlrock/data/files/backup/code_backup/hobby/algoritms/LabExecutor/app/./doc_src/images/450564617.jpg}

\includegraphics[width=0.8\linewidth]{/home/fizlrock/data/files/backup/code_backup/hobby/algoritms/LabExecutor/app/./doc_src/images/1232348326.jpg}
\pagebreak
\paragraph{Задание 3.1}

Закодировать сообщение методом LZ77\\
Строка:СКЛАД\_КЛАД\_КЛАДЕЗЬ\\
Результат: <0,0,С> <0,0,К> <0,0,Л> <0,0,А> <0,0,Д> <0,0,\_> <5,5,К> <0,3,Е> <0,0,З> <0,0,Ь>\\
\begin{table}[h!]
\centering
\begin{tabular}{|c|c|c|c|c|c|c|c|c|c|c|c|c|c|c|c|c|} 
\hline
\multicolumn{10}{|c|}{Cловарь} & \multicolumn{6}{c|}{Буфер} & Код  \\ \hline
  &   &   &   &   &   &   &   &   &   & \cellcolor[HTML]{8CE4F6} С & К & Л & А & Д &   & <0,0,С>
\\ \hline
  &   &   &   &   &   &   &   &   & С & \cellcolor[HTML]{8CE4F6} К & Л & А & Д &   & К & <0,0,К>
\\ \hline
  &   &   &   &   &   &   &   & С & К & \cellcolor[HTML]{8CE4F6} Л & А & Д &   & К & Л & <0,0,Л>
\\ \hline
  &   &   &   &   &   &   & С & К & Л & \cellcolor[HTML]{8CE4F6} А & Д &   & К & Л & А & <0,0,А>
\\ \hline
  &   &   &   &   &   & С & К & Л & А & \cellcolor[HTML]{8CE4F6} Д &   & К & Л & А & Д & <0,0,Д>
\\ \hline
  &   &   &   &   & С & К & Л & А & Д & \cellcolor[HTML]{8CE4F6}   & К & Л & А & Д &   & <0,0,\_>
\\ \hline
  &   &   &   & С & \cellcolor[HTML]{FFFF00} К & \cellcolor[HTML]{FFFF00} Л & \cellcolor[HTML]{FFFF00} А & \cellcolor[HTML]{FFFF00} Д & \cellcolor[HTML]{FFFF00}   & \cellcolor[HTML]{FFFF00} К & \cellcolor[HTML]{FFFF00} Л & \cellcolor[HTML]{FFFF00} А & \cellcolor[HTML]{FFFF00} Д & \cellcolor[HTML]{FFFF00}   & \cellcolor[HTML]{8CE4F6} К & <5,5,К>
\\ \hline
\cellcolor[HTML]{FFFF00} Л & \cellcolor[HTML]{FFFF00} А & \cellcolor[HTML]{FFFF00} Д &   & К & Л & А & Д &   & К & \cellcolor[HTML]{FFFF00} Л & \cellcolor[HTML]{FFFF00} А & \cellcolor[HTML]{FFFF00} Д & \cellcolor[HTML]{8CE4F6} Е & З & Ь & <0,3,Е>
\\ \hline
К & Л & А & Д &   & К & Л & А & Д & Е & \cellcolor[HTML]{8CE4F6} З & Ь &   &   &   &   & <0,0,З>
\\ \hline
Л & А & Д &   & К & Л & А & Д & Е & З & \cellcolor[HTML]{8CE4F6} Ь &   &   &   &   &   & <0,0,Ь>
\\ \hline
\end{tabular}
\end{table}

\paragraph{Задание 3.2}

Закодировать сообщение методом LZSS\\
Строка:СКЛАД\_КЛАД\_КЛАДЕЗЬ\\
Результат: 0'С' 0'К' 0'Л' 0'А' 0'Д' 0'\_' 1<5,5> 1<0,4> 0'Е' 0'З' 0'Ь'\\
\begin{table}[h!]
\centering
\begin{tabular}{|c|c|c|c|c|c|c|c|c|c|c|c|c|c|c|c|c|}
\hline
\multicolumn{10}{|c|}{Cловарь} & \multicolumn{6}{c|}{Буфер} & Код  \\ \hline
  &   &   &   &   &   &   &   &   &   & С & К & Л & А & Д & \_ & 0'С'\\ \hline
  &   &   &   &   &   &   &   &   & С & К & Л & А & Д & \_ & К & 0'К'\\ \hline
  &   &   &   &   &   &   &   & С & К & Л & А & Д & \_ & К & Л & 0'Л'\\ \hline
  &   &   &   &   &   &   & С & К & Л & А & Д & \_ & К & Л & А & 0'А'\\ \hline
  &   &   &   &   &   & С & К & Л & А & Д & \_ & К & Л & А & Д & 0'Д'\\ \hline
  &   &   &   &   & С & К & Л & А & Д & \_ & К & Л & А & Д & \_ & 0'\_'\\ \hline
  &   &   &   & С & \cellcolor[HTML]{FFFF00} К & \cellcolor[HTML]{FFFF00} Л & \cellcolor[HTML]{FFFF00} А & \cellcolor[HTML]{FFFF00} Д & \cellcolor[HTML]{FFFF00} \_ & \cellcolor[HTML]{FFFF00} К & \cellcolor[HTML]{FFFF00} Л & \cellcolor[HTML]{FFFF00} А & \cellcolor[HTML]{FFFF00} Д & \cellcolor[HTML]{FFFF00} \_ & К & 1<5,5>\\ \hline
\cellcolor[HTML]{FFFF00} К & \cellcolor[HTML]{FFFF00} Л & \cellcolor[HTML]{FFFF00} А & \cellcolor[HTML]{FFFF00} Д & \_ & К & Л & А & Д & \_ & \cellcolor[HTML]{FFFF00} К & \cellcolor[HTML]{FFFF00} Л & \cellcolor[HTML]{FFFF00} А & \cellcolor[HTML]{FFFF00} Д & Е & З & 1<0,4>\\ \hline
\_ & К & Л & А & Д & \_ & К & Л & А & Д & Е & З & Ь &   &   &   & 0'Е'\\ \hline
К & Л & А & Д & \_ & К & Л & А & Д & Е & З & Ь &   &   &   &   & 0'З'\\ \hline
Л & А & Д & \_ & К & Л & А & Д & Е & З & Ь &   &   &   &   &   & 0'Ь'\\ \hline
\end{tabular}
\end{table}

\paragraph{Задание 3.3}

Закодировать сообщение методом LZ78\\
Строка:СКЛАД\_КЛАД\_КЛАДЕЗЬ\\
\begin{table}[h!]
\centering
\begin{tabular}{|c|c|c|} 
\hline
 Входная фраза (в словарь) & Код & Позиция словаря \\ \hline

 &  & 0 \\ \hline
С & 0'С' & 1 \\ \hline
К & 0'К' & 2 \\ \hline
Л & 0'Л' & 3 \\ \hline
А & 0'А' & 4 \\ \hline
Д & 0'Д' & 5 \\ \hline
\_ & 0'\_' & 6 \\ \hline
КЛ & 2'Л' & 7 \\ \hline
АД & 4'Д' & 8 \\ \hline
\_К & 6'К' & 9 \\ \hline
ЛА & 3'А' & 10 \\ \hline
ДЕ & 5'Е' & 11 \\ \hline
З & 0'З' & 12 \\ \hline
Ь & 0'Ь' & 13 \\ \hline
\end{tabular}
\end{table}

Результат: 0'С' 0'К' 0'Л' 0'А' 0'Д' 0'\_' 2'Л' 4'Д' 6'К' 3'А' 5'Е' 0'З' 0'Ь'\\
\pagebreak
\paragraph{Задание 4. Арифметическое кодирование\\}

Исходная строка: КРЫЛЛЛЛЫРР\
\begin{center}
 \begin{tabular}{ |c|c| } 
  \hline
     Буква & Вероятность \\ \hline
Л & 0.40\\\hline
Р & 0.30\\\hline
Ы & 0.20\\\hline
К & 0.10
\\ \hline \end{tabular}
\end{center}
\begin{center}
 \begin{tabular}{ |c|c|c| } 
  \hline
     Буква & Начало & Конец \\ \hline
Л & 0.00 & 0.40\\\hline
Р & 0.40 & 0.70\\\hline
Ы & 0.70 & 0.90\\\hline
К & 0.90 & 1.00
\\ \hline \end{tabular}
\end{center}
\begin{center}
 \begin{tabular}{ |c|c|c|c| } 
  \hline
     Буква & delta & min & max \\ \hline
К & 0.1000000000 & 0.9000000000 & 1.0000000000\\\hline
Р & 0.0300000000 & 0.9400000000 & 0.9700000000\\\hline
Ы & 0.0060000000 & 0.9610000000 & 0.9670000000\\\hline
Л & 0.0024000000 & 0.9610000000 & 0.9634000000\\\hline
Л & 0.0009600000 & 0.9610000000 & 0.9619600000\\\hline
Л & 0.0003840000 & 0.9610000000 & 0.9613840000\\\hline
Л & 0.0001536000 & 0.9610000000 & 0.9611536000\\\hline
Ы & 0.0000307200 & 0.9611075200 & 0.9611382400\\\hline
Р & 0.0000092160 & 0.9611198080 & 0.9611290240\\\hline
Р & 0.0000027648 & 0.9611234944 & 0.9611262592
\\ \hline \end{tabular}
\end{center}
Результат: 961124
\pagebreak
\paragraph{Задание 5.1}

\\ 

Декодировать сообщение методом адаптивного хаффмана \\
Строка: 
'S'0'K'00'T'100'R'10111101110111110\\
Результат: SKTRSSRKKSS

\includegraphics[width=0.8\linewidth]{/home/fizlrock/data/files/backup/code_backup/hobby/algoritms/LabExecutor/app/./doc_src/images/1151153770.jpg}

\includegraphics[width=0.8\linewidth]{/home/fizlrock/data/files/backup/code_backup/hobby/algoritms/LabExecutor/app/./doc_src/images/386009110.jpg}

\includegraphics[width=0.8\linewidth]{/home/fizlrock/data/files/backup/code_backup/hobby/algoritms/LabExecutor/app/./doc_src/images/94226536.jpg}

\includegraphics[width=0.8\linewidth]{/home/fizlrock/data/files/backup/code_backup/hobby/algoritms/LabExecutor/app/./doc_src/images/87327961.jpg}

\includegraphics[width=0.8\linewidth]{/home/fizlrock/data/files/backup/code_backup/hobby/algoritms/LabExecutor/app/./doc_src/images/1314757611.jpg}

\includegraphics[width=0.8\linewidth]{/home/fizlrock/data/files/backup/code_backup/hobby/algoritms/LabExecutor/app/./doc_src/images/957446004.jpg}

\includegraphics[width=0.8\linewidth]{/home/fizlrock/data/files/backup/code_backup/hobby/algoritms/LabExecutor/app/./doc_src/images/1537710351.jpg}

\includegraphics[width=0.8\linewidth]{/home/fizlrock/data/files/backup/code_backup/hobby/algoritms/LabExecutor/app/./doc_src/images/363140132.jpg}

\includegraphics[width=0.8\linewidth]{/home/fizlrock/data/files/backup/code_backup/hobby/algoritms/LabExecutor/app/./doc_src/images/1420239551.jpg}

\includegraphics[width=0.8\linewidth]{/home/fizlrock/data/files/backup/code_backup/hobby/algoritms/LabExecutor/app/./doc_src/images/63087908.jpg}

\includegraphics[width=0.8\linewidth]{/home/fizlrock/data/files/backup/code_backup/hobby/algoritms/LabExecutor/app/./doc_src/images/843056159.jpg}
\pagebreak
\paragraph{Задание 5.3 Декодировать строку(LZSS)\\}

Исходная строка: [0'б'] [0'а'] [0'т'] [0'у'] [1<8,1>] [0' '] [1<6,3>] [1<4,1>] [1<5,1>] [1<2,2>] [0'к'] [0'а'] [1<0,3>] [1<5,2>] [0'н']\\
\begin{table}[h!]
\centering
\begin{tabular}{|c|c|c|}
\hline
 Cловарь & Буфер & Код  \\ \hline
0'б' & [ ,  ,  ,  ,  ,  ,  ,  ,  , б] & б
\\ \hline
0'а' & [ ,  ,  ,  ,  ,  ,  ,  , б, а] & а
\\ \hline
0'т' & [ ,  ,  ,  ,  ,  ,  , б, а, т] & т
\\ \hline
0'у' & [ ,  ,  ,  ,  ,  , б, а, т, у] & у
\\ \hline
1<8,1> & [ ,  ,  ,  ,  , б, а, т, у, т] & т
\\ \hline
0' ' & [ ,  ,  ,  , б, а, т, у, т,  ] &  
\\ \hline
1<6,3> & [ , б, а, т, у, т,  , т, у, т] & тут
\\ \hline
1<4,1> & [б, а, т, у, т,  , т, у, т, у] & у
\\ \hline
1<5,1> & [а, т, у, т,  , т, у, т, у,  ] &  
\\ \hline
1<2,2> & [у, т,  , т, у, т, у,  , у, т] & ут
\\ \hline
0'к' & [т,  , т, у, т, у,  , у, т, к] & к
\\ \hline
0'а' & [ , т, у, т, у,  , у, т, к, а] & а
\\ \hline
1<0,3> & [т, у,  , у, т, к, а,  , т, у] &  ту
\\ \hline
1<5,2> & [ , у, т, к, а,  , т, у, к, а] & ка
\\ \hline
0'н' & [у, т, к, а,  , т, у, к, а, н] & н
\\ \hline
\end{tabular}
\end{table}

Результат: батут туту утка тукан
\pagebreak
\paragraph{Задание 5.4 Декодировать строку(LZ78)\\}

Исходная строка: [0'к'] [0'л'] [0'а'] [0'д'] [0' '] [0'с'] [1'л'] [3'д'] [5'л'] [8' '] [2'а'] [4'ь'] [0'я']\\
\begin{table}[h!]
\centering
\begin{tabular}{|c|c|c|}
\hline
 Cловарь & Буфер & Код  \\ \hline
 & [] & 
\\ \hline
0'к' & [, к] & к
\\ \hline
0'л' & [, к, л] & л
\\ \hline
0'а' & [, к, л, а] & а
\\ \hline
0'д' & [, к, л, а, д] & д
\\ \hline
0' ' & [, к, л, а, д,  ] &  
\\ \hline
0'с' & [, к, л, а, д,  , с] & с
\\ \hline
1'л' & [, к, л, а, д,  , с, кл] & кл
\\ \hline
3'д' & [, к, л, а, д,  , с, кл, ад] & ад
\\ \hline
5'л' & [, к, л, а, д,  , с, кл, ад,  л] &  л
\\ \hline
8' ' & [, к, л, а, д,  , с, кл, ад,  л, ад ] & ад 
\\ \hline
2'а' & [, к, л, а, д,  , с, кл, ад,  л, ад , ла] & ла
\\ \hline
4'ь' & [, к, л, а, д,  , с, кл, ад,  л, ад , ла, дь] & дь
\\ \hline
0'я' & [, к, л, а, д,  , с, кл, ад,  л, ад , ла, дь, я] & я
\\ \hline
\end{tabular}
\end{table}

Результат: клад склад лад ладья
\pagebreak
\subsection{Вариант №0}
\paragraph{Задание 1. Блочный хаффман \\}

Строка ОККОЛТКККК, размер блока: 2
\begin{center}
 \begin{tabular}{ |c|c|l| } 
  \hline
     Буква & Вероятность & Код\\ \hline
К & 0.60 & 1\\\hline
О & 0.20 & 00\\\hline
Т & 0.10 & 010\\\hline
Л & 0.10 & 011
\\ \hline \end{tabular}
\end{center}
Энтропия алфавита: 1.5710
\begin{center}
 \begin{tabular}{ |c|c|l| } 
  \hline
     Блок & Вероятность & Код\\ \hline
КК & 0.36 & 11\\\hline
КО & 0.12 & 010\\\hline
ОК & 0.12 & 011\\\hline
КЛ & 0.06 & 1000\\\hline
КТ & 0.06 & 1001\\\hline
ТК & 0.06 & 1010\\\hline
ЛК & 0.06 & 1011\\\hline
ОО & 0.04 & 0000\\\hline
ЛО & 0.02 & 00010\\\hline
ОТ & 0.02 & 00011\\\hline
ТО & 0.02 & 00100\\\hline
ОЛ & 0.02 & 00101\\\hline
ТТ & 0.01 & 001100\\\hline
ЛЛ & 0.01 & 001101\\\hline
ЛТ & 0.01 & 001110\\\hline
ТЛ & 0.01 & 001111
\\ \hline \end{tabular}
\end{center}
Бит на символ при посимвольном кодировании: 1.6000, при блочном: 1.6000

\includegraphics[width=0.5\linewidth]{/home/fizlrock/data/files/backup/code_backup/hobby/algoritms/LabExecutor/app/./doc_src/images/438221091.jpg}

\includegraphics[width=0.9\linewidth]{/home/fizlrock/data/files/backup/code_backup/hobby/algoritms/LabExecutor/app/./doc_src/images/1451856705.jpg}
\pagebreak
\paragraph{Задание 2. Сжать адаптивным хаффманом\\}

Строка: 
ABCCDDDDBB\\
Результат: 'A' 0'B' 00'C' 101 110'D' 1101 10 0 1011 111

\includegraphics[width=0.8\linewidth]{/home/fizlrock/data/files/backup/code_backup/hobby/algoritms/LabExecutor/app/./doc_src/images/787847712.jpg}

\includegraphics[width=0.8\linewidth]{/home/fizlrock/data/files/backup/code_backup/hobby/algoritms/LabExecutor/app/./doc_src/images/16237899.jpg}

\includegraphics[width=0.8\linewidth]{/home/fizlrock/data/files/backup/code_backup/hobby/algoritms/LabExecutor/app/./doc_src/images/1396736571.jpg}

\includegraphics[width=0.8\linewidth]{/home/fizlrock/data/files/backup/code_backup/hobby/algoritms/LabExecutor/app/./doc_src/images/520129913.jpg}

\includegraphics[width=0.8\linewidth]{/home/fizlrock/data/files/backup/code_backup/hobby/algoritms/LabExecutor/app/./doc_src/images/1821158385.jpg}

\includegraphics[width=0.8\linewidth]{/home/fizlrock/data/files/backup/code_backup/hobby/algoritms/LabExecutor/app/./doc_src/images/692676724.jpg}

\includegraphics[width=0.8\linewidth]{/home/fizlrock/data/files/backup/code_backup/hobby/algoritms/LabExecutor/app/./doc_src/images/1273060481.jpg}

\includegraphics[width=0.8\linewidth]{/home/fizlrock/data/files/backup/code_backup/hobby/algoritms/LabExecutor/app/./doc_src/images/458013996.jpg}

\includegraphics[width=0.8\linewidth]{/home/fizlrock/data/files/backup/code_backup/hobby/algoritms/LabExecutor/app/./doc_src/images/1234777694.jpg}

\includegraphics[width=0.8\linewidth]{/home/fizlrock/data/files/backup/code_backup/hobby/algoritms/LabExecutor/app/./doc_src/images/2058182225.jpg}
\pagebreak
\paragraph{Задание 3.1}

Закодировать сообщение методом LZ77\\
Строка:СКЛАД\_КЛАД\_КЛАДЕЗЬ\\
Результат: <0,0,С> <0,0,К> <0,0,Л> <0,0,А> <0,0,Д> <0,0,\_> <5,5,К> <0,3,Е> <0,0,З> <0,0,Ь>\\
\begin{table}[h!]
\centering
\begin{tabular}{|c|c|c|c|c|c|c|c|c|c|c|c|c|c|c|c|c|} 
\hline
\multicolumn{10}{|c|}{Cловарь} & \multicolumn{6}{c|}{Буфер} & Код  \\ \hline
  &   &   &   &   &   &   &   &   &   & \cellcolor[HTML]{8CE4F6} С & К & Л & А & Д &   & <0,0,С>
\\ \hline
  &   &   &   &   &   &   &   &   & С & \cellcolor[HTML]{8CE4F6} К & Л & А & Д &   & К & <0,0,К>
\\ \hline
  &   &   &   &   &   &   &   & С & К & \cellcolor[HTML]{8CE4F6} Л & А & Д &   & К & Л & <0,0,Л>
\\ \hline
  &   &   &   &   &   &   & С & К & Л & \cellcolor[HTML]{8CE4F6} А & Д &   & К & Л & А & <0,0,А>
\\ \hline
  &   &   &   &   &   & С & К & Л & А & \cellcolor[HTML]{8CE4F6} Д &   & К & Л & А & Д & <0,0,Д>
\\ \hline
  &   &   &   &   & С & К & Л & А & Д & \cellcolor[HTML]{8CE4F6}   & К & Л & А & Д &   & <0,0,\_>
\\ \hline
  &   &   &   & С & \cellcolor[HTML]{FFFF00} К & \cellcolor[HTML]{FFFF00} Л & \cellcolor[HTML]{FFFF00} А & \cellcolor[HTML]{FFFF00} Д & \cellcolor[HTML]{FFFF00}   & \cellcolor[HTML]{FFFF00} К & \cellcolor[HTML]{FFFF00} Л & \cellcolor[HTML]{FFFF00} А & \cellcolor[HTML]{FFFF00} Д & \cellcolor[HTML]{FFFF00}   & \cellcolor[HTML]{8CE4F6} К & <5,5,К>
\\ \hline
\cellcolor[HTML]{FFFF00} Л & \cellcolor[HTML]{FFFF00} А & \cellcolor[HTML]{FFFF00} Д &   & К & Л & А & Д &   & К & \cellcolor[HTML]{FFFF00} Л & \cellcolor[HTML]{FFFF00} А & \cellcolor[HTML]{FFFF00} Д & \cellcolor[HTML]{8CE4F6} Е & З & Ь & <0,3,Е>
\\ \hline
К & Л & А & Д &   & К & Л & А & Д & Е & \cellcolor[HTML]{8CE4F6} З & Ь &   &   &   &   & <0,0,З>
\\ \hline
Л & А & Д &   & К & Л & А & Д & Е & З & \cellcolor[HTML]{8CE4F6} Ь &   &   &   &   &   & <0,0,Ь>
\\ \hline
\end{tabular}
\end{table}

\paragraph{Задание 3.2}

Закодировать сообщение методом LZSS\\
Строка:СКЛАД\_КЛАД\_КЛАДЕЗЬ\\
Результат: 0'С' 0'К' 0'Л' 0'А' 0'Д' 0'\_' 1<5,5> 1<0,4> 0'Е' 0'З' 0'Ь'\\
\begin{table}[h!]
\centering
\begin{tabular}{|c|c|c|c|c|c|c|c|c|c|c|c|c|c|c|c|c|}
\hline
\multicolumn{10}{|c|}{Cловарь} & \multicolumn{6}{c|}{Буфер} & Код  \\ \hline
  &   &   &   &   &   &   &   &   &   & С & К & Л & А & Д & \_ & 0'С'\\ \hline
  &   &   &   &   &   &   &   &   & С & К & Л & А & Д & \_ & К & 0'К'\\ \hline
  &   &   &   &   &   &   &   & С & К & Л & А & Д & \_ & К & Л & 0'Л'\\ \hline
  &   &   &   &   &   &   & С & К & Л & А & Д & \_ & К & Л & А & 0'А'\\ \hline
  &   &   &   &   &   & С & К & Л & А & Д & \_ & К & Л & А & Д & 0'Д'\\ \hline
  &   &   &   &   & С & К & Л & А & Д & \_ & К & Л & А & Д & \_ & 0'\_'\\ \hline
  &   &   &   & С & \cellcolor[HTML]{FFFF00} К & \cellcolor[HTML]{FFFF00} Л & \cellcolor[HTML]{FFFF00} А & \cellcolor[HTML]{FFFF00} Д & \cellcolor[HTML]{FFFF00} \_ & \cellcolor[HTML]{FFFF00} К & \cellcolor[HTML]{FFFF00} Л & \cellcolor[HTML]{FFFF00} А & \cellcolor[HTML]{FFFF00} Д & \cellcolor[HTML]{FFFF00} \_ & К & 1<5,5>\\ \hline
\cellcolor[HTML]{FFFF00} К & \cellcolor[HTML]{FFFF00} Л & \cellcolor[HTML]{FFFF00} А & \cellcolor[HTML]{FFFF00} Д & \_ & К & Л & А & Д & \_ & \cellcolor[HTML]{FFFF00} К & \cellcolor[HTML]{FFFF00} Л & \cellcolor[HTML]{FFFF00} А & \cellcolor[HTML]{FFFF00} Д & Е & З & 1<0,4>\\ \hline
\_ & К & Л & А & Д & \_ & К & Л & А & Д & Е & З & Ь &   &   &   & 0'Е'\\ \hline
К & Л & А & Д & \_ & К & Л & А & Д & Е & З & Ь &   &   &   &   & 0'З'\\ \hline
Л & А & Д & \_ & К & Л & А & Д & Е & З & Ь &   &   &   &   &   & 0'Ь'\\ \hline
\end{tabular}
\end{table}

\paragraph{Задание 3.3}

Закодировать сообщение методом LZ78\\
Строка:СКЛАД\_КЛАД\_КЛАДЕЗЬ\\
\begin{table}[h!]
\centering
\begin{tabular}{|c|c|c|} 
\hline
 Входная фраза (в словарь) & Код & Позиция словаря \\ \hline

 &  & 0 \\ \hline
С & 0'С' & 1 \\ \hline
К & 0'К' & 2 \\ \hline
Л & 0'Л' & 3 \\ \hline
А & 0'А' & 4 \\ \hline
Д & 0'Д' & 5 \\ \hline
\_ & 0'\_' & 6 \\ \hline
КЛ & 2'Л' & 7 \\ \hline
АД & 4'Д' & 8 \\ \hline
\_К & 6'К' & 9 \\ \hline
ЛА & 3'А' & 10 \\ \hline
ДЕ & 5'Е' & 11 \\ \hline
З & 0'З' & 12 \\ \hline
Ь & 0'Ь' & 13 \\ \hline
\end{tabular}
\end{table}

Результат: 0'С' 0'К' 0'Л' 0'А' 0'Д' 0'\_' 2'Л' 4'Д' 6'К' 3'А' 5'Е' 0'З' 0'Ь'\\
\pagebreak
\paragraph{Задание 4. Арифметическое кодирование\\}

Исходная строка: ABCCDDDDBB\
\begin{center}
 \begin{tabular}{ |c|c| } 
  \hline
     Буква & Вероятность \\ \hline
D & 0.40\\\hline
B & 0.30\\\hline
C & 0.20\\\hline
A & 0.10
\\ \hline \end{tabular}
\end{center}
\begin{center}
 \begin{tabular}{ |c|c|c| } 
  \hline
     Буква & Начало & Конец \\ \hline
D & 0.00 & 0.40\\\hline
B & 0.40 & 0.70\\\hline
C & 0.70 & 0.90\\\hline
A & 0.90 & 1.00
\\ \hline \end{tabular}
\end{center}
\begin{center}
 \begin{tabular}{ |c|c|c|c| } 
  \hline
     Буква & delta & min & max \\ \hline
A & 0.1000000000 & 0.9000000000 & 1.0000000000\\\hline
B & 0.0300000000 & 0.9400000000 & 0.9700000000\\\hline
C & 0.0060000000 & 0.9610000000 & 0.9670000000\\\hline
C & 0.0012000000 & 0.9652000000 & 0.9664000000\\\hline
D & 0.0004800000 & 0.9652000000 & 0.9656800000\\\hline
D & 0.0001920000 & 0.9652000000 & 0.9653920000\\\hline
D & 0.0000768000 & 0.9652000000 & 0.9652768000\\\hline
D & 0.0000307200 & 0.9652000000 & 0.9652307200\\\hline
B & 0.0000092160 & 0.9652122880 & 0.9652215040\\\hline
B & 0.0000027648 & 0.9652159744 & 0.9652187392
\\ \hline \end{tabular}
\end{center}
Результат: 965216
\pagebreak
\paragraph{Задание 5.1}

\\ 

Декодировать сообщение методом адаптивного хаффмана \\
Строка: 
'S'0'K'00'T'100'R'10111101110111110\\
Результат: SKTRSSRKKSS

\includegraphics[width=0.8\linewidth]{/home/fizlrock/data/files/backup/code_backup/hobby/algoritms/LabExecutor/app/./doc_src/images/859772202.jpg}

\includegraphics[width=0.8\linewidth]{/home/fizlrock/data/files/backup/code_backup/hobby/algoritms/LabExecutor/app/./doc_src/images/734319816.jpg}

\includegraphics[width=0.8\linewidth]{/home/fizlrock/data/files/backup/code_backup/hobby/algoritms/LabExecutor/app/./doc_src/images/1648819435.jpg}

\includegraphics[width=0.8\linewidth]{/home/fizlrock/data/files/backup/code_backup/hobby/algoritms/LabExecutor/app/./doc_src/images/594986735.jpg}

\includegraphics[width=0.8\linewidth]{/home/fizlrock/data/files/backup/code_backup/hobby/algoritms/LabExecutor/app/./doc_src/images/943958581.jpg}

\includegraphics[width=0.8\linewidth]{/home/fizlrock/data/files/backup/code_backup/hobby/algoritms/LabExecutor/app/./doc_src/images/271850073.jpg}

\includegraphics[width=0.8\linewidth]{/home/fizlrock/data/files/backup/code_backup/hobby/algoritms/LabExecutor/app/./doc_src/images/1444140606.jpg}

\includegraphics[width=0.8\linewidth]{/home/fizlrock/data/files/backup/code_backup/hobby/algoritms/LabExecutor/app/./doc_src/images/1635713955.jpg}

\includegraphics[width=0.8\linewidth]{/home/fizlrock/data/files/backup/code_backup/hobby/algoritms/LabExecutor/app/./doc_src/images/941664167.jpg}

\includegraphics[width=0.8\linewidth]{/home/fizlrock/data/files/backup/code_backup/hobby/algoritms/LabExecutor/app/./doc_src/images/745040039.jpg}

\includegraphics[width=0.8\linewidth]{/home/fizlrock/data/files/backup/code_backup/hobby/algoritms/LabExecutor/app/./doc_src/images/1604652891.jpg}
\pagebreak
\paragraph{Задание 5.3 Декодировать строку(LZSS)\\}

Исходная строка: [0'б'] [0'а'] [0'т'] [0'у'] [1<8,1>] [0' '] [1<6,3>] [1<4,1>] [1<5,1>] [1<2,2>] [0'к'] [0'а'] [1<0,3>] [1<5,2>] [0'н']\\
\begin{table}[h!]
\centering
\begin{tabular}{|c|c|c|}
\hline
 Cловарь & Буфер & Код  \\ \hline
0'б' & [ ,  ,  ,  ,  ,  ,  ,  ,  , б] & б
\\ \hline
0'а' & [ ,  ,  ,  ,  ,  ,  ,  , б, а] & а
\\ \hline
0'т' & [ ,  ,  ,  ,  ,  ,  , б, а, т] & т
\\ \hline
0'у' & [ ,  ,  ,  ,  ,  , б, а, т, у] & у
\\ \hline
1<8,1> & [ ,  ,  ,  ,  , б, а, т, у, т] & т
\\ \hline
0' ' & [ ,  ,  ,  , б, а, т, у, т,  ] &  
\\ \hline
1<6,3> & [ , б, а, т, у, т,  , т, у, т] & тут
\\ \hline
1<4,1> & [б, а, т, у, т,  , т, у, т, у] & у
\\ \hline
1<5,1> & [а, т, у, т,  , т, у, т, у,  ] &  
\\ \hline
1<2,2> & [у, т,  , т, у, т, у,  , у, т] & ут
\\ \hline
0'к' & [т,  , т, у, т, у,  , у, т, к] & к
\\ \hline
0'а' & [ , т, у, т, у,  , у, т, к, а] & а
\\ \hline
1<0,3> & [т, у,  , у, т, к, а,  , т, у] &  ту
\\ \hline
1<5,2> & [ , у, т, к, а,  , т, у, к, а] & ка
\\ \hline
0'н' & [у, т, к, а,  , т, у, к, а, н] & н
\\ \hline
\end{tabular}
\end{table}

Результат: батут туту утка тукан
\pagebreak
\paragraph{Задание 5.4 Декодировать строку(LZ78)\\}

Исходная строка: [0'к'] [0'л'] [0'а'] [0'д'] [0' '] [0'с'] [1'л'] [3'д'] [5'л'] [8' '] [2'а'] [4'ь'] [0'я']\\
\begin{table}[h!]
\centering
\begin{tabular}{|c|c|c|}
\hline
 Cловарь & Буфер & Код  \\ \hline
 & [] & 
\\ \hline
0'к' & [, к] & к
\\ \hline
0'л' & [, к, л] & л
\\ \hline
0'а' & [, к, л, а] & а
\\ \hline
0'д' & [, к, л, а, д] & д
\\ \hline
0' ' & [, к, л, а, д,  ] &  
\\ \hline
0'с' & [, к, л, а, д,  , с] & с
\\ \hline
1'л' & [, к, л, а, д,  , с, кл] & кл
\\ \hline
3'д' & [, к, л, а, д,  , с, кл, ад] & ад
\\ \hline
5'л' & [, к, л, а, д,  , с, кл, ад,  л] &  л
\\ \hline
8' ' & [, к, л, а, д,  , с, кл, ад,  л, ад ] & ад 
\\ \hline
2'а' & [, к, л, а, д,  , с, кл, ад,  л, ад , ла] & ла
\\ \hline
4'ь' & [, к, л, а, д,  , с, кл, ад,  л, ад , ла, дь] & дь
\\ \hline
0'я' & [, к, л, а, д,  , с, кл, ад,  л, ад , ла, дь, я] & я
\\ \hline
\end{tabular}
\end{table}

Результат: клад склад лад ладья
\pagebreak
\end{document}