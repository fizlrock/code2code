
\documentclass[a4paper, 12pt]{article}
\usepackage[left=3cm,right=1.5cm,top=2cm,bottom=2cm]{geometry}
\usepackage[T2A]{fontenc}
\usepackage{graphicx}
\usepackage[table]{xcolor}

%Hyphenation rules
%--------------------------------------
\usepackage{hyphenat}
\hyphenation{ма-те-ма-ти-ка вос-ста-нав-ли-вать}
%--------------------------------------
\usepackage[english, russian]{babel}
\begin{document}
 
\tableofcontents

\begin{abstract}
  Это вводный абзац в начале документа.
\end{abstract}
 
\section{Задание}
\begin{enumerate}
\item Составить таблицу кодов блоков для метода Хаффмана с блокированием. Вероятности букв считать по фрагменту сообщения в задании. Длина блока указана. Вычислить EX, ML(X), ML(Xбл). Здесь EX – энтропия алфавита из букв сообщения, ML(X) – среднее количество элементарных символов на букву при сжатии методом Хаффмана, ML(Xбл) – среднее количество элементарных символов на букву при сжатии методом Хаффмана с блокированием. 
\item Сжать сообщение адаптивным методом Хаффмана. 
\item Сжать сообщение методами LZ77, LZSS, LZ78  Для методов LZ77, LZSS размер словаря – 10 символов, буфера – 6 символов. Для метода LZ78 размер словаря 32 записи. 
\item Сжать сообщение из задания №2 арифметическим методом. 
\item Распаковать сообщения, сжатые адаптивным методом Хаффмана, методами LZ77, LZSS, LZ78 и арифметическим методом. Для методов LZ77, LZSS размер словаря – 10 символов. Для метода LZ78 размер словаря – 16 записей. При декодировании таблица состоит из следующих столбцов: «Код», «Словарь» и «Выходной поток».
\end{enumerate}
\pagebreak
\section{Решение}
